\documentclass[12pt,a4paper]{article} 
\usepackage[T1]{fontenc}
\usepackage{textcomp}
\usepackage{makeidx}

\usepackage[sc, noBBpl]{mathpazo}
\usepackage{dsfont} % for indicator function
\usepackage[deluxe]{otf}
\usepackage[utf8]{inputenc}

\usepackage{amsmath}
\usepackage{amsfonts}
\usepackage{amssymb}
\usepackage{amsthm}
\usepackage{mathtools}
\usepackage{mathrsfs}

\usepackage{listings} % for programming codes

\usepackage{bm}
\usepackage[dvipdfmx]{color}
\pagestyle{plain} 
\usepackage{float}
\usepackage[dvipdfmx]{graphicx}

\usepackage[left=15mm, right=18mm, top=20mm, bottom=20mm]{geometry}

\setlength{\footskip}{40pt} 
\setlength{\abovecaptionskip}{0pt}
\setlength{\belowcaptionskip}{0pt}
\renewcommand{\baselinestretch}{1.05}

\newcommand{\argmax}{\mathop{\rm arg max}\limits}
\newcommand{\argmin}{\mathop{\rm arg min}\limits}
\newcommand{\st}{\mathop{\rm s.t.}\limits}
\newcommand{\fsize}[1]{\fontsize{#1}{#1}\selectfont}

% custom color
\usepackage[dvipdfmx]{xcolor}

% colors
\definecolor{darkred}{rgb}{.40,.00,.00}
\definecolor{navy}{rgb}{.00,.00,.40}
\definecolor{darknavy}{rgb}{.00,.00,.30}
\definecolor{lightnavy}{rgb}{.10,.10,.40}

\newcommand{\navy}{\color{navy}}
\newcommand{\deepred}{\color{darkred}}
\newcommand{\emred}[1]{\textbf{\deepred #1}}
\newcommand{\emblue}[1]{\textbf{\navy #1}}

% title and tableofcontents
\usepackage[titles]{tocloft}
\setlength{\cftbeforesecskip}{2pt}
\setlength{\cftbeforesubsecskip}{0pt}
\renewcommand{\cftsecfont}{\normalsize\mcfamily}
\renewcommand{\cftsubsecfont}{\normalsize\mcfamily}
\renewcommand{\cftsubsecfont}{\normalsize\mcfamily}
\renewcommand{\contentsname}{\large\mcfamily\bfseries \S\hspace{0.5\baselineskip}Contents\\[-\baselineskip]}
\cftsetindents{section}{10.0pt}{15.0pt}
\cftsetindents{subsection}{20.0pt}{30.0pt}
\cftpagenumbersoff{section}
\cftpagenumbersoff{subsection}
\cftpagenumbersoff{subsubsection}

\makeatletter
\def\vhrulefill#1{\leavevmode\leaders\hrule\@height#1\hfill \kern\z@}
\makeatother

\makeatletter
\def\maketitle{%
	\begin{center}\leavevmode
	\normalfont
	{\normalsize\raggedleft \@date\par}%
	\vskip -0.7\baselineskip
	\vhrulefill{0.2pt}\par
	\vskip 0.3\baselineskip
	\renewcommand{\baselinestretch}{0.80}
	{\color{darknavy}\bfseries\Large\raggedright \@title\par}%
	\renewcommand{\baselinestretch}{1.00}
	\vskip 0.2\baselineskip
	\ifx\@subtitle\@empty\else%
	{\normalsize\raggedright \@subtitle\par}%
	\fi
	\ifx\@author\@empty\else%
	\vskip 0.5\baselineskip
	{\normalsize\raggedleft \@author\par}%
	\fi
	\end{center}%
	\vskip 1.0\baselineskip
}
\def\subtitle#1{\gdef\@subtitle{#1}}
\let\@subtitle\@empty
\makeatother

% section
\newcommand{\Secindent}{0.0\baselineskip}
\newcommand{\Subsecindent}{0.0\baselineskip}
\newcommand{\lmargini}{1.3\baselineskip}
\newcommand{\lmarginii}{0.5\baselineskip}
\newcommand{\lmarginiii}{0.2\baselineskip}
\def\theenumi{\arabic{enumi}}
\def\theenumii{\alph{enumii}}
\renewcommand{\labelenumi}{\hfill\hspace{60pt}\theenumi.}
\renewcommand{\labelenumii}{\hspace{30pt}\theenumii.}
\renewcommand{\labelitemi}{\small\textbullet\hspace{2pt}}
\renewcommand{\labelitemii}{\color{gray}\small$\circ$\hspace{2pt}}
\renewcommand{\labelitemiii}{\textendash\hspace{1pt}}
\makeatletter
\renewcommand{\section}{\@startsection%
	{section}{1}{\Secindent}% name, depth, indent \z@=0pt
	{-1.0\baselineskip \@plus1.5\baselineskip \@minus0.5\baselineskip}% beforeskip (no indent if negative)
	{0.5\baselineskip \@plus0.8\baselineskip \@minus0.3\baselineskip}% afterskip (no line break if negative)
	{\reset@font\large\bfseries\scshape\color{darknavy}}% font
}
\renewcommand{\subsection}{\@startsection%
	{subsection}{2}{\Subsecindent}%
	{-0.5\baselineskip \@plus0.7\baselineskip \@minus0.4\baselineskip}%
	{0.3\baselineskip \@plus0.5\baselineskip \@minus0.2\baselineskip}%
	{\reset@font\normalsize\bfseries\scshape\color{darknavy}}% 
}
\renewcommand{\thesection}{\@arabic\c@section} % arabic, roman, Roman, alph, Alph, fnsymbol
\renewcommand{\thesubsection}{\thesection\hspace{0pt}.\hspace{0pt}\@arabic\c@subsection}
\makeatother

\makeatletter
\def\@listi{%
	\topsep = 0.2\baselineskip
	\partopsep = 0.0\baselineskip
	\parsep = 0.0\baselineskip
	\itemsep = 0.5\baselineskip
	\leftmargin = 1.5zw
	\labelsep = 0.5zw
	\itemindent = 0pt
	\labelwidth\leftmargin \advance\labelwidth-\labelsep
	\listparindent = 0zw
	\rightmargin = 0pt
	\reset@font\normalsize\normalfont
}
\let\@listI\@listi
\@listi
\def\@listii{%
	\topsep = 0.25\baselineskip
	\partopsep = 0.0\baselineskip
	\parsep = 0.0\baselineskip
	\itemsep = 0.3\baselineskip
	\leftmargin = 0.5zw
	\labelsep = .5zw
	\itemindent = 0pt
	\labelwidth\leftmargin \advance\labelwidth-\labelsep
	\listparindent = 0zw
	\rightmargin = 0pt
	\reset@font\normalsize\normalfont%
}
\def\@listiii{%
	\topsep = 0.30\baselineskip
	\partopsep = 0.0\baselineskip
	\parsep = 0.1\baselineskip
	\itemsep = 0.05\baselineskip
	\leftmargin = 1.1zw
	\labelsep = .5zw
	\itemindent = 0pt
	\labelwidth\leftmargin \advance\labelwidth-\labelsep
	\listparindent = 0zw
	\rightmargin = 0pt
	\reset@font\normalsize\normalfont%
}
\def\@listiv{%
	\topsep = 0.0\baselineskip
	\partopsep = 0.0\baselineskip
	\parsep = 0.0\baselineskip
	\itemsep = 0.0\baselineskip
	\leftmargin = 0.9zw
	\labelsep = .5zw
	\itemindent = 0pt
	\labelwidth\leftmargin \advance\labelwidth-\labelsep
	\listparindent = 0zw
	\rightmargin = 0pt
	\reset@font\normalsize\normalfont%
}
\makeatother

\newcommand{\R}{\ensuremath{\mathbb{R}}}
\newcommand{\E}{\ensuremath{\mathbb{E}}}
\newcommand{\V}{\ensuremath{\mathbb{V}}}
\newcommand{\Cov}{\ensuremath{\text{Cov}}}
\newcommand{\N}{\ensuremath{\mathbb{N}}}
\newcommand{\Z}{\ensuremath{\mathbb{Z}}}
\let\epsilon\varepsilon

\let\origitem\item
\renewcommand{\item}{\normalfont\origitem}

\usepackage[%
backref=page,
dvipdfmx,
bookmarks=true,bookmarksnumbered=true,bookmarkstype=toc,
colorlinks=true,
linkcolor=navy,
citecolor=navy,
filecolor=blue,
pagecolor=blue,
urlcolor=blue
]{hyperref}
\AtBeginDvi{\special{pdf:tounicode EUC-UCS2}}

\renewcommand*{\backref}[1]{}
\renewcommand*{\backrefalt}[4]{%
  \ifcase #1 (Not cited.)%
  \or        [#2]%
  \else      [#2]%
  \fi}
 

\title{Gaussian distribution}
\subtitle{Introduction to dynamical systems~\#10}
\author{Hiroaki Sakamoto}
\date{\today}

\begin{document}
\maketitle
\tableofcontents

\section{Elementary statistics}

\subsection{Density}

\begin{itemize}

\item \textbf{Probability and density}
  \begin{itemize}
  \item Let $\bm{x} = (x_{1},x_{2},\ldots, x_{m})\in \R^{m}$ be a random vector (i.e., values of $x_{1}, x_{2},\ldots, x_{m}$ are randomly determined)
  \item For any subset $A\subset \R^{m}$,
    we write
    \begin{equation}\nonumber%\label{eq:}%
      \text{Pr}(\bm{x}\in A) := \text{probability of $\bm{x}$ being in the set $A$}
    \end{equation}
  \item If there exists a function $f:\R^{m}\to \R$ such that
    \begin{equation}\nonumber%\label{eq:}%
      \int_{A}f(\bm{x})d\bm{x}
      =
      \text{Pr}(\bm{x}\in A)
      \quad \forall A\subset \R^{m},
    \end{equation}
    we call it the (joint) \emph{density} of $\bm{x}$
    and write $p_{X}(\bm{x}):=f(\bm{x})$
  \item Density $p_{X}(\bar{\bm{x}})$
    represents the likelihood of $\bm{x}$ taking a particular value $\bar{\bm{x}}$
    in the sense that
    \begin{equation}
      \lim_{\Delta \bm{x} \to \bm{0}}\frac{\text{Pr}(\bm{x}\in \Delta(\bar{\bm{x}}))}{|\Delta(\bar{\bm{x}})|}
      =
      \lim_{\Delta x_{m} \to 0}\cdots \lim_{\Delta x_{1} \to 0}\frac{\int_{\bar{x}_{m}}^{\bar{x}_{m}+\Delta x_{m}}\cdots \int_{\bar{x}_{1}}^{\bar{x}_{1}+\Delta x_{1}}p_{X}(\bm{x})dx_{1}\cdots dx_{m}}{\Delta x_{1}\times \Delta x_{2}\times \cdots \times\Delta x_{m}}
      = p_{X}(\bar{\bm{x}})
    \nonumber%\label{eq:}%
    \end{equation}
    where $\Delta(\bar{\bm{x}}) := \prod_{i=1}^{m}[\bar{x}_{i}, \bar{x}_{i}+\Delta x_{i}]$
  \end{itemize}

\item \textbf{Expectation and variance}
  \begin{itemize}
  \item For a random vector $\bm{x}\in \R^{m}$
    and a function $\psi:\R^{m}\to\R^{n}$,
    the \emph{expectation} of $\psi(\bm{x})$ is
    \begin{equation}\nonumber%\label{eq:}%
      \mathbb{E}[\psi(\bm{x})] := \int_{\R^{m}}\psi(\bm{x})p_{X}(\bm{x})d\bm{x} \in \R^{n}
    \end{equation}
    and the \emph{(co)variance} of $\psi(\bm{x})$ is
    \begin{equation}\nonumber%\label{eq:}%
      \mathbb{V}[\psi(\bm{x})] := \mathbb{E}\left[(\psi(\bm{x}) - \mathbb{E}[\psi(\bm{x})])(\psi(\bm{x}) - \mathbb{E}[\psi(\bm{x})])^{\top}\right] \in \R^{n\times n}
    \end{equation}
  \item Note that the probability can be expressed using expectation:
    \begin{equation}\nonumber%\label{eq:}%
      \text{Pr}(\bm{x}\in A)
      = \int_{A}p_{X}(\bm{x})d\bm{x}
      = \int_{\R^{m}}\mathds{1}_{A}(\bm{x})p_{X}(\bm{x})d\bm{x}
      = \mathbb{E}\left[\mathds{1}_{A}(\bm{x})\right]
      \quad \forall A\subset \R^{m}
    \end{equation}

  \end{itemize}

\end{itemize}

\subsection{Conditional density}

\begin{itemize}

\item \textbf{Conditional probability and conditional density}
  \begin{itemize}
  \item Let $\bm{x} = (x_{1},x_{2},\ldots, x_{m})\in \R^{m}$ be a random vector
  \item For each $B\subset \R^{m}$ such that $\text{Pr}(\bm{x}\in B)\neq 0$,
    we define the \emph{conditional probability} of $\bm{x}\in A$ given that $\bm{x}\in B$ as
    \begin{equation}\nonumber%\label{eq:}%
      \text{Pr}(\bm{x}\in A|\bm{x}\in B)
      := \frac{\text{Pr}(\bm{x}\in A\cap B)}{\text{Pr}(\bm{x}\in B)}
      \quad \forall A\subset \R^{m}
    \end{equation}
  \item For a given set $B\subset \R^{m}$,
    if there exists a function
    $f:\R^{m}\to \R$ such that
    \begin{equation}\label{eq:xB}%
      \int_{A}f(\bm{x})d\bm{x}
      =
      \text{Pr}(\bm{x}\in A|\bm{x}\in B)
      \quad \forall A\subset \R^{m},
    \end{equation}
    we call it the \emph{conditional density} of $\bm{x}$ given $\bm{x}\in B$
    and write $p_{X}(\bm{x}|\bm{x}\in B) := f(\bm{x})$
  \item For any $B\subset \R^{m}$ such that $\text{Pr}(\bm{x}\in B)\neq 0$, we have
    \begin{equation}
      \int_{A}\frac{\mathds{1}_{B}(\bm{x})p_{X}(\bm{x})}{\int_{B}p_{X}(\bm{x})d\bm{x}}d\bm{x}
      = 
      \frac{\int_{A\cap B}p_{X}(\bm{x})d\bm{x}}{\int_{B}p_{X}(\bm{x})d\bm{x}}
      = 
      \frac{\text{Pr}(\bm{x}\in A\cap B)}{\text{Pr}(\bm{x}\in B)}
      =
      \text{Pr}(\bm{x}\in A|\bm{x}\in B)
      \nonumber%\label{eq:}%
      \quad \forall A\subset \R^{m},
    \end{equation}
    meaning that the conditional density is given by
    \begin{equation}\label{eq:pxB}%
      p_{X}(\bm{x}|\bm{x}\in B)
      = \frac{\mathds{1}_{B}(\bm{x})p_{X}(\bm{x})}{\int_{B}p_{X}(\bm{x})d\bm{x}}
      =
      \begin{cases}
        \frac{p_{X}(\bm{x})}{\int_{B}p_{X}(\bm{x})d\bm{x}} & \text{if $\bm{x}\in B$} \\
        0 & \text{otherwise} \\
      \end{cases}
    \end{equation}
  \item Obviously,
    \begin{equation}\nonumber%\label{eq:}%
      \text{Pr}(\bm{x}\in A|\bm{x}\in \R^{m})
      = 
      \frac{\text{Pr}(\bm{x}\in A\cap \R^{m})}{\text{Pr}(\bm{x}\in \R^{m})}
      =
      \text{Pr}(\bm{x}\in A)
      \quad \forall A \subset \R^{m}
    \end{equation}
    and
    \begin{equation}\nonumber%\label{eq:}%
      p_{X}(\bm{x}|\bm{x}\in \R^{m})
      = \frac{\mathds{1}_{\R^{m}}(\bm{x})p_{X}(\bm{x})}{\int_{\R^{m}}p_{X}(\bm{x})d\bm{x}}
      = p_{X}(\bm{x})
      \quad \forall \bm{x} \in \R^{m}
    \end{equation}
  \end{itemize}

\item \textbf{Conditional expectation}
  \begin{itemize}

  \item For any function $\psi:\R^{m}\to\R^{n}$,
    the \emph{conditional expectation} of $\psi(\bm{x})$ given $\bm{x}\in B\subset \R^{m}$ is
    \begin{equation}\label{eq:ExB}%
      \mathbb{E}\left[\psi(\bm{x})| \bm{x}\in B \right]
      := \int_{\R^{m}}\psi(\bm{x})p_{X}(\bm{x}|\bm{x}\in B)d\bm{x}
      =
      \frac{1}{\int_{B}p_{X}(\bm{x})d\bm{x}}\int_{B}\psi(\bm{x})p_{X}(\bm{x})d\bm{x}
    \end{equation}
  \item For any partition $B_{1},\ldots, B_{I}$ of $\R^{m}$ (i.e., $B_{i}\cap B_{j}=\varnothing$ for any $i\neq j$ and $\cup_{i=1}^{I}B_{i}=\R^{m_{}}$),
    \begin{equation}\nonumber%\label{eq:}%
      \sum_{i=1}^{I}\mathbb{E}\left[\psi(\bm{x})| \bm{x}\in B_{i} \right]\text{Pr}(\bm{x}\in B_{i})
      =
      \sum_{i=1}^{I}\int_{B_{i}}\psi(\bm{x})p_{X}(\bm{x})d\bm{x}
      =
      \int_{\cup_{i=1}^{I}B_{i}}\psi(\bm{x})p_{X}(\bm{x})d\bm{x}
      = \mathbb{E}\left[\psi(\bm{x})\right]
    \end{equation}
  \item Obviously
    \begin{equation}\nonumber%\label{eq:}%
      \mathbb{E}\left[\psi(\bm{x})| \bm{x}\in \R^{m} \right]
      =
      \int_{\R^{m}}\psi(\bm{x})p_{X}(\bm{x}|\bm{x}\in \R^{m})d\bm{x}
      =
      \int_{\R^{m}}\psi(\bm{x})p_{X}(\bm{x})d\bm{x}
      =
      \mathbb{E}\left[\psi(\bm{x}) \right]
    \end{equation}
  \end{itemize}

\end{itemize}

\subsection{Marginal density}

\begin{itemize}

\item \textbf{Marginal density and independence}
  \begin{itemize}
  \item Let $\bm{x} = (x_{1},x_{2},\ldots, x_{m})\in \R^{m}$ be a random vector
    with density $p_{X}(\bm{x})$
  \item Split $\bm{x}$ as $\bm{x}=(\bm{x}_{1},\bm{x}_{2})$
    where $\bm{x}_{1}\in \R^{m_{1}}$ and $\bm{x}_{2}\in \R^{m_{2}}$ with $m_{1}+m_{2}=m$
  \item If there exists a function $f:\R^{m_{i}}\to \R$ ($i=1,2$) such that
    \begin{equation}\nonumber%\label{eq:}%
      \int_{A_{i}}f(\bm{x}_{i})d\bm{x}_{i}
      =
      \text{Pr}(\bm{x}_{i}\in A_{i})
      \quad \forall A_{i}\subset \R^{m_{i}},
    \end{equation}
    we call it the \emph{(marginal) density} of $\bm{x}_{i}$
    and write $p_{X_{i}}(\bm{x}_{i}):=f(\bm{x}_{i})$
  \item Note
    \begin{equation}\nonumber%\label{eq:}%
      \text{Pr}(\bm{x}_{1}\in A_{1})
      =
      \text{Pr}((\bm{x}_{1},\bm{x}_{2})\in A_{1}\times \R^{m_{2}})
      = \int_{A_{1}}\int_{\R^{m_{2}}}p_{X}(\bm{x}_{1},\bm{x}_{2})d\bm{x}_{2}d\bm{x}_{1}
      \quad \forall A_{1}\subset \R^{m_{1}},
    \end{equation}
    meaning that
    the density of $\bm{x}_{i}$ ($i=1,2$) is
    \begin{equation}\nonumber%\label{eq:}%
      p_{X_{i}}(\bm{x}_{i}) = \int_{\R^{m_{j}}}p_{X}(\bm{x}_{i},\bm{x}_{j})d\bm{x}_{j} \quad \forall \bm{x}_{i}\in \R^{m_{i}},
      \quad j\neq i
    \end{equation}
  \item We say that $\bm{x}_{1}$ and $\bm{x}_{2}$ are \emph{independent} if
    \begin{equation}\nonumber%\label{eq:}%
      p_{X}(\bm{x}_{1},\bm{x}_{2}) =
      p_{X_{1}}(\bm{x}_{1})p_{X_{2}}(\bm{x}_{2})
      \quad \forall (\bm{x}_{1},\bm{x}_{2})\in \R^{m}
    \end{equation}
  \end{itemize}

\item \textbf{Marginal density and conditional density}

  \begin{itemize}
  \item For each $A_{2}\subset \R^{m_{2}}$ such that $\text{Pr}(\bm{x}_{2}\in A_{2})\neq 0$,
    we define the conditional probability of $\bm{x}_{1}\in A_{1}$ given $\bm{x}_{2}\in A_{2}$ is
    \begin{align}
      \text{Pr}(\bm{x}_{1}\in A_{1}|\bm{x}_{2}\in A_{2})
        & := \text{Pr}((\bm{x}_{1},\bm{x}_{2})\in A_{1}\times \R^{m_{2}}|(\bm{x}_{1},\bm{x}_{2})\in \R^{m_{1}}\times A_{2}) \nonumber \\
        & = \frac{\text{Pr}((\bm{x}_{1},\bm{x}_{2})\in (A_{1}\times \R^{m_{2}})\cap (\R^{m_{1}}\times A_{2}))}{\text{Pr}((\bm{x}_{1},\bm{x}_{2})\in \R^{m_{1}}\times A_{2})}  \nonumber \\
        & = \frac{\text{Pr}((\bm{x}_{1},\bm{x}_{2})\in A_{1}\times A_{2})}{\text{Pr}(\bm{x}_{2}\in A_{2})}
    \nonumber%\label{eq:}%
    \end{align}
  \item For a given set $A_{2}\subset \R^{m_{2}}$,
    if there exists a function
    $f:\R^{m_{1}}\to \R$ such that
    \begin{equation}\nonumber%\label{eq:x1x2A2}%
      \int_{A_{1}}f(\bm{x}_{1})d\bm{x}_{1}
      =
      \text{Pr}(\bm{x}_{1}\in A_{1}|\bm{x}_{2}\in A_{2})
      \quad \forall A_{1}\subset \R^{m_{1}},
    \end{equation}
    we call it the conditional density of $\bm{x}_{1}$ given $\bm{x}_{2}\in A_{1}$
    and write $p_{X_{1}|X_{2}}(\bm{x}_{1}|\bm{x}_{2}\in A_{2}):=f(\bm{x}_{1})$
  \item Notice that
    \begin{align}
      \int_{A_{1}}\int_{\R^{m_{2}}}p_{X}(\bm{x}|(\bm{x}_{1},\bm{x}_{2})\in \R^{m_{1}}\times A_{2})d\bm{x}_{2}d\bm{x}_{1}
        & = \int_{A_{1}\times\R^{m_{2}}}p_{X}(\bm{x}|(\bm{x}_{1},\bm{x}_{2})\in \R^{m_{1}}\times A_{2})d\bm{x} \nonumber \\
        & = \text{Pr}((\bm{x}_{1},\bm{x}_{2})\in A_{1}\times \R^{m_{2}}|(\bm{x}_{1},\bm{x}_{2})\in \R^{m_{1}}\times A_{2}) \nonumber \\
        & = \text{Pr}(\bm{x}_{1}\in A_{1}|\bm{x}_{2}\in A_{2})
      \quad \forall A_{1}\subset \R^{m_{1}},
    \nonumber%\label{eq:}%
    \end{align}
    meaning that the conditional density is more explicitly expressed as
    \begin{align}
      p_{X_{1}|X_{2}}(\bm{x}_{1}|\bm{x}_{2}\in A_{2})
        & = \int_{\R^{m_{2}}}p_{X}(\bm{x}|(\bm{x}_{1},\bm{x}_{2})\in \R^{m_{1}}\times A_{2})d\bm{x}_{2} \nonumber \\
        & = \int_{\R^{m_{2}}}\frac{\mathds{1}_{\R^{m_{1}}\times A_{2}}(\bm{x})p_{X}(\bm{x})}{\int_{\R^{m_{1}}\times A_{2}}p_{X}(\bm{x})d\bm{x}} d\bm{x}_{2} \nonumber \\
        & = 
          \begin{cases}
            \frac{\int_{A_{2}}p_{X}(\bm{x}_{1},\bm{x}_{2})d\bm{x}_{2}}{\int_{A_{2}}p_{X_{2}}(\bm{x}_{2})d\bm{x}_{2}} & \text{if $\text{Pr}(\bm{x}_{2}\in A_{2}) = \int_{A_{2}}p_{X_{2}}(\bm{x}_{2})d\bm{x}_{2} \neq 0$} \\
            \frac{p_{X}(\bm{x}_{1},\bm{x}_{2})}{p_{X_{2}}(\bm{x}_{2})} =:p_{X_{1}|X_{2}}(\bm{x}_{1}|\bm{x}_{2}) & \text{if $A_{2} = \{\bm{x}_{2}\}$ and $p_{X_{2}}(\bm{x}_{2})\neq 0$}  \\
          \end{cases}
          \nonumber%\label{eq:}%
    \end{align}
  \end{itemize}

\item \textbf{Marginal density and conditional expectation}
  \begin{itemize}
  \item For any function $\psi:\R^{m}\to\R^{n}$,
    the expectation of $\psi(\bm{x})$ conditional on $\bm{x}_{2}\in A_{2}\subset \R^{m_{2}}$ is
    \begin{equation}\nonumber%\label{eq:}%
      \mathbb{E}\left[\psi(\bm{x})|\bm{x}_{2}\in A_{2} \right]
      := \mathbb{E}\left[ \psi(\bm{x})| (\bm{x}_{1},\bm{x}_{2})\in \R^{m_{1}}\times A_{2}\right]
      = \frac{1}{\int_{\R^{m_{1}}\times A_{2}}p_{X}(\bm{x})d\bm{x}}\int_{\R^{m_{1}}\times A_{2}}\psi(\bm{x})p_{X}(\bm{x})d\bm{x}
    \end{equation}
  \item In particular, defining $\psi(\bm{x})=\bm{x}_{1}$ gives %We define the expectation of $\bm{x}_{1}$ conditional on $\bm{x}_{2}\in A_{2}\subset \R^{m_{2}}$ as
    \begin{align}
      \mathbb{E}\left[ \bm{x}_{1}| \bm{x}_{2}\in A_{2} \right]
        & = \frac{1}{\int_{\R^{m_{1}}\times A_{2}}p_{X}(\bm{x})d\bm{x}}\int_{\R^{m_{1}}\times A_{2}}\bm{x}_{1}p_{X}(\bm{x})d\bm{x} \nonumber \\
        & = 
          \begin{cases}
            \frac{1}{\int_{A_{2}}p_{X_{2}}(\bm{x}_{2})d\bm{x}_{2}}\int_{\R^{m_{1}}}\bm{x}_{1}\int_{A_{2}}p_{X}(\bm{x}_{1},\bm{x}_{2})d\bm{x}_{2}d\bm{x}_{1} & \text{if $\int_{A_{2}}p_{X_{2}}(\bm{x}_{2})d\bm{x}_{2} \neq 0$} \\
            \frac{1}{p_{X_{2}}(\bm{x}_{2})}\int_{\R^{m_{1}}}\bm{x}_{1}p_{X}(\bm{x}_{1},\bm{x}_{2})d\bm{x}_{1} & \text{if $A_{2} = \{\bm{x}_{2}\}$}  \\
          \end{cases}
          \nonumber \\
        & = \int_{\R^{m_{1}}}\bm{x}_{1}p_{X_{1}|X_{2}}(\bm{x}_{1}|\bm{x}_{2}\in A_{2})d\bm{x}_{1}
    \nonumber%\label{eq:}%
    \end{align}

  \item For a singleton set $A_{2}=\{\bm{x}_{2}\}$ for a particular $\bm{x}_{2}\in \R^{m_{2}}$,
    we write
    \begin{equation}\nonumber%\label{eq:}%
      \mathbb{E}\left[ \bm{x}_{1}| \bm{x}_{2}\right]
      := \int_{\R^{m_{1}}}\bm{x}_{1}\frac{p_{X}(\bm{x}_{1},\bm{x}_{2})}{p_{X_{2}}(\bm{x}_{2})}d\bm{x}_{1},
    \end{equation}
    which is a function of $\bm{x}_{2}$ (and therefore $\mathbb{E}\left[ \bm{x}_{1}| \bm{x}_{2}\right]$ is a random variable)

  \item For any partition $B_{1},\ldots, B_{I}$ of $\R^{m_{2}}$,
    \begin{equation}\nonumber%\label{eq:}%
      \sum_{i=1}^{I}\mathbb{E}\left[\bm{x}_{1}| \bm{x}_{2}\in B_{i} \right]\text{Pr}(\bm{x}_{2}\in B_{i})
      =
      \sum_{i=1}^{I}\int_{\R^{m_{1}}\times B_{i}}\bm{x}_{1}p_{X}(\bm{x})d\bm{x}
      =
      \int_{\R^{m_{1}}\times\cup_{i=1}^{I}B_{i}}\bm{x}_{1}p_{X}(\bm{x})d\bm{x}
      = \mathbb{E}\left[\bm{x}_{1}\right],
    \end{equation}
    the limit case of which (where each $B_{i}$ is a singleton set) gives
    \begin{equation}\nonumber%\label{eq:}%
      \int_{\R^{m_{2}}}\mathbb{E}\left[ \bm{x}_{1}|\bm{x}_{2} \right]p_{X_{2}}(\bm{x}_{2})d\bm{x}_{2}
      = \mathbb{E}\left[\bm{x}_{1}\right],
    \end{equation}
    and therefore
    \begin{equation}\label{eq:lte}%
      \mathbb{E}\left[\mathbb{E}\left[ \bm{x}_{1}|\bm{x}_{2} \right]\right]
      = \int_{\R^{m}}\mathbb{E}\left[ \bm{x}_{1}|\bm{x}_{2} \right]p_{X}(\bm{x})d\bm{x}
      = \int_{\R^{m_{2}}}\mathbb{E}\left[ \bm{x}_{1}|\bm{x}_{2} \right]\underbrace{\int_{\R^{m_{1}}}p_{X}(\bm{x})d\bm{x}_{1}}_{p_{X_{2}}(\bm{x}_{2})}d\bm{x}_{2}
      = \mathbb{E}\left[\bm{x}_{1}\right],
    \end{equation}
    which is called the \emph{law of total expectation} or the \emph{law of iterated expectation}
  \item Also notice that, for any $A_{2}\subset \R^{m_{2}}$ such that $\text{Pr}(\bm{x}_{2}\in A_{2})\neq 0$, we have
    \begin{align}
      \mathbb{E}\left[\mathbb{E}\left[\bm{x}_{1}|\bm{x}_{2}\right] | \bm{x}_{2}\in A_{2} \right]
        & = \frac{1}{\int_{\R^{m_{1}}\times A_{2}}p_{X}(\bm{x})d\bm{x}}\int_{\R^{m_{1}}\times A_{2}}\mathbb{E}\left[\bm{x}_{1}|\bm{x}_{2}\right]p_{X}(\bm{x})d\bm{x} \nonumber \\
        & = \frac{1}{\int_{A_{2}}p_{X_{2}}(\bm{x}_{2})d\bm{x}_{2}}\int_{A_{2}}\mathbb{E}\left[\bm{x}_{1}|\bm{x}_{2}\right]p_{X_{2}}(\bm{x}_{2})d\bm{x}_{2} \nonumber \\
        & = \frac{1}{\int_{A_{2}}p_{X_{2}}(\bm{x}_{2})d\bm{x}_{2}}\int_{A_{2}}\int_{\R^{m_{1}}}\bm{x}_{1} \frac{p_{X}(\bm{x}_{1},\bm{x}_{2})}{p_{X_{2}}(\bm{x}_{2})}d\bm{x}_{1} p_{X_{2}}(\bm{x}_{2})d\bm{x}_{2}  \nonumber \\
        & = \int_{\R^{m_{1}}}\bm{x}_{1}\frac{\int_{A_{2}}p_{X}(\bm{x}_{1},\bm{x}_{2})d\bm{x}_{2}}{\int_{A_{2}}p_{X_{2}}(\bm{x}_{2})d\bm{x}_{2}}d\bm{x}_{1} \nonumber \\
        & = \mathbb{E}\left[\bm{x}_{1} | \bm{x}_{2}\in A_{2} \right],
    \nonumber%\label{eq:}%
    \end{align}
    which is a generalization of the law of total expectation
    (i.e., setting $A_{2}=\R^{m_{2}}$ gives \eqref{eq:lte})

  \end{itemize}
  
\end{itemize}

\section{Gaussian distribution}

\subsection{Univariate Gaussian distribution}

\begin{itemize}

\item \textbf{Standard Gaussian distribution}

  \begin{itemize}

  \item We say that a random variable $z\in \R$ has
    the \emph{standard Gaussian distribution} (also called the \emph{standard normal distribution}) if
    its density is
    \begin{equation}\nonumber%\label{eq:}%
      p_{Z}(z)
      = \frac{1}{2^{\frac{1}{2}}\pi^{\frac{1}{2}}}e^{- \frac{1}{2}z^{2}}
      \quad \forall z\in \R
    \end{equation}
    and we write $z\sim \mathcal{N}(0,1)$

  \item Notice that:
    \begin{itemize}
    \item the mean of $z\sim \mathcal{N}(0,1)$ is $0$:
      \begin{equation}\nonumber%\label{eq:}%
        \mathbb{E}\left[z\right]
        := \int_{-\infty}^{\infty}zp_{Z}(z)dz
        = \frac{1}{2^{\frac{1}{2}}\pi^{\frac{1}{2}}}\int_{-\infty}^{\infty}ze^{- \frac{1}{2}z^{2}}dz
        = \frac{1}{2^{\frac{1}{2}}\pi^{\frac{1}{2}}}\int_{-\infty}^{\infty}\frac{d}{dz}\left\{-e^{- \frac{1}{2}z^{2}}\right\}dz
        = 0
      \end{equation}
    \item the variance of $z\sim \mathcal{N}(0,1)$ is $1$:
      \begin{equation}\nonumber%\label{eq:}%
        \mathbb{V}\left[z\right]
        := \mathbb{E}\left[ (z-\mathbb{E}[z])^{2} \right]
        = \frac{1}{2^{\frac{1}{2}}\pi^{\frac{1}{2}}}\int_{-\infty}^{\infty}z^{2}e^{- \frac{1}{2}z^{2}}dz
        = \frac{1}{2^{\frac{1}{2}}\pi^{\frac{1}{2}}}\int_{-\infty}^{\infty}
        \left(e^{- \frac{1}{2}z^{2}} - \frac{d}{dz}ze^{- \frac{1}{2}z^{2}}\right)
        dz = 1
      \end{equation}
    \end{itemize}

  \end{itemize}

\item \textbf{Gaussian distribution}

  \begin{itemize}

  \item We say that a random variable $x\in \R$ has a Gaussian (or normal) distribution if
    there exists
    a standard Gaussian $z\sim \mathcal{N}(0,1)$
    and constants $(\mu,\sigma) \in \R\times \R_{+}$
    such that
    \begin{equation}\nonumber%\label{eq:}%
      x = \mu + \sigma z
    \end{equation}
  and we write $x\sim \mathcal{N}(\mu, \sigma^{2})$

  \item Notice that:
    \begin{itemize}
    \item the mean of $x\sim \mathcal{N}(\mu, \sigma^{2})$ is
      $\mathbb{E}\left[x\right]=\mathbb{E}\left[ \mu + \sigma z \right]=\mu+\sigma\mathbb{E}[z]=\mu$
    \item the variance of $x\sim \mathcal{N}(\mu, \sigma^{2})$ is
      $\mathbb{V}\left[x\right] = \mathbb{E}\left[ (x-\mathbb{E}[x])^{2} \right]=\mathbb{E}\left[ (\sigma z)^{2} \right] =\sigma^{2}\mathbb{E}\left[ z^{2} \right] = \sigma^{2}$
    \end{itemize}

  \item If $\sigma\neq 0$,
    the Gaussian distribution has a density function:
    \begin{itemize}
      \item Define
        \begin{equation}\nonumber%\label{eq:}%
          \phi(z):=\mu + \sigma z
          \quad\text{or} \quad
          \psi(x):=\phi^{-1}(x)= \frac{x-\mu}{\sigma}
        \end{equation}
      \item It then follows from the change of variable formula that
        \begin{equation}\nonumber%\label{eq:}%
          \text{Pr}(x\in A)
          =
          \text{Pr}(z \in \psi(A))
          =
          \int_{\psi(A)}p_{Z}(z)dz
          =
          \int_{A}p_{Z}(\psi(x)) \left| \frac{d\psi(x)}{dx} \right| dx,
          \quad \forall A\subset \R
        \end{equation}
        meaning that the density of $x$ is
        \begin{equation}\nonumber%\label{eq:}%
          p_{X}(x)
          =
          p_{Z}(\psi(x)) \left| \frac{d\psi(x)}{dx} \right|
          =
          \frac{1}{2^{\frac{1}{2}}\pi^{\frac{1}{2}}}e^{- \frac{1}{2}(\psi(x))^{2}}
          \left| \frac{1}{\sigma}\right|
          =
          \frac{1}{2^{\frac{1}{2}}\pi^{\frac{1}{2}}\sigma}e^{- \frac{1}{2}\left(\frac{x-\mu}{\sigma}\right)^{2}}
        \end{equation}
      \end{itemize}
    \end{itemize}
\item \textbf{Independent Gaussian vector}
  
  \begin{itemize}

    \item If
      $z_{i}\sim \mathcal{N}(0,1)$
      are independent standard Gaussian,
      the density of $\bm{z}=(z_{1},\ldots, z_{n})$ is
      \begin{equation}\nonumber%\label{eq:}%
        p_{Z}(\bm{z})
        = p_{Z_{1}}(z_{1})p_{Z_{2}}(z_{2})\cdots p_{Z_{n}}(z_{n}) 
        = 
        \frac{1}{2^{\frac{n}{2}}\pi^{\frac{n}{2}}}
        e^{-\frac{1}{2}\bm{z}^{\top}\bm{z}}
      \end{equation}
    \item The expectation of $\bm{z}$ is $\mathbb{E}\left[\bm{z}\right]=\bm{0}$
    \item The variance-covariance matrix of $\bm{z}$ is $\mathbb{V}[\bm{z}]=\bm{I}$

  \end{itemize}
\end{itemize}

\subsection{Multivariate Gaussian distribution}

\begin{itemize}

\item \textbf{Multivariate Gaussian}
    
  \begin{itemize}

    \item We say that $\bm{x}=(x_{1},x_{2},\ldots,x_{m})\in\R^{m}$ has a
      \emph{multivariate Gaussian distribution} if there exist
      $n$ independent standard Gaussian variables $\bm{z}=(z_{1}, z_{2},\ldots,z_{n})\in\R^{n}$,
      $\bm{\mu}=(\mu_{1},\mu_{2},\ldots,\mu_{m})\in\R^{m}$,
      and
      $\bm{S}\in\R^{m\times n}$
      such that
      \begin{equation}\label{eq:xmusz}%
        \bm{x} = \bm{\mu} + \bm{S} \bm{z}
      \end{equation}
      and we write $\bm{x}\sim \mathcal{N}(\bm{\mu}, \bm{\Sigma})$
      where $\bm{\Sigma}:=\bm{S}\bm{S}^{\top}$ (a symmetric positive semidefinite matrix)

  \item Notice that:
    \begin{itemize}
    \item the expectation of $\bm{x}\sim\mathcal{N}(\bm{\mu},\bm{\Sigma})$ is
      $\mathbb{E}\left[\bm{x} \right] = \mathbb{E}\left[\bm{\mu} + \bm{S}\bm{z}\right] = \bm{\mu} + \bm{S}\mathbb{E}\left[\bm{z}\right] = \bm{\mu}$
    \item the variance-covariance matrix of $\bm{x}\sim\mathcal{N}(\bm{\mu},\bm{\Sigma})$ is
      \begin{equation}\nonumber%\label{eq:}%
        \mathbb{V}[\bm{x}]
        =
        \mathbb{E}\left[(\bm{x} - \bm{\mu})(\bm{x} - \bm{\mu})^{\top}\right]
        =
        \mathbb{E}\left[(\bm{S}\bm{z})(\bm{S}\bm{z})^{\top}\right]
        =
        \bm{S}\mathbb{E}\left[\bm{z}\bm{z}^{\top}\right]\bm{S}^{\top}
        =
        \bm{S}\bm{S}^{\top} =\bm{\Sigma}
      \end{equation}
    \item there cay be multiple combinations of $(\bm{S}, \bm{z})$ that give the same distribution of $\bm{x}$;
      the distribution is the same as long as $\bm{\Sigma}=\bm{S}\bm{S}^{\top}$ is the same
    \end{itemize}

  \item If $\bm{\Sigma}=\bm{S}\bm{S}^{\top}$ is non-singular,
    the Gaussian random vector has a density function:
    \begin{itemize}
    \item Since $\bm{\Sigma}$ is a symmetric positive semidefinite matrix,
      the non-singularity implies that $\bm{\Sigma}$ is positive definite,
      meaning that it is diagonalizable as
      \begin{equation}\nonumber%\label{eq:}%
        \bm{\Sigma} =
        \bm{V}
        \begin{bmatrix}
          \lambda_{1} & 0 & \cdots & 0 \\
          0 & \lambda_{2} & \cdots & 0 \\
          \vdots & \vdots & \ddots & \vdots \\
          0 & 0 & \cdots & \lambda_{n} \\
        \end{bmatrix}
        \bm{V}^{-1}
      \end{equation}
      where $\lambda_{i}>0$ for all $i=1,\ldots, n$
    \item Define a symmetric positive definite matrix $\bm{\Sigma}^{1/2}\in \R^{n\times n}$ by
      \begin{equation}\nonumber%\label{eq:}%
        \bm{\Sigma}^{1/2}:=
        \bm{V}
        \begin{bmatrix}
          \lambda_{1}^{1/2} & 0 & \cdots & 0 \\
          0 & \lambda_{2}^{1/2} & \cdots & 0 \\
          \vdots & \vdots & \ddots & \vdots \\
          0 & 0 & \cdots & \lambda_{n}^{1/2} \\
        \end{bmatrix}
        \bm{V}^{-1}
      \end{equation}
      so that
      $\bm{\Sigma}^{1/2}(\bm{\Sigma}^{1/2})^{\top}=\bm{\Sigma}$
      and
      $|\bm{\Sigma}^{1/2}|=|\bm{\Sigma}|^{1/2}$
    \item Notice that
      \begin{equation}\nonumber%\label{eq:}%
        \bm{x} = \bm{\mu} + \bm{\Sigma}^{1/2}\bm{z},
        \quad\text{where $\bm{z}\in \R^{m}$ is $m$ independent standard Gaussian}
      \end{equation}
      has the same distribution as \eqref{eq:xmusz} because $\bm{\Sigma}^{1/2}(\bm{\Sigma}^{1/2})^{\top}=\bm{\Sigma} = \bm{S}\bm{S}^{\top}$

      \item Define
        \begin{equation}\nonumber%\label{eq:}%
          \phi(\bm{z}) := \bm{\mu} + \bm{\Sigma}^{1/2}\bm{z}
          \quad\text{or} \quad
          \psi(\bm{x}):=\phi^{-1}(\bm{x})= (\bm{\Sigma}^{1/2})^{-1}(\bm{x}-\bm{\mu})
        \end{equation}
    
      \item It then follows from the change of variable formula that
        \begin{equation}\nonumber%\label{eq:}%
          \text{Pr}(\bm{x}\in A)
          =
          \text{Pr}(\bm{z} \in \psi(A))
          =
          \int_{\psi(A)}p_{Z}(\bm{z})d\bm{z}
          =
          \int_{A}p_{Z}(\psi(\bm{x})) \left|\left| \frac{d\psi(\bm{x})}{d\bm{x}} \right| \right| d\bm{x},
          \quad \forall A\subset \R^{m}
        \end{equation}
        meaning that the density of $\bm{x}$ is
        \begin{equation}\nonumber%\label{eq:}%
          p_{X}(\bm{x})
          =
          p_{Z}(\psi(\bm{x})) \left| \left| \frac{d\psi(\bm{x})}{d\bm{x}} \right| \right|
          =
          \frac{1}{2^{\frac{m}{2}}\pi^{\frac{m}{2}}}e^{- \frac{1}{2}\psi(\bm{x})^{\top}\psi(\bm{x})}
          ||(\bm{\Sigma}^{1/2})^{-1}||
          =
          \frac{1}{2^{\frac{m}{2}}\pi^{\frac{m}{2}}|\bm{\Sigma}|^{\frac{1}{2}}} e^{- \frac{1}{2}(\bm{x}-\bm{\mu})^{\top}\bm{\Sigma}^{-1}(\bm{x}-\bm{\mu})}
        \end{equation}
    \end{itemize}

  \end{itemize}

\item \textbf{Affine transformation}
  \begin{itemize}
  \item If $\bm{x}\sim \mathcal{N}(\bm{\mu},\bm{\Sigma})$
    is $m$-dimensional Gaussian,
    then for any $\bm{A}\in \R^{l\times m}$ and $\bm{b}\in \R^{l}$,
    \begin{equation}\nonumber%\label{eq:}%
      \bm{A}\bm{x} + \bm{b} \sim \mathcal{N}(\bm{A}\bm{\mu} + \bm{b},\bm{A}\bm{\Sigma} \bm{A}^{\top})
    \end{equation}
  \item Note that $\bm{x}\sim \mathcal{N}(\bm{\mu},\bm{\Sigma})$ means that
    there exist $\bm{S}$ with $\bm{\Sigma}=\bm{S}\bm{S}^{\top}$
    and independent Gaussian vector $\bm{z}$ such that
    \begin{equation}\nonumber%\label{eq:}%
      \bm{x} = \bm{\mu} + \bm{S}\bm{z}
    \end{equation}
    and therefore
    \begin{equation}\nonumber%\label{eq:}%
      \bm{A}\bm{x} + \bm{b}
      =
      \bm{A}(\bm{\mu} +\bm{S}\bm{z}) + \bm{b}
      =
      (\bm{A}\bm{\mu} + \bm{b}) + (\bm{A}\bm{S})\bm{z}
    \end{equation}
    with
    \begin{equation}\nonumber%\label{eq:}%
      (\bm{A}\bm{S})(\bm{A}\bm{S})^{\top}
      = \bm{A}\bm{S}\bm{S}^{\top}\bm{A}^{\top}
      = \bm{A}\bm{\Sigma} \bm{A}^{\top},
    \end{equation}
    which means $\bm{A}\bm{x} + \bm{b} \sim \mathcal{N}(\bm{A}\bm{\mu} + \bm{b},\bm{A}\bm{\Sigma} \bm{A}^{\top})$
  \item For example, if
    \begin{equation}\nonumber%\label{eq:}%
      \begin{bmatrix}
        x_{1} \\ x_{2}
      \end{bmatrix}
      \sim \mathcal{N} \bigg(
      \underbrace{
        \begin{bmatrix}
          \mu_{1} \\ \mu_{2}
        \end{bmatrix}}_{\bm{\mu}},
      \underbrace{
        \begin{bmatrix}
          \sigma_{11} & \sigma_{12} \\
          \sigma_{21} & \sigma_{22} \\
        \end{bmatrix}}_{\bm{\Sigma}}
      \bigg)
    \end{equation}
    then
    \begin{equation}\nonumber%\label{eq:}%
      a_{1}x_{1} + a_{2}x_{2} + b
      =
      \underbrace{
      \begin{bmatrix}
        a_{1} & a_{2}
      \end{bmatrix}}_{\bm{A}}
      \underbrace{
      \begin{bmatrix}
        x_{1} \\ x_{2}
      \end{bmatrix}}_{\bm{x}}
    + b
      \sim
      \mathcal{N}
      \big(
      \underbrace{a_{1}\mu_{1}+a_{2}\mu_{2}}_{\bm{A}\bm{\mu}} + b,
      \underbrace{
        a_{1}^{2}\sigma_{11} + 2a_{1}a_{2}\sigma_{12}\sigma_{21} + a_{2}^{2}\sigma_{22}^{2}
        }_{\bm{A}\bm{\Sigma}\bm{A}^{\top}}
      \big)
    \end{equation}
  \end{itemize}

\end{itemize}

\subsection{Marginal and conditional distribution}

\begin{itemize}

\item \textbf{Marginal distribution}

  \begin{itemize}

  \item If $\bm{x}_{1}\in \R^{m_{1}}$ and $\bm{x}_{2}\in \R^{m_{2}}$ jointly have a normal distribution
    \begin{equation}\label{eq:x1x2normal}%
      \underbrace{
      \begin{bmatrix}
        \bm{x}_{1}\\
        \bm{x}_{2}\\
      \end{bmatrix}}_{\bm{x}}
      \sim \mathcal{N} \bigg(
        \underbrace{\begin{bmatrix}
          \bm{\mu}_{1} \\
          \bm{\mu}_{2} \\
        \end{bmatrix}}_{\bm{\mu}},
      \underbrace{
        \begin{bmatrix}
          \bm{\Sigma}_{11} & \bm{\Sigma}_{12} \\
          \bm{\Sigma}_{21} & \bm{\Sigma}_{22} \\
        \end{bmatrix}}_{\bm{\Sigma}}
      \bigg),
    \end{equation}
    then the marginal distributions of $\bm{x}_{1}$ and $\bm{x}_{2}$ are
    \begin{equation}\nonumber%\label{eq:}%
      \bm{x}_{1} = \bm{A}_{1}\bm{x}
      \,\,\text{where}\,\,
      \bm{A}_{1}:=
      \begin{bmatrix}
        \bm{I}_{m_{1}} & \bm{0} \\
      \end{bmatrix}
      \implies
      \bm{x}_{1}
      \sim \mathcal{N}(\bm{A}_{1}\bm{\mu}, \bm{A}_{1}\Sigma \bm{A}_{1}^{\top})
      = \mathcal{N}(\bm{\mu}_{1}, \bm{\Sigma}_{11})
    \end{equation}
    \begin{equation}\nonumber%\label{eq:}%
      \bm{x}_{2} = \bm{A}_{2}\bm{x}
      \,\,\text{where}\,\,
      \bm{A}_{2}:=
      \begin{bmatrix}
        \bm{0} & \bm{I}_{m_{2}}\\
      \end{bmatrix}
      \implies
      \bm{x}_{2}
      \sim N(\bm{A}_{2}\bm{\mu}, \bm{A}_{2}\Sigma \bm{A}_{2}^{\top})
      = \mathcal{N}(\bm{\mu}_{2}, \bm{\Sigma}_{22})
    \end{equation}

  \end{itemize}

\item \textbf{Independence}
  \begin{itemize}

  \item If $\bm{x}_{1}\in \R^{m_{1}}$ and $\bm{x}_{2}\in \R^{m_{2}}$ jointly have a normal distribution~\eqref{eq:x1x2normal},
    then
    \begin{equation}\nonumber%\label{eq:}%
      \text{$\bm{x}_{1}$ and $\bm{x}_{2}$ are independent}
      \iff
      \bm{\Sigma}_{12} = \bm{\Sigma}_{21} = \bm{O}
    \end{equation}
  \item This is because
    \begin{align}
      p_{X_{1}}(\bm{x}_{1})p_{X_{2}}(\bm{x}_{2})
      & = \frac{1}{(2\pi)^{\frac{m_{1}}{2}}|\bm{\Sigma}_{11}|^{\frac{1}{2}}}e^{- \frac{1}{2}(\bm{x}_{1}-\bm{\mu}_{1})^{\top}\bm{\Sigma}_{11}^{-1}(\bm{x}_{1}-\bm{\mu}_{1})}\frac{1}{(2\pi)^{\frac{m_{2}}{2}}|\bm{\Sigma}_{22}|^{\frac{1}{2}}}e^{- \frac{1}{2}(\bm{x}_{2}-\bm{\mu}_{2})^{\top}\bm{\Sigma}_{22}^{-1}(\bm{x}_{2}-\bm{\mu}_{2})}  \nonumber \\
      & =
        \underbrace{
      \frac{1}{(2\pi)^{\frac{m_{1}+m_{2}}{2}}\begin{vmatrix}\bm{\Sigma}_{11}&\bm{O}\\\bm{O}&\bm{\Sigma}_{22}\end{vmatrix}^{\frac{1}{2}}}
      e^{- \frac{1}{2}
        \begin{bmatrix}
          \bm{x}_{1}-\bm{\mu}_{1}\\
          \bm{x}_{2}-\bm{\mu}_{2}\\
        \end{bmatrix}^{\top}
        \begin{bmatrix}
          \bm{\Sigma}_{11}&\bm{O}\\
          \bm{O} &\bm{\Sigma}_{22}
        \end{bmatrix}^{-1}
        \begin{bmatrix}
          \bm{x}_{1}-\bm{\mu}_{1}\\
          \bm{x}_{2}-\bm{\mu}_{2}\\
        \end{bmatrix}}
        }_{p_{X}(\bm{x}_{1},\bm{x}_{2})\big|_{\bm{\Sigma}_{12}=\bm{\Sigma}_{21}=\bm{O}}}
      \nonumber%\label{eq:}%
    \end{align}

  \end{itemize}

\item \textbf{Conditional distribution}
  \begin{itemize}
  \item If $\bm{x}_{1}\in \R^{m_{1}}$ and $\bm{x}_{2}\in \R^{m_{2}}$ jointly have a normal distribution~\eqref{eq:x1x2normal},
    then the conditional distribution is
    \begin{equation}\nonumber%\label{eq:}%
      \bm{x}_{1}|\bm{x}_{2} \sim
      \mathcal{N} \left(\bm{\mu}_{1|2}, \bm{\Sigma}_{1|2}\right),
    \end{equation}
    where
    \begin{equation}\nonumber%\label{eq:}%
      \bm{\mu}_{1|2}:=
      \bm{\mu}_{1}+\bm{\Sigma}_{12}\bm{\Sigma}_{22}^{-1}(\bm{x}_{2}-\bm{\mu}_{2}),
      \quad
      \bm{\Sigma}_{1|2}:=\bm{\Sigma}_{11}- \bm{\Sigma}_{12}\bm{\Sigma}_{22}^{-1}\bm{\Sigma}_{21}
    \end{equation}
  \item This is because
    \begin{align}
      p_{X_{1}|X_{2}}(\bm{x}_{1}|\bm{x}_{2})
        & = \frac{p_{X}(\bm{x}_{1},\bm{x}_{2})}{p_{X_{2}}(\bm{x}_{2})}  \nonumber \\
        & = \frac{e^{(\bm{x}_{2}-\bm{\mu}_{2})^{\top}\bm{\Sigma}_{22}^{-1}\bm{\Sigma}_{21}\bm{\Sigma}_{1|2}^{-1}\bm{\Sigma}_{12}\bm{\Sigma}_{22}^{-1}(\bm{x}_{2}-\bm{\mu}_{2})}}{(2\pi)^{\frac{m_{1}}{2}} |\bm{\Sigma}_{1|2}|^{\frac{1}{2}}}e^{(\bm{x}_{1}-\bm{\mu}_{1})^{\top}\bm{\Sigma}_{1|2}^{-1}(\bm{x}_{1}-\bm{\mu}_{1})-2(\bm{x}_{2}-\bm{\mu}_{2})^{\top}\bm{\Sigma}_{22}^{-1}\bm{\Sigma}_{21}\bm{\Sigma}_{1|2}^{-1}(\bm{x}_{1}-\bm{\mu}_{1})} \nonumber \\
        & = \frac{1}{(2\pi)^{\frac{m_{1}}{2}} |\bm{\Sigma}_{1|2}|^{\frac{1}{2}}}e^{(\bm{x}_{1}-\bm{\mu}_{1|2})^{\top}\bm{\Sigma}_{1|2}^{-1}(\bm{x}_{1}-\bm{\mu}_{1|2})},
    \nonumber%\label{eq:}%
    \end{align}
    which is the density of $\mathcal{N}(\bm{\mu}_{1|2}, \bm{\Sigma}_{1|2})$
  \item Note that $p_{X_{1}|X_{2}}(\bm{x}_{1}|\bm{x}_{2})=p_{X_{1}}(\bm{x}_{1})$
    if $\bm{x}_{1}$ and $\bm{x}_{2}$ are independent ($\bm{\Sigma}_{12}=\bm{\Sigma}_{12}=\bm{O}$)
  \end{itemize}

\item \textbf{Example~1}
  \begin{itemize}
  \item Consider a Gaussian random vector
    \begin{equation}\nonumber%\label{eq:}%
      \bm{x}
      =
      \begin{bmatrix}
        x_{1} \\ x_{2}
      \end{bmatrix}
      \sim \mathcal{N} \left(
        \begin{bmatrix}
          \mu_{1} \\ \mu_{2}
        \end{bmatrix},
        \begin{bmatrix}
          \sigma_{11} & \sigma_{12} \\
          \sigma_{21} & \sigma_{22} \\
        \end{bmatrix}
      \right),
      \quad\text{where}\quad \sigma_{21} = \sigma_{12}
    \end{equation}
  \item Suppose we can only observe the realization of $x_{2}$
    and we want to infer the value of $x_{1}$
  \item The distribution of $x_{1}$ conditional on a particular observation $x_{2}$ is
    \begin{equation}\nonumber%\label{eq:}%
      x_{1}|x_{2} \sim
      \mathcal{N}\left(
        \mu_{1} + \frac{\sigma_{12}}{\sigma_{22}}(x_{2}-\mu_{2}),
        \sigma_{11} - \frac{\sigma_{12}\sigma_{21}}{\sigma_{22}}
      \right),
    \end{equation}
  \item Observe
    \begin{equation}\nonumber%\label{eq:}%
      \mathbb{E}\left[\mathbb{E}\left[x_{1}|x_{2}\right]\right]
      =
      \mathbb{E}\left[\mu_{1} + \frac{\sigma_{12}}{\sigma_{22}}(x_{2}-\mu_{2})\right]
      =
      \mu_{1} + \frac{\sigma_{12}}{\sigma_{22}}(\mathbb{E}\left[x_{2}\right]-\mu_{2})
      =
      \mu_{1}
      =
      \mathbb{E}[x_{1}],
    \end{equation}
    which confirms the law of total expectation
  \end{itemize}
\item \textbf{Example~2}
  \begin{itemize}
  \item Consider an independent Gaussian random vector
    \begin{equation}\nonumber%\label{eq:}%
      \begin{bmatrix}
        x_{1} \\
        x_{2} \\
      \end{bmatrix}
      =
        \begin{bmatrix}
          \mu_{1} \\
          \mu_{2}
        \end{bmatrix}
        +
        \begin{bmatrix}
          \sigma_{1} & 0 \\
          0 & \sigma_{2}
        \end{bmatrix}
      \begin{bmatrix}
       z_{1} \\ z_{2}
      \end{bmatrix}
      \sim \mathcal{N}
      \left(
        \begin{bmatrix}
          \mu_{1} \\
          \mu_{2}
        \end{bmatrix},
        \begin{bmatrix}
          \sigma_{1}^{2} & 0 \\
          0 & \sigma_{2}^{2}
        \end{bmatrix}
      \right),
      \quad\text{where}\quad
      \begin{bmatrix}
       z_{1} \\ z_{2}
      \end{bmatrix}
      \sim \mathcal{N}\left(\bm{0}, \bm{I} \right)
    \end{equation}
  \item Suppose that we cannot directly observe the realization of $\bm{x}=(x_{1},x_{2})$,
    but we instead observe the sum of $x_{1}$ and $x_{2}$
  \item We want to infer the values of $x_{1}$ and $x_{2}$ based on the observation of $y=x_{1}+x_{2}$
  \item Since
    \begin{equation}\nonumber%\label{eq:}%
      y = \underbrace{x_{1}}_{\mu_{1}+\sigma_{1}z_{1}} + \underbrace{x_{2}}_{\mu_{2}+\sigma_{2}z_{2}}
      = \mu_{1} + \mu_{2} +
      \begin{bmatrix}
       \sigma_{1} & \sigma_{2} 
      \end{bmatrix}
      \begin{bmatrix}
       z_{1} \\ z_{2}
      \end{bmatrix},
    \end{equation}
    the joint distribution is
    \begin{equation}\nonumber%\label{eq:}%
      \begin{bmatrix}
        x_{1} \\
        x_{2} \\
        y
      \end{bmatrix}
      =
      \begin{bmatrix}
        \mu_{1} \\
        \mu_{2} \\
        \mu_{1}+\mu_{2}
      \end{bmatrix}
      +
      \begin{bmatrix}
        \sigma_{1} & 0 \\
        0 & \sigma_{2} \\
        \sigma_{1} & \sigma_{2}
      \end{bmatrix}
      \begin{bmatrix}
       z_{1} \\ z_{2}
      \end{bmatrix}
      \sim
      \mathcal{N}
      \Bigg(
        \underbrace{
      \begin{bmatrix}
        \mu_{1} \\
        \mu_{2} \\
        \mu_{1}+\mu_{2}
      \end{bmatrix}}_{
      \begin{bmatrix}
        \bm{\mu}_{x}\\\mu_{y}
      \end{bmatrix}},
      \underbrace{
      \begin{bmatrix}
        \sigma_{1}^{2} & 0 & \sigma_{1}^{2} \\
        0 & \sigma_{2}^{2} & \sigma_{2}^{2} \\
        \sigma_{1}^{2} & \sigma_{2}^{2} & \sigma_{1}^{2}+\sigma_{2}^{2} \\
      \end{bmatrix}}_{
      \begin{bmatrix}
        \bm{\Sigma}_{xx} & \bm{\sigma}_{xy} \\
        \bm{\sigma}_{xy}^{\top} & \sigma_{yy} \\
      \end{bmatrix}}
      \Bigg)
    \end{equation}
  \item Hence, the conditional distribution is
    \begin{align}\nonumber%\label{eq:}%
      \begin{bmatrix}
        x_{1}\\x_{2}
      \end{bmatrix}
      \bigg| y
      & \sim
        \mathcal{N}
        \left(
        \bm{\mu}_{x} + \bm{\sigma}_{xy}\frac{1}{\sigma_{yy}}(y-\mu_{y}),
        \bm{\Sigma}_{xx} - \bm{\sigma}_{xy}\frac{1}{\sigma_{yy}}\bm{\sigma}_{xy}^{\top}
        \right) \nonumber \\
      & =
        \mathcal{N}
        \left(
        \begin{bmatrix}
          \mu_{1} \\ \mu_{2}
        \end{bmatrix}
        +
        \begin{bmatrix}
          \sigma_{1}^{2} \\ \sigma_{2}^{2}
        \end{bmatrix}
        \frac{y-(\mu_{1}+\mu_{2})}{\sigma_{1}^{2}+\sigma_{2}^{2}},
        \begin{bmatrix}
          \sigma_{1}^{2} & 0 \\
          0 & \sigma_{2}^{2} \\
        \end{bmatrix}
        -
        \frac{1}{\sigma_{1}^{2}+\sigma_{2}^{2}}
        \begin{bmatrix}
          \sigma_{1}^{2} \\ \sigma_{2}^{2}
        \end{bmatrix}
        \begin{bmatrix}
          \sigma_{1}^{2} && \sigma_{2}^{2}
        \end{bmatrix}
        \right) \nonumber \\
      & =
        \mathcal{N}
      \left(
        \begin{bmatrix}
          \frac{\sigma_{1}^{-2}}{\sigma_{1}^{-2}+\sigma_{2}^{-2}}\mu_{1}
          +
          \frac{\sigma_{2}^{-2}}{\sigma_{1}^{-2}+\sigma_{2}^{-2}}(y-\mu_{2})
          \\
          \frac{\sigma_{2}^{-2}}{\sigma_{1}^{-2}+\sigma_{2}^{-2}}\mu_{2}
          +
          \frac{\sigma_{1}^{-2}}{\sigma_{1}^{-2}+\sigma_{2}^{-2}}(y-\mu_{1})
        \end{bmatrix},        
        \begin{bmatrix}
          \frac{1}{\sigma_{1}^{-2}+\sigma_{2}^{-2}} & -\frac{1}{\sigma_{1}^{-2}+\sigma_{2}^{-2}} \\
          -\frac{1}{\sigma_{1}^{-2}+\sigma_{2}^{-2}} & \frac{1}{\sigma_{1}^{-2}+\sigma_{2}^{-2}} \\
        \end{bmatrix}
      \right)\nonumber
    \end{align}
  \item Observe
    \begin{equation}\nonumber%\label{eq:}%
      \mathbb{E}\left[\mathbb{E}\left[x_{i}|y\right]\right]
      =
        \frac{\sigma_{i}^{-2}}{\sigma_{1}^{-2}+\sigma_{2}^{-2}}\mu_{i}
        +
        \frac{\sigma_{j}^{-2}}{\sigma_{1}^{-2}+\sigma_{2}^{-2}}\underbrace{(\mathbb{E}\left[y\right]-\mu_{j})}_{\mu_{i}}
        = \mu_{i} = \mathbb{E}\left[ x_{i} \right]
    \end{equation}

  \end{itemize}

\item \textbf{Example~3 (Gaussian signal extraction)}
  \begin{itemize}
  \item Let $\bm{x}=(x_{1},\ldots, x_{m})\sim \mathcal{N}(\bm{\mu},\bm{\Sigma})$ be a multivariate Gaussian random vector
  \item Suppose that we cannot directly observe the realization of $\bm{x}$,
    but we instead observe a `signal' $\bm{y}\in\R^{l}$, the value of which is determined by
    \begin{equation}\nonumber%\label{eq:yx}%
      \bm{y}
      =
      \begin{bmatrix}
        y_{1} \\
        y_{2} \\
        \vdots \\
        y_{l} \\
      \end{bmatrix}
      =
      \underbrace{
      \begin{bmatrix}
        a_{1,1} & a_{1,2} & \cdots & a_{1,m} \\
        a_{2,1} & a_{2,2} & \cdots & a_{2,m} \\
        \vdots & \vdots & \ddots & \vdots \\
        a_{l,1} & a_{l,2} & \cdots & a_{l,m} \\
      \end{bmatrix}}_{\bm{A}}
      \underbrace{
      \begin{bmatrix}
        x_{1} \\
        x_{2} \\
        \vdots \\
        x_{m} \\
      \end{bmatrix}}_{\bm{x}}
      +
      \underbrace{
      \begin{bmatrix}
        b_{1} \\
        b_{2} \\
        \vdots \\
        b_{l} \\
      \end{bmatrix}}_{\bm{b}},
    \quad\text{where}\quad \mathrm{rank}(\bm{A})<m
    \end{equation}
  \item We want to infer the value of $\bm{x}$ conditional on the observed value of $\bm{y}$
  \item Note that Example~3 is a generalization of Example~1 and Example~2
    \begin{itemize}
    \item Example~1: $l=1$, $m=2$, $a_{1,1}=0$, $a_{1,2}=1$, and $b=0$
    \item Example~2: $l=1$, $m=2$, $a_{1,1}= a_{1,2}=1$ and $b=0$
    \end{itemize}
  \item Since $\bm{x}\sim \mathcal{N}(\bm{\mu},\bm{\Sigma})$,
    there exists $\bm{S}\in \R^{m\times n}$ and
    independent standard Gaussian variables $\bm{z}\in \R^{n}$ such that
    \begin{equation}\nonumber%\label{eq:}%
      \bm{x} = \bm{\mu} + \bm{S}\bm{z}
      \quad \text{with}\quad \bm{S}\bm{S}^{\top} = \bm{\Sigma}
    \end{equation}
    and thus the joint distribution of $\bm{x}$ and $\bm{y}$ is
    \begin{equation}\nonumber%\label{eq:}%
      \begin{bmatrix}
        \bm{x} \\ \bm{y}
      \end{bmatrix}
      =
      \begin{bmatrix}
        \bm{\mu} \\ \bm{A}\bm{\mu} + \bm{b}
      \end{bmatrix}
      +
      \begin{bmatrix}
        \bm{S}\\
        \bm{A}\bm{S}
      \end{bmatrix}
      \bm{z}
      \sim
      \mathcal{N}
      \left(
      \begin{bmatrix}
        \bm{\mu} \\ \bm{A}\bm{\mu} + \bm{b}
      \end{bmatrix},
      \begin{bmatrix}
        \bm{\Sigma} & \bm{\Sigma}\bm{A}^{\top} \\
        \bm{A}\bm{\Sigma} & \bm{A}\bm{\Sigma}\bm{A}^{\top}
      \end{bmatrix}
      \right)
    \end{equation}
  \item The distribution of $\bm{x}$ conditional on $\bm{y}$ is therefore
    \begin{equation}\nonumber%\label{eq:}%
      \bm{x}|\bm{y} \sim \mathcal{N}
      \left(
        \bm{\mu}+\bm{\Sigma}\bm{A}^{\top}(\bm{A}\bm{\Sigma}\bm{A}^{\top})^{-1}(\bm{y}-\bm{A}\bm{\mu} - \bm{b}),
        \bm{\Sigma} - \bm{\Sigma}\bm{A}^{\top}(\bm{A}\bm{\Sigma}\bm{A}^{\top})^{-1}\bm{A}\bm{\Sigma}
      \right)
    \end{equation}

  \end{itemize}
  
\item \textbf{Example~4}
  \begin{itemize}
  \item Suppose $z\sim \mathcal{N}(0,1)$
  \item What is the expectation of $z$ conditional on $z\geq c$ for some constant $c\in \R$?
    \begin{equation}\nonumber%\label{eq:}%
      \mathbb{E}\left[ z | z \geq c \right]
      = \frac{1}{\int_{c}^{\infty}\frac{1}{\sqrt{2\pi}}e^{- \frac{1}{2}z^{2}}dz}\int_{c}^{\infty}z\frac{1}{\sqrt{2\pi}}e^{- \frac{1}{2}z^{2}}dz
      = \frac{\Phi'(c)}{1-\Phi(c)},
    \end{equation}
    where $\Phi(\cdot)$ is the cumulative distribution function of standard Gaussian:
    \begin{equation}\nonumber%\label{eq:}%
      \Phi(c)
      :=
      \text{Pr}(z\leq c)
      =
      \int_{-\infty}^{c}\frac{1}{\sqrt{2\pi}}e^{- \frac{1}{2}z^{2}}dz
    \end{equation}
  \item Similarly
    \begin{equation}\nonumber%\label{eq:}%
      \mathbb{E}\left[ z | z < c \right]
      = \frac{1}{\int_{-\infty}^{c}\frac{1}{\sqrt{2\pi}}e^{- \frac{1}{2}z^{2}}dz}\int_{-\infty}^{c}z\frac{1}{\sqrt{2\pi}}e^{- \frac{1}{2}z^{2}}dz
      = -\frac{\Phi'(c)}{\Phi(c)}
    \end{equation}
  \item Observe
    \begin{equation}\nonumber%\label{eq:}%
      \mathbb{E}\left[ z | z \geq c \right]\text{Pr}(z\geq c)
      +
      \mathbb{E}\left[ z | z < c \right]\text{Pr}(z< c)
      =
      \frac{\Phi'(c)}{1-\Phi(c)}(1-\Phi(c))
      -
      \frac{\Phi'(c)}{\Phi(c)}\Phi(c) = 0
      = \mathbb{E}[z]
    \end{equation}
  \end{itemize}

\item \textbf{Example~5}
  \begin{itemize}
  \item Suppose $x\sim \mathcal{N}(\mu,\sigma^{2})$ with $\sigma >0$
  \item What is the expectation of $x$ conditional on $x\geq c$ for some constant $c\in \R$?
  \item There exists $z\sim \mathcal{N}(0,1)$ such that $x=\mu+\sigma z$ and thus
    \begin{equation}\nonumber%\label{eq:}%
      \mathbb{E}\left[ x | x \geq c \right]
      = \mathbb{E}\left[ \mu + \sigma z | \mu + \sigma z \geq c \right]
      = \mu + \sigma\mathbb{E}\left[z \bigg| z \geq \frac{c - \mu}{\sigma} \right]
      = \mu + \sigma \frac{\Phi'\left(\frac{c - \mu}{\sigma}\right) }{1-\Phi\left(\frac{c - \mu}{\sigma}\right)}
    \end{equation}
  \item Similarly
    \begin{equation}\nonumber%\label{eq:}%
      \mathbb{E}\left[ x | x < c \right]
      = \mathbb{E}\left[ \mu + \sigma z | \mu + \sigma z < c \right]
      = \mu + \sigma\mathbb{E}\left[z \bigg| z < \frac{c - \mu}{\sigma} \right]
      = \mu - \sigma \frac{\Phi'\left(\frac{c - \mu}{\sigma}\right) }{\Phi\left(\frac{c - \mu}{\sigma}\right)}
    \end{equation}
  \end{itemize}

\item \textbf{Example~6}
  \begin{itemize}
  \item Suppose $(x_{1},\ldots ,x_{m})\sim \mathcal{N}(\bm{\mu},\bm{\Sigma})$ with non-singular $\bm{\Sigma}\in \R^{m\times m}$ and let
    \begin{equation}\nonumber%\label{eq:}%
      y := 
      a_{1}x_{1} + a_{2}x_{2} + \cdots + a_{m}x_{m}
      =
      \underbrace{
      \begin{bmatrix}
        a_{1} & a_{2} & \cdots & a_{m}
      \end{bmatrix}}_{\bm{A}}
    \underbrace{
      \begin{bmatrix}
        x_{1} \\
        x_{2} \\
        \vdots \\
        x_{m} \\
      \end{bmatrix}}_{\bm{x}}
    \sim
    \mathcal{N}
    \left(\bm{A}\bm{\mu}, \bm{A}\bm{\Sigma}\bm{A}^{\top}\right)
    \end{equation}
  \item What is the expectation of $\bm{x}$ conditional on $y\geq c$ for some constant $c\in \R$?
  \item The examples above imply
    \begin{equation}\nonumber%\label{eq:}%
      \mathbb{E}\left[y| y\geq c \right]
      =
      \bm{A}\bm{\mu} + (\bm{A}\bm{\Sigma}\bm{A}^{\top})^{1/2} \frac{\Phi'\left(\frac{c - \bm{A}\bm{\mu}}{(\bm{A}\bm{\Sigma}\bm{A}^{\top})^{1/2}}\right) }{1-\Phi\left(\frac{c - \bm{A}\bm{\mu}}{(\bm{A}\bm{\Sigma}\bm{A}^{\top})^{1/2}}\right)}
    \end{equation}
  \item Since we know
    \begin{equation}\nonumber%\label{eq:}%
      \bm{x}|y
      \sim \mathcal{N}
      \left(
        \bm{\mu}+\bm{\Sigma}\bm{A}^{\top}(\bm{A}\bm{\Sigma}\bm{A}^{\top})^{-1}(y-\bm{A}\bm{\mu}),
        \bm{\Sigma} - \bm{\Sigma}\bm{A}^{\top}(\bm{A}\bm{\Sigma}\bm{A}^{\top})^{-1}\bm{A}\bm{\Sigma}
      \right),
    \end{equation}
    the law of total expectation gives
    \begin{align}
      \mathbb{E}\left[\bm{x}|y\geq c \right]
        & = \mathbb{E}\left[\mathbb{E}\left[\bm{x}|y\right]|y\geq c \right] \nonumber \\
        & = \mathbb{E}\left[\bm{\mu}+\bm{\Sigma}\bm{A}^{\top}(\bm{A}\bm{\Sigma}\bm{A}^{\top})^{-1}(y-\bm{A}\bm{\mu}) \Big| y\geq c \right] \nonumber \\
        & = \bm{\mu}+\bm{\Sigma}\bm{A}^{\top}(\bm{A}\bm{\Sigma}\bm{A}^{\top})^{-1}(\mathbb{E}\left[y| y\geq c \right]-\bm{A}\bm{\mu}) \nonumber \\
        & = \bm{\mu}+\bm{\Sigma}\bm{A}^{\top}(\bm{A}\bm{\Sigma}\bm{A}^{\top})^{-1}(\bm{A}\bm{\Sigma}\bm{A}^{\top})^{1/2} \frac{\Phi'\left(\frac{c - \bm{A}\bm{\mu}}{(\bm{A}\bm{\Sigma}\bm{A}^{\top})^{1/2}}\right) }{1-\Phi\left(\frac{c - \bm{A}\bm{\mu}}{(\bm{A}\bm{\Sigma}\bm{A}^{\top})^{1/2}}\right)} \nonumber \\
        & = \bm{\mu}+\frac{1}{(\bm{A}\bm{\Sigma}\bm{A}^{\top})^{1/2}}\bm{\Sigma}\bm{A}^{\top}\frac{\Phi'\left(\frac{c - \bm{A}\bm{\mu}}{(\bm{A}\bm{\Sigma}\bm{A}^{\top})^{1/2}}\right) }{1-\Phi\left(\frac{c - \bm{A}\bm{\mu}}{(\bm{A}\bm{\Sigma}\bm{A}^{\top})^{1/2}}\right)}
    \nonumber%\label{eq:}%
    \end{align}
  \item Similarly
    \begin{align}
      \mathbb{E}\left[\bm{x}|y< c \right]
        & = \mathbb{E}\left[\mathbb{E}\left[\bm{x}|y\right]|y< c \right] \nonumber \\
        & = \mathbb{E}\left[\bm{\mu}+\bm{\Sigma}\bm{A}^{\top}(\bm{A}\bm{\Sigma}\bm{A}^{\top})^{-1}(y-\bm{A}\bm{\mu}) \Big| y< c \right] \nonumber \\
        & = \bm{\mu}+\bm{\Sigma}\bm{A}^{\top}(\bm{A}\bm{\Sigma}\bm{A}^{\top})^{-1}(\mathbb{E}\left[y| y< c \right]-\bm{A}\bm{\mu}) \nonumber \\
        & = \bm{\mu}-\bm{\Sigma}\bm{A}^{\top}(\bm{A}\bm{\Sigma}\bm{A}^{\top})^{-1}(\bm{A}\bm{\Sigma}\bm{A}^{\top})^{1/2} \frac{\Phi'\left(\frac{c - \bm{A}\bm{\mu}}{(\bm{A}\bm{\Sigma}\bm{A}^{\top})^{1/2}}\right) }{\Phi\left(\frac{c - \bm{A}\bm{\mu}}{(\bm{A}\bm{\Sigma}\bm{A}^{\top})^{1/2}}\right)} \nonumber \\
        & = \bm{\mu}-\frac{1}{(\bm{A}\bm{\Sigma}\bm{A}^{\top})^{1/2}}\bm{\Sigma}\bm{A}^{\top}\frac{\Phi'\left(\frac{c - \bm{A}\bm{\mu}}{(\bm{A}\bm{\Sigma}\bm{A}^{\top})^{1/2}}\right) }{\Phi\left(\frac{c - \bm{A}\bm{\mu}}{(\bm{A}\bm{\Sigma}\bm{A}^{\top})^{1/2}}\right)}
    \nonumber%\label{eq:}%
    \end{align}
    
  \end{itemize}

\end{itemize}

\end{document}
