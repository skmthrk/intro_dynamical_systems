\documentclass[12pt,a4paper]{article} 
\usepackage[T1]{fontenc}
\usepackage{textcomp}
\usepackage{makeidx}

\usepackage[sc, noBBpl]{mathpazo}
\usepackage{dsfont} % for indicator function
\usepackage[deluxe]{otf}
\usepackage[utf8]{inputenc}

\usepackage{amsmath}
\usepackage{amsfonts}
\usepackage{amssymb}
\usepackage{amsthm}
\usepackage{mathtools}
\usepackage{mathrsfs}

\usepackage{listings} % for programming codes

\usepackage{bm}
\usepackage[dvipdfmx]{color}
\pagestyle{plain} 
\usepackage{float}
\usepackage[dvipdfmx]{graphicx}

\usepackage[left=15mm, right=18mm, top=20mm, bottom=20mm]{geometry}

\setlength{\footskip}{40pt} 
\setlength{\abovecaptionskip}{0pt}
\setlength{\belowcaptionskip}{0pt}
\renewcommand{\baselinestretch}{1.05}

\newcommand{\argmax}{\mathop{\rm arg max}\limits}
\newcommand{\argmin}{\mathop{\rm arg min}\limits}
\newcommand{\st}{\mathop{\rm s.t.}\limits}
\newcommand{\fsize}[1]{\fontsize{#1}{#1}\selectfont}

% custom color
\usepackage[dvipdfmx]{xcolor}

% colors
\definecolor{darkred}{rgb}{.40,.00,.00}
\definecolor{navy}{rgb}{.00,.00,.40}
\definecolor{darknavy}{rgb}{.00,.00,.30}
\definecolor{lightnavy}{rgb}{.10,.10,.40}

\newcommand{\navy}{\color{navy}}
\newcommand{\deepred}{\color{darkred}}
\newcommand{\emred}[1]{\textbf{\deepred #1}}
\newcommand{\emblue}[1]{\textbf{\navy #1}}

% title and tableofcontents
\usepackage[titles]{tocloft}
\setlength{\cftbeforesecskip}{2pt}
\setlength{\cftbeforesubsecskip}{0pt}
\renewcommand{\cftsecfont}{\normalsize\mcfamily}
\renewcommand{\cftsubsecfont}{\normalsize\mcfamily}
\renewcommand{\cftsubsecfont}{\normalsize\mcfamily}
\renewcommand{\contentsname}{\large\mcfamily\bfseries \S\hspace{0.5\baselineskip}Contents\\[-\baselineskip]}
\cftsetindents{section}{10.0pt}{15.0pt}
\cftsetindents{subsection}{20.0pt}{30.0pt}
\cftpagenumbersoff{section}
\cftpagenumbersoff{subsection}
\cftpagenumbersoff{subsubsection}

\makeatletter
\def\vhrulefill#1{\leavevmode\leaders\hrule\@height#1\hfill \kern\z@}
\makeatother

\makeatletter
\def\maketitle{%
	\begin{center}\leavevmode
	\normalfont
	{\normalsize\raggedleft \@date\par}%
	\vskip -0.7\baselineskip
	\vhrulefill{0.2pt}\par
	\vskip 0.3\baselineskip
	\renewcommand{\baselinestretch}{0.80}
	{\color{darknavy}\bfseries\Large\raggedright \@title\par}%
	\renewcommand{\baselinestretch}{1.00}
	\vskip 0.2\baselineskip
	\ifx\@subtitle\@empty\else%
	{\normalsize\raggedright \@subtitle\par}%
	\fi
	\ifx\@author\@empty\else%
	\vskip 0.5\baselineskip
	{\normalsize\raggedleft \@author\par}%
	\fi
	\end{center}%
	\vskip 1.0\baselineskip
}
\def\subtitle#1{\gdef\@subtitle{#1}}
\let\@subtitle\@empty
\makeatother

% section
\newcommand{\Secindent}{0.0\baselineskip}
\newcommand{\Subsecindent}{0.0\baselineskip}
\newcommand{\lmargini}{1.3\baselineskip}
\newcommand{\lmarginii}{0.5\baselineskip}
\newcommand{\lmarginiii}{0.2\baselineskip}
\def\theenumi{\arabic{enumi}}
\def\theenumii{\alph{enumii}}
\renewcommand{\labelenumi}{\hfill\hspace{60pt}\theenumi.}
\renewcommand{\labelenumii}{\hspace{30pt}\theenumii.}
\renewcommand{\labelitemi}{\small\textbullet\hspace{2pt}}
\renewcommand{\labelitemii}{\color{gray}\small$\circ$\hspace{2pt}}
\renewcommand{\labelitemiii}{\textendash\hspace{1pt}}
\makeatletter
\renewcommand{\section}{\@startsection%
	{section}{1}{\Secindent}% name, depth, indent \z@=0pt
	{-1.0\baselineskip \@plus1.5\baselineskip \@minus0.5\baselineskip}% beforeskip (no indent if negative)
	{0.5\baselineskip \@plus0.8\baselineskip \@minus0.3\baselineskip}% afterskip (no line break if negative)
	{\reset@font\large\bfseries\scshape\color{darknavy}}% font
}
\renewcommand{\subsection}{\@startsection%
	{subsection}{2}{\Subsecindent}%
	{-0.5\baselineskip \@plus0.7\baselineskip \@minus0.4\baselineskip}%
	{0.3\baselineskip \@plus0.5\baselineskip \@minus0.2\baselineskip}%
	{\reset@font\normalsize\bfseries\scshape\color{darknavy}}% 
}
\renewcommand{\thesection}{\@arabic\c@section} % arabic, roman, Roman, alph, Alph, fnsymbol
\renewcommand{\thesubsection}{\thesection\hspace{0pt}.\hspace{0pt}\@arabic\c@subsection}
\makeatother

\makeatletter
\def\@listi{%
	\topsep = 0.2\baselineskip
	\partopsep = 0.0\baselineskip
	\parsep = 0.0\baselineskip
	\itemsep = 0.5\baselineskip
	\leftmargin = 1.5zw
	\labelsep = 0.5zw
	\itemindent = 0pt
	\labelwidth\leftmargin \advance\labelwidth-\labelsep
	\listparindent = 0zw
	\rightmargin = 0pt
	\reset@font\normalsize\normalfont
}
\let\@listI\@listi
\@listi
\def\@listii{%
	\topsep = 0.25\baselineskip
	\partopsep = 0.0\baselineskip
	\parsep = 0.0\baselineskip
	\itemsep = 0.3\baselineskip
	\leftmargin = 0.5zw
	\labelsep = .5zw
	\itemindent = 0pt
	\labelwidth\leftmargin \advance\labelwidth-\labelsep
	\listparindent = 0zw
	\rightmargin = 0pt
	\reset@font\normalsize\normalfont%
}
\def\@listiii{%
	\topsep = 0.30\baselineskip
	\partopsep = 0.0\baselineskip
	\parsep = 0.1\baselineskip
	\itemsep = 0.05\baselineskip
	\leftmargin = 1.1zw
	\labelsep = .5zw
	\itemindent = 0pt
	\labelwidth\leftmargin \advance\labelwidth-\labelsep
	\listparindent = 0zw
	\rightmargin = 0pt
	\reset@font\normalsize\normalfont%
}
\def\@listiv{%
	\topsep = 0.0\baselineskip
	\partopsep = 0.0\baselineskip
	\parsep = 0.0\baselineskip
	\itemsep = 0.0\baselineskip
	\leftmargin = 0.9zw
	\labelsep = .5zw
	\itemindent = 0pt
	\labelwidth\leftmargin \advance\labelwidth-\labelsep
	\listparindent = 0zw
	\rightmargin = 0pt
	\reset@font\normalsize\normalfont%
}
\makeatother

\newcommand{\R}{\ensuremath{\mathbb{R}}}
\newcommand{\E}{\ensuremath{\mathbb{E}}}
\newcommand{\V}{\ensuremath{\mathbb{V}}}
\newcommand{\Cov}{\ensuremath{\text{Cov}}}
\newcommand{\N}{\ensuremath{\mathbb{N}}}
\newcommand{\Z}{\ensuremath{\mathbb{Z}}}
\let\epsilon\varepsilon

\let\origitem\item
\renewcommand{\item}{\normalfont\origitem}

\usepackage[%
backref=page,
dvipdfmx,
bookmarks=true,bookmarksnumbered=true,bookmarkstype=toc,
colorlinks=true,
linkcolor=navy,
citecolor=navy,
filecolor=blue,
pagecolor=blue,
urlcolor=blue
]{hyperref}
\AtBeginDvi{\special{pdf:tounicode EUC-UCS2}}

\renewcommand*{\backref}[1]{}
\renewcommand*{\backrefalt}[4]{%
  \ifcase #1 (Not cited.)%
  \or        [#2]%
  \else      [#2]%
  \fi}
 

\title{Preliminaries}
\subtitle{Introduction to dynamical systems~\#1}
\author{Hiroaki Sakamoto}
\date{\today}

\DeclareMathOperator{\vect}{vec}
 
\begin{document}
\maketitle
\tableofcontents

\section{Notations and definitions}

\begin{itemize}
\item \textbf{Vector}
  \begin{itemize}
  \item An $m$-dimensional vector is denoted as
    \begin{equation}\nonumber%\label{eq:}%
      \bm{v} =
      \begin{bmatrix}
        v_{1} \\
        v_{2} \\
        \vdots \\
        v_{m} \\
      \end{bmatrix},
      \quad\text{where}\quad
      v_{i}\in \R
    \end{equation}
  \item The set of all $m$-dimensional vectors is denoted by $\R^{m}$
  \item For $\bm{v}, \bm{u}\in \R^{m}$ and $\alpha \in \R$,
    we define scalar multiplication $\alpha\bm{v}$ and addition $\bm{v}+\bm{u}$ as
    \begin{equation}\nonumber%\label{eq:}%
      \alpha \bm{v}
      := 
      \begin{bmatrix}
        \alpha v_{1} \\
        \alpha v_{2} \\
        \vdots \\
        \alpha v_{m} \\
      \end{bmatrix},
      \quad
      \bm{v} + \bm{u}
      := 
      \begin{bmatrix}
        v_{1} + u_{1} \\
        v_{2} + u_{2} \\
        \vdots \\
        v_{m} + u_{m} \\
      \end{bmatrix}
    \end{equation}
  \end{itemize}


\item \textbf{Matrix}
  \begin{itemize}
  \item A real matrix of dimension $m\times n$ is denoted as
    \begin{equation}\nonumber%\label{eq:}%
      \bm{A} =
      \begin{bmatrix}
        a_{11} & a_{12} & \ldots & a_{1n} \\
        a_{21} & a_{22} & \ldots & a_{2n} \\
        \vdots & \vdots & \ddots & \vdots \\
        a_{m1} & a_{m2} & \ldots & a_{mn} \\
      \end{bmatrix}
      = 
      \begin{bmatrix}
        \bm{a}_{1} & \bm{a}_{2} & \ldots & \bm{a}_{n} \\
      \end{bmatrix},
      \quad\text{where}\quad
      \bm{a}_{j} :=
      \begin{bmatrix}
        a_{1j} \\
        a_{2j} \\
        \vdots \\
        a_{mj} \\
      \end{bmatrix}\in \R^{m}
    \end{equation}
  \item The set of all $m\times n$ real matrices is denoted as $\R^{m\times n}$
  \item $\bm{A}^{\top}$ denotes the transpose of $\bm{A}$
  \item If $\bm{A}$ has an inverse, we denote it by $\bm{A}^{-1}$
  \item The determinant of $\bm{A}$ is denoted by $|\bm{A}|$ 
  \item We use $\bm{I}$ for an identity matrix
    and $\bm{O}$ for a null matrix
    \begin{equation}\nonumber%\label{eq:}%
      \bm{I} :=
      \begin{bmatrix}
        1 & 0 & \ldots & 0 \\
        0 & 1 & \ldots & 0 \\
        \vdots & \vdots & \ddots & \vdots \\
        0 & 0 & \ldots & 1 \\
      \end{bmatrix},
      \quad
      \bm{O} :=
      \begin{bmatrix}
        0 & 0 & \ldots & 0 \\
        0 & 0 & \ldots & 0 \\
        \vdots & \vdots & \ddots & \vdots \\
        0 & 0 & \ldots & 0 \\
      \end{bmatrix}
    \end{equation}
  \end{itemize}

\item \textbf{Some definitions}
  \begin{itemize}
  \item For $\bm{A}\in \R^{m\times n}$ and $\bm{v}\in \R^{n}$,
    we define $\bm{A}\bm{v}\in \R^{m}$ as
    \begin{equation}\nonumber%\label{eq:}%
      \bm{A}\bm{v} := \sum_{j=1}^{n}v_{j}\bm{a}_{j}
      =
      v_{1}
      \begin{bmatrix}
        a_{11}\\
        a_{21}\\
        \vdots \\
        a_{m1}\\
      \end{bmatrix}
      +
      v_{2}
      \begin{bmatrix}
        a_{12}\\
        a_{22}\\
        \vdots \\
        a_{m2}\\
      \end{bmatrix}
      + \cdots +
      v_{n}
      \begin{bmatrix}
        a_{1n}\\
        a_{2n}\\
        \vdots \\
        a_{mn}\\
      \end{bmatrix}
    \end{equation}
  \item For $\bm{A}\in \R^{m\times n}$ and $\alpha \in \R$, we define $\alpha\bm{A}\in \R^{m\times n}$ as
    \begin{equation}\nonumber%\label{eq:}%
      \alpha\bm{A} := 
      \begin{bmatrix}
        \alpha\bm{a}_{1} & \alpha\bm{a}_{2} & \ldots & \alpha\bm{a}_{n} \\
      \end{bmatrix}
    \end{equation}
  \item For $\bm{A}\in \R^{m\times n}$ and $\bm{B}\in \R^{m\times n}$,
    we define $\bm{A}+\bm{B}\in \R^{m\times n}$ as
    \begin{equation}\nonumber%\label{eq:}%
      \bm{A} + \bm{B} :=
      \begin{bmatrix}
        \bm{a}_{1}+\bm{b}_{1} & \bm{a}_{2}+\bm{b}_{2} & \ldots & \bm{a}_{n}+\bm{b}_{n} \\
      \end{bmatrix}
    \end{equation}
  \item For $\bm{A} \in \R^{m\times n}$ and $\bm{B}\in \R^{l\times m}$
    we define $\bm{B}\bm{A}\in \R^{l\times n}$ as
    \begin{equation}\nonumber%\label{eq:}%
      \bm{B}\bm{A} :=
      \begin{bmatrix}
        \bm{B}\bm{a}_{1} & \bm{B}\bm{a}_{2} & \ldots & \bm{B}\bm{a}_{n}
      \end{bmatrix}
    \end{equation}
  \end{itemize}

\item \textbf{Some facts}
  \begin{itemize}
  \item $\bm{C}(\bm{B}\bm{A}) = (\bm{C}\bm{B})\bm{A}$
  \item $\bm{C}(\bm{B}+\bm{A}) = \bm{C}\bm{B} + \bm{C}\bm{A}$
  \item $(\bm{C}+\bm{B})\bm{A} = \bm{C}\bm{A} + \bm{B}\bm{A}$
  \item $(\bm{A}\bm{B})^{\top} = \bm{B}^{\top}\bm{A}^{\top}$
  \item $(\bm{A}\bm{B})^{-1} = \bm{B}^{-1}\bm{A}^{-1}$
  \item $|\bm{A}\bm{B}| = |\bm{A}||\bm{B}|$
  \item $|\bm{A}^{-1}| = |\bm{A}|^{-1}$
  \item $|\bm{A}| \neq 0$ if and only if $\bm{A}$ is non-singular (i.e., $\bm{A}^{-1}$ exists)
  \item For any $\bm{A}\in\R^{n\times n}$,
    \begin{equation}\nonumber%\label{eq:}%
      |\bm{A}| = \sum_{i=1}^{n}a_{ij}\tilde{a}_{ij} \quad \forall j,
      \qquad
      |\bm{A}| = \sum_{j=1}^{n}a_{ij}\tilde{a}_{ij} \quad \forall i,
    \end{equation}
    where $\tilde{a}_{ij}$ is the cofactor of $a_{ij}$, defined by
    \begin{equation}\nonumber%\label{eq:}%
      \tilde{a}_{ij}:=
      (-1)^{i+j}
      \begin{vmatrix}
        a_{11} & \cdots & a_{1j-1} &  a_{1j+1} & \cdots & a_{1n} \\
        \vdots & \ddots & \vdots & \vdots & \ddots & \vdots \\
        a_{i-1\, 1} & \cdots & a_{i-1j-1} & a_{i-1j+1} & \cdots & a_{i-1n} \\
        a_{i+1\, 1} & \cdots & a_{i+1j-1} & a_{i+1j+1} & \cdots & a_{i+1n} \\
        \vdots & \ddots & \vdots & \vdots & \ddots & \vdots \\
        a_{m1} & \cdots & a_{mj-1} & a_{mj+1} & \cdots & a_{mn} \\
      \end{vmatrix}\nonumber
    \end{equation}
  \item If $\bm{A}\in \R^{n\times n}$ is nonsingular,
    \begin{equation}\nonumber%\label{eq:}%
      \bm{A}^{-1} = \frac{1}{|\bm{A}|}\tilde{\bm{A}},
      \quad\text{where}\quad
      \tilde{\bm{A}}:=
      \underbrace{
      \begin{bmatrix}
        \tilde{a}_{11} & \tilde{a}_{21} & \cdots & \tilde{a}_{n1} \\
        \tilde{a}_{12} & \tilde{a}_{22} & \cdots & \tilde{a}_{n2} \\
        \vdots & \vdots & \ddots & \vdots \\
        \tilde{a}_{1n} & \tilde{a}_{2n} & \cdots & \tilde{a}_{nn} \\
      \end{bmatrix}}_{\text{cofactor matrix}}
    \end{equation}
  \end{itemize}

\item \textbf{Partitioned matrices}
  
  \begin{itemize}

  \item Consider a matrix of the form
    \begin{equation}\nonumber%\label{eq:}%
      \bm{A} = 
      \begin{bmatrix}
        \bm{A}_{11}&\bm{A}_{12}\\
        \bm{A}_{21}&\bm{A}_{22}
      \end{bmatrix}
    \end{equation}

  \item Then,
    the determinant of $\bm{A}$ is
    \begin{equation}\nonumber%\label{eq:}%
      |\bm{A}|
      =
      \begin{vmatrix}
        \bm{A}_{11}&\bm{A}_{12}\\
        \bm{A}_{21}&\bm{A}_{22}\end{vmatrix}
      = |\bm{A}_{11}||\bm{A}_{2|1}|
      = |\bm{A}_{22}||\bm{A}_{1|2}|
    \end{equation}
    and the inverse of $\bm{A}$ is
    \begin{align}\nonumber%\label{eq:}%
      \begin{bmatrix}
        \bm{A}_{11}&\bm{A}_{12}\\
        \bm{A}_{21}&\bm{A}_{22}
      \end{bmatrix}^{-1}
      & =
      \begin{bmatrix}
        \bm{A}_{1|2}^{-1} & -\bm{A}_{1|2}^{-1}\bm{A}_{12}\bm{A}_{22}^{-1} \\
        - \bm{A}_{22}^{-1}\bm{A}_{21}\bm{A}_{1|2}^{-1} & \bm{A}_{22}^{-1} + \bm{A}_{22}^{-1}\bm{A}_{21}\bm{A}_{1|2}^{-1}\bm{A}_{12}\bm{A}_{22}^{-1} \\
      \end{bmatrix}\\
      & =
      \begin{bmatrix}
        \bm{A}_{11}^{-1} + \bm{A}_{11}^{-1}\bm{A}_{12}\bm{A}_{2|1}^{-1}\bm{A}_{21}\bm{A}_{11}^{-1} & - \bm{A}_{11}^{-1}\bm{A}_{12}\bm{A}_{2|1}^{-1} \\
        -\bm{A}_{2|1}^{-1}\bm{A}_{21}\bm{A}_{11}^{-1} & \bm{A}_{2|1}^{-1} \\
      \end{bmatrix}
      \nonumber
    \end{align}
    where
    \begin{equation}\nonumber%\label{eq:}%
      \bm{A}_{1|2}:=\bm{A}_{11}- \bm{A}_{12}\bm{A}_{22}^{-1}\bm{A}_{21},
      \quad
      \bm{A}_{2|1}:=\bm{A}_{22}- \bm{A}_{21}\bm{A}_{11}^{-1}\bm{A}_{12}
    \end{equation}
  \end{itemize}

\item \textbf{Linear independence and determinant}
  \begin{itemize}
  \item We say that $\bm{v}_{1}, \ldots, \bm{v}_{k}\in \R^{n}$ are linearly independent if
    \begin{equation}\nonumber%\label{eq:}%
      c_{1}\bm{v}_{1} + \cdots + c_{k}\bm{v}_{k} = 0
      \implies
      c_{1} = \cdots = c_{k} = 0
    \end{equation}
  \item For a function $f(\bm{x}):=\bm{A}\bm{x}$ with a matrix $\bm{A}\in\R^{n\times k}$,    
    \begin{itemize}
    \item the following are equivalent:
      \begin{itemize}
      \item $f$ is an injective function ($f(\bm{x})= f(\bm{x}')$ implies $\bm{x}=\bm{x}'$)
      \item the column vectors of $\bm{A}$ are linearly independent
      \end{itemize}
    \item the following are equivalent:
      \begin{itemize}
      \item $f$ is a surjective function ($f(\R^{k})= \R^{n}$)
      \item the column vectors of $\bm{A}$ spans $\R^{n}$
      \end{itemize}
    \end{itemize}
  \item For a function $f(\bm{x}):=\bm{A}\bm{x}$ with a square matrix $\bm{A}\in\R^{n\times n}$,
    the following are equivalent:
    \begin{itemize}
    \item $f$ is an injective function
    \item $f$ is a surjective function
    \item $f$ is a bijective function (one-to-one)
    \end{itemize}
  \item For a square matrix $\bm{A}\in\R^{n\times n}$,
    the following are equivalent:
    \begin{itemize}
    \item column vectors of $\bm{A}$ are linearly independent
    \item $\bm{A}$ is non-singular
    \item $|\bm{A}|\neq 0$
    \end{itemize}
  \end{itemize}

\item \textbf{Sign of matrices}
  \begin{itemize}
  \item For a square matrix $\bm{A}\in \R^{n\times n}$,
    \begin{itemize}
    \item $\bm{A}$ is positive definite if $\bm{x}^{\top}\bm{A}\bm{x}>0$ for any $\bm{x}\neq \bm{0}$
    \item $\bm{A}$ is positive semidefinite if $\bm{x}^{\top}\bm{A}\bm{x}\geq 0$ for any $\bm{x}$
    \item $\bm{A}$ is negative definite if $\bm{x}^{\top}\bm{A}\bm{x}<0$ for any $\bm{x}\neq \bm{0}$
    \item $\bm{A}$ is negative semidefinite if $\bm{x}^{\top}\bm{A}\bm{x}\leq 0$ for any $\bm{x}$
    \end{itemize}
  \end{itemize}

\item \textbf{Symmetric matrices}
  \begin{itemize}
  \item A matrix $\bm{A}\in \R^{n\times n}$ is called a symmetric matrix if $\bm{A}^{\top} = \bm{A}$
  \item A matrix $\bm{P}\in \R^{n\times n}$ is called an orthogonal matrix if $\bm{P}^{\top} = \bm{P}^{-1}$
  \item For a  matrix $\bm{A}\in \R^{n\times n}$, the following are equivalent:
    \begin{itemize}
    \item $\bm{A}$ is a symmetric matrix
    \item $\bm{A}$ is decomposed as $\bm{A}=\bm{P}\bm{\Lambda}\bm{P}^{-1}$
      where $\bm{\Lambda}$ is a diagonal matrix and $\bm{P}$ is an orthogonal matrix
    \end{itemize}
  \item For a symmetric matrix $\bm{A}=\bm{P}\bm{\Lambda}\bm{P}^{-1}$,
    \begin{itemize}
    \item $\bm{A}$ is positive definite if and only if every diagonal element of $\bm{\Lambda}$ is positive
    \item $\bm{A}$ is positive semidefinite if and only if every diagonal element of $\bm{\Lambda}$ is nonnegative
    \item $\bm{A}$ is negative definite if and only if every diagonal element of $\bm{\Lambda}$ is negative
    \item $\bm{A}$ is negative semidefinite if and only if every diagonal element of $\bm{\Lambda}$ is nonpositive
    \end{itemize}
  \end{itemize}

\end{itemize}

\section{Matrix differentiation and integration}

\begin{itemize}

\item \textbf{Differentiation}
  \begin{itemize}
  \item For a function $f:\R^{n}\to \R$, we define
    \begin{equation}\nonumber%\label{eq:}%
      \frac{df(\bm{x})}{d\bm{x}} :=
      \begin{bmatrix}
        \frac{\partial f(\bm{x})}{\partial x_{1}}
        & \frac{\partial f(\bm{x})}{\partial x_{2}}
        & \cdots
        & \frac{\partial f(\bm{x})}{\partial x_{n}}
      \end{bmatrix}\in \R^{1\times n}
    \end{equation}
  \item For a function $\bm{f}:\R^{n}\to \R^{m}$, we define
    \begin{equation}\nonumber%\label{eq:}%
      \bm{f}(\bm{x})
      =
      \begin{bmatrix}
        f_{1}(\bm{x}) \\
        f_{2}(\bm{x}) \\
        \vdots \\
        f_{m}(\bm{x}) \\
      \end{bmatrix}
      \implies
      \frac{d\bm{f}(\bm{x})}{d\bm{x}} :=
      \begin{bmatrix}
        \frac{\partial f_{1}(\bm{x})}{\partial x_{1}}
        & \frac{\partial f_{1}(\bm{x})}{\partial x_{2}}
        & \cdots 
        & \frac{\partial f_{1}(\bm{x})}{\partial x_{n}} \\
        \frac{\partial f_{2}(\bm{x})}{\partial x_{1}}
        & \frac{\partial f_{2}(\bm{x})}{\partial x_{2}}
        & \cdots 
        & \frac{\partial f_{2}(\bm{x})}{\partial x_{n}} \\
        \vdots & \vdots & \ddots & \vdots \\
        \frac{\partial f_{m}(\bm{x})}{\partial x_{1}}
        & \frac{\partial f_{m}(\bm{x})}{\partial x_{2}}
        & \cdots 
        & \frac{\partial f_{m}(\bm{x})}{\partial x_{n}} \\
      \end{bmatrix}
      \in \R^{m\times n},
    \end{equation}
    which is often called the Jacobian matrix of $\bm{f}$
  \item For a function $\bm{F}:\R \to \R^{m\times n}$, we define
    \begin{equation}\nonumber%\label{eq:}%
      \bm{F}(x) =
      \begin{bmatrix}
        f_{11}(x) & f_{12}(x) & \cdots & f_{1n}(x) \\
        f_{21}(x) & f_{22}(x) & \cdots & f_{2n}(x) \\
        \vdots & \vdots & \ddots & \vdots \\
        f_{m1}(x) & f_{m2}(x) & \cdots & f_{mn}(x) \\
      \end{bmatrix}
      \implies
      \frac{d\bm{F}(x)}{dx} :=
      \begin{bmatrix}
        \frac{df_{11}(x)}{dx} & \frac{df_{12}(x)}{dx} & \cdots & \frac{df_{1n}(x)}{dx} \\
        \frac{df_{21}(x)}{dx} & \frac{df_{22}(x)}{dx} & \cdots & \frac{df_{2n}(x)}{dx} \\
        \vdots & \vdots & \ddots & \vdots \\
        \frac{df_{m1}(x)}{dx} & \frac{df_{m2}(x)}{dx} & \cdots & \frac{df_{mn}(x)}{dx} \\
      \end{bmatrix}
    \end{equation}
  \end{itemize}

\item \textbf{Some facts}
  \begin{itemize}
  \item For $\bm{A}\in \R^{m\times n}$,
    \begin{equation}\nonumber%\label{eq:}%
      \frac{d\bm{A}\bm{x}}{d\bm{x}} = \bm{A}
    \end{equation}
  \item For $\bm{A}\in \R^{m\times l}$ and $\bm{g}:\R^{n}\to \R^{l}$,
    \begin{equation}\nonumber%\label{eq:}%
      \frac{d\bm{A}\bm{g}(\bm{x})}{d\bm{x}} = \bm{A} \frac{d\bm{g}(\bm{x})}{d\bm{x}}
    \end{equation}
  \item For $\bm{F}:\R \to \R^{m\times l}$ and  $\bm{G}:\R \to \R^{l\times n}$,
    \begin{equation}\nonumber%\label{eq:}%
      \frac{d\bm{F}(x)\bm{G}(x)}{dx} =
      \frac{d \bm{F}(x)}{dx}\bm{G}(x)
      +
      \bm{F}(x)\frac{d\bm{G}(x)}{dx}
    \end{equation}
  \end{itemize}

\item \textbf{Integration}
  \begin{itemize}
  \item For a function $\bm{F}:\R \to \R^{m\times n}$, we define
    \begin{align}
      & \bm{F}(x) =
      \begin{bmatrix}
        f_{11}(x) & f_{12}(x) & \cdots & f_{1n}(x) \\
        f_{21}(x) & f_{22}(x) & \cdots & f_{2n}(x) \\
        \vdots & \vdots & \ddots & \vdots \\
        f_{m1}(x) & f_{m2}(x) & \cdots & f_{mn}(x) \\
      \end{bmatrix} \nonumber \\
      & \implies
      \int\bm{F}(x)dx :=
      \begin{bmatrix}
        \int f_{11}(x)dx & \int f_{12}(x)dx & \cdots & \int f_{1n}(x)dx \\
        \int f_{21}(x)dx & \int f_{22}(x)dx & \cdots & \int f_{2n}(x)dx \\
        \vdots & \vdots & \ddots & \vdots \\
        \int f_{m1}(x)dx & \int f_{m2}(x)dx & \cdots & \int f_{mn}(x)dx \\
      \end{bmatrix}
    \nonumber%\label{eq:}%
    \end{align}
  \end{itemize}

\end{itemize}

\section{Kronecker product and vectorization}
\label{sec:Kronecker_product}

\begin{itemize}

\item \textbf{Kronecker product}
  
  \begin{itemize}

  \item For any pair of matrices $\bm{A}\in \R^{m\times n}$ and $\bm{B}\in \R^{p\times q}$,
    we define their \emph{Kronecker product} $\bm{A}\otimes \bm{B} \in \R^{mp\times nq}$ by
    \begin{equation}\nonumber%\label{eq:}%
      \bm{A}\otimes \bm{B}
      :=
      \begin{bmatrix}
        a_{11}\bm{B} & a_{12}\bm{B} & \ldots & a_{1n}\bm{B} \\
        a_{21}\bm{B} & a_{22}\bm{B} & \ldots & a_{2n}\bm{B} \\
        \vdots & \vdots & \ddots & \vdots \\
        a_{m1}\bm{B} & a_{m2}\bm{B} & \ldots & a_{mn}\bm{B} \\
      \end{bmatrix}
    \end{equation}

  \item Example:
    \begin{equation}\nonumber%\label{eq:}%
      \begin{bmatrix}
        a_{11} & a_{12} \\
        a_{21} & a_{22} \\
      \end{bmatrix}
      \otimes
      \begin{bmatrix}
        b_{11} & b_{12} & b_{13} \\
        b_{21} & b_{22} & b_{23} \\
        b_{31} & b_{32} & b_{33} \\
      \end{bmatrix}
      = 
%      \begin{bmatrix}
%        a_{11}\bm{B} & a_{12}\bm{B} \\
%        a_{21}\bm{B} & a_{22}\bm{B} \\
%      \end{bmatrix}
%      =
      \begin{bmatrix}
        a_{11}b_{11} & a_{11}b_{12} & a_{11}b_{13} & a_{12}b_{11} & a_{12}b_{12} & a_{12}b_{13} \\
        a_{11}b_{21} & a_{11}b_{22} & a_{11}b_{23} & a_{12}b_{21} & a_{12}b_{22} & a_{12}b_{23} \\
        a_{11}b_{31} & a_{11}b_{32} & a_{11}b_{33} & a_{12}b_{31} & a_{12}b_{32} & a_{12}b_{33} \\
        a_{21}b_{11} & a_{21}b_{12} & a_{21}b_{13} & a_{22}b_{11} & a_{22}b_{12} & a_{22}b_{13} \\
        a_{21}b_{21} & a_{21}b_{22} & a_{21}b_{23} & a_{22}b_{21} & a_{22}b_{22} & a_{22}b_{23} \\
        a_{21}b_{31} & a_{21}b_{32} & a_{21}b_{33} & a_{22}b_{31} & a_{22}b_{32} & a_{22}b_{33} \\
      \end{bmatrix}
    \end{equation}

  \item Properties

    \begin{itemize}
    \item $(\bm{A}\otimes \bm{B}) (\bm{C}\otimes \bm{D}) = (\bm{A}\bm{C})\otimes (\bm{B}\bm{D})$
    \item $\bm{A}^{k}\otimes \bm{B}^{k} = (\bm{A}\otimes \bm{B})^{k}$ for any $k\in \N$
    \item $(\alpha \bm{A}\otimes \beta \bm{B}) = \alpha \beta (\bm{A}\otimes \bm{B})$ for any $\alpha, \beta \in \R$
    \end{itemize}

  \end{itemize}

\end{itemize}

\begin{itemize}
\item \textbf{Vectorization}
  
  \begin{itemize}

  \item We define $\vect(\bm{A})$ as the column vector created by stacking the column vectors of $\bm{A}$,
    namely,
    \begin{equation}\nonumber%\label{eq:}%
      \bm{A} =
      \begin{bmatrix}
        \bm{a}_{1} & \bm{a}_{2} & \cdots & \bm{a}_{n}
      \end{bmatrix}
      \implies 
      \vect(\bm{A}) :=
      \begin{bmatrix}
        \bm{a}_{1}\\
        \bm{a}_{2}\\
        \vdots\\
        \bm{a}_{n}\\
      \end{bmatrix}
      \in \R^{mn}
    \end{equation}
  \item Example:
    \begin{equation}\nonumber%\label{eq:}%
      \vect
      \left(
      \begin{bmatrix}
        a_{11} & a_{12} & a_{13} \\
        a_{21} & a_{22} & a_{23} \\
        a_{31} & a_{32} & a_{33} \\
        a_{41} & a_{42} & a_{43} \\
      \end{bmatrix}
    \right)
    = 
      \begin{bmatrix}
        a_{11} \\
        a_{21} \\
        a_{31} \\
        a_{41} \\
        a_{12} \\
        a_{22} \\
        a_{32} \\
        a_{42} \\
        a_{13} \\
        a_{23} \\
        a_{33} \\
        a_{43} \\
      \end{bmatrix}
    \end{equation}
  \item Properties:
    \begin{itemize}
    \item $\vect(\bm{A} + \bm{B}) = \vect(\bm{A}) + \vect(\bm{B})$ (by definition)
    \item $\vect(\bm{A}\bm{B}\bm{C}) = (\bm{C}^{\top}\otimes \bm{A})\vect(\bm{B})$ %(See the textbook) %, for example, \cite{Abadir2005-aa})
    \item $\vect(\sum_{k=0}^{t}\bm{A}\bm{B}\bm{A}^{\top}) = \sum_{k=0}^{t}(\bm{A}\otimes \bm{A})^{k}\vect(\bm{B})$ %(See the textbook) %, for example, \cite{Abadir2005-aa})
    \end{itemize}
  \item Example:
    \begin{align}
      & \vect \left(
        \begin{bmatrix}
          a_{11} & a_{12} \\
          a_{21} & a_{22} \\
          a_{31} & a_{32} \\
        \end{bmatrix}
        \begin{bmatrix}
          b_{11} & b_{12} \\
          b_{21} & b_{22} \\
        \end{bmatrix}
        \begin{bmatrix}
          c_{11} & c_{12} & c_{13} \\
          c_{21} & c_{22} & c_{23}\\
        \end{bmatrix}
        \right)
     = 
     \begin{bmatrix}
        a_{11}b_{11}c_{11} + a_{12}b_{21}c_{11} + a_{11}b_{12}c_{21} + a_{12}b_{22}c_{21} \\
        a_{21}b_{11}c_{11} + a_{22}b_{21}c_{11} + a_{21}b_{12}c_{21} + a_{22}b_{22}c_{21} \\
        a_{31}b_{11}c_{11} + a_{32}b_{21}c_{11} + a_{31}b_{12}c_{21} + a_{32}b_{22}c_{21} \\
        a_{11}b_{11}c_{12} + a_{12}b_{21}c_{12} + a_{11}b_{12}c_{22} + a_{12}b_{22}c_{22} \\
        a_{21}b_{11}c_{12} + a_{22}b_{21}c_{12} + a_{21}b_{12}c_{22} + a_{22}b_{22}c_{22} \\
        a_{31}b_{11}c_{12} + a_{32}b_{21}c_{12} + a_{31}b_{12}c_{22} + a_{32}b_{22}c_{22} \\
        a_{11}b_{11}c_{13} + a_{12}b_{21}c_{13} + a_{11}b_{12}c_{23} + a_{12}b_{22}c_{23} \\
        a_{21}b_{11}c_{13} + a_{22}b_{21}c_{13} + a_{21}b_{12}c_{23} + a_{22}b_{22}c_{23} \\
        a_{31}b_{11}c_{13} + a_{32}b_{21}c_{13} + a_{31}b_{12}c_{23} + a_{32}b_{22}c_{23} \\
     \end{bmatrix} \nonumber \\
      & \qquad
        =   
     \begin{bmatrix}
       c_{11}a_{11} & c_{11}a_{12} & c_{21}a_{11} & c_{21}a_{12} \\
       c_{11}a_{21} & c_{11}a_{22} & c_{21}a_{21} & c_{21}a_{22} \\
       c_{11}a_{31} & c_{11}a_{32} & c_{21}a_{31} & c_{21}a_{32} \\
       c_{12}a_{11} & c_{12}a_{12} & c_{22}a_{11} & c_{22}a_{12} \\
       c_{12}a_{21} & c_{12}a_{22} & c_{22}a_{21} & c_{22}a_{22} \\
       c_{12}a_{31} & c_{12}a_{32} & c_{22}a_{31} & c_{22}a_{32} \\
       c_{13}a_{11} & c_{13}a_{12} & c_{23}a_{11} & c_{23}a_{12} \\
       c_{13}a_{21} & c_{13}a_{22} & c_{23}a_{21} & c_{23}a_{22} \\
       c_{13}a_{31} & c_{13}a_{32} & c_{23}a_{31} & c_{23}a_{32} \\
     \end{bmatrix} 
     \begin{bmatrix}
       b_{11} \\
       b_{21} \\
       b_{12} \\
       b_{22} \\
     \end{bmatrix}
       =
       \left(
     \begin{bmatrix}
       c_{11} & c_{21} \\
       c_{12} & c_{22} \\
       c_{13} & c_{23} \\
     \end{bmatrix}
     \otimes
     \begin{bmatrix}
       a_{11} & a_{12} \\
       a_{21} & a_{22} \\
       a_{31} & a_{32} \\
     \end{bmatrix}
       \right)       
     \begin{bmatrix}
       b_{11} \\
       b_{21} \\
       b_{12} \\
       b_{22} \\
     \end{bmatrix}
    \nonumber%\label{eq:}%
    \end{align}
  \end{itemize}

\end{itemize}

\end{document}
