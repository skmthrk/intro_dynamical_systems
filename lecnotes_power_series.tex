\documentclass[12pt,a4paper]{article} 
\usepackage[T1]{fontenc}
\usepackage{textcomp}
\usepackage{makeidx}

\usepackage[sc, noBBpl]{mathpazo}
\usepackage{dsfont} % for indicator function
\usepackage[deluxe]{otf}
\usepackage[utf8]{inputenc}

\usepackage{amsmath}
\usepackage{amsfonts}
\usepackage{amssymb}
\usepackage{amsthm}
\usepackage{mathtools}
\usepackage{mathrsfs}

\usepackage{listings} % for programming codes

\usepackage{bm}
\usepackage[dvipdfmx]{color}
\pagestyle{plain} 
\usepackage{float}
\usepackage[dvipdfmx]{graphicx}

\usepackage[left=15mm, right=18mm, top=20mm, bottom=20mm]{geometry}

\setlength{\footskip}{40pt} 
\setlength{\abovecaptionskip}{0pt}
\setlength{\belowcaptionskip}{0pt}
\renewcommand{\baselinestretch}{1.05}

\newcommand{\argmax}{\mathop{\rm arg max}\limits}
\newcommand{\argmin}{\mathop{\rm arg min}\limits}
\newcommand{\st}{\mathop{\rm s.t.}\limits}
\newcommand{\fsize}[1]{\fontsize{#1}{#1}\selectfont}

% custom color
\usepackage[dvipdfmx]{xcolor}

% colors
\definecolor{darkred}{rgb}{.40,.00,.00}
\definecolor{navy}{rgb}{.00,.00,.40}
\definecolor{darknavy}{rgb}{.00,.00,.30}
\definecolor{lightnavy}{rgb}{.10,.10,.40}

\newcommand{\navy}{\color{navy}}
\newcommand{\deepred}{\color{darkred}}
\newcommand{\emred}[1]{\textbf{\deepred #1}}
\newcommand{\emblue}[1]{\textbf{\navy #1}}

% title and tableofcontents
\usepackage[titles]{tocloft}
\setlength{\cftbeforesecskip}{2pt}
\setlength{\cftbeforesubsecskip}{0pt}
\renewcommand{\cftsecfont}{\normalsize\mcfamily}
\renewcommand{\cftsubsecfont}{\normalsize\mcfamily}
\renewcommand{\cftsubsecfont}{\normalsize\mcfamily}
\renewcommand{\contentsname}{\large\mcfamily\bfseries \S\hspace{0.5\baselineskip}Contents\\[-\baselineskip]}
\cftsetindents{section}{10.0pt}{15.0pt}
\cftsetindents{subsection}{20.0pt}{30.0pt}
\cftpagenumbersoff{section}
\cftpagenumbersoff{subsection}
\cftpagenumbersoff{subsubsection}

\makeatletter
\def\vhrulefill#1{\leavevmode\leaders\hrule\@height#1\hfill \kern\z@}
\makeatother

\makeatletter
\def\maketitle{%
	\begin{center}\leavevmode
	\normalfont
	{\normalsize\raggedleft \@date\par}%
	\vskip -0.7\baselineskip
	\vhrulefill{0.2pt}\par
	\vskip 0.3\baselineskip
	\renewcommand{\baselinestretch}{0.80}
	{\color{darknavy}\bfseries\Large\raggedright \@title\par}%
	\renewcommand{\baselinestretch}{1.00}
	\vskip 0.2\baselineskip
	\ifx\@subtitle\@empty\else%
	{\normalsize\raggedright \@subtitle\par}%
	\fi
	\ifx\@author\@empty\else%
	\vskip 0.5\baselineskip
	{\normalsize\raggedleft \@author\par}%
	\fi
	\end{center}%
	\vskip 1.0\baselineskip
}
\def\subtitle#1{\gdef\@subtitle{#1}}
\let\@subtitle\@empty
\makeatother

% section
\newcommand{\Secindent}{0.0\baselineskip}
\newcommand{\Subsecindent}{0.0\baselineskip}
\newcommand{\lmargini}{1.3\baselineskip}
\newcommand{\lmarginii}{0.5\baselineskip}
\newcommand{\lmarginiii}{0.2\baselineskip}
\def\theenumi{\arabic{enumi}}
\def\theenumii{\alph{enumii}}
\renewcommand{\labelenumi}{\hfill\hspace{60pt}\theenumi.}
\renewcommand{\labelenumii}{\hspace{30pt}\theenumii.}
\renewcommand{\labelitemi}{\small\textbullet\hspace{2pt}}
\renewcommand{\labelitemii}{\color{gray}\small$\circ$\hspace{2pt}}
\renewcommand{\labelitemiii}{\textendash\hspace{1pt}}
\makeatletter
\renewcommand{\section}{\@startsection%
	{section}{1}{\Secindent}% name, depth, indent \z@=0pt
	{-1.0\baselineskip \@plus1.5\baselineskip \@minus0.5\baselineskip}% beforeskip (no indent if negative)
	{0.5\baselineskip \@plus0.8\baselineskip \@minus0.3\baselineskip}% afterskip (no line break if negative)
	{\reset@font\large\bfseries\scshape\color{darknavy}}% font
}
\renewcommand{\subsection}{\@startsection%
	{subsection}{2}{\Subsecindent}%
	{-0.5\baselineskip \@plus0.7\baselineskip \@minus0.4\baselineskip}%
	{0.3\baselineskip \@plus0.5\baselineskip \@minus0.2\baselineskip}%
	{\reset@font\normalsize\bfseries\scshape\color{darknavy}}% 
}
\renewcommand{\thesection}{\@arabic\c@section} % arabic, roman, Roman, alph, Alph, fnsymbol
\renewcommand{\thesubsection}{\thesection\hspace{0pt}.\hspace{0pt}\@arabic\c@subsection}
\makeatother

\makeatletter
\def\@listi{%
	\topsep = 0.2\baselineskip
	\partopsep = 0.0\baselineskip
	\parsep = 0.0\baselineskip
	\itemsep = 0.5\baselineskip
	\leftmargin = 1.5zw
	\labelsep = 0.5zw
	\itemindent = 0pt
	\labelwidth\leftmargin \advance\labelwidth-\labelsep
	\listparindent = 0zw
	\rightmargin = 0pt
	\reset@font\normalsize\normalfont
}
\let\@listI\@listi
\@listi
\def\@listii{%
	\topsep = 0.25\baselineskip
	\partopsep = 0.0\baselineskip
	\parsep = 0.0\baselineskip
	\itemsep = 0.3\baselineskip
	\leftmargin = 0.5zw
	\labelsep = .5zw
	\itemindent = 0pt
	\labelwidth\leftmargin \advance\labelwidth-\labelsep
	\listparindent = 0zw
	\rightmargin = 0pt
	\reset@font\normalsize\normalfont%
}
\def\@listiii{%
	\topsep = 0.30\baselineskip
	\partopsep = 0.0\baselineskip
	\parsep = 0.1\baselineskip
	\itemsep = 0.05\baselineskip
	\leftmargin = 1.1zw
	\labelsep = .5zw
	\itemindent = 0pt
	\labelwidth\leftmargin \advance\labelwidth-\labelsep
	\listparindent = 0zw
	\rightmargin = 0pt
	\reset@font\normalsize\normalfont%
}
\def\@listiv{%
	\topsep = 0.0\baselineskip
	\partopsep = 0.0\baselineskip
	\parsep = 0.0\baselineskip
	\itemsep = 0.0\baselineskip
	\leftmargin = 0.9zw
	\labelsep = .5zw
	\itemindent = 0pt
	\labelwidth\leftmargin \advance\labelwidth-\labelsep
	\listparindent = 0zw
	\rightmargin = 0pt
	\reset@font\normalsize\normalfont%
}
\makeatother

\newcommand{\R}{\ensuremath{\mathbb{R}}}
\newcommand{\E}{\ensuremath{\mathbb{E}}}
\newcommand{\V}{\ensuremath{\mathbb{V}}}
\newcommand{\Cov}{\ensuremath{\text{Cov}}}
\newcommand{\N}{\ensuremath{\mathbb{N}}}
\newcommand{\Z}{\ensuremath{\mathbb{Z}}}
\let\epsilon\varepsilon

\let\origitem\item
\renewcommand{\item}{\normalfont\origitem}

\usepackage[%
backref=page,
dvipdfmx,
bookmarks=true,bookmarksnumbered=true,bookmarkstype=toc,
colorlinks=true,
linkcolor=navy,
citecolor=navy,
filecolor=blue,
pagecolor=blue,
urlcolor=blue
]{hyperref}
\AtBeginDvi{\special{pdf:tounicode EUC-UCS2}}

\renewcommand*{\backref}[1]{}
\renewcommand*{\backrefalt}[4]{%
  \ifcase #1 (Not cited.)%
  \or        [#2]%
  \else      [#2]%
  \fi}
 

\title{Matrix series}
\subtitle{Introduction to dynamical systems~\#4}
\author{Hiroaki Sakamoto}
\date{\today}

\begin{document}
\maketitle
\tableofcontents

\section{Numerical series and convergence}

\subsection{Sequences and series}

\begin{itemize}
\item \textbf{Cauchy sequences}
  \begin{itemize}
  \item A sequence $\{a_{k}\}$ in $\R$ is said to be a \emph{Cauchy sequence}
    if for every $\epsilon>0$, there is an integer $n\in \N$ such that
    \begin{equation}\nonumber%\label{eq:}%
      l\geq m\geq n  \implies |a_{l} - a_{m}|<\epsilon
    \end{equation}
  \item Cauchy criterion: for a sequence $\{a_{k}\}$ in $\R$, the following are euqivalent:
    \begin{itemize}
    \item it converges to some $a \in \R$: for every $\epsilon > 0$, there exists $n\in \N$ such that
      $|a_{k}-a|<\epsilon$ for all $k\geq n$
    \item it is a Cauchy sequence
    \end{itemize}
  \end{itemize}

\item \textbf{Series}
  \begin{itemize}
  \item Given a sequence $\{a_{k}\}$ in $\R$,
    we define another sequence $\{s_{t}\}$ in $\R$ by
    \begin{equation}\nonumber%\label{eq:}%
      \textstyle s_{t} := \sum_{k=0}^{t}a_{k} = a_{0} + a_{1} + \ldots + a_{t}, \quad \forall t = 0, 1, 2, \ldots,
    \end{equation}
    which is called a \emph{series} and denoted as $\sum_{k} a_{k}$
  \item We say that a series $\sum_{k} a_{k}$ converges if $\{s_{t}\}$ converges
  \item By the Cauchy criterion, the following are equivalent:
    \begin{itemize}
    \item a series $\sum_{k} a_{k}$ converges
    \item for any $\epsilon > 0$, there exists $n \in \N$ such that
    \begin{equation}\label{eq:series_Cauchy}%
      \textstyle l\geq m\geq n  \implies \left| \sum_{k=m}^{l}a_{k}\right|< \epsilon
    \end{equation}
    \end{itemize}
  \item It immediately follows that
    $\lim_{k\to\infty}a_{k}=0$
    whenever $\sum_{k} a_{k}$ converges:
    \begin{equation}\nonumber%\label{eq:}%
      \text{$\sum_{k} a_{k}$ is convergent}
      \implies
      \text{\eqref{eq:series_Cauchy} with $l=m$}
      \implies
      \lim_{k\to\infty} a_{k} = 0
    \end{equation}
  \end{itemize}

\item \textbf{Example}
  \begin{itemize}
  \item If $a_{k}:=x$ for all $k$ and $|x|<1$, then the series $\sum_{k}a_{k}$ converges:
    \begin{equation}\label{eq:xk}%
      \textstyle \sum_{k=0}^{\infty}a_{k}
      =
      x^{0} + 
      x^{1} + 
      x^{2} + \ldots
      =
      (1-x)^{-1}
    \end{equation}
  \end{itemize}
 
\end{itemize}

\subsection{Tests for convergence}

\begin{itemize}

\item \textbf{Comparison test}
  \begin{itemize}
  \item Consider a sequence $\{a_{k}\}$ in $\R$
  \item If there exists another sequence $\{b_{k}\}$ in $\R$ such that
    \begin{itemize}
    \item $\sum_{k}b_{k}$ converges, and
    \item $|a_{k}|\leq b_{k}$ for all $k\geq n_{0}$ for some fixed $n_{0}\in \N$,
    \end{itemize}
    then $\sum_{k}a_{k}$ converges
  \item Proof:
    \begin{itemize}
    \item Fix $\epsilon > 0$
    \item Since $\sum_{k}b_{k}$ converges, there exists $n_{1}\in \N$ such that
      \begin{equation}\nonumber%\label{eq:}%
        l\geq m\geq n_{1}  \implies \left| \sum_{k=m}^{l}b_{k}\right|< \epsilon
      \end{equation}
    \item Letting $n:=\max\{n_{0}, n_{1}\}$,
      we then have $b_{k} \geq |a_{k}|\geq 0$ for all $k\geq n$ and thus
      \begin{equation}\nonumber%\label{eq:}%
        l\geq m\geq n  \implies
        \left| \sum_{k=m}^{l}a_{k}\right|
        \leq
        \sum_{k=m}^{l}\left|a_{k}\right|
        \leq
        \sum_{k=m}^{l}b_{k}
        =
        \left| \sum_{k=m}^{l}b_{k}\right|< \epsilon,
      \end{equation}
      which implies $\sum_{k}a_{k}$ converges
    \end{itemize}
  \end{itemize}

\item \textbf{Root test}
  \begin{itemize}
  \item Consider a sequence $\{a_{k}\}$ such that the limit
    \begin{equation}\nonumber%\label{eq:}%
      r:=\lim_{k\to\infty} \left|a_{k}\right|^{1/k}
    \end{equation}
    exists in $\R\cup\{\infty\}$
  \item If $r<1$, then $\sum_{k}a_{k}$ converges because:
    \begin{itemize}
    \item Since $r<1$, one can choose $\beta\in \R$ such that $r < \beta < 1$
    \item Since $\lim_{k\to\infty}\left|a_{k}\right|^{1/k}=r$, there exists $n_{0}\in \N$ such that
      \begin{equation}\nonumber%\label{eq:}%
        k\geq n_{0} \implies |a_{k}|^{1/k}< \beta \implies |a_{k}| < \beta^{k}
      \end{equation}
    \item Since $\sum_{k}\beta^{k}$ is convergent,
      so is $\sum_{k}a_{k}$ by the comparison test
    \end{itemize}
  \item If $r>1$, then $\sum_{k}a_{k}$ does not converge because:
    \begin{itemize}
    \item Since $r>1$ and since $\lim_{k\to\infty}\left|a_{k}\right|^{1/k}=r$,
      there exists $n\in \N$ such that
      \begin{equation}\nonumber%\label{eq:}%
        k\geq n \implies |a_{k}|^{1/k} > 1 \implies |a_{k}| > 1,
      \end{equation}
      which means $\lim_{k\to\infty}a_{k}\neq 0$, violating the necessary condition for series convergence
    \end{itemize}
  \end{itemize}

\item \textbf{Ratio test}
  \begin{itemize}
  \item Consider a sequence $\{a_{k}\}$ such that the limit
    \begin{equation}\nonumber%\label{eq:}%
      r:=\lim_{k\to\infty} \left| \frac{a_{k+1}}{a_{k}} \right|
    \end{equation}
    exists in $\R\cup\{\infty\}$
  \item If $r < 1$, then $\sum_{k}a_{k}$ converges because:
    \begin{itemize}
    \item Since $r<1$, one can choose $\beta\in \R$ such that $r < \beta < 1$
    \item Since $\lim_{k\to\infty}\left|a_{k+1}/a_{k}\right|=r$, there exists $n\in \N$ such that
      \begin{equation}\nonumber%\label{eq:}%
        k\geq n
        \implies |a_{k+1}/a_{k}| < \beta
        \implies |a_{k+1}| < \beta|a_{k}|,
      \end{equation}
      which implies $|a_{k}|< |a_{n}|\beta^{k-n}$ for all $k\geq n+1$
    \item Since $\sum_{k}|a_{n}|\beta^{k-n}$ is convergent,
      so is $\sum_{k}a_{k}$ by the comparison test
    \end{itemize}
    
  \item If $r > 1$, then $\sum_{k}a_{k}$ does not converge because:
    \begin{itemize}
    \item Since $r>1$ and
      since $\lim_{k\to\infty}\left|a_{k+1}/a_{k}\right|=r$, there exists $n\in \N$ such that
      \begin{equation}\nonumber%\label{eq:}%
        k\geq n
        \implies |a_{k+1}/a_{k}| > 1
        \implies |a_{k+1}| > |a_{k}|,
      \end{equation}
      which means $\lim_{k\to\infty}a_{k}\neq 0$, violating the necessary condition for series convergence
    \end{itemize}
  \end{itemize}

\item \textbf{Power series and radius of convergence}
  \begin{itemize}
  \item A series $\sum_{k}a_{k}$ of the form
    \begin{equation}\nonumber%\label{eq:}%
      a_{k}:= \alpha_{k}x^{k} \quad \forall k = 0, 1, 2, \ldots,
    \end{equation}
    is called a \emph{power series}
  \item \textbf{Root test}: consider a power series $\sum_{k}\alpha_{k}x^{k}$ such that
    \begin{equation}\nonumber%\label{eq:}%
      \alpha := \lim_{k\to\infty}|\alpha_{k}|^{1/k}
    \end{equation}
    exists in $\R\cup\{\infty\}$
    \begin{itemize}
    \item Define $R\in \R\cup\{\infty\}$ by
      \begin{equation}\nonumber%\label{eq:}%
        R :=
        \begin{cases}
          0 & \text{if $\alpha\in \{-\infty, \infty\}$} \\
          \infty & \text{if $\alpha = 0$} \\
          1/\alpha & \text{otherwise} \\
        \end{cases},
      \end{equation}
      which is called the \emph{radius of convergence} of the power series $\sum_{k}\alpha_{k}x^{k}$
    \item If $|x|<R$, then $\sum_{k}\alpha_{k}x^{k}$ converges because
      $\lim_{k\to\infty}|\alpha_{k}x^{k}|^{1/k} = \lim_{k\to\infty}|\alpha_{k}|^{1/k}|x|= \alpha|x|$
    \item Similarly, if $|x|>R$, then $\sum_{k}\alpha_{k}x^{k}$ does not converge
    \end{itemize}
  \item \textbf{Ratio test}: consider a power series $\sum_{k}\alpha_{k}x^{k}$ such that
    \begin{equation}\nonumber%\label{eq:}%
      \alpha := \lim_{k\to\infty}|\alpha_{k+1}/\alpha_{k}|
    \end{equation}
    exists in $\R\cup\{\infty\}$
    \begin{itemize}
    \item The radius of convergence is given by $R:=1/\alpha$ because
      \begin{equation}\nonumber%\label{eq:}%
        \lim_{k\to\infty}\left| \frac{\alpha_{k+1}x^{k+1}}{\alpha_{k}x^{k}}\right|
        =
        \lim_{k\to\infty}\left| \frac{\alpha_{k+1}}{\alpha_{k}}\right||x|
        =
        \alpha|x|
      \end{equation}
    \end{itemize}
  \end{itemize}

\item \textbf{Examples}
  \begin{itemize}
  \item What is the radius of convergence of the series defined in \eqref{eq:xk}?
  \item Does the following series converge? What is the radius of convergence?
    \begin{equation}\label{eq:ex}%
      \sum_{k=0}^{\infty} \frac{1}{k!}x^{k}
      := 1 + x + \frac{1}{2!}x^{2} + \frac{1}{3!}x^{3} + \ldots
    \end{equation}
  \end{itemize}

\end{itemize}

\section{Matrix series and convergence}

\subsection{Powers of matrices}

\begin{itemize}
\item \textbf{Powers of Jordan blocks}
  \begin{itemize}
  \item Let $\bm{J}_{m}(\lambda)$ be a Jordan block of size $m$:
    \begin{equation}\nonumber%\label{eq:}%
      \bm{J}_{m}(\lambda) :=
      \begin{bmatrix}
        \lambda & 1 & 0 & \cdots & 0 & 0 \\
        0 & \lambda & 1 & \cdots & 0 & 0 \\
        0 & 0 & \lambda & \ddots & 0 & 0 \\
        \vdots & \vdots & \vdots & \ddots & \ddots & \vdots \\
        0 & 0 & 0 & \cdots & \lambda & 1 \\
        0 & 0 & 0  & \cdots & 0 &  \lambda \\
      \end{bmatrix}
      =
      \lambda \bm{I}
      +
      \underbrace{
      \begin{bmatrix}
        0 & 1 & 0 & \cdots & 0 & 0 \\
        0 & 0 & 1 & \cdots & 0 & 0 \\
        0 & 0 & 0 & \ddots & 0 & 0 \\
        \vdots & \vdots & \vdots & \ddots & \ddots & \vdots \\
        0 & 0 & 0 & \cdots & 0 & 1 \\
        0 & 0 & 0  & \cdots & 0 & 0 \\
      \end{bmatrix}}_{=:\bm{Z}}
    \end{equation}
  \item It follows form the binomial theorem\footnote{
      For any $a, b \in \R$ and $k\in \N$, we have
      $(a+b)^{k}=
      a^{k} + ka^{k-1}b + \frac{k(k-1)}{2}a^{k-2}b^{2}
      + \cdots
      + \frac{k(k-1)}{2}a^{2}b^{k-2} + kab^{k-1} + b^{k}
      =
      \sum_{l=0}^{k}\frac{k!}{l!(k-l)!}a^{k-l}b^{l}$.
      The binomial theorem is also true for square matrices $\bm{A}, \bm{B} \in \R^{n\times n}$
      provided that $\bm{A}\bm{B} = \bm{B}\bm{A}$.
    } that, for any $k\in \N$,
    \begin{equation}\nonumber%\label{eq:}%
      \textstyle (\bm{J}_{m}(\lambda))^{k}
      = (\lambda\bm{I} + \bm{Z})^{k}
      = \sum_{l=0}^{k}\frac{k!}{l!(k-l)!}\lambda^{k-l}\bm{Z}^{l},
    \end{equation}
    where
    \begin{equation}\nonumber%\label{eq:}%
      \begin{aligned}
      &  \bm{Z}^{0} = 
      \begin{bmatrix}
        \bm{e}_{1} & \bm{e}_{2} & \bm{e}_{3} & \bm{e}_{4} & \bm{e}_{5} & \bm{e}_{6} & \cdots & \bm{e}_{m} 
      \end{bmatrix} = \bm{I}\\
      &  \bm{Z}^{1} = 
      \begin{bmatrix}
        \bm{0} & \bm{e}_{1} & \bm{e}_{2} & \bm{e}_{3} & \bm{e}_{4} & \bm{e}_{5} & \cdots & \bm{e}_{m-1} 
      \end{bmatrix}\\
      & \bm{Z}^{2} = 
      \begin{bmatrix}
        \bm{0} & \bm{0} & \bm{e}_{1} & \bm{e}_{2} & \bm{e}_{3} & \bm{e}_{4} & \cdots & \bm{e}_{m-2} 
      \end{bmatrix}\\
      & \quad\vdots \\
      & \bm{Z}^{m-1} = 
      \begin{bmatrix}
        \bm{0} & \bm{0} & \bm{0} & \bm{0} & \bm{0} & \bm{0} & \cdots & \bm{e}_{1} 
      \end{bmatrix}\\
      & \bm{Z}^{l} = 
      \begin{bmatrix}
        \bm{0} & \bm{0} & \bm{0} & \bm{0} & \bm{0} & \bm{0} & \cdots & \bm{0}
      \end{bmatrix} = \bm{O} \quad \forall l\geq m\\
      \end{aligned}
    \end{equation}
  \item Consider the case with $m = 4$, for example:
    \begin{equation}\nonumber%\label{eq:}%
      \bm{Z} =
      \begin{bmatrix}
        0 & 1 & 0 & 0 \\
        0 & 0 & 1 & 0 \\
        0 & 0 & 0 & 1 \\
        0 & 0 & 0 & 0 \\
      \end{bmatrix},
      \quad
      \bm{Z}^{2} =
      \begin{bmatrix}
        0 & 0 & 1 & 0 \\
        0 & 0 & 0 & 1 \\
        0 & 0 & 0 & 0 \\
        0 & 0 & 0 & 0 \\
      \end{bmatrix},
      \quad
      \bm{Z}^{3} =
      \begin{bmatrix}
        0 & 0 & 0 & 1 \\
        0 & 0 & 0 & 0 \\
        0 & 0 & 0 & 0 \\
        0 & 0 & 0 & 0 \\
      \end{bmatrix},
      \quad
      \bm{Z}^{4} =
      \begin{bmatrix}
        0 & 0 & 0 & 0 \\
        0 & 0 & 0 & 0 \\
        0 & 0 & 0 & 0 \\
        0 & 0 & 0 & 0 \\
      \end{bmatrix}
    \end{equation}
    and thus
    \begin{align}
      (\bm{J}_{4}(\lambda))^{k}
        & = \frac{k!}{0!(k-0)!}\lambda^{k-0}\bm{Z}^{0} + \frac{k!}{1!(k-1)!}\lambda^{k-1}\bm{Z}^{1} \nonumber \\
        & \quad + \frac{k!}{2!(k-2)!}\lambda^{k-2}\bm{Z}^{2} +  \frac{k!}{3!(k-1)!}\lambda^{k-3}\bm{Z}^{3} + \bm{O}  \nonumber \\
        & = \frac{k!}{0!(k-0)!}\lambda^{k-0}
          \begin{bmatrix}
            1 & 0 & 0 & 0 \\
            0 & 1 & 0 & 0 \\
            0 & 0 & 1 & 0 \\
            0 & 0 & 0 & 1 \\
          \end{bmatrix}
          + \frac{k!}{1!(k-1)!}\lambda^{k-1}
          \begin{bmatrix}
            0 & 1 & 0 & 0 \\
            0 & 0 & 1 & 0 \\
            0 & 0 & 0 & 1 \\
            0 & 0 & 0 & 0 \\
          \end{bmatrix}
          \nonumber \\
        & \quad + \frac{k!}{2!(k-2)!}\lambda^{k-2}
          \begin{bmatrix}
            0 & 0 & 1 & 0 \\
            0 & 0 & 0 & 1 \\
            0 & 0 & 0 & 0 \\
            0 & 0 & 0 & 0 \\
          \end{bmatrix}
        + \frac{k!}{3!(k-3)!}\lambda^{k-3}
          \begin{bmatrix}
            0 & 0 & 0 & 1 \\
            0 & 0 & 0 & 0 \\
            0 & 0 & 0 & 0 \\
            0 & 0 & 0 & 0 \\
          \end{bmatrix}
          \nonumber \\
    \nonumber%\label{eq:}%
    \end{align}
  \item In general,
    \begin{equation}\nonumber%\label{eq:Jnit}%
      (\bm{J}_{m}(\lambda))^{k}
      =
      \begin{bmatrix}
        c_{0}(k) & c_{1}(k) & c_{2}(k) & \cdots & c_{m-1}(k) \\
        0 & c_{0}(k) & c_{1}(k) & \cdots & c_{m-2}(k) \\
        0 & 0 & c_{0}(k) & \ddots & \vdots \\
        \vdots & \vdots & \vdots & \ddots & \vdots \\
        0 & 0 & 0  & 0 & c_{0}(k) \\
      \end{bmatrix},
      \quad
      c_{l}(k) :=
      \begin{cases}
        \frac{k!}{l!(k-l)!}\lambda^{k-l} & k\geq l \\
        0 & k < l \\
      \end{cases}
    \end{equation}
  \item Notice that for each $l=1,\ldots, m-1$
    \begin{equation}\nonumber%\label{eq:}%
      c_{l}(k)
      = \frac{k!}{l!(k-l)!}\lambda^{k-l}
      = \frac{k(k-1)\cdots (k-l+1)}{l(l-1)\cdots (l-l+1)}\lambda^{k-l}
      \leq k^{l}\lambda^{k-l}
      \quad \forall k \geq l,
    \end{equation}
    which, since $\lim_{k\to\infty} k^{l}\lambda^{k}=0$ iff $|\lambda|<1$, implies
    \begin{equation}\nonumber%\label{eq:}%
      \lim_{k\to\infty}c_{l}(k) = 0
      \iff
      |\lambda| < 1
    \end{equation}
  \item Therefore
    \begin{equation}\nonumber%\label{eq:}%
      \lim_{k\to\infty}(\bm{J}_{m}(\lambda))^{k} = \bm{O}
      \iff
      |\lambda| < 1
    \end{equation}
  \end{itemize}

\item \textbf{Powers of Jordan matrices}
  \begin{itemize}
  \item Let $\bm{J}\in \R^{n\times n}$ be a Jordan matrix of the following form:
    \begin{equation}\nonumber%\label{eq:}%
      \bm{J} =
      \begin{bmatrix}
        \bm{J}_{n_{1}}(\lambda_{1}) & \bm{O} & \cdots & \bm{O} \\
        \bm{O} & \bm{J}_{n_{2}}(\lambda_{2}) & \cdots & \bm{O} \\
        \vdots & \vdots & \ddots & \vdots \\
        \bm{O} & \bm{O} & \cdots & \bm{J}_{n_{d}}(\lambda_{d}) \\
      \end{bmatrix},
    \end{equation}
    where $n_{1}+n_{2}+\ldots + n_{d} = n$
  \item The $k$th power of $\bm{J}$ is
    \begin{equation}\nonumber%\label{eq:}%
      \bm{J}^{k}
      =
      \begin{bmatrix}
        (\bm{J}_{n_{1}}(\lambda_{1}))^{k} & \bm{O} & \ldots & \bm{O} \\
        \bm{O} & (\bm{J}_{n_{2}}(\lambda_{2}))^{k} & \ldots & \bm{O} \\
        \vdots & \vdots & \ddots & \vdots \\
        \bm{O} & \bm{O} & \ldots & (\bm{J}_{n_{d}}(\lambda_{d}))^{k} \\
      \end{bmatrix}
    \end{equation}
    and therefore
    \begin{equation}\nonumber%\label{eq:}%
      \lim_{k\to\infty}\bm{J}^{k} = \bm{O}
      \iff
      \max\{|\lambda_{1}|,|\lambda_{2}|, \ldots, |\lambda_{d}|\} < 1
    \end{equation}
  \end{itemize}

\item \textbf{Powers of general matrices}
  \begin{itemize}
  \item Consider a square matrix $\bm{A}\in \R^{n\times n}$
  \item Denote by $\rho(\bm{A})\in \R_{+}$ the largest (in absolute terms) eigenvalue of $\bm{A}$, i.e.,
    \begin{equation}\nonumber%\label{eq:}%
      \rho(\bm{A}) := \max\left\{ |\lambda|\in\R_{+} \,|\, \text{$\lambda$ is an eigenvalue of $\bm{A}$} \right\},
    \end{equation}
    which is called the \emph{spectral radius} of $\bm{A}$
  \item The $k$th power of $\bm{A}$ is
    \begin{equation}\nonumber%\label{eq:}%
      \bm{A}^{k}
      = (\bm{V} \bm{J} \bm{V}^{-1})^{k}
      = \bm{V} \bm{J}^{k} \bm{V}^{-1}
    \end{equation}
    and therefore
    \begin{equation}\nonumber%\label{eq:}%
      \lim_{k\to\infty}\bm{A}^{k} = \bm{O}
      \iff
      \bm{V} \left(\lim_{k\to \infty}\bm{J}^{k}\right)\bm{V}^{-1} = \bm{O}
      \iff
      \rho(\bm{A}) < 1
    \end{equation}
   \item Note: if $\rho(\bm{A})<1$ and $\rho(\bm{B})<1$, then $\rho(\bm{A}\otimes \bm{B})<1$
  \end{itemize}

\end{itemize}

\subsection{Matrix series and its convergence}

\begin{itemize}

\item \textbf{Geometric series and Neumann series lemma}
  \begin{itemize}
  \item Let $\bm{A}\in \R^{n\times n}$ be an arbitrary square matrix
  \item We define the \emph{geometric series} generated by $\bm{A}$ as
    \begin{equation}\nonumber%\label{eq:}%
      \sum_{k=0}^{t}\bm{A}^{k}
      := \bm{A}^{0} + \bm{A}^{1} + \bm{A}^{2} + \ldots + \bm{A}^{t},
      \quad
      t = 0, 1, 2, 3, \ldots
    \end{equation}
  \item The following result is called the \emph{Neumann series lemma}:
    \begin{equation}\nonumber%\label{eq:}%
      \text{
        $\sum_{k=0}^{t}\bm{A}^{k}$ converges
      }
      \iff
      \rho(\bm{A}) < 1
    \end{equation}
  \item In particular, if $\rho(\bm{A}) < 1$, then $\sum_{k}\bm{A}^{k}$ converges and
    \begin{equation}\nonumber%\label{eq:}%
      \lim_{t\to\infty}\sum_{k=0}^{t}\bm{A}^{k} = (\bm{I}-\bm{A})^{-1}
    \end{equation}
  \end{itemize}

\item \textbf{Sufficiency ($\Longleftarrow$)}
  \begin{itemize}
  \item Notice that
    \begin{equation}\nonumber%\label{eq:}%
      \sum_{k=0}^{t}\bm{A}^{k}\left(\bm{I}-\bm{A}\right)
      =
      \left(\bm{A}^{0} + \bm{A}^{1} + \bm{A}^{2} + \ldots + \bm{A}^{t}\right)\left(\bm{I}-\bm{A}\right)
      =
      \bm{I}-\bm{A}^{t+1}
    \end{equation}
  \item Hence
    \begin{equation}\nonumber%\label{eq:}%
      \rho(\bm{A}) < 0
      \implies
      \lim_{t\to\infty}\bm{A}^{t+1}=\bm{O}
      \implies
      \lim_{t\to\infty}\sum_{k=0}^{t}\bm{A}^{k}\left(\bm{I}-\bm{A}\right) = \bm{I}
      \implies
      \lim_{t\to\infty}\sum_{k=0}^{t}\bm{A}^{k} = \left(\bm{I}-\bm{A}\right)^{-1}
    \end{equation}
  \end{itemize}

\item \textbf{Necessity ($\Longrightarrow$)}
  \begin{itemize}
  \item If $(\lambda,\bm{v})$ is an eigenpair of $\bm{A}$,
    we have $\bm{A}^{k}\bm{v}=\lambda^{k}\bm{v}$ and thus
    \begin{equation}\nonumber%\label{eq:}%
      \left(\sum_{k=0}^{t}\bm{A}^{k}\right)\bm{v}
      =
      \sum_{k=0}^{t}\left(\bm{A}^{k}\bm{v}\right)
      =
      \sum_{k=0}^{t}\left(\lambda^{k}\bm{v}\right)
      =
      \left(\sum_{k=0}^{t}\lambda^{k}\right)\bm{v},
    \end{equation}
    which means that (because $\bm{v}\neq \bm{0}$)
    \begin{equation}\nonumber%\label{eq:}%
      \text{$\sum_{k=0}^{t}\bm{A}^{k}$ converges}
      \implies
      \text{$\sum_{k=0}^{t}\lambda^{k}$ converges}
      \implies
      |\lambda|<1
    \end{equation}
  \item Since this must be true for any eigenpair of $\bm{A}$,
    we conclude that $\rho(\bm{A})<1$
  \end{itemize}

\item \textbf{Examples}
  \begin{itemize}
  \item Consider a square matrix
    \begin{equation}\nonumber%\label{eq:}%
      \bm{A} :=
      \begin{bmatrix}
        5/8 & -1/4 \\
        -1/16 & 5/8 \\
      \end{bmatrix}
    \end{equation}
  \item Does the series $\sum_{k}\bm{A}^{k}$ converge? If so, what is the limit?
  \item The characteristic polynomial of $\bm{A}$ is
    \begin{equation}\nonumber%\label{eq:}%
      \phi_{\bm{A}}(t) = 
      \begin{vmatrix}
        5/8 - t & -1/4 \\
        -1/16 & 5/8 - t \\
      \end{vmatrix}
      = (3/4-t)(1/2-t),
    \end{equation}
    which means that
    the eigenvalues of $\bm{A}$ are $\lambda_{1}:=3/4$ and $\lambda_{2}:=1/2$
  \item Since $\rho(\bm{A}) = \max\{|\lambda_{1}|,|\lambda_{2}|\} = 3/4 < 1$,
    we know that $\sum_{k}\bm{A}^{k}$ must converge to
    \begin{equation}\nonumber%\label{eq:}%
      (\bm{I}-\bm{A})^{-1}
      =
      \begin{bmatrix}
        1 - 5/8 & 1/4 \\
        1/16 & 1 - 5/8 \\
      \end{bmatrix}^{-1}
      =
      \begin{bmatrix}
        3 & -2 \\
        -1/2 & 3 \\
      \end{bmatrix}
    \end{equation}
  \item To verify this, 
    we decompose $\bm{A}$ through eigenvectors:
    \begin{equation}\nonumber%\label{eq:}%
      (\bm{A}-\lambda_{1}\bm{I})\bm{v}
      = \bm{0}
      \iff
      \begin{bmatrix}
        5/8 - 3/4 & - 1/4 \\
        -1/16 & 5/8 - 3/4 \\
      \end{bmatrix}
      \begin{bmatrix}
        v_{1} \\
        v_{2} \\
      \end{bmatrix}
      =
      \begin{bmatrix}
        0 \\
        0 \\
      \end{bmatrix}
      \iff
      \bm{v} = \alpha
      \begin{bmatrix}
        1 \\
        -\frac{1}{2}
      \end{bmatrix}
      \quad \forall \alpha
    \end{equation}
    and
    \begin{equation}\nonumber%\label{eq:}%
      (\bm{A}-\lambda_{2}\bm{I})\bm{v}
      = \bm{0}
      \iff
      \begin{bmatrix}
        5/8 - 1/2 & - 1/4 \\
        -1/16 & 5/8 - 1/2 \\
      \end{bmatrix}
      \begin{bmatrix}
        v_{1} \\
        v_{2} \\
      \end{bmatrix}
      =
      \begin{bmatrix}
        0 \\
        0 \\
      \end{bmatrix}
      \iff
      \bm{v} = \alpha
      \begin{bmatrix}
        1 \\
        \frac{1}{2}
      \end{bmatrix}
      \quad \forall \alpha
    \end{equation}
    so we choose
    \begin{equation}\nonumber%\label{eq:}%
      \bm{\Lambda}:=
      \begin{bmatrix}
       \lambda_{1} & 0\\
       0 & \lambda_{2}
      \end{bmatrix},
      \quad
      \bm{v}_{1} :=
      \begin{bmatrix}
        1 \\
        - \frac{1}{2}
      \end{bmatrix},
      \quad
      \bm{v}_{2} :=
      \begin{bmatrix}
        1 \\
        \frac{1}{2}
      \end{bmatrix}
    \end{equation}
    and
    \begin{equation}\nonumber%\label{eq:}%
      \bm{V}:=
      \begin{bmatrix}
        \bm{v}_{1} & \bm{v}_{2}
      \end{bmatrix}
      =
      \begin{bmatrix}
        1 & 1 \\
        -1/2 & 1/2 \\
      \end{bmatrix}
      \implies
      \bm{V}^{-1} =
      \begin{bmatrix}
        1/2 & -1 \\
        1/2 & 1 \\
      \end{bmatrix}
    \end{equation}
  \item It follows that
    \begin{equation}\nonumber%\label{eq:}%
      \bm{A}^{k}
      =
      \left(\bm{V}\bm{\Lambda}\bm{V}^{-1}\right)^{k}
      =
      \bm{V}\bm{\Lambda}^{k}\bm{V}^{-1}
      =
      \begin{bmatrix}
        1 & 1 \\
        -1/2 & 1/2 \\
      \end{bmatrix}
      \begin{bmatrix}
        (3/4)^{k} & 0 \\
        0 & (1/2)^{k} \\
      \end{bmatrix}
      \begin{bmatrix}
        1/2 & -1 \\
        1/2 & 1 \\
      \end{bmatrix}
    \end{equation}
    and therefore
    \begin{align}
      \sum_{k=0}^{t}\bm{A}^{k}
        & = \sum_{k=0}^{t}\left(\bm{V}\bm{\Lambda}^{k}\bm{V}^{-1}\right)  \nonumber \\
        & = \bm{V}\left(\sum_{k=0}^{t}\bm{\Lambda}^{k}\right)\bm{V}^{-1}  \nonumber \\
        & =
          \begin{bmatrix}
            1 & 1 \\
            -1/2 & 1/2 \\
          \end{bmatrix}
          \begin{bmatrix}
            \sum_{k=0}^{t}(3/4)^{k} & 0 \\
            0 & \sum_{k=0}^{t}(1/2)^{k} \\
          \end{bmatrix}
          \begin{bmatrix}
            1/2 & -1 \\
            1/2 & 1 \\
          \end{bmatrix}
          \nonumber \\
        & \to
          \begin{bmatrix}
            1 & 1 \\
            -1/2 & 1/2 \\
          \end{bmatrix}
          \begin{bmatrix}
            4 & 0 \\
            0 & 2 \\
          \end{bmatrix}
          \begin{bmatrix}
            1/2 & -1 \\
            1/2 & 1 \\
          \end{bmatrix}
          \quad (t \to \infty)
          \nonumber \\
        & = 
          \begin{bmatrix}
            3 & -2 \\
            -1/2 & 3 \\
          \end{bmatrix}
          \nonumber
    \end{align}
    
  \item What about the following matrix?
    \begin{equation}\nonumber%\label{eq:}%
      \bm{A} :=
      \begin{bmatrix}
        3/4 & -1/2 \\
        -1/8 & 3/4 \\
      \end{bmatrix}
    \end{equation}
  \end{itemize}
  \clearpage

\item \textbf{Convergence of general power series}
  \begin{itemize}
  \item Consider a more general power series
    \begin{equation}\nonumber%\label{eq:}%
      \sum_{k=0}^{t}\alpha_{k}\bm{A}^{k}
      := \alpha_{0}\bm{A}^{0} + \alpha_{1}\bm{A}^{1} + \alpha_{2}\bm{A}^{2} + \ldots + \alpha_{t}\bm{A}^{t},
      \quad
      t = 0, 1, 2, 3, \ldots,
    \end{equation}
    where $\alpha_{k}$ is not necessarily $1$
  \item We claim that
    \begin{equation}\nonumber%\label{eq:}%
      \sum_{k=0}^{t}|\alpha_{k}(\rho(\bm{A}))^{k}| \text{ converges }
      \implies
      \sum_{k=0}^{t}\alpha_{k}\bm{A}^{k} \text{ converges}
    \end{equation}
  \item To prove this, let $R$ be the radius of convergence of the series $\sum_{k} \alpha_{k}\rho^{k}$ so that
    the function
    \begin{equation}\nonumber%\label{eq:}%
      f(\rho) :=
      \lim_{t\to\infty}\sum_{k=0}^{t} \alpha_{k}\rho^{k}
      \quad \forall \rho \in (-R, R)
    \end{equation}
    is well-defined, differentiable on $(-R, R)$, and
    \begin{equation}\nonumber%\label{eq:}%
      \frac{d^{l} f(\rho)}{d\rho^{l}} = 
      \lim_{t\to\infty}\sum_{k=0}^{t} \frac{k!}{(k-l)!}\alpha_{k}\rho^{k-l}
      \quad \forall \rho \in (-R, R),
      \quad \forall l=1,2,\ldots
    \end{equation}
    meaning that the limit on the right-hand side exists for any $\rho \in (-R, R)$
  \item Hence,
    \begin{equation}\nonumber%\label{eq:}%
      \sum_{k=0}^{t}|\alpha_{k}(\rho(\bm{A}))^{k}| \text{ converges }
      \implies
      \rho(\bm{A}) < R
      \implies
      \sum_{k=0}^{t} \frac{k!}{(k-l)!}|\alpha_{k}|(\rho(\bm{A}))^{k-l}
      \text{ converges }
    \end{equation}
  \item Then, the above claim follows form the observation that
    \begin{align}
      \sum_{k=0}^{t}\alpha_{k}\bm{A}^{k}
        & = \bm{V}\left(\sum_{k=0}^{t}\alpha_{k}\bm{J}^{k}\right)\bm{V}^{-1} \nonumber \\
        & = 
      \bm{V}\left(
      \begin{bmatrix}
        \sum_{k=0}^{t}\alpha_{k}(\bm{J}_{n_{1}}(\lambda_{1}))^{k} & \bm{O} & \cdots & \bm{O} \\
        \bm{O} & \sum_{k=0}^{t}\alpha_{k}(\bm{J}_{n_{2}}(\lambda_{2}))^{k} & \cdots & \bm{O} \\
        \vdots & \vdots & \ddots & \vdots \\
        \bm{O} & \bm{O} & \cdots & \sum_{k=0}^{t}\alpha_{k}(\bm{J}_{n_{d}}(\lambda_{d}))^{k} \\
      \end{bmatrix}
      \right)\bm{V}^{-1},
    \nonumber%\label{eq:}%
    \end{align}
    where a typical element of $\sum_{k=0}^{t}\alpha_{k}(\bm{J}_{n_{i}}(\lambda_{i}))^{k}$ satisfies
    \begin{equation}\nonumber%\label{eq:}%
      \left|
      \sum_{k=0}^{t}\frac{k!}{l!(k-l)!}\alpha_{k}\lambda_{i}^{k-l}
      \right|
      \leq
      \frac{1}{l!}\sum_{k=0}^{t}\frac{k!}{(k-l)!}|\alpha_{k}||\lambda_{i}|^{k-l}
      \leq
      \frac{1}{l!}\sum_{k=0}^{t}\frac{k!}{(k-l)!}|\alpha_{k}|(\rho(\bm{A}))^{k-l},
    \end{equation}
    which means that each element of $\sum_{k=0}^{t}\alpha_{k}(\bm{J}_{n_{i}}(\lambda_{i}))^{k}$
    converges due to the comparison test
    
  \end{itemize}

\item \textbf{Example}
  \begin{itemize}
  \item Consider a matrix sequence of the form
    \begin{equation}\nonumber%\label{eq:}%
      \sum_{k=0}^{\infty} \frac{1}{k!}\bm{A}^{k}
      := \bm{I} + \bm{A} + \frac{1}{2!}\bm{A}^{2} + \frac{1}{3!}\bm{A}^{3} + \ldots
    \end{equation}
  \item Does this series converge? For any matrix $\bm{A}$?
  \end{itemize}


\end{itemize}

\end{document}
