\documentclass[12pt,a4paper]{article} 
\usepackage[T1]{fontenc}
\usepackage{textcomp}
\usepackage{makeidx}

\usepackage[sc, noBBpl]{mathpazo}
\usepackage{dsfont} % for indicator function
\usepackage[deluxe]{otf}
\usepackage[utf8]{inputenc}

\usepackage{amsmath}
\usepackage{amsfonts}
\usepackage{amssymb}
\usepackage{amsthm}
\usepackage{mathtools}
\usepackage{mathrsfs}

\usepackage{listings} % for programming codes

\usepackage{bm}
\usepackage[dvipdfmx]{color}
\pagestyle{plain} 
\usepackage{float}
\usepackage[dvipdfmx]{graphicx}

\usepackage[left=15mm, right=18mm, top=20mm, bottom=20mm]{geometry}

\setlength{\footskip}{40pt} 
\setlength{\abovecaptionskip}{0pt}
\setlength{\belowcaptionskip}{0pt}
\renewcommand{\baselinestretch}{1.05}

\newcommand{\argmax}{\mathop{\rm arg max}\limits}
\newcommand{\argmin}{\mathop{\rm arg min}\limits}
\newcommand{\st}{\mathop{\rm s.t.}\limits}
\newcommand{\fsize}[1]{\fontsize{#1}{#1}\selectfont}

% custom color
\usepackage[dvipdfmx]{xcolor}

% colors
\definecolor{darkred}{rgb}{.40,.00,.00}
\definecolor{navy}{rgb}{.00,.00,.40}
\definecolor{darknavy}{rgb}{.00,.00,.30}
\definecolor{lightnavy}{rgb}{.10,.10,.40}

\newcommand{\navy}{\color{navy}}
\newcommand{\deepred}{\color{darkred}}
\newcommand{\emred}[1]{\textbf{\deepred #1}}
\newcommand{\emblue}[1]{\textbf{\navy #1}}

% title and tableofcontents
\usepackage[titles]{tocloft}
\setlength{\cftbeforesecskip}{2pt}
\setlength{\cftbeforesubsecskip}{0pt}
\renewcommand{\cftsecfont}{\normalsize\mcfamily}
\renewcommand{\cftsubsecfont}{\normalsize\mcfamily}
\renewcommand{\cftsubsecfont}{\normalsize\mcfamily}
\renewcommand{\contentsname}{\large\mcfamily\bfseries \S\hspace{0.5\baselineskip}Contents\\[-\baselineskip]}
\cftsetindents{section}{10.0pt}{15.0pt}
\cftsetindents{subsection}{20.0pt}{30.0pt}
\cftpagenumbersoff{section}
\cftpagenumbersoff{subsection}
\cftpagenumbersoff{subsubsection}

\makeatletter
\def\vhrulefill#1{\leavevmode\leaders\hrule\@height#1\hfill \kern\z@}
\makeatother

\makeatletter
\def\maketitle{%
	\begin{center}\leavevmode
	\normalfont
	{\normalsize\raggedleft \@date\par}%
	\vskip -0.7\baselineskip
	\vhrulefill{0.2pt}\par
	\vskip 0.3\baselineskip
	\renewcommand{\baselinestretch}{0.80}
	{\color{darknavy}\bfseries\Large\raggedright \@title\par}%
	\renewcommand{\baselinestretch}{1.00}
	\vskip 0.2\baselineskip
	\ifx\@subtitle\@empty\else%
	{\normalsize\raggedright \@subtitle\par}%
	\fi
	\ifx\@author\@empty\else%
	\vskip 0.5\baselineskip
	{\normalsize\raggedleft \@author\par}%
	\fi
	\end{center}%
	\vskip 1.0\baselineskip
}
\def\subtitle#1{\gdef\@subtitle{#1}}
\let\@subtitle\@empty
\makeatother

% section
\newcommand{\Secindent}{0.0\baselineskip}
\newcommand{\Subsecindent}{0.0\baselineskip}
\newcommand{\lmargini}{1.3\baselineskip}
\newcommand{\lmarginii}{0.5\baselineskip}
\newcommand{\lmarginiii}{0.2\baselineskip}
\def\theenumi{\arabic{enumi}}
\def\theenumii{\alph{enumii}}
\renewcommand{\labelenumi}{\hfill\hspace{60pt}\theenumi.}
\renewcommand{\labelenumii}{\hspace{30pt}\theenumii.}
\renewcommand{\labelitemi}{\small\textbullet\hspace{2pt}}
\renewcommand{\labelitemii}{\color{gray}\small$\circ$\hspace{2pt}}
\renewcommand{\labelitemiii}{\textendash\hspace{1pt}}
\makeatletter
\renewcommand{\section}{\@startsection%
	{section}{1}{\Secindent}% name, depth, indent \z@=0pt
	{-1.0\baselineskip \@plus1.5\baselineskip \@minus0.5\baselineskip}% beforeskip (no indent if negative)
	{0.5\baselineskip \@plus0.8\baselineskip \@minus0.3\baselineskip}% afterskip (no line break if negative)
	{\reset@font\large\bfseries\scshape\color{darknavy}}% font
}
\renewcommand{\subsection}{\@startsection%
	{subsection}{2}{\Subsecindent}%
	{-0.5\baselineskip \@plus0.7\baselineskip \@minus0.4\baselineskip}%
	{0.3\baselineskip \@plus0.5\baselineskip \@minus0.2\baselineskip}%
	{\reset@font\normalsize\bfseries\scshape\color{darknavy}}% 
}
\renewcommand{\thesection}{\@arabic\c@section} % arabic, roman, Roman, alph, Alph, fnsymbol
\renewcommand{\thesubsection}{\thesection\hspace{0pt}.\hspace{0pt}\@arabic\c@subsection}
\makeatother

\makeatletter
\def\@listi{%
	\topsep = 0.2\baselineskip
	\partopsep = 0.0\baselineskip
	\parsep = 0.0\baselineskip
	\itemsep = 0.5\baselineskip
	\leftmargin = 1.5zw
	\labelsep = 0.5zw
	\itemindent = 0pt
	\labelwidth\leftmargin \advance\labelwidth-\labelsep
	\listparindent = 0zw
	\rightmargin = 0pt
	\reset@font\normalsize\normalfont
}
\let\@listI\@listi
\@listi
\def\@listii{%
	\topsep = 0.25\baselineskip
	\partopsep = 0.0\baselineskip
	\parsep = 0.0\baselineskip
	\itemsep = 0.3\baselineskip
	\leftmargin = 0.5zw
	\labelsep = .5zw
	\itemindent = 0pt
	\labelwidth\leftmargin \advance\labelwidth-\labelsep
	\listparindent = 0zw
	\rightmargin = 0pt
	\reset@font\normalsize\normalfont%
}
\def\@listiii{%
	\topsep = 0.30\baselineskip
	\partopsep = 0.0\baselineskip
	\parsep = 0.1\baselineskip
	\itemsep = 0.05\baselineskip
	\leftmargin = 1.1zw
	\labelsep = .5zw
	\itemindent = 0pt
	\labelwidth\leftmargin \advance\labelwidth-\labelsep
	\listparindent = 0zw
	\rightmargin = 0pt
	\reset@font\normalsize\normalfont%
}
\def\@listiv{%
	\topsep = 0.0\baselineskip
	\partopsep = 0.0\baselineskip
	\parsep = 0.0\baselineskip
	\itemsep = 0.0\baselineskip
	\leftmargin = 0.9zw
	\labelsep = .5zw
	\itemindent = 0pt
	\labelwidth\leftmargin \advance\labelwidth-\labelsep
	\listparindent = 0zw
	\rightmargin = 0pt
	\reset@font\normalsize\normalfont%
}
\makeatother

\newcommand{\R}{\ensuremath{\mathbb{R}}}
\newcommand{\E}{\ensuremath{\mathbb{E}}}
\newcommand{\V}{\ensuremath{\mathbb{V}}}
\newcommand{\Cov}{\ensuremath{\text{Cov}}}
\newcommand{\N}{\ensuremath{\mathbb{N}}}
\newcommand{\Z}{\ensuremath{\mathbb{Z}}}
\let\epsilon\varepsilon

\let\origitem\item
\renewcommand{\item}{\normalfont\origitem}

\usepackage[%
backref=page,
dvipdfmx,
bookmarks=true,bookmarksnumbered=true,bookmarkstype=toc,
colorlinks=true,
linkcolor=navy,
citecolor=navy,
filecolor=blue,
pagecolor=blue,
urlcolor=blue
]{hyperref}
\AtBeginDvi{\special{pdf:tounicode EUC-UCS2}}

\renewcommand*{\backref}[1]{}
\renewcommand*{\backrefalt}[4]{%
  \ifcase #1 (Not cited.)%
  \or        [#2]%
  \else      [#2]%
  \fi}
 

\title{Integration and change of variables}
\subtitle{Introduction to dynamical systems~\#9}
\author{Hiroaki Sakamoto}
\date{\today}

\begin{document}
\maketitle
\tableofcontents

\section{Integrals: recap}

\subsection{Single dimensional case}

\begin{itemize}

\item \textbf{Integral over intervals}
  \begin{itemize}
  \item Consider a function $f:X \to \R$ defined over $X\subset \R$ 
  \item For an interval $R:=[a, b]\subset X$ and $n\in \N$,
    the number
    \begin{equation}\nonumber%\label{eq:}%
      \sum_{k=0}^{n-1}f(x_{k})(x_{k+1}-x_{k}),
      \quad x_{k}:= a + \frac{k}{n}(b-a)
    \end{equation}
    is called a (left) Riemann sum
  \item We define the integral of $f$ over $R$ as the limit of this number for $n\to \infty$
    \begin{equation}\nonumber%\label{eq:q}%
      \int_{R}f(x)dx := \lim_{n\to \infty}\sum_{k=0}^{n-1}f(x_{k})(x_{k+1}-x_{k})
    \end{equation}
  \item We also write this integral as $\int_{a}^{b}f(x)dx$
  \end{itemize}

\item \textbf{Examples}
  \begin{itemize}
  \item The integral of $f(x):=x$ over $R:=[0,x]$ is
    \begin{equation}\nonumber
      \int_{0}^{x}f(t)dt
      =
      \int_{0}^{x}tdt
      =
      \lim_{n\to\infty}\sum_{k=0}^{n-1}\left(k\frac{x}{n}\right)\frac{x}{n}
      =
      \lim_{n\to\infty}\frac{x^{2}}{n^{2}}\underbrace{\sum_{k=0}^{n-1}k}_{\frac{1}{2}n(n-1)}
      =
      \frac{1}{2}x^{2}
    \end{equation}
  \item The integral of $f(x) := x^{2}$ over $R:=[0,x]$ is
    \begin{equation}\nonumber
      \int_{0}^{x}f(t)dt
      =
      \int_{0}^{x}t^{2}dt
      =
      \lim_{n\to\infty}\sum_{k=0}^{n-1}\left(k\frac{x}{n}\right)^{2}\frac{x}{n}
      =
      \lim_{n\to\infty}\frac{x^{3}}{n^{3}}\hspace{-15pt}\underbrace{\sum_{k=0}^{n-1}k^{2}}_{\frac{1}{3}n(n-1)\left(n-\frac{1}{2}\right)}\hspace{-15pt}
      =
      \frac{1}{3}x^{3}
    \end{equation}
  \end{itemize}

\item \textbf{Antiderivatives}
  \begin{itemize}
  \item We say that a function $F:X\to \R$ is an \emph{antiderivative} or \emph{primitive function} of another function $f:X\to\R$
    if $F'(x) = f(x)$ for all $x\in X$
  \item If $F(x)$ is an antiderivative of $f(x)$,
    then the integral of $f(x)$ over $R=[a,b]$ is
    \begin{equation}\nonumber%\label{eq:}%
      \int_{R}f(x)dx
      = \lim_{n\to \infty}\sum_{k=0}^{n-1}\underbrace{f(x_{k})}_{F'(x_{k})}(x_{k+1}-x_{k})
      = \lim_{n\to \infty}\sum_{k=0}^{n-1}(F(x_{k+1})-F(x_{k}))
      = F(b)-F(a)
    \end{equation}
    
  \end{itemize}

\item \textbf{Example}
  \begin{itemize}
  \item Consider a function $f(x):=1/x$
  \item Since $F(x):=\ln(x)$ is an antiderivative of $f(x)$,
    the integral of $f(x)$ over $[1,x]$ is
    \begin{equation}\nonumber%\label{eq:}%
      \int_{1}^{x}\frac{1}{t}dt
      = \ln(x) - \ln(1)
      = \ln(x)
    \end{equation}
  \item In fact, this can be seen as a definition of $\ln(x)$ when read from right to left
  \end{itemize}

\item \textbf{Measure of sets}
  \begin{itemize}
  \item For any subset $D\subset \R$, we define
    \begin{equation}\nonumber%\label{eq:}%
      \mathds{1}_{D}(x) :=
      \begin{cases}
        1 & \text{if $x\in D$} \\
        0 & \text{otherwise} \\
      \end{cases}
      \quad \forall x \in \R,
    \end{equation}
    which is called the \emph{indicator function} of $D$
  \item Let $D\subset \R$ be an arbitrary set and $R=[a,b]\subset \R$ be an interval such that $D\subset R$
  \item We define the (Lebesgue) measure of $D$ as
    \begin{equation}\nonumber%\label{eq:}%
      |D|:=\int_{R}\mathds{1}_{D}(x)dx
      := \lim_{n\to\infty}\sum_{k=0}^{n-1}\mathds{1}_{D}(x_{k})(x_{k+1}-x_{k})
    \end{equation}
  \item The measure of an interval $R=[a,b]$ coincides with the length of the interval
    \begin{equation}\nonumber%\label{eq:}%
      |R|
      = \lim_{n\to\infty}\sum_{k=0}^{n-1}\mathds{1}_{R}(x_{k})(x_{k+1}-x_{k})
      = \lim_{n\to\infty}\sum_{k=0}^{n-1}(x_{k+1}-x_{k})
      = b-a
    \end{equation}
  \end{itemize}

\item \textbf{Integral over arbitrary sets}
  \begin{itemize}
  \item Consider a function $f:X \to \R$ defined over $X\subset \R$
  \item For any subset $D\subset X$, 
    we define the integral of $f$ over $D$ as
    \begin{equation}\nonumber%\label{eq:q}%
      \int_{D}f(x)dx := \int_{R}\mathds{1}_{D}(x)f(x)dx := \lim_{n\to \infty}\sum_{k=0}^{n-1}\mathds{1}_{D}(x_{k})f(x_{k})(x_{k+1}-x_{k}),
    \end{equation}
    where $R$ is an interval that contains $D$, provided that the limit exists
  \item Note that $\mathds{1}_{D}(x)f(x)$ can be seen as a single function defined over $X$
    \begin{equation}\nonumber%\label{eq:}%
      \mathds{1}_{D}(x)f(x) =
      \begin{cases}
        f(x) & \text{if $x\in D$} \\
        0 & \text{if $x\notin D$}
      \end{cases}
      \quad \forall x\in X
     \end{equation}
  \end{itemize}

\end{itemize}

\clearpage
\subsection{Two dimensional case}

\begin{itemize}

\item \textbf{Integral over rectangles}
  \begin{itemize}
  \item Consider a function $f:X \to \R$ defined over $X\subset \R^{2}$
  \item For a rectangle $R:=[a_{1}, b_{1}]\times [a_{2}, b_{2}]\subset X$ and $n_{1},n_{2}\in \N$,
    the Riemann sum is
    \begin{equation}\nonumber%\label{eq:}%
      \sum_{k_{2}=0}^{n_{2}-1}\sum_{k_{1}=0}^{n_{1}-1}f(x_{1,k_{1}},x_{2,k_{2}})(x_{1,k_{1}+1}-x_{1,k_{1}})(x_{2,k_{2}+1}-x_{2,k_{2}}),
      \quad x_{i,k_{i}}:= a_{i} + \frac{k_{i}}{n_{i}}(b_{i}-a_{i})
    \end{equation}
  \item We define the integral of $f$ over $R$ as
    \begin{align}
      \int_{R}f(\bm{x})d\bm{x}
      & := \lim_{n_{2}\to \infty}\lim_{n_{1}\to \infty}\sum_{k_{2}=0}^{n_{2}-1}\sum_{k_{1}=0}^{n_{1}-1}f(x_{1,k_{1}},x_{2,k_{2}})(x_{1,k_{1}+1}-x_{1,k_{1}})(x_{2,k_{2}+1}-x_{2,k_{2}}) \nonumber \\
      & = \lim_{n_{2}\to \infty}\sum_{k_{2}=0}^{n_{2}-1}\underbrace{\left(\lim_{n_{1}\to \infty}\sum_{k_{1}=0}^{n_{1}-1}f(x_{1,k_{1}},x_{2,k_{2}})(x_{1,k_{1}+1}-x_{1,k_{1}})\right)}_{\int_{a_{1}}^{b_{1}}f(x_{1},x_{2,k_{2}})dx_{1}}(x_{2,k_{2}+1}-x_{2,k_{2}}) \nonumber \\
      & = \int_{a_{2}}^{b_{2}}\left(\int_{a_{1}}^{b_{1}}f(x_{1},x_{2})dx_{1}\right)dx_{2} \nonumber%\label{eq:}
    \end{align}
  \item We may write this integral more explicitly as
    \begin{equation}\nonumber%\label{eq:}%
      \iint_{R}f(x_{1},x_{2})dx_{1}dx_{2},
      \quad\text{or}\quad
      \int_{a_{2}}^{b_{2}}\int_{a_{1}}^{b_{1}}f(x_{1},x_{2})dx_{1}dx_{2}
    \end{equation}
  \end{itemize}

\item \textbf{Example}
  \begin{itemize}
  \item The integral of $f(x_{1},x_{2}):=x_{1}x_{2}$ over $R:=[0,x_{1}]\times [0, x_{2}]$ is
    \begin{equation}\nonumber%\label{eq:}%
      \int_{0}^{x_{2}}\int_{0}^{x_{1}}t_{1}t_{2}dt_{1}dt_{2}
      = 
      \int_{0}^{x_{2}}\left(\int_{0}^{x_{1}}t_{1}t_{2}dt_{1}\right)dt_{2}
      = 
      \int_{0}^{x_{2}}\left(\frac{1}{2}x_{1}^{2}t_{2}\right)dt_{2}
      = 
      \frac{1}{2}\frac{1}{2}x_{1}^{2}x_{2}^{2}
      = 
      \frac{1}{4}x_{1}^{2}x_{2}^{2}
    \end{equation}
  \end{itemize}

\item \textbf{Measure of sets}
  \begin{itemize}
  \item For any subset $D\subset \R^{2}$, we define
    \begin{equation}\nonumber%\label{eq:}%
      \mathds{1}_{D}(\bm{x}) :=
      \begin{cases}
        1 & \text{if $\bm{x}\in D$} \\
        0 & \text{otherwise} \\
      \end{cases}
      \quad \forall \bm{x} \in \R^{2},
    \end{equation}
    which is called the \emph{indicator function} of $D$
  \item Let $R=[a_{1},b_{2}]\times [a_{2},b_{2}]\subset \R^{2}$ be a rectangle that contains $D\subset R^{2}$
  \item We define the (Lebesgue) measure of $D$ as
    \begin{equation}\nonumber%\label{eq:}%
      |D|:=\int_{R}\mathds{1}_{D}(\bm{x})d\bm{x} :=
      \lim_{n_{2}\to\infty}\lim_{n_{1}\to\infty}
      \sum_{k_{2}=0}^{n_{2}-1}\sum_{k_{1}=0}^{n_{1}-1}\mathds{1}_{D}(x_{1,k_{1}},x_{2,k_{2}})(x_{1,k_{1}+1}-x_{1,k_{1}})(x_{2,k_{2}+1}-x_{2,k_{2}})
    \end{equation}
    provided that the limit exists
  \item The measure of a rectangle $R=[a_{1},b_{1}]\times[a_{2},b_{2}]$ coincides with the area of the rectangle
    \begin{equation}\nonumber%\label{eq:}%
      |R| =
      \lim_{n_{2}\to\infty}\lim_{n_{1}\to\infty}
      \sum_{k_{1}=0}^{n_{1}-1}(x_{1,k_{1}+1}-x_{1,k_{1}})\sum_{k_{2}=0}^{n_{2}-1}(x_{2,k_{2}+1}-x_{2,k_{2}})
      = (b_{1}-a_{1})(b_{2}-a_{2})
    \end{equation}
  \end{itemize}

\item \textbf{Integral over arbitrary sets}
  \begin{itemize}
  \item Consider a function $f:X \to \R$ defined over $X\subset \R^{2}$
  \item For any subset $D\subset X$, 
    we define the integral of $f$ over $D$ as
    \begin{equation}\nonumber%\label{eq:q}%
      \int_{D}f(\bm{x})d\bm{x} := \int_{R}\mathds{1}_{D}(\bm{x})f(\bm{x})d\bm{x}
    \end{equation}
    where $R$ is a rectangle that contains $D$, provided that the limit exists
  \end{itemize}

\end{itemize}

\subsection{Higher dimensional case}

\begin{itemize}

\item \textbf{Integral over hyperrectangles}
  \begin{itemize}
  \item Consider a function $f:X \to \R$ defined over $X\subset \R^{m}$
  \item For an $m$-dimensional hyperrectangle $R:=\prod_{i=1}^{m}[a_{i}, b_{i}]\subset X$ and $n_{1},n_{2},\ldots, n_{i}\in \N$,
    the number
    \begin{equation}\nonumber%\label{eq:}%
      \sum_{k_{m}=0}^{n_{m}-1}\cdots\sum_{k_{1}=0}^{n_{1}-1}f(x_{1,k_{1}},\ldots, x_{m,k_{m}})\prod_{i=1}^{m}(x_{i,k_{i}+1}-x_{i,k_{i}}),
      \quad x_{i,k_{i}}:= a_{i} + \frac{k_{i}}{n_{i}}(b_{i}-a_{i})
    \end{equation}
    is called a (left) Riemann sum
  \item We define the integral of $f$ over $R$ as
    \begin{equation}\nonumber%\label{eq:q}%
      \int_{R}f(\bm{x})d\bm{x} := \lim_{n_{m}\to \infty}\cdots\lim_{n_{1}\to \infty}
      \left(\sum_{k_{m}=0}^{n_{m}-1}\cdots\sum_{k_{1}=0}^{n_{1}-1}f(x_{1,k_{1}},\ldots, x_{m,k_{m}})\prod_{i=1}^{m}(x_{i,k_{i}+1}-x_{i,k_{i}})\right)
    \end{equation}
    provided that the limit exists
  \item We may write this integral more explicitly as
    \begin{equation}\nonumber%\label{eq:}%
      \idotsint_{R}f(x_{1},\ldots, x_{m})dx_{1}\ldots dx_{m}
      \quad\text{or}\quad
      \int_{a_{m}}^{b_{m}}\cdots\int_{a_{1}}^{b_{2}}(x_{1},\ldots, x_{m})dx_{1}\ldots dx_{m}
    \end{equation}
  \end{itemize}

\item \textbf{Measure of sets}
  \begin{itemize}
  \item For any subset $D\subset \R^{m}$, we define
    \begin{equation}\nonumber%\label{eq:}%
      \mathds{1}_{D}(\bm{x}) :=
      \begin{cases}
        1 & \text{if $\bm{x}\in D$} \\
        0 & \text{otherwise} \\
      \end{cases}
      \quad \forall \bm{x} \in \R^{m},
    \end{equation}
    which is called the \emph{indicator function} of $D$
  \item Let $R=\prod_{i=1}^{m}[a_{i},b_{i}]\subset \R^{m}$ be a hyperrectangle that contains $D\subset R^{m}$
  \item We define the (Lebesgue) measure of $D$ as
    \begin{equation}\nonumber%\label{eq:}%
      |D|:=\int_{R}\mathds{1}_{D}(\bm{x})d\bm{x}
    \end{equation}
    provided that the limit on the right-hand side exists
  \item The measure of a hyperrectangle $R=\prod_{i=1}^{m}[a_{i},b_{i}]$ is
    \begin{equation}\nonumber%\label{eq:}%
      |R|
      = (b_{1}-a_{1})(b_{2}-a_{2})\cdots(b_{m}-a_{m})
    \end{equation}
  \end{itemize}

\item \textbf{Integral over arbitrary sets}
  \begin{itemize}
  \item Consider a function $f:X \to \R$ defined over $X\subset \R^{m}$
  \item For any subset $D\subset X$, 
    we define the integral of $f$ over $D$ as
    \begin{equation}\nonumber%\label{eq:q}%
      \int_{D}f(\bm{x})d\bm{x} := \int_{R}\mathds{1}_{D}(\bm{x})f(\bm{x})d\bm{x}
    \end{equation}
    where $R$ is a hyperrectangle that contains $D$, provided that the limit exists
  \end{itemize}

\end{itemize}

\section{Change of variables}

\subsection{Determinant revisited}

\begin{itemize}

\item \textbf{Determinant of orthogonal matrices}
  \begin{itemize}
  \item Let $\bm{a}_{1}, \bm{a}_{2},\ldots \bm{a}_{m} \in \R^{m}$ be $m$-dimensional orthogonal vectors,
    i.e.,
    \begin{equation}\nonumber%\label{eq:}%
      \bm{a}_{i}^{\top}\bm{a}_{j} = 0 \quad \forall i\neq j
    \end{equation}
    and define $\bm{A}:= \begin{bmatrix} \bm{a}_{1} & \bm{a}_{2} & \cdots & \bm{a}_{m}\end{bmatrix}$
  \item Then
    \begin{equation}\nonumber%\label{eq:}%
      \prod_{i=1}^{m}\|\bm{a}_{i}\| = \left|\det(\bm{A}) \right|
    \end{equation}
    because
    \begin{equation}\nonumber%\label{eq:}%
      (|\bm{A}|)^{2}
      = 
      |\bm{A}||\bm{A}|
      =
      |\bm{A}^{\top}||\bm{A}|
      =
      |\bm{A}^{\top}\bm{A}|
      = \begin{vmatrix}
          \bm{a}_{1}^{\top}\bm{a}_{1} & \bm{a}_{1}^{\top}\bm{a}_{2} & \cdots & \bm{a}_{1}^{\top}\bm{a}_{m} \\
          \bm{a}_{2}^{\top}\bm{a}_{1} & \bm{a}_{2}^{\top}\bm{a}_{2} & \cdots & \bm{a}_{2}^{\top}\bm{a}_{m} \\
          \vdots & \vdots & \ddots & \vdots \\
          \bm{a}_{m}^{\top}\bm{a}_{1} & \bm{a}_{m}^{\top}\bm{a}_{2} & \cdots & \bm{a}_{m}^{\top}\bm{a}_{m} \\
      \end{vmatrix}
      = \prod_{i=1}^{n}\|\bm{a}_{i}\|^{2}
      = \left(\prod_{i=1}^{n}\|\bm{a}_{i}\|\right)^{2}
    \end{equation}
  \item In other words,
    if $\bm{a}_{1},\ldots,\bm{a}_{m}$ are orthogonal,
    \begin{equation}\nonumber%\label{eq:}%
      \text{measure of $m$-dimensional hyperrectangle made by $\bm{a}_{1},\ldots,\bm{a}_{m}$}
      = \text{absolute value of $\det(\bm{A})$}
    \end{equation}
  \end{itemize}

\item \textbf{Skew translation and determinant}
  \begin{itemize}
  \item Let $\bm{a}_{1}, \bm{a}_{2},\ldots \bm{a}_{m} \in \R^{m}$ be orthogonal vectors
  \item Now take the $j$-th column vector $\bm{a}_{j}$ and add $c_{j}\bm{a}_{j}$ with $c_{j}\neq 0$ to the $i$-th column to create another matrix
    \begin{equation}\nonumber%\label{eq:}%
      \bm{A}' := 
    \begin{bmatrix}
      \bm{a}_{1} & \cdots & \bm{a}_{i} + c_{j}\bm{a}_{j} & \cdots & \bm{a}_{j} & \cdots & \bm{a}_{m},
    \end{bmatrix}
    \quad i \neq j
    \end{equation}
    which geometrically represents an $m$-dimensional parallelepiped (i.e., an object you obtain by skewing the hyperrectangle made by $\bm{A}$ in the direction of $\bm{a}_{j}$)
  \item Such a skew translation does not change the measure of the object, so we know that
    \begin{equation}\nonumber%\label{eq:}%
      \text{measure of object made by columns of $\bm{A}'$}
      = \text{measure of object made by columns of $\bm{A}$}
    \end{equation}
  \item Since
    \begin{align}
      |\bm{A}'|
      & =
        \begin{vmatrix}
          \bm{a}_{1} & \cdots & \bm{a}_{i} + c_{j}\bm{a}_{j} & \cdots & \bm{a}_{j} & \cdots & \bm{a}_{m}
        \end{vmatrix}
        \nonumber \\
      & =
        \begin{vmatrix}
          \bm{a}_{1} & \cdots & \bm{a}_{i} & \cdots & \bm{a}_{j} & \cdots & \bm{a}_{m}
        \end{vmatrix}
        + c_{j}
        \underbrace{
        \begin{vmatrix}
          \bm{a}_{1} & \cdots & \bm{a}_{j} & \cdots & \bm{a}_{j} & \cdots & \bm{a}_{m}
        \end{vmatrix}}_{=0}
       = |\bm{A}|
    \nonumber%\label{eq:}%
    \end{align}
    we conclude that
    \begin{align}
      &\text{measure of parallelepiped made by columns of $\bm{A}'$} \nonumber \\
        &\qquad = \text{measure of hyperrectangle made by columns of $\bm{A}$} \nonumber \\
        &\qquad = \text{absolute value of $\det(\bm{A})$}  \nonumber \\
        &\qquad = \text{absolute value of $\det(\bm{A}')$}
    \nonumber%\label{eq:}%
    \end{align}
  \item This observation should hold more generally
    because one can repeatedly apply skew translations without invalidating the argument
    
  \end{itemize}
  \clearpage

\item \textbf{Determinant of general $m\times m$ matrices}
  \begin{itemize}
  \item Let $\bm{a}_{1}, \ldots, \bm{a}_{m} \in \R^{m}$ be a set of any vectors
    and define $\bm{A} := \begin{bmatrix} \bm{a}_{1} & \bm{a}_{2} & \cdots & \bm{a}_{m} \end{bmatrix}$
  \item We want to show that:
    \begin{equation}\nonumber%\label{eq:}%
      \text{measure of parallelepiped made by $\bm{a}_{1},\ldots, \bm{a}_{m}\in \R^{m}$}
      = \text{absolute value of $\det(\bm{A})$}
    \end{equation}
  \item To show this, we apply a set of skew translations (i.e., the Gram-Schmidt orthogonalization) to transform the parallelepiped into a hyperrectangle (\figurename~\ref{fig:integration_2}):
    \begin{align}
      \bm{a}_{1}' & := \bm{a}_{1} \nonumber \\
      \bm{a}_{2}' & := \bm{a}_{2} - (\bm{a}_{2}^{\top}\bm{a}_{1}')\frac{\bm{a}_{1}'}{\|\bm{a}_{1}'\|^{2}} \nonumber \\
      \bm{a}_{3}' & := \bm{a}_{3} - (\bm{a}_{3}^{\top}\bm{a}_{1}')\frac{\bm{a}_{1}'}{\|\bm{a}_{1}'\|^{2}} - (\bm{a}_{3}^{\top}\bm{a}_{2}')\frac{\bm{a}_{2}'}{\|\bm{a}_{2}'\|^{2}} \nonumber \\
       \vdots &  \nonumber \\
      \bm{a}_{m}' & := \bm{a}_{m} - (\bm{a}_{m}^{\top}\bm{a}_{1}')\frac{\bm{a}_{1}'}{\|\bm{a}_{1}'\|^{2}}
                    - (\bm{a}_{m}^{\top}\bm{a}_{2}')\frac{\bm{a}_{2}'}{\|\bm{a}_{2}'\|^{2}}
                    \cdots - (\bm{a}_{m}^{\top}\bm{a}_{m-1}')\frac{\bm{a}_{m-1}'}{\|\bm{a}_{m-1}'\|^{2}}
                    \nonumber
    \end{align}
    and let
    $\bm{A}' := \begin{bmatrix}\bm{a}_{1}' & \bm{a}_{2}' & \cdots & \bm{a}_{m}'\end{bmatrix}$
      \begin{figure}[t]\centering%
        \includegraphics[width=480pt]%
        {figures/fig_integration_2.png}
        \caption{%
          Skew translation and determinant
        } 
        \label{fig:integration_2}
      \end{figure} 
  \item Observe:
    \begin{itemize}
    \item the column vectors of $\bm{A}'$ are orthogonal to each other so:
      \begin{equation}\nonumber%\label{eq:}%
        \text{measure of hyperrectangle made by $\bm{a}_{1}',\ldots,\bm{a}_{m}'$}
        = \text{absolute value of $\det(\bm{A}')$}
      \end{equation}
    \item since the skew translations should not change the measure of the object:
      \begin{align}
        & \text{measure of hyperrectangle made by $\bm{a}_{1}',\ldots,\bm{a}_{m}'$} \nonumber \\
        & \quad = \text{measure of parallelepiped made by $\bm{a}_{1},\ldots,\bm{a}_{m}$}
          \nonumber%\label{eq:}% 
      \end{align}
    \item On the other hand, we know from the properties of determinant that $\det(\bm{A}') = \det(\bm{A})$
    \end{itemize}
    \clearpage
  \item Therefore
    \begin{align}
      & \text{measure of parallelepiped made by $\bm{a}_{1}, \bm{a}_{2},\ldots \bm{a}_{m}$} \nonumber \\
      & \quad = \text{measure of hyperrectangle made by $\bm{a}_{1}',\ldots,\bm{a}_{m}'$} \nonumber \\
      & \quad = \text{absolute value of $\det(\bm{A}')$} \nonumber \\
      & \quad = \text{absolute value of $\det(\bm{A})$}
    \nonumber%\label{eq:}% 
    \end{align}
    \end{itemize}

\item \textbf{Example: $2\times 2$ matrix}
  \begin{itemize}
  \item Let $\bm{a}_{1}, \bm{a}_{2}\in \R^{2}$ be the column vectors of $\bm{A}\in \R^{2\times 2}$
  \item Then
    \begin{align}
      (|\bm{A}|)^{2}
        & = |\bm{A}||\bm{A}|
        = |\bm{A}^{\top}||\bm{A}|
        = |\bm{A}^{\top}\bm{A}|
        = \left|
          \begin{bmatrix}
            \bm{a}_{1}^{\top} \\
            \bm{a}_{2}^{\top} \\
          \end{bmatrix}
          \begin{bmatrix}
            \bm{a}_{1} & \bm{a}_{2}
          \end{bmatrix}
          \right|
        = \left|
          \begin{bmatrix}
            \bm{a}_{1}^{\top}\bm{a}_{1} & \bm{a}_{1}^{\top}\bm{a}_{2} \\
            \bm{a}_{2}^{\top}\bm{a}_{1} & \bm{a}_{2}^{\top}\bm{a}_{2} \\
          \end{bmatrix}
          \right|  \nonumber \\
        & = \bm{a}_{1}^{\top}\bm{a}_{1}\bm{a}_{2}^{\top}\bm{a}_{2}
          - \bm{a}_{1}^{\top}\bm{a}_{2}\bm{a}_{2}^{\top}\bm{a}_{1}
         = \|\bm{a}_{1}\|^{2}\|\bm{a}_{2}\|^{2} - (\bm{a}_{1}^{\top}\bm{a}_{2})^{2} \nonumber \\
        & = \|\bm{a}_{1}\|^{2}\|\bm{a}_{2}\|^{2} \left(1 - \left(\frac{(\bm{a}_{1}^{\top}\bm{a}_{2})}{\|\bm{a}_{1}\|\|\bm{a}_{2}\|}\right)^{2}\right)  \nonumber \\
        & = \|\bm{a}_{1}\|^{2}\|\bm{a}_{2}\|^{2} \left(1 - \cos^{2}(\theta)\right) \quad\text{where}\quad \text{$\theta$ is the angle between $\bm{a}_{1}$ and $\bm{a}_{2}$}  \nonumber \\
        & = \|\bm{a}_{1}\|^{2}\|\bm{a}_{2}\|^{2}\sin^{2}(\theta)
        = \left(\|\bm{a}_{1}\|\|\bm{a}_{2}\|\sin(\theta)\right)^{2}  \nonumber \\
        & = \left(\text{area of parallelogram made by $\bm{a}_{1}$ and $\bm{a}_{2}$}\right)^{2},
    \nonumber%\label{eq:}%
    \end{align}
    which confirms
    \begin{equation}\nonumber%\label{eq:}%
      \text{area of parallelogram made by $\bm{a}_{1}$ and $\bm{a}_{2}$}
      = \text{absolute value of $\det(\bm{A})$}
    \end{equation}
  \end{itemize}

\end{itemize}

\subsection{Change of variable formula}

\begin{itemize}
\item \textbf{What we want to do}
  \begin{itemize}
  \item Let us say that
    we want to compute the integral
    \begin{equation}\nonumber%\label{eq:}%
      \int_{Z}f_{Z}(\bm{z})d\bm{z}
    \end{equation}
    for some function $f_{Z}:\R^{m}\to \R$
    over some subset $Z\subset \R^{m}$
  \item Suppose that the variable $\bm{z}\in Z$ can be transformed into another variable $\bm{x}\in \R^{m}$
    through a bijective function $\phi:Z\to \phi(Z)$
    \begin{equation}\nonumber%\label{eq:}%
      \bm{x} = \phi(\bm{z}) \quad \forall \bm{z}\in Z\subset \R^{n}
    \end{equation}
    or
    \begin{equation}\nonumber%\label{eq:}%
      \bm{z} = \phi^{-1}(\bm{x})=:\psi(\bm{x}) \quad \forall \bm{x}\in X:=\phi(Z)
    \end{equation}
  \item We want to find a function $f_{X}:X\to\R$ such that
    \begin{equation}\label{eq:fzfx}%
      \int_{B}f_{Z}(\bm{z})d\bm{z}
      = \int_{\phi(B)}f_{X}(\bm{x})d\bm{x}
      \quad \forall B\subset Z
    \end{equation}
    or equivalently 
    \begin{equation}\label{eq:fxfz}%
      \int_{A}f_{X}(\bm{x})d\bm{x} = \int_{\psi(A)}f_{Z}(\bm{z})d\bm{z} \quad \forall A\subset X
    \end{equation}
  \end{itemize}

\item \textbf{Case with $m=1$}
  \begin{itemize}
  \item The function $f_{X}$ that satisfies \eqref{eq:fzfx} or \eqref{eq:fxfz} is
    \begin{equation}\label{eq:fx1}%
      f_{X}(x) = f_{Z}(\psi(x))\left| \frac{d\psi(x)}{dx}\right| \quad \forall x \in X
    \end{equation}
  \item To see this, fix $\bar{x}\in X$
  \item If $f_{X}$ satisfies \eqref{eq:fxfz}, we must have
    \begin{equation}\label{eq:fxfz1}%
      \int_{[\bar{x}, \bar{x}+\Delta x]} f_{X}(x)dx
      = 
      \int_{\psi([\bar{x}, \bar{x}+\Delta x])} f_{Z}(z)dz
    \quad \forall \Delta x\geq 0
    \end{equation}
  \item Notice that for each $\Delta x> 0$,
    \begin{itemize}
    \item there exists $\tilde{x}(\Delta x) \in [\bar{x},\bar{x}+\Delta x]$ such that
      \begin{equation}\label{eq:fxdx1}%
        \int_{[\bar{x}, \bar{x}+\Delta x]} f_{X}(x)dx
        =
        \int_{[\bar{x}, \bar{x}+\Delta x]} f_{X}(\tilde{x}(\Delta x))dx
        =
        f_{X}(\tilde{x}(\Delta x))\underbrace{|[\bar{x},\bar{x}+\Delta x]|}_{\Delta x},
      \end{equation}
      which is illustrated in \figurename~\ref{fig:integration_1}
      \begin{figure}[t]\centering%
        \includegraphics[width=400pt]%
        {figures/fig_integration_1.png}
        \caption{%
          Choice of $\tilde{x}(\Delta x)\in [\bar{x},\bar{x}+\Delta x]$ and $\tilde{z}(\Delta x)\in \psi([\bar{x},\bar{x}+\Delta x])$
          that satisfies \eqref{eq:fxdx1} and \eqref{eq:fzdzpsi1}
        } 
        \label{fig:integration_1}
      \end{figure} 
    \item there exists $\tilde{z}(\Delta x) \in \psi([\bar{x},\bar{x}+\Delta x])$ such that
      \begin{equation}\label{eq:fzdzpsi1}%
        \int_{\psi([\bar{x}, \bar{x}+\Delta x])} f_{Z}(z)dz
        =
        \int_{\psi([\bar{x}, \bar{x}+\Delta x])} f_{Z}(\tilde{z}(\Delta x))dz
        =
        f_{Z}(\tilde{z}(\Delta x))|\psi([\bar{x},\bar{x}+\Delta x])|
      \end{equation}
      where $|\psi([\bar{x},\bar{x}+\Delta x])|$ is the measure of the set $\psi([\bar{x},\bar{x}+\Delta x])$
    \end{itemize}
  \item It follows from \eqref{eq:fxfz1}, \eqref{eq:fxdx1}, and \eqref{eq:fzdzpsi1} that
    \begin{equation}\nonumber%\label{eq:}%
      f_{X}(\tilde{x}(\Delta x))
      = f_{Z}(\tilde{z}(\Delta x)) \frac{|\psi([\bar{x},\bar{x}+\Delta x])|}{\Delta x},
      \quad \forall \Delta x >0
    \end{equation}
  \item Obviously,
    \begin{equation}\nonumber%\label{eq:}%
      \lim_{\Delta x \to 0}\tilde{x}(\Delta x) = \bar{x},
      \quad
      \lim_{\Delta x \to 0}\tilde{z}(\Delta x) = \psi(\bar{x}),
    \end{equation}
    \begin{equation}\nonumber%\label{eq:}%
      |\psi([\bar{x},\bar{x}+\Delta x])| =  \left|\psi(\bar{x}+\Delta x) - \psi(\bar{x}) \right|
      \quad \text{for sufficiently small $\Delta x$},
    \end{equation}
    and therefore
    \begin{align}
      f_{X}(\bar{x})
        & = \lim_{\Delta x \to 0}f_{X}(\tilde{x}(\Delta x))  \nonumber \\
        & = \lim_{\Delta x \to 0} f_{Z}(\tilde{z}(\Delta x))\frac{|\psi([\bar{x},\bar{x}+\Delta x])|}{\Delta x}  \nonumber \\
        & = f_{Z}(\psi(\bar{x}))\lim_{\Delta x \to 0} \frac{|\psi([\bar{x},\bar{x}+\Delta x])|}{\Delta x}  \nonumber \\
        & = f_{Z}(\psi(\bar{x}))\lim_{\Delta x \to 0} \left|\frac{\psi(\bar{x}+\Delta x) - \psi(\bar{x})}{\Delta x}\right| \nonumber \\
        & = f_{Z}(\psi(\bar{x}))\left| \frac{d\psi(\bar{x})}{dx}\right|,
    \nonumber%\label{eq:}%
    \end{align}
    where the absolute value is necessary because $\psi(\bar{x}+\Delta x)$ can be smaller than $\psi(\bar{x})$
  \item Since the argument above does not depend on the choice of $\bar{x}$,
    we conclude that the function $f_{X}$ must be given by \eqref{eq:fx1}
  \end{itemize}

\item \textbf{Case with $m=2$ and higher $m$}
  \begin{itemize}
  \item The function $f_{X}$ that satisfies \eqref{eq:fzfx} or \eqref{eq:fxfz} is
    \begin{equation}\label{eq:fx2}%
      f_{X}(\bm{x}) =
      f_{Z}(\psi(\bm{x}))\left|\left| \frac{d\psi(\bm{x})}{d\bm{x}}\right|\right|
      \quad \forall \bm{x} \in X
    \end{equation}
  \item To see this, fix $\bar{\bm{x}} = (\bar{x}_{1},\bar{x}_{2})\in X$
  \item If $f_{X}$ satisfies \eqref{eq:fxfz}, we must have
    \begin{equation}\label{eq:fxfz2}%
      \int_{[\bar{\bm{x}}, \bar{\bm{x}}+\Delta \bm{x}]} f_{X}(\bm{x})d\bm{x}
      = 
      \int_{\psi([\bar{\bm{x}}, \bar{\bm{x}}+\Delta \bm{x}])} f_{Z}(\bm{z})d\bm{z}
      \quad \forall \Delta \bm{x} = (\Delta x_{1}, \Delta x_{2})\geq \bm{0}
    \end{equation}
    where $[\bar{\bm{x}}, \bar{\bm{x}}+\Delta \bm{x}] := [\bar{x}_{1},\bar{x}_{1}+\Delta x_{1}]\times[\bar{x}_{2},\bar{x}_{2}+\Delta x_{2}]$
  \item Notice that for each $\Delta \bm{x}>\bm{0}$,
    \begin{itemize}
    \item there exists $\tilde{\bm{x}}(\Delta \bm{x}) \in [\bar{\bm{x}},\bar{\bm{x}}+\Delta \bm{x}]$ such that
      \begin{equation}\label{eq:fxdx2}%
        \int_{[\bar{\bm{x}}, \bar{\bm{x}}+\Delta \bm{x}]} f_{X}(\bm{x})d\bm{x}
        =
        \int_{[\bar{\bm{x}}, \bar{\bm{x}}+\Delta \bm{x}]} f_{X}(\tilde{\bm{x}}(\Delta\bm{x}))d\bm{x}
        =
        f_{X}(\tilde{\bm{x}}(\Delta \bm{x}))\underbrace{|[\bar{\bm{x}},\bar{\bm{x}}+\Delta \bm{x}]|}_{\Delta x_{1}\Delta x_{2}}
      \end{equation}
    \item there exists $\tilde{\bm{z}}(\Delta \bm{x}) \in \psi([\bar{\bm{x}},\bar{\bm{x}}+\Delta \bm{x}])$ such that
      \begin{equation}\label{eq:fzdzpsi2}%
        \int_{\psi([\bar{\bm{x}}, \bar{\bm{x}}+\Delta \bm{x}])} f_{Z}(\bm{z})d\bm{z}
        =
        \int_{\psi([\bar{\bm{x}}, \bar{\bm{x}}+\Delta \bm{x}])} f_{Z}(\tilde{\bm{z}}(\Delta \bm{x}))d\bm{z}
        =
        f_{Z}(\tilde{\bm{z}}(\Delta \bm{x}))|\psi([\bar{\bm{x}},\bar{\bm{x}}+\Delta \bm{x}])|
      \end{equation}
      where $|\psi([\bar{\bm{x}},\bar{\bm{x}}+\Delta \bm{x}])|$
        is the measure of the set $\psi([\bar{\bm{x}},\bar{\bm{x}}+\Delta \bm{x}])$
    \end{itemize}
  \item It follows from \eqref{eq:fxfz2}, \eqref{eq:fxdx2}, and \eqref{eq:fzdzpsi2} that
    \begin{equation}\nonumber%\label{eq:}%
      f_{X}(\tilde{\bm{x}}(\Delta \bm{x})) = f_{Z}(\tilde{\bm{z}}(\Delta \bm{x})) \frac{|\psi([\bar{\bm{x}},\bar{\bm{x}}+\Delta \bm{x}])|}{\Delta x_{1}\Delta x_{2}},
      \quad \forall \Delta \bm{x} >0
    \end{equation}
  \item We know that for $\Delta \bm{x}$ close enough to $\bm{0}$
    \begin{align}
      |\psi([\bar{\bm{x}},\bar{\bm{x}}+\Delta \bm{x}])|
      & \approx \textstyle \text{area of parallelogram made by
        $\bm{a}_{1}:=
        \begin{bmatrix}
          \frac{\partial \psi_{1}(\bar{\bm{x}})}{\partial x_{1}}\\
          \frac{\partial\psi_{2}(\bar{\bm{x}})}{\partial x_{1}}
        \end{bmatrix}\Delta x_{1}$
        and
        $\bm{a}_{2}:=
        \begin{bmatrix}
          \frac{\partial \psi_{1}(\bar{\bm{x}})}{\partial x_{2}}\\
          \frac{\partial\psi_{2}(\bar{\bm{x}})}{\partial x_{2}}
        \end{bmatrix}\Delta x_{2}$
        } \nonumber \\
        & = \left|\det(\begin{bmatrix} \bm{a}_{1} & \bm{a}_{2}\end{bmatrix})\right|\nonumber \\
      & = \left|
        \begin{vmatrix}
          \frac{\partial \psi_{1}(\bar{\bm{x}})}{\partial x_{1}}\Delta x_{1} & \frac{\partial \psi_{1}(\bar{\bm{x}})}{\partial x_{2}}\Delta x_{2} \\
          \frac{\partial \psi_{2}(\bar{\bm{x}})}{\partial x_{1}}\Delta x_{1} & \frac{\partial \psi_{2}(\bar{\bm{x}})}{\partial x_{2}}\Delta x_{2} \\
        \end{vmatrix}\right|
       = \left|
        \begin{vmatrix}
          \frac{\partial \psi_{1}(\bar{\bm{x}})}{\partial x_{1}} & \frac{\partial \psi_{1}(\bar{\bm{x}})}{\partial x_{2}} \\
          \frac{\partial \psi_{2}(\bar{\bm{x}})}{\partial x_{1}} & \frac{\partial \psi_{2}(\bar{\bm{x}})}{\partial x_{2}} \\
        \end{vmatrix}
        \Delta x_{1}\Delta x_{2}
        \right|\nonumber \\
      & = \left|\left| \frac{d\psi(\bar{\bm{x}})}{d\bm{x}}\right|\right|\Delta x_{1} \Delta x_{2}
    \nonumber%\label{eq:}%
    \end{align}
    where $\left|\left| \frac{d\psi(\bar{\bm{x}})}{d\bm{x}}\right|\right|$
    is the absolute value of the determinant of the Jacobian matrix $\frac{d\psi(\bar{\bm{x}})}{d\bm{x}}$
    (See \figurename~\ref{fig:integration_3} for an illustration)
      \begin{figure}[t]\centering%
        \includegraphics[width=480pt]%
        {figures/fig_integration_3.png}
        \caption{%
          Derivation of $|\psi([\bar{\bm{x}},\bar{\bm{x}}+\Delta\bm{x}])|$
          based on Example~2
        }
        \label{fig:integration_3}
      \end{figure} 
  \item Obviously,
    \begin{equation}\nonumber%\label{eq:}%
      \lim_{\Delta \bm{x} \to \bm{0}}\tilde{\bm{x}}(\Delta \bm{x}) = \bar{\bm{x}},
      \quad
      \lim_{\Delta \bm{x} \to \bm{0}}\tilde{\bm{z}}(\Delta \bm{x}) = \psi(\bar{\bm{x}})
    \end{equation}
    and therefore
    \begin{align}
      f_{X}(\bar{\bm{x}})
        & = \lim_{\Delta \bm{x} \to \bm{0}}f_{X}(\tilde{\bm{x}}(\Delta \bm{x}))  \nonumber \\
        & = \lim_{\Delta \bm{x} \to \bm{0}} f_{Z}(\tilde{\bm{z}}(\Delta \bm{x}))\frac{|\psi([\bar{\bm{x}},\bar{\bm{x}}+\Delta \bm{x}])|}{\Delta x_{1}\Delta x_{2}}  \nonumber \\
        & = f_{Z}(\psi(\bar{\bm{x}}))\lim_{\Delta \bm{x} \to \bm{0}} \frac{|\psi([\bar{\bm{x}},\bar{\bm{x}}+\Delta \bm{x}])|}{\Delta x_{1}\Delta x_{2}}  \nonumber \\
        & = f_{Z}(\psi(\bar{\bm{x}}))\left|\left| \frac{d\psi(\bar{\bm{x}})}{d\bm{x}}\right|\right|
    \nonumber%\label{eq:}%
    \end{align}

  \item Since the argument above does not depend on the choice of $\bar{\bm{x}}$,
    we conclude that the function $f_{X}$ must be given by \eqref{eq:fx2}
  \item The same is true of higher $m$

  \end{itemize}
  \clearpage

\item \textbf{Example~1}

  \begin{itemize}
  \item Consider the integral
    \begin{equation}\nonumber%\label{eq:}%
      \int_{Z} f_{Z}(z)dz,
      \quad
      f_{Z}(z) = \frac{(z^{-\alpha}+\gamma)^{\beta}}{z^{\alpha+1}}
      \quad Z:=[a, b],
      \quad b>a>0
    \end{equation}
    for some $\alpha, \beta,\gamma>0$
  \item Transform $z$ into $x$ by defining a function $\phi:Z\to \R$ as
    \begin{equation}\nonumber%\label{eq:}%
      x = \phi(z) := z^{-\alpha} + \gamma
      \quad \forall z \in Z
    \end{equation}
  \item Notice that $\phi:Z\to X$ is a bijective (monotonically decreasing) function with
    \begin{equation}\nonumber%\label{eq:}%
      X:=
      \phi(Z) =
      [\phi(b), \phi(a)]
      =
      [b^{-\alpha} + \gamma, a^{-\alpha} + \gamma]
    \end{equation}
    and the inverse function $\psi:=\phi^{-1}$ is given by
    \begin{equation}\nonumber%\label{eq:}%
      z = \psi(x):=(x - \gamma)^{- \frac{1}{\alpha}}
      \quad \forall x \in X
    \end{equation}
  \item Using the change of variable formula, one would obtain
    \begin{align}
      \int_{Z}f_{Z}(z)dz
        & = \int_{X}f_{X}(x)dx,\quad\text{where}\quad f_{X}(x):=f_{Z}(\psi(x))\left| \frac{d\psi(x)}{dx}\right|  \nonumber \\
        & = \int_{X} (x-\gamma)^{\frac{\alpha+1}{\alpha}}x^{\beta}\left|-\frac{1}{\alpha}(x-\gamma)^{- \frac{\alpha+1}{\alpha}}\right|dx  \nonumber \\
        & = \int_{X} \frac{1}{\alpha} x^{\beta}dx  \nonumber \\
        & = \int_{b^{-\alpha} + \gamma}^{a^{-\alpha} + \gamma} \frac{d}{dx}\left\{\frac{1}{\alpha \beta} x^{\beta} \right\}dx  \nonumber \\
        & = \frac{1}{\alpha \beta} \left((a^{-\alpha} + \gamma)^{\beta} - (b^{-\alpha} + \gamma)^{\beta} \right)
    \nonumber%\label{eq:}%
    \end{align}

  \end{itemize}

\item \textbf{Example~2}

  \begin{itemize}
  \item Consider the integral
    \begin{equation}\nonumber%\label{eq:}%
      \int_{Z}f_{Z}(\bm{z})d\bm{z},
      \quad\text{where}\quad
      f_{Z}(\bm{z}):=e^{-z_{1}^{2}-z_{2}^{2}},
      \quad
      Z:= \left\{ (z_{1},z_{2})\in \R^{2}\setminus\{\bm{0}\} \,\Big|\, z_{1}^{2} + z_{2}^{2} \leq a^{2} \right\}
    \end{equation}
  \item Transform $\bm{z}=(z_{1},z_{2})$ into $\bm{x}=(x_{1},x_{2})$ by defining a function $\phi:Z \to \R^{2}$ as
    \begin{equation}\nonumber%\label{eq:}%
      \begin{bmatrix}
        x_{1} \\
        x_{2}
      \end{bmatrix}
      =
      \begin{bmatrix}
        \phi_{1}(z_{1},z_{2}) \\
        \phi_{2}(z_{1},z_{2})
      \end{bmatrix}
      :=
      \begin{bmatrix}
        (z_{1}^{2} + z_{2}^{2})^{1/2} \\
        \arctan2(z_{2},z_{1})
      \end{bmatrix}
      \quad \forall \bm{z}\in Z
    \end{equation}
  \item Notice that $\phi:Z \to X$ is bijective with
    \begin{equation}\nonumber%\label{eq:}%
      X:= \phi(Z) = \left\{ (x_{1},x_{2}) \,|\, a \geq x_{1}> 0, \pi\geq x_{2} \geq -\pi \right\}
    \end{equation}
    and the inverse function $\psi:=\phi^{-1}$ is given by
    \begin{equation}\nonumber%\label{eq:}%
      \begin{bmatrix}
        z_{1} \\
        z_{2}
      \end{bmatrix}
      =
      \begin{bmatrix}
        \psi_{1}(x_{1},x_{2}) \\
        \psi_{2}(x_{1},x_{2})
      \end{bmatrix}
      :=
      \begin{bmatrix}
        x_{1}\cos(x_{2}) \\
        x_{1}\sin(x_{2})
      \end{bmatrix}
      \quad \forall \bm{x} \in X
    \end{equation}
  \item Using the change of variable formula,
    \begin{align}
      \int_{Z}f_{Z}(\bm{z})d\bm{z}
      & = \int_{X}f_{X}(\bm{x})d\bm{x}, \quad \text{where}\quad
        f_{X}(\bm{x}) = f_{Z}(\psi(\bm{x})) \left|\left| \frac{d\psi(\bm{x})}{d\bm{x}}\right|\right|
        \nonumber \\
        & = \int_{X}f_{Z}(\psi(\bm{x})) \left|\left| \frac{d\psi(\bm{x})}{d\bm{x}}\right|\right| d\bm{x} \nonumber \\
      & = \int_{-\pi}^{\pi}\int_{0}^{a}e^{-x_{1}^{2}\cos^{2}(x_{2})-x_{1}^{2}\sin^{2}(x_{2})}
        \left|
        \begin{vmatrix}
          \cos(x_{2}) & -x_{1}\sin(x_{2}) \\
          \sin(x_{2}) & x_{1}\cos(x_{2}) \\
        \end{vmatrix}
        \right| dx_{1}dx_{2} \nonumber \\
      & = \int_{-\pi}^{\pi}\int_{0}^{a}e^{-x_{1}^{2}\cos^{2}(x_{2})-x_{1}^{2}\sin^{2}(x_{2})}
        \left| x_{1}\cos^{2}(x_{2}) + x_{1}\sin^{2}(x_{2}) \right| dx_{1}dx_{2} \nonumber \\
      & = \int_{-\pi}^{\pi}\int_{0}^{a}e^{-x_{1}^{2}}x_{1} dx_{1}dx_{2} \nonumber \\
      & = \int_{-\pi}^{\pi} \frac{1}{2}\left(1 - e^{-a^{2}}\right)  dx_{2} \nonumber \\
      & = \pi\left(1 - e^{-a^{2}}\right)
    \nonumber%\label{eq:}%
    \end{align}
  \item Since $\lim_{a\to\infty}Z=\R^{2}\setminus\{\bm{0}\}$, we have
    \begin{equation}\nonumber%\label{eq:}%
      \int_{\R^{2}}e^{-z_{1}^{2}-z_{2}^{2}}d\bm{z}
      =\int_{\R^{2}\setminus\{\bm{0}\}}e^{-z_{1}^{2}-z_{2}^{2}}d\bm{z}
      = \lim_{a\to\infty}\pi\left(1 - e^{-a^{2}}\right)
      = \pi,
    \end{equation}
    which in turn allows us to compute the Gaussian integral:
    \begin{equation}\label{eq:gaussianintegral}%
      \int_{\R}e^{-z^{2}}dz
      =
      \left(
      \left(\int_{\R}e^{-z_{1}^{2}}dz_{1}\right)
      \left(\int_{\R}e^{-z_{2}^{2}}dz_{2}\right)
      \right)^{1/2}
      =
      \left(
      \int_{\R}\int_{\R}e^{-z_{1}^{2}}e^{-z_{2}^{2}}dz_{1}dz_{2}
      \right)^{1/2}
      = \sqrt{\pi}
    \end{equation}
  \end{itemize}

\item \textbf{Example~3}

  \begin{itemize}
  \item Consider the integral
    \begin{equation}\nonumber%\label{eq:}%
      \int_{Z}f_{Z}(\bm{z})d\bm{z},
      \quad
      Z \subset \R^{m}
    \end{equation}
    for some function $f_{Z}:\R^{m}\to \R$
  \item Transform $\bm{z}\in \R^{m}$ into $\bm{x}\in \R^{m}$ by defining a function $\phi:\R^{m} \to \R^{m}$ as
    \begin{equation}\nonumber%\label{eq:}%
      \bm{x} = \phi(\bm{z}) := \bm{A}\bm{z} + \bm{b}
      \quad \forall \bm{z}\in \R^{m}
    \end{equation}
    for some $\bm{A}\in \R^{m\times m}$ and $\bm{b}\in \R^{m}$
  \item Notice that, as long as $\bm{A}$ is non-singular,
    the function $\phi: Z \to X$ is bijective
    and the inverse function $\psi:=\phi^{-1}$ is
    \begin{equation}\nonumber%\label{eq:}%
      \bm{z}
      = \psi(\bm{x})
      := \bm{A}^{-1}(\bm{x}-\bm{b})
      \quad \forall \bm{x} \in X:= \phi(Z) = \left\{ \bm{x}\in \R^{m} \,|\, \bm{x}=\bm{A}\bm{z} + \bm{b}\right\}
    \end{equation}
  \item Using the change of variable formula, we arrive at
    \begin{align}
      \int_{Z}f_{Z}(\bm{z})d\bm{z}
      & = \int_{X}f_{X}(\bm{x})d\bm{x}, \quad \text{where}\quad
        f_{X}(\bm{x}) = f_{Z}(\psi(\bm{x})) \left|\left| \frac{d\psi(\bm{x})}{d\bm{x}}\right|\right|
        \nonumber \\
      & = \int_{X}f_{Z}(\psi(\bm{x})) \left|\left| \frac{d\psi(\bm{x})}{d\bm{x}}\right|\right| d\bm{x} \nonumber \\
      & = \int_{X}f_{Z}(\bm{A}^{-1}(\bm{x}-\bm{b}))\left|\left|\bm{A}^{-1}\right|\right| d\bm{x} \nonumber \\
      & = \int_{X}f_{Z}(\bm{A}^{-1}(\bm{x}-\bm{b}))\frac{1}{||\bm{A}||}d\bm{x}
    \nonumber%\label{eq:}%
    \end{align}
  \end{itemize}
\item \textbf{Example~4}
  \begin{itemize}
  \item Let us say we want to compute the integral
    \begin{equation}\nonumber%\label{eq:}%
      \int_{-\infty}^{\infty}e^{-\frac{1}{2}z^{2}}dz
    \end{equation}
  \item With the application of Gaussian integral \eqref{eq:gaussianintegral} in mind,
    let us consider the following ``change of variable''
    \begin{equation}\nonumber%\label{eq:}%
      x = \underbrace{\frac{1}{\sqrt{2}} z}_{\phi(z)}
      \qquad \text{or equivalently} \qquad
      z = \underbrace{\sqrt{2}x}_{\psi(x)}
    \end{equation}
  \item Then the change of variable formula implies
    \begin{equation}\nonumber%\label{eq:}%
      \int_{-\infty}^{\infty}e^{-\frac{1}{2}z^{2}}dz
      =
      \int_{-\infty}^{\infty}e^{-\frac{1}{2}(\psi(x))^{2}}\underbrace{|\psi'(x)|}_{\sqrt{2}}dx
      =
      \sqrt{2}\underbrace{\int_{-\infty}^{\infty}e^{-x^{2}}dx}_{\sqrt{\pi}}
      =
      \sqrt{2\pi}
    \end{equation}
  \item Note that this is a special case of Example~3, where $\bm{A}=1/\sqrt{2}$ and $\bm{b}=0$
  \end{itemize}

\end{itemize}

\end{document}
