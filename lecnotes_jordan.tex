\documentclass[12pt,a4paper]{article} 
\usepackage[T1]{fontenc}
\usepackage{textcomp}
\usepackage{makeidx}

\usepackage[sc, noBBpl]{mathpazo}
\usepackage{dsfont} % for indicator function
\usepackage[deluxe]{otf}
\usepackage[utf8]{inputenc}

\usepackage{amsmath}
\usepackage{amsfonts}
\usepackage{amssymb}
\usepackage{amsthm}
\usepackage{mathtools}
\usepackage{mathrsfs}

\usepackage{listings} % for programming codes

\usepackage{bm}
\usepackage[dvipdfmx]{color}
\pagestyle{plain} 
\usepackage{float}
\usepackage[dvipdfmx]{graphicx}

\usepackage[left=15mm, right=18mm, top=20mm, bottom=20mm]{geometry}

\setlength{\footskip}{40pt} 
\setlength{\abovecaptionskip}{0pt}
\setlength{\belowcaptionskip}{0pt}
\renewcommand{\baselinestretch}{1.05}

\newcommand{\argmax}{\mathop{\rm arg max}\limits}
\newcommand{\argmin}{\mathop{\rm arg min}\limits}
\newcommand{\st}{\mathop{\rm s.t.}\limits}
\newcommand{\fsize}[1]{\fontsize{#1}{#1}\selectfont}

% custom color
\usepackage[dvipdfmx]{xcolor}

% colors
\definecolor{darkred}{rgb}{.40,.00,.00}
\definecolor{navy}{rgb}{.00,.00,.40}
\definecolor{darknavy}{rgb}{.00,.00,.30}
\definecolor{lightnavy}{rgb}{.10,.10,.40}

\newcommand{\navy}{\color{navy}}
\newcommand{\deepred}{\color{darkred}}
\newcommand{\emred}[1]{\textbf{\deepred #1}}
\newcommand{\emblue}[1]{\textbf{\navy #1}}

% title and tableofcontents
\usepackage[titles]{tocloft}
\setlength{\cftbeforesecskip}{2pt}
\setlength{\cftbeforesubsecskip}{0pt}
\renewcommand{\cftsecfont}{\normalsize\mcfamily}
\renewcommand{\cftsubsecfont}{\normalsize\mcfamily}
\renewcommand{\cftsubsecfont}{\normalsize\mcfamily}
\renewcommand{\contentsname}{\large\mcfamily\bfseries \S\hspace{0.5\baselineskip}Contents\\[-\baselineskip]}
\cftsetindents{section}{10.0pt}{15.0pt}
\cftsetindents{subsection}{20.0pt}{30.0pt}
\cftpagenumbersoff{section}
\cftpagenumbersoff{subsection}
\cftpagenumbersoff{subsubsection}

\makeatletter
\def\vhrulefill#1{\leavevmode\leaders\hrule\@height#1\hfill \kern\z@}
\makeatother

\makeatletter
\def\maketitle{%
	\begin{center}\leavevmode
	\normalfont
	{\normalsize\raggedleft \@date\par}%
	\vskip -0.7\baselineskip
	\vhrulefill{0.2pt}\par
	\vskip 0.3\baselineskip
	\renewcommand{\baselinestretch}{0.80}
	{\color{darknavy}\bfseries\Large\raggedright \@title\par}%
	\renewcommand{\baselinestretch}{1.00}
	\vskip 0.2\baselineskip
	\ifx\@subtitle\@empty\else%
	{\normalsize\raggedright \@subtitle\par}%
	\fi
	\ifx\@author\@empty\else%
	\vskip 0.5\baselineskip
	{\normalsize\raggedleft \@author\par}%
	\fi
	\end{center}%
	\vskip 1.0\baselineskip
}
\def\subtitle#1{\gdef\@subtitle{#1}}
\let\@subtitle\@empty
\makeatother

% section
\newcommand{\Secindent}{0.0\baselineskip}
\newcommand{\Subsecindent}{0.0\baselineskip}
\newcommand{\lmargini}{1.3\baselineskip}
\newcommand{\lmarginii}{0.5\baselineskip}
\newcommand{\lmarginiii}{0.2\baselineskip}
\def\theenumi{\arabic{enumi}}
\def\theenumii{\alph{enumii}}
\renewcommand{\labelenumi}{\hfill\hspace{60pt}\theenumi.}
\renewcommand{\labelenumii}{\hspace{30pt}\theenumii.}
\renewcommand{\labelitemi}{\small\textbullet\hspace{2pt}}
\renewcommand{\labelitemii}{\color{gray}\small$\circ$\hspace{2pt}}
\renewcommand{\labelitemiii}{\textendash\hspace{1pt}}
\makeatletter
\renewcommand{\section}{\@startsection%
	{section}{1}{\Secindent}% name, depth, indent \z@=0pt
	{-1.0\baselineskip \@plus1.5\baselineskip \@minus0.5\baselineskip}% beforeskip (no indent if negative)
	{0.5\baselineskip \@plus0.8\baselineskip \@minus0.3\baselineskip}% afterskip (no line break if negative)
	{\reset@font\large\bfseries\scshape\color{darknavy}}% font
}
\renewcommand{\subsection}{\@startsection%
	{subsection}{2}{\Subsecindent}%
	{-0.5\baselineskip \@plus0.7\baselineskip \@minus0.4\baselineskip}%
	{0.3\baselineskip \@plus0.5\baselineskip \@minus0.2\baselineskip}%
	{\reset@font\normalsize\bfseries\scshape\color{darknavy}}% 
}
\renewcommand{\thesection}{\@arabic\c@section} % arabic, roman, Roman, alph, Alph, fnsymbol
\renewcommand{\thesubsection}{\thesection\hspace{0pt}.\hspace{0pt}\@arabic\c@subsection}
\makeatother

\makeatletter
\def\@listi{%
	\topsep = 0.2\baselineskip
	\partopsep = 0.0\baselineskip
	\parsep = 0.0\baselineskip
	\itemsep = 0.5\baselineskip
	\leftmargin = 1.5zw
	\labelsep = 0.5zw
	\itemindent = 0pt
	\labelwidth\leftmargin \advance\labelwidth-\labelsep
	\listparindent = 0zw
	\rightmargin = 0pt
	\reset@font\normalsize\normalfont
}
\let\@listI\@listi
\@listi
\def\@listii{%
	\topsep = 0.25\baselineskip
	\partopsep = 0.0\baselineskip
	\parsep = 0.0\baselineskip
	\itemsep = 0.3\baselineskip
	\leftmargin = 0.5zw
	\labelsep = .5zw
	\itemindent = 0pt
	\labelwidth\leftmargin \advance\labelwidth-\labelsep
	\listparindent = 0zw
	\rightmargin = 0pt
	\reset@font\normalsize\normalfont%
}
\def\@listiii{%
	\topsep = 0.30\baselineskip
	\partopsep = 0.0\baselineskip
	\parsep = 0.1\baselineskip
	\itemsep = 0.05\baselineskip
	\leftmargin = 1.1zw
	\labelsep = .5zw
	\itemindent = 0pt
	\labelwidth\leftmargin \advance\labelwidth-\labelsep
	\listparindent = 0zw
	\rightmargin = 0pt
	\reset@font\normalsize\normalfont%
}
\def\@listiv{%
	\topsep = 0.0\baselineskip
	\partopsep = 0.0\baselineskip
	\parsep = 0.0\baselineskip
	\itemsep = 0.0\baselineskip
	\leftmargin = 0.9zw
	\labelsep = .5zw
	\itemindent = 0pt
	\labelwidth\leftmargin \advance\labelwidth-\labelsep
	\listparindent = 0zw
	\rightmargin = 0pt
	\reset@font\normalsize\normalfont%
}
\makeatother

\newcommand{\R}{\ensuremath{\mathbb{R}}}
\newcommand{\E}{\ensuremath{\mathbb{E}}}
\newcommand{\V}{\ensuremath{\mathbb{V}}}
\newcommand{\Cov}{\ensuremath{\text{Cov}}}
\newcommand{\N}{\ensuremath{\mathbb{N}}}
\newcommand{\Z}{\ensuremath{\mathbb{Z}}}
\let\epsilon\varepsilon

\let\origitem\item
\renewcommand{\item}{\normalfont\origitem}

\usepackage[%
backref=page,
dvipdfmx,
bookmarks=true,bookmarksnumbered=true,bookmarkstype=toc,
colorlinks=true,
linkcolor=navy,
citecolor=navy,
filecolor=blue,
pagecolor=blue,
urlcolor=blue
]{hyperref}
\AtBeginDvi{\special{pdf:tounicode EUC-UCS2}}

\renewcommand*{\backref}[1]{}
\renewcommand*{\backrefalt}[4]{%
  \ifcase #1 (Not cited.)%
  \or        [#2]%
  \else      [#2]%
  \fi}
 

\title{Jordan normal form}
\subtitle{Introduction to dynamical systems~\#3}
\author{Hiroaki Sakamoto}
\date{\today}

\begin{document}
\maketitle
\tableofcontents

\section{Jordan normal form}

\subsection{Jordan block and Jordan matrix}

\begin{itemize}

\item \textbf{Jordan block}
  \begin{itemize}
  \item A square matrix of the following form is called a \emph{Jordan block} of size $m$:
    \begin{equation}\nonumber%\label{eq:}%
      \bm{J}_{m}(\lambda) :=
      \begin{bmatrix}
        \lambda & 1 & 0 & \cdots & 0 & 0 \\
        0 & \lambda & 1 & \cdots & 0 & 0 \\
        0 & 0 & \lambda & \ddots & 0 & 0 \\
        \vdots & \vdots & \vdots & \ddots & \ddots & \vdots \\
        0 & 0 & 0 & \cdots & \lambda & 1 \\
        0 & 0 & 0  & \cdots & 0 &  \lambda \\
      \end{bmatrix} \in \R^{m\times m}
    \end{equation}
    
  \item For example,
    \begin{equation}\nonumber%\label{eq:}%
      \bm{J}_{1}(\lambda) = \lambda,
      \quad
      \bm{J}_{2}(\lambda) =
      \begin{bmatrix}
        \lambda & 1 \\
        0 & \lambda \\
      \end{bmatrix},
      \quad
      \bm{J}_{3}(\lambda) =
      \begin{bmatrix}
        \lambda & 1 & 0 \\
        0 & \lambda & 1 \\
        0 & 0 & \lambda \\
      \end{bmatrix}
    \end{equation}
  \end{itemize}

\item \textbf{Jordan matrix}
  \begin{itemize}
  \item A block diagonal matrix of the form
    \begin{equation}\nonumber%\label{eq:}%
      \bm{J} =
      \begin{bmatrix}
        \bm{J}_{n_{1}}(\lambda_{1}) & 0 & \cdots & 0 \\
        0 & \bm{J}_{n_{2}}(\lambda_{2}) & \cdots & 0 \\
        \vdots & \vdots & \ddots & \vdots \\
        0 & 0 & \cdots & \bm{J}_{n_{d}}(\lambda_{d}) \\
      \end{bmatrix}
    \end{equation}
    is called a \emph{Jordan matrix},
    where $\lambda_{1}, \lambda_{2},\ldots, \lambda_{d}$ can be the same value or different values
  \item Examples of a Jordan matrix:
    \begin{equation}\nonumber%\label{eq:}%
      \bm{J} =
      \begin{bmatrix}
        2 & 1 \\
        0 & 2 \\
      \end{bmatrix},
      \quad
      \bm{J} =
      \begin{bmatrix}
        2 & 0 \\
        0 & 5 \\
      \end{bmatrix},
      \quad
      \bm{J} =
      \begin{bmatrix}
        3 & 1 & 0 \\
        0 & 3 & 0 \\
        0 & 0 & 2 \\
      \end{bmatrix}
      \quad
      \bm{J} =
      \begin{bmatrix}
        2 & 1 & 0 & 0 \\
        0 & 2 & 0 & 0 \\
        0 & 0 & 2 & 1 \\
        0 & 0 & 0 & 2 \\
      \end{bmatrix}
    \end{equation}
  \end{itemize}

\end{itemize}

\subsection{Generalized eigenvectors}

\begin{itemize}

\item \textbf{Definition}
  \begin{itemize}
    \item Let $\lambda$ be an eigenvalue of $\bm{A}\in \R^{n\times n}$
    \item We call $\bm{v}\in \R^{n}$ a \emph{generalized eigenvector of rank $m\in \N$} associated with $\lambda$ if
      \begin{equation}\nonumber%\label{eq:}%
        (\bm{A}-\lambda \bm{I})^{m}\bm{v} = \bm{0}
        \quad\text{and}\quad
        (\bm{A}-\lambda \bm{I})^{m-1}\bm{v} \neq \bm{0}
      \end{equation}
    \item Note that an eigenvector is nothing but a generalized eigenvector of rank $m=1$
      \begin{equation}\nonumber%\label{eq:}%
        (\bm{A}-\lambda \bm{I})^{1}\bm{v} = \bm{0}
        \quad\text{and}\quad
        \bm{v} = (\bm{A}-\lambda \bm{I})^{1-1}\bm{v} \neq \bm{0}
      \end{equation}
    \item Important fact:
      \begin{itemize}
      \item An eigenvalue does \textbf{not always} have as many linearly independent \textbf{eigenvectors} as its algebraic multiplicity
      \item Any eigenvalue \textbf{always} has as many linearly independent \textbf{generalized eigenvectors} as its algebraic multiplicity
      \end{itemize}
  \end{itemize}

\item \textbf{Example~1}
  \begin{itemize}
  \item Consider a $3\times 3$ square matrix
    \begin{equation}\nonumber%\label{eq:}%
      \bm{A} :=
      \begin{bmatrix}
        3 & -1 & 1 \\
        2 & 0 & 2 \\
        1 & -1 & 3 \\
      \end{bmatrix},
    \end{equation}
    whose characteristic polynomial is
    \begin{equation}\nonumber%\label{eq:}%
      \phi_{\bm{A}}(t) =
      \begin{vmatrix}
        3-t & -1 & 1 \\
        2 & 0-t & 2 \\
        1 & -1 & 3-t \\
      \end{vmatrix}
      = (2-t)^{3},
    \end{equation}
    meaning that $\lambda:=1$ is the unique eigenvalue of $\bm{A}$ (with algebraic multiplicity of $3$)
  \item Then
    \begin{equation}\nonumber%\label{eq:}%
      \bm{v} :=
      \begin{bmatrix}
        1 \\ 0 \\ 0
      \end{bmatrix}
    \end{equation}
    is a generalized eigenvector of rank $2$ associated with $\lambda=2$,
    because
    \begin{equation}\nonumber%\label{eq:}%
      (\bm{A}-\lambda\bm{I})^{2}\bm{v} = 
      \begin{bmatrix}
        3-2 & -1 & 1 \\
        2 & 0-2 & 2 \\
        1 & -1 & 3-2 \\
      \end{bmatrix}^{2}
      \begin{bmatrix}
        1 \\ 0 \\ 0
      \end{bmatrix}
      = 
      \begin{bmatrix}
        0 & 0 & 0 \\
        0 & 0 & 0 \\
        0 & 0 & 0 \\
      \end{bmatrix}
      \begin{bmatrix}
        1 \\ 0 \\ 0
      \end{bmatrix}
      = \bm{0}
    \end{equation}
    and
    \begin{equation}\nonumber%\label{eq:}%
      (\bm{A}-\lambda\bm{I})\bm{v} = 
      \begin{bmatrix}
        3-2 & -1 & 1 \\
        2 & 0-2 & 2 \\
        1 & -1 & 3-2 \\
      \end{bmatrix}
      \begin{bmatrix}
        1 \\ 0 \\ 0
      \end{bmatrix}
      = 
      \begin{bmatrix}
        1 \\ 2 \\ 1
      \end{bmatrix}
      \neq \bm{0}
    \end{equation}
  \item In fact,
    since $(\bm{A}-\lambda \bm{I})^{2}=\bm{O}$,
    any vector $\bm{v}=(v_{1}, v_{2}, v_{3})^{\top}$ is
    a generalized eigenvector of rank $2$ associated with $\lambda=2$
    as long as $(\bm{A}-\lambda \bm{I})\bm{v}\neq \bm{0}$
    (i.e., as long as $\bm{v}$ is not an eigenvector itself)
  \end{itemize}

\end{itemize}

\subsection{Jordan decomposition}

\begin{itemize}
\item \textbf{Jordan decomposition theorem}
  \begin{itemize}
  \item Consider a square matrix $\bm{A} \in\R^{n\times n}$
  \item There always exists
    a Jordan matrix $\bm{J}\in \R^{n\times n}$ and
    a nonsingular matrix $\bm{V}\in \R^{n\times n}$ (consisting of $n$ generalized eigenvectors) such that
    \begin{equation}\nonumber%\label{eq:}%
      \bm{A} = \bm{V} \bm{J} \bm{V}^{-1},
      \quad\text{where}\quad
      \bm{J} =
      \begin{bmatrix}
        \bm{J}_{n_{1}}(\lambda_{1}) & 0 & \cdots & 0 \\
        0 & \bm{J}_{n_{2}}(\lambda_{2}) & \cdots & 0 \\
        \vdots & \vdots & \ddots & \vdots \\
        0 & 0 & \cdots & \bm{J}_{n_{d}}(\lambda_{d}) \\
      \end{bmatrix}
    \end{equation}
    and $\lambda_{1},\lambda_{2},\ldots, \lambda_{d}$ are eigenvalues (with possibility of duplicates) of $\bm{A}$
  \item The matrix $\bm{J}$ above is called the \emph{Jordan normal form} of $\bm{A}$
  \end{itemize}

\item \textbf{More details}
  \begin{itemize}
  \item There exist $\lambda_{1},\lambda_{2},\ldots, \lambda_{d}\in \R$
    and
    $n_{1}, n_{2}, \ldots, n_{d} \in \N$
    such that
    \begin{enumerate}
    \item $\lambda_{1},\lambda_{2},\ldots, \lambda_{d}$ are eigenvalues of $\bm{A}$ (with possibility of duplicates)
    \item $n_{1} + n_{2} + \ldots + n_{d} = n$
    \item for each $i=1,\ldots, d$,
      there exist $n_{i}$ generalized eigenvectors $\bm{v}_{i,1}, \bm{v}_{i,2}, \ldots, \bm{v}_{i,n_{i}}$
      associated with $\lambda_{i}$
      such that
      \begin{itemize}
      \item $\bm{v}_{i,1}$ is a generalized eigenvector of rank $1$ satisfying $(\bm{A} - \lambda_{i}\bm{I})\bm{v}_{i,1}= \bm{0}$ and $\bm{v}_{i,1}\neq \bm{0}$
      \item $\bm{v}_{i,2}$ is a generalized eigenvector of rank $2$ satisfying $(\bm{A} - \lambda_{i}\bm{I})\bm{v}_{i,2} = \bm{v}_{i,1}$
      \item $\bm{v}_{i,3}$ is a generalized eigenvector of rank $3$ satisfying $(\bm{A} - \lambda_{i}\bm{I})\bm{v}_{i,3} = \bm{v}_{i,2}$
      \item[] $\vdots$
      \item $\bm{v}_{i,n_{i}}$ is a generalized eigenvector of rank $n_{i}$ satisfying $(\bm{A} - \lambda_{i}\bm{I})\bm{v}_{i,n_{i}} = \bm{v}_{i,n_{i}-1}$
      \end{itemize}
    \end{enumerate}
  \item Then $\bm{v}_{1,1},\bm{v}_{1,2},\ldots, \bm{v}_{1,n_{1}},\bm{v}_{2,1},\ldots, \bm{v}_{2,n_{2}},\ldots, \bm{v}_{d,n_{d}}$
    are $n$ linearly independent vectors
  \item Note that for each $i = 1, 2, \ldots, d$,
    \begin{equation}\nonumber%\label{eq:}%
      (\bm{A} - \lambda_{i}\bm{I})
      \underbrace{
      \begin{bmatrix}
        \bm{v}_{i,1} & \bm{v}_{i,2} & \ldots & \bm{v}_{i,n_{i}}
      \end{bmatrix}}_{=:\bm{V}_{i}\in \R^{n\times n_{i}}}
      =
      \begin{bmatrix}
        \bm{0} & \bm{v}_{i,1} & \ldots & \bm{v}_{i,n_{i}-1}
      \end{bmatrix},
    \end{equation}
    which implies
    \begin{equation}
      \bm{V}_{i} \bm{J}_{n_{i}}(\lambda_{i})
      = \bm{V}_{i} \left(\lambda_{i}\bm{I}
        +
        \begin{bmatrix}
          0 & 1 & 0 & \ldots & 0 \\
          0 & 0 & 1 & \ldots & 0 \\
          \vdots & \vdots & \ddots & \ddots & \vdots \\
          0 & 0 & \ldots & 0 & 1 \\
          0 & 0 & \ldots & 0 & 0 \\
        \end{bmatrix}
        \right)
      = \lambda_{i}\bm{V}_{i} + 
      \begin{bmatrix}
        \bm{0} & \bm{v}_{i,1} & \ldots & \bm{v}_{i,n_{i}-1}
      \end{bmatrix}
      = \bm{A}\bm{V}_{i}
    \nonumber%\label{eq:}%
    \end{equation}
  \item Hence, defining $\bm{V}:=[\bm{V}_{1}\, \bm{V}_{2} \, \ldots \, \bm{V}_{d}]$ gives 
    \begin{equation}\nonumber
      \bm{V}
      \begin{bmatrix}
        \bm{J}_{n_{1}}(\lambda_{1}) & 0 & \cdots & 0 \\
        0 & \bm{J}_{n_{2}}(\lambda_{2}) & \cdots & 0 \\
        \vdots & \vdots & \ddots & \vdots \\
        0 & 0 & \cdots & \bm{J}_{n_{d}}(\lambda_{d}) \\
      \end{bmatrix}
      =
        \big[
        \underbrace{\bm{V}_{1}\bm{J}_{n_{1}}(\lambda_{1})}_{\bm{A}\bm{V}_{1}}
        \,\,\,
        \underbrace{\bm{V}_{2}\bm{J}_{n_{2}}(\lambda_{2})}_{\bm{A}\bm{V}_{2}}
        \,\,\,
        \ldots
        \,\,\,
        \underbrace{\bm{V}_{d}\bm{J}_{n_{d}}(\lambda_{d})}_{\bm{A}\bm{V}_{d}}
        \big]
        = \bm{A} \bm{V}
    \end{equation}
  \item Since $\bm{V}$ is nonsingular, we therefore obtain
    \begin{equation}\nonumber%\label{eq:}%
      \bm{A} = \bm{A}\bm{V}\bm{V}^{-1} = \bm{V}\bm{J}\bm{V}^{-1}
    \end{equation}
  \end{itemize}

\end{itemize}

\section{Examples}

\subsection{2$\times$2 matrix}

\begin{itemize}

\item \textbf{Example~2}
  \begin{itemize}
  \item Consider a square matrix
    \begin{equation}\nonumber%\label{eq:}%
      \bm{A} :=
      \begin{bmatrix}
        5 & 4 \\
        -1 & 1 \\
      \end{bmatrix}
    \end{equation}
  \item We know that
    \begin{itemize}
    \item the characteristic polynomial of $\bm{A}$ is
      \begin{equation}\nonumber%\label{eq:}%
        \phi_{\bm{A}}(t) =
        \begin{vmatrix}
          5 - t & 4 \\
          -1 & 1 - t \\
        \end{vmatrix}
        = (3-t)^{2}
      \end{equation}
    \item $\bm{A}$ only has one eigenvalue $\lambda_{1}:=3$
      with $\bm{v}_{1,1}:=(2, -1)^{\top}$ being an associated eigenvector
    \item $\bm{A}$ is not diagonalizable (no other linearly independent eigenvector)
    \end{itemize}
  \item But the Jordan decomposition theorem shows that
    $\bm{A}$ is ``semi-diagonalizable''
    in the sense that 
    there exists a non-zero vector $\bm{v}_{1,2}\in \R^{2}$ such that
    $\bm{v}_{1,1}, \bm{v}_{1,2}$ are linearly independent and
    \begin{equation}\nonumber%\label{eq:}%
      \bm{A} =
      \begin{bmatrix}
        \bm{v}_{1,1} & \bm{v}_{1,2}
      \end{bmatrix}
      \underbrace{
      \begin{bmatrix}
        \lambda_{1} & 1 \\
        0 & \lambda_{1} \\
      \end{bmatrix}}_{\bm{J}}
      \begin{bmatrix}
        \bm{v}_{1,1} & \bm{v}_{1,2}
      \end{bmatrix}^{-1},
    \end{equation}
    where $\bm{J}$ is the Jordan normal form of $\bm{A}$
  \item Procedure to find such $\bm{v}_{1,2}$
    \begin{enumerate}
    \item The discussion above indicates that
      $\bm{v}_{1,2}$ is a generalized eigenvector of rank $2$ associated with $\lambda_{1}$, satisfying
      \begin{equation}\nonumber%\label{eq:}%
        (\bm{A} - \lambda_{1}\bm{I})\bm{v}_{1,2} = \bm{v}_{1,1}
      \end{equation}
    \item Hence, we solve the following system of equations for $\bm{v}$:
      \begin{equation}\nonumber%\label{eq:}%
        (\bm{A} - \lambda_{1}\bm{I})\bm{v} = \bm{v}_{1,1}
        \implies
        \bm{v} = \ldots
      \end{equation}
    \item For the current example,
      \begin{equation}\nonumber%\label{eq:}%
        \begin{bmatrix}
          5 - 3 & 4 \\
          -1 & 1 - 3 \\
        \end{bmatrix}
        \begin{bmatrix}
          v_{1} \\
          v_{2} \\
        \end{bmatrix}
        = 
        \begin{bmatrix}
          2 \\
          -1 \\
        \end{bmatrix}
        \iff
        \begin{matrix*}[l]
          v_{1} + 2v_{2} = 1  \\
        \end{matrix*}
        \iff
        \begin{bmatrix}
          v_{1} \\
          v_{2} \\
        \end{bmatrix}
        =
        \begin{bmatrix}
          1 \\
          0 \\
        \end{bmatrix}
        +
        \alpha
        \begin{bmatrix}
          -2 \\
          1 \\
        \end{bmatrix}
        \quad \forall \alpha\in \R
      \end{equation}
    \item So we choose the following $\bm{v}_{1,2}$ as a generalized eigenvector associated with $\lambda_{1}$:
      \begin{equation}\nonumber%\label{eq:}%
        \bm{v}_{1,2} := 
        \begin{bmatrix}
          1 \\
          0 \\
        \end{bmatrix}
      \end{equation}
    \end{enumerate}
  \item Defining
    \begin{equation}\nonumber%\label{eq:}%
      \bm{V} :=
      \begin{bmatrix}
        \bm{v}_{1,1} & \bm{v}_{1,2}
      \end{bmatrix}
      =
      \begin{bmatrix}
        2 & 1 \\
        -1 & 0
      \end{bmatrix}
      \implies
      \bm{V}^{-1}
      = 
      \begin{bmatrix}
        0 & -1 \\
        1 & 2
      \end{bmatrix},
    \end{equation}
    one can verify that
    \begin{equation}\nonumber%\label{eq:}%
      \bm{V}\bm{J}\bm{V}^{-1}
      =
      \begin{bmatrix}
        2 & 1 \\
        -1 & 0
      \end{bmatrix}
      \begin{bmatrix}
        3 & 1 \\
        0 & 3
      \end{bmatrix}
      \begin{bmatrix}
        0 & -1 \\
        -1 & 2
      \end{bmatrix}
      =
      \begin{bmatrix}
        5 & 4 \\
        -1 & 1
      \end{bmatrix}
      = \bm{A}
    \end{equation}

  \end{itemize}

\clearpage
\item \textbf{Example~3}
  \begin{itemize}
  \item Consider a square matrix
    \begin{equation}\nonumber%\label{eq:}%
      \bm{A} :=
      \begin{bmatrix}
        4 & 1 \\
        -1 & 2 \\
      \end{bmatrix}
    \end{equation}
  \item The characteristic polynomial is
    \begin{equation}\nonumber%\label{eq:}%
      \phi_{\bm{A}}(t) = 
      \begin{vmatrix}
        4 -t & 1 \\
        -1 & 2-t \\
      \end{vmatrix}
      = (4-t)(2-t) + 1
      = (3-t)^{2},
    \end{equation}
    which means $\bm{A}$ has only one eigenvalue $\lambda_{1}:=3$
  \item Solving $(\bm{A}-\lambda_{1}\bm{I})\bm{v}=\bm{0}$ for $\bm{v}$ yields
    \begin{equation}\nonumber%\label{eq:}%
      \begin{bmatrix}
        4 - 3 & 1 \\
        - 1 & 2 - 3 \\
      \end{bmatrix}
      \begin{bmatrix}
        v_{1} \\
        v_{2} \\
      \end{bmatrix}
      = 
      \begin{bmatrix}
        0 \\
        0 \\
      \end{bmatrix}
      \iff
      \begin{matrix*}[l]
        v_{1} = -v_{2} \\
      \end{matrix*}
      \iff
      \begin{bmatrix}
        v_{1} \\
        v_{2} \\
      \end{bmatrix}
      =
      \alpha
      \begin{bmatrix}
        1 \\
        -1 \\
      \end{bmatrix}
      \quad \forall \alpha
    \end{equation}
    so we choose the following $\bm{v}_{1,1}$ as an eigenvector associated with $\lambda_{1}$:
    \begin{equation}\nonumber%\label{eq:}%
      \bm{v}_{1,1} :=
      \begin{bmatrix}
        1 \\
        -1 \\
      \end{bmatrix}
    \end{equation}
  \item Since there is no other linearly independent eigenvector,
    we seek to find a rank-$2$ generalized eigenvector $\bm{v}_{1,2}$ associated with $\lambda_{1}$
    such that
    \begin{equation}\nonumber%\label{eq:}%
      (\bm{A}-\lambda_{1}\bm{I})\bm{v}_{1,2}=\bm{v}_{1,1},
    \end{equation}
    which, once we found such $\bm{v}_{1,2}$, would imply
    \begin{equation}\nonumber%\label{eq:}%
      \bm{A}\bm{v}_{1,1} = \lambda_{1}\bm{v}_{1,1}
      =
      \begin{bmatrix}
        \bm{v}_{1,1} & \bm{v}_{1,2}
      \end{bmatrix}
      \begin{bmatrix}
        \lambda_{1}\\
        0 \\
      \end{bmatrix}
      \quad\text{and}\quad
      \bm{A}\bm{v}_{1,2} = \bm{v}_{1,1} + \lambda_{1}\bm{v}_{1,2}
      =
      \begin{bmatrix}
        \bm{v}_{1,1} & \bm{v}_{1,2}
      \end{bmatrix}
      \begin{bmatrix}
        1 \\
        \lambda_{1}
      \end{bmatrix}
    \end{equation}
    and thus
    \begin{equation}\nonumber%\label{eq:}%
      \bm{A}
      \begin{bmatrix}
        \bm{v}_{1,1} & \bm{v}_{1,2}
      \end{bmatrix}
      =
      \begin{bmatrix}
        \bm{v}_{1,1} & \bm{v}_{1,2}
      \end{bmatrix}
      \begin{bmatrix}
        \lambda_{1} & 1 \\
        0 & \lambda_{1}
      \end{bmatrix}
      \implies
      \bm{A}
      =
      \begin{bmatrix}
        \bm{v}_{1,1} & \bm{v}_{1,2}
      \end{bmatrix}
      \begin{bmatrix}
        \lambda_{1} & 1 \\
        0 & \lambda_{1}
      \end{bmatrix}
      \begin{bmatrix}
        \bm{v}_{1,1} & \bm{v}_{1,2}
      \end{bmatrix}^{-1}
    \end{equation}
    
  \item Solving $(\bm{A}-\lambda_{1}\bm{I})\bm{v}=\bm{v}_{1,1}$ for $\bm{v}$ yields
    \begin{equation}\nonumber%\label{eq:}%
      \begin{bmatrix}
        4 - 3 & 1 \\
        - 1 & 2 - 3 \\
      \end{bmatrix}
      \begin{bmatrix}
        v_{1} \\
        v_{2} \\
      \end{bmatrix}
      = 
      \begin{bmatrix}
        1 \\
        -1 \\
      \end{bmatrix}
      \iff
      \begin{matrix*}[l]
        v_{1} = 1 -v_{2} \\
      \end{matrix*}
      \iff
      \begin{bmatrix}
        v_{1} \\
        v_{2} \\
      \end{bmatrix}
      =
      \begin{bmatrix}
        1 \\
        0 \\
      \end{bmatrix}
      +
      \alpha
      \begin{bmatrix}
        1 \\
        -1 \\
      \end{bmatrix}
      \quad \forall \alpha
    \end{equation}
    so we choose the following $\bm{v}_{1,2}$ as a generalized eigenvector associated with $\lambda_{1}$:
    \begin{equation}\nonumber%\label{eq:}%
      \bm{v}_{1,2} :=
      \begin{bmatrix}
        1 \\
        0 \\
      \end{bmatrix}
    \end{equation}
  \item Defining
    \begin{equation}\nonumber%\label{eq:}%
      \bm{V} :=
      \begin{bmatrix}
        \bm{v}_{1,1} & \bm{v}_{1,2}
      \end{bmatrix}
      =
      \begin{bmatrix}
        1 & 1 \\
        -1 & 0
      \end{bmatrix}
      \implies
      \bm{V}^{-1}
      = 
      \begin{bmatrix}
        0 & -1 \\
        1 & 1
      \end{bmatrix},
    \end{equation}
    one can verify that
    \begin{equation}\nonumber%\label{eq:}%
      \bm{V}\bm{J}\bm{V}^{-1}
      =
      \begin{bmatrix}
        1 & 1 \\
        -1 & 0
      \end{bmatrix}
      \underbrace{
      \begin{bmatrix}
        3 & 1 \\
        0 & 3
      \end{bmatrix}}_{=\bm{J}}
      \begin{bmatrix}
        0 & -1 \\
        1 & 1
      \end{bmatrix}
      =
      \begin{bmatrix}
        4 & 1 \\
        -1 & 2
      \end{bmatrix}
      = \bm{A}
    \end{equation}
    
  \end{itemize}

\end{itemize}

\subsection{3$\times$3 matrix}

\begin{itemize}

\item \textbf{Example~4}
  \begin{itemize}
  \item Consider a square matrix
    \begin{equation}\nonumber%\label{eq:}%
      \bm{A} :=
      \begin{bmatrix}
        1 & 3 & 0 \\
        0 & 1 & 0 \\
        2 & 1 & 5 \\
      \end{bmatrix}
    \end{equation}
  \item We know that
    \begin{itemize}
    \item $\phi_{\bm{A}}(t) = -(t-1)^{2}(t-5)$ so eigenvalues are $\lambda_{1}:=1$ (twice) and $\lambda_{2}:=5$
    \item Only one linearly independent eigenvector associated with $\lambda_{1}$, such as $\bm{v}_{1,1}:=(2, 0, -1)^{\top}$
    \item Only one linearly independent eigenvector associated with $\lambda_{2}$, such as $\bm{v}_{2,1}:=(0, 0, 1)^{\top}$
    \item $\bm{A}$ is not diagonalizable
    \end{itemize}
  \item We need another generalized eigenvector which, together with $\bm{v}_{1,1}$ and $\bm{v}_{2,1}$,
    gives us 3 linearly independent vectors in total
  \item Since $\lambda_{1}$ has the algebraic multiplicity of $2$,
    it must have two generalized eigenvectors
    \begin{enumerate}
    \item Use $\bm{v}_{1,1}$ as the first generalized eigenvector (of rank $1$)
    \item Find the second generalized eigenvector, $\bm{v}_{1,2}$, such that
      \begin{equation}\nonumber%\label{eq:}%
        (\bm{A}-\lambda_{1}\bm{I})\bm{v}_{1,2} = \bm{v}_{1,1},
      \end{equation}
      so that $\bm{A}$ would be decomposed as
      \begin{equation}\nonumber%\label{eq:}%
        \bm{A} =
        \begin{bmatrix}
          \bm{v}_{1,1} & \bm{v}_{1,2} & \bm{v}_{2,1}
        \end{bmatrix}
        \begin{bmatrix}
          \lambda_{1} & 1 & 0 \\
          0 & \lambda_{1} & 0 \\
          0 & 0 & \lambda_{2} \\
        \end{bmatrix}
        \begin{bmatrix}
          \bm{v}_{1,1} & \bm{v}_{1,2} & \bm{v}_{2,1}
        \end{bmatrix}^{-1}
      \end{equation}
    \end{enumerate}
  \item Solving $(\bm{A}-\lambda_{1}\bm{I})\bm{v} = \bm{v}_{1,1}$ for $\bm{v}$ yields
    \begin{equation}\nonumber%\label{eq:}%
      \begin{bmatrix}
        1-1 & 3 & 0 \\
        0 & 1-1 & 0 \\
        2 & 1 & 5-1 \\
      \end{bmatrix}
      \begin{bmatrix}
        v_{1} \\
        v_{2} \\
        v_{3} \\
      \end{bmatrix}
      =
      \begin{bmatrix}
        2 \\
        0 \\
        -1 \\
      \end{bmatrix}
      \iff
      \begin{matrix*}[l]
        v_{2} = \frac{2}{3}  \\
        v_{1} + 2v_{3} = - \frac{5}{6} \\
      \end{matrix*}
      \iff
      \begin{bmatrix}
        v_{1} \\
        v_{2} \\
        v_{3} \\
      \end{bmatrix}
      =
      \begin{bmatrix}
        - \frac{5}{6} \\
        \frac{2}{3} \\
        0 \\
      \end{bmatrix}
      +
      \alpha
      \begin{bmatrix}
        -2 \\
        0 \\
        1 \\
      \end{bmatrix}
      \quad \forall \alpha
    \end{equation}
  \item So we choose the following $\bm{v}_{1,2}$ as a generalized eigenvector of rank $2$ associated with $\lambda_{1}$:
    \begin{equation}\nonumber%\label{eq:}%
      \bm{v}_{1,2} := 
      \begin{bmatrix}
        -\frac{5}{6} \\
        \frac{2}{3} \\
        0 \\
      \end{bmatrix}
    \end{equation}

  \item Defining
    \begin{equation}\nonumber%\label{eq:}%
      \bm{V} :=
      \begin{bmatrix}
        \bm{v}_{1,1} & \bm{v}_{1,2} & \bm{v}_{2,1}
      \end{bmatrix}
      =
      \begin{bmatrix}
        2 & - 5/6 & 0 \\
        0 & 2/3 & 0 \\
        -1 & 0 & 1 \\
      \end{bmatrix}
      \implies
      \bm{V}^{-1}
      = 
      \begin{bmatrix}
        1/2 & 5/8 & 0 \\
        0 & 3/2 & 0 \\
        1/2 & 5/8 & 1 \\
      \end{bmatrix},
    \end{equation}
    one can verify that
    \begin{equation}\nonumber%\label{eq:}%
      \bm{V}\bm{J}\bm{V}^{-1}
      =
      \begin{bmatrix}
        2 & - 5/6 & 0 \\
        0 & 2/3 & 0 \\
        -1 & 0 & 1 \\
      \end{bmatrix}
      \underbrace{
      \begin{bmatrix}
        1 & 1 & 0 \\
        0 & 1 & 0 \\
        0 & 0 & 5 \\
      \end{bmatrix}}_{=\bm{J}}
      \begin{bmatrix}
        1/2 & 5/8 & 0 \\
        0 & 3/2 & 0 \\
        1/2 & 5/8 & 1 \\
      \end{bmatrix}
      =
      \begin{bmatrix}
        1 & 3 & 0 \\
        0 & 1 & 0 \\
        2 & 1 & 5 \\
      \end{bmatrix}
      = \bm{A}
    \end{equation}

  \end{itemize}

\item \textbf{Example~5}
  \begin{itemize}
  \item Consider a square matrix
    \begin{equation}\nonumber%\label{eq:}%
      \bm{A} :=
      \begin{bmatrix}
        2 & 1 & 1 \\
        1 & 3 & 2 \\
        0 & -1 & 1 \\
      \end{bmatrix}
    \end{equation}
  \item The characteristic polynomial is
    \begin{equation}\nonumber%\label{eq:}%
      \phi_{\bm{A}}(t) = 
      \begin{vmatrix}
        2-t & 1 & 1 \\
        1 & 3-t & 2 \\
        0 & -1 & 1-t \\
      \end{vmatrix}
      = - (t-2)^{3}
    \end{equation}
    which means $\bm{A}$ has only one eigenvalue $\lambda_{1}:=2$ (three times)

  \item Solving $(\bm{A}-\lambda_{1}\bm{I})\bm{v}=\bm{0}$ for $\bm{v}$ yields
    \begin{equation}\nonumber%\label{eq:}%
      \begin{bmatrix}
        2-2 & 1 & 1 \\
        1 & 3-2 & 2 \\
        0 & -1 & 1-2 \\
      \end{bmatrix}
      \begin{bmatrix}
        v_{1} \\
        v_{2} \\
        v_{3} \\
      \end{bmatrix}
      = 
      \begin{bmatrix}
        0 \\
        0 \\
        0 \\
      \end{bmatrix}
      \iff
      \begin{matrix*}[l]
        v_{1} = v_{2} \\
        v_{3} = -v_{2} \\
      \end{matrix*}
      \iff
      \begin{bmatrix}
        v_{1} \\
        v_{2} \\
        v_{3} \\
      \end{bmatrix}
      =
      \alpha
      \begin{bmatrix}
        1 \\
        1 \\
        -1 \\
      \end{bmatrix}
      \quad \forall \alpha
    \end{equation}
    so we choose the following $\bm{v}_{1,1}$ as an eigenvector associated with $\lambda_{1}$:
    \begin{equation}\nonumber%\label{eq:}%
      \bm{v}_{1,1} :=
      \begin{bmatrix}
        1 \\
        1 \\
        -1 \\
      \end{bmatrix}
    \end{equation}
  \item No other linearly independent eigenvector, thus
    we seek to find two more generalized eigenvectors $\bm{v}_{1,2}, \bm{v}_{1,3}$ associated with $\lambda_{1}$ so that
    \begin{equation}\nonumber%\label{eq:}%
      \begin{bmatrix}
        \bm{v}_{1,1} & \bm{v}_{1,2} & \bm{v}_{1,3}
      \end{bmatrix}
      \begin{bmatrix}
        \lambda_{1} & 1 & 0 \\
        0 & \lambda_{1} & 1 \\
        0 & 0 & \lambda_{1} \\
      \end{bmatrix}
      \begin{bmatrix}
        \bm{v}_{1,1} & \bm{v}_{1,2} & \bm{v}_{1,3}
      \end{bmatrix}^{-1}
      = \bm{A}
    \end{equation}
    
  \item Such generalized eigenvectors must satisfy
    \begin{equation}\nonumber%\label{eq:}%
      (\bm{A}-\lambda_{1}\bm{I})\bm{v}_{1,2} = \bm{v}_{1,1},
      \quad\text{and}\quad
      (\bm{A}-\lambda_{1}\bm{I})\bm{v}_{1,3} = \bm{v}_{1,2}
    \end{equation}

  \item First, solving $(\bm{A}-\lambda_{1}\bm{I})\bm{v} = \bm{v}_{1,1}$ for $\bm{v}$ yields
    \begin{equation}\nonumber%\label{eq:}%
      \begin{bmatrix}
        2-2 & 1 & 1 \\
        1 & 3-2 & 2 \\
        0 & -1 & 1-2 \\
      \end{bmatrix}
      \begin{bmatrix}
        v_{1} \\
        v_{2} \\
        v_{3} \\
      \end{bmatrix}
      = 
      \begin{bmatrix}
        1 \\
        1 \\
        -1 \\
      \end{bmatrix}
      \iff
      \begin{matrix*}[l]
        v_{2} + v_{3} = 1 \\
        v_{1} + v_{3} = 0 \\
      \end{matrix*}
      \iff
      \begin{bmatrix}
        v_{1} \\
        v_{2} \\
        v_{3} \\
      \end{bmatrix}
      =
      \begin{bmatrix}
        0 \\
        1 \\
        0 \\
      \end{bmatrix}
      +
      \alpha
      \begin{bmatrix}
        -1 \\
        -1 \\
        1 \\
      \end{bmatrix}
      \quad \forall \alpha
    \end{equation}
    so we choose the following $\bm{v}_{1,2}$ as another generalized eigenvector associated with $\lambda_{1}$:
    \begin{equation}\nonumber%\label{eq:}%
      \bm{v}_{1,2} :=
      \begin{bmatrix}
        0 \\
        1 \\
        0 \\
      \end{bmatrix}
    \end{equation}

  \item Given this $\bm{v}_{1,2}$, we solve $(\bm{A}-\lambda_{1}\bm{I})\bm{v} = \bm{v}_{1,2}$ for $\bm{v}$ to obtain
    \begin{equation}\nonumber%\label{eq:}%
      \begin{bmatrix}
        2-2 & 1 & 1 \\
        1 & 3-2 & 2 \\
        0 & -1 & 1-2 \\
      \end{bmatrix}
      \begin{bmatrix}
        v_{1} \\
        v_{2} \\
        v_{3} \\
      \end{bmatrix}
      = 
      \begin{bmatrix}
        0 \\
        1 \\
        0 \\
      \end{bmatrix}
      \iff
      \begin{matrix*}[l]
        v_{2} + v_{3} = 0 \\
        v_{1} + v_{3} = 1 \\
      \end{matrix*}
      \iff
      \begin{bmatrix}
        v_{1} \\
        v_{2} \\
        v_{3} \\
      \end{bmatrix}
      =
      \begin{bmatrix}
        1 \\
        0 \\
        0 \\
      \end{bmatrix}
      +
      \alpha
      \begin{bmatrix}
        -1 \\
        -1 \\
        1 \\
      \end{bmatrix}
      \quad \forall \alpha
    \end{equation}
    so we choose the following $\bm{v}_{1,3}$ as yet another generalized eigenvector associated with $\lambda_{1}$:
    \begin{equation}\nonumber%\label{eq:}%
      \bm{v}_{1,3} :=
      \begin{bmatrix}
        1 \\
        0 \\
        0 \\
      \end{bmatrix}
    \end{equation}
    
  \item Defining
    \begin{equation}\nonumber%\label{eq:}%
      \bm{V} :=
      \begin{bmatrix}
        \bm{v}_{1,1} & \bm{v}_{1,2} & \bm{v}_{1,3}
      \end{bmatrix}
      =
      \begin{bmatrix}
        1 & 0 & 1 \\
        1 & 1 & 0 \\
        -1 & 0 & 0 \\
      \end{bmatrix}
      \implies
      \bm{V}^{-1}
      = 
      \begin{bmatrix}
        0 & 0 & -1 \\
        0 & 1 & 1 \\
        1 & 0 & 1 \\
      \end{bmatrix},
    \end{equation}
    one can verify that
    \begin{equation}\nonumber%\label{eq:}%
      \bm{V}\bm{J}\bm{V}^{-1}
      =
      \begin{bmatrix}
        1 & 0 & 1 \\
        1 & 1 & 0 \\
        -1 & 0 & 0 \\
      \end{bmatrix}
      \underbrace{
      \begin{bmatrix}
        2 & 1 & 0 \\
        0 & 2 & 1 \\
        0 & 0 & 2 \\
      \end{bmatrix}}_{=\bm{J}}
      \begin{bmatrix}
        0 & 0 & -1 \\
        0 & 1 & 1 \\
        1 & 0 & 1 \\
      \end{bmatrix}
      =
      \begin{bmatrix}
        2 & 1 & 1 \\
        1 & 3 & 2 \\
        0 & -1 & 1 \\
      \end{bmatrix}
      = \bm{A}
    \end{equation}

  \end{itemize}

\item \textbf{Example~1}

  \begin{itemize}
  \item Let us revisit Example~1 and consider a square matrix
    \begin{equation}\nonumber%\label{eq:}%
      \bm{A} :=
      \begin{bmatrix}
        3 & -1 & 1 \\
        2 & 0 & 2 \\
        1 & -1 & 3 \\
      \end{bmatrix},
    \end{equation}
    for which we already know that
    $\lambda:=1$ is the unique eigenvalue
  \item Solving $(\bm{A}-\lambda \bm{I})\bm{v} = \bm{0}$ for $\bm{v}$ yields
    \begin{equation}\nonumber%\label{eq:}%
      (\bm{A}-\lambda \bm{I})\bm{v} = \bm{0}
      \iff
      \begin{bmatrix}
        3-2 & -1 & 1 \\
        2 & 0-2 & 2 \\
        1 & -1 & 3-2 \\
      \end{bmatrix}
      \begin{bmatrix}
        v_{1} \\
        v_{2} \\
        v_{3} \\
      \end{bmatrix}
      =
      \begin{bmatrix}
        0 \\
        0 \\
        0 \\
      \end{bmatrix}
      \iff
      v_{2} = v_{1} + v_{3},
    \end{equation}
    implying that any non-zero vector of the form
    \begin{equation}\nonumber%\label{eq:}%
      \bm{v}
      =
      \begin{bmatrix}
        \alpha \\ \alpha + \beta \\ \beta
      \end{bmatrix}
      =
      \alpha
      \begin{bmatrix}
        1 \\ 1 \\ 0
      \end{bmatrix}
      +
      \beta
      \begin{bmatrix}
        0 \\ 1 \\ 1
      \end{bmatrix}
      \quad \forall \alpha, \beta 
    \end{equation}
    is an eigenvector associated with $\lambda$
  \item Observe that
    \begin{itemize}
    \item one can choose two linearly independent eigenvectors associated with $\lambda$, such as
      \begin{equation}\nonumber%\label{eq:}%
        \bm{v}\big|_{(\alpha,\beta)=(1,0)}
        =
        \begin{bmatrix}
          1 \\ 1 \\ 0
        \end{bmatrix}
        \quad\text{and}\quad
        \bm{v}\big|_{(\alpha,\beta)=(0,1)}
        =
        \begin{bmatrix}
          0 \\ 1 \\ 1
        \end{bmatrix}
      \end{equation}
    \item there are many other ways of choosing two linearly independent eigenvectors, like
      \begin{equation}\nonumber%\label{eq:}%
        \bm{v}\big|_{(\alpha,\beta)=(1,1)}
        =
        \begin{bmatrix}
          1 \\ 2 \\ 1
        \end{bmatrix}
        \quad\text{and}\quad
        \bm{v}\big|_{(\alpha,\beta)=(1,-1)}
        =
        \begin{bmatrix}
          1 \\ 0 \\ -1
        \end{bmatrix}
      \end{equation}
    \end{itemize}
  \item When the eigenspace has multiple degrees of freedom (here we have two free parameters, $\alpha$ and $\beta$),
    you need to carefully choose eigenvectors; otherwise you would not be able to find generalized eigenvectors of higher ranks
    \begin{itemize}
    \item To demonstrate the point, let us say we choose
      \begin{equation}\label{eq:eigenvector_fail}%
        \bm{v}_{1,1}
        =
        \begin{bmatrix}
          1 \\ 1 \\ 0
        \end{bmatrix}
        \quad\text{and}\quad
        \bm{v}_{2,1}
        =
        \begin{bmatrix}
          0 \\ 1 \\ 1
        \end{bmatrix}
      \end{equation}
      as two linearly independent eigenvectors associated with $\lambda=2$
    \item Then, there does not exist $\bm{v}_{1,2}$ such that $(\bm{A}-\lambda \bm{I})\bm{v}_{1,2}=\bm{v}_{1,1}$ because
      \begin{equation}\nonumber%\label{eq:}%
        (\bm{A}-\lambda \bm{I})\bm{v} = \bm{v}_{1,1}
        \iff
      \begin{bmatrix}
        1 & -1 & 1 \\
        2 & -2 & 2 \\
        1 & -1 & 1 \\
      \end{bmatrix}
      \begin{bmatrix}
        v_{1} \\
        v_{2} \\
        v_{3} \\
      \end{bmatrix}
      =
      \begin{bmatrix}
        1 \\
        1 \\
        0 \\
      \end{bmatrix}
      \iff
      \begin{matrix*}[l]
      v_{1} - v_{2} + v_{3} = 1 \\
      v_{1} - v_{2} + v_{3} = 1/2 \\
      v_{1} - v_{2} + v_{3} = 0 \\
      \end{matrix*}
      \end{equation}

    \item There is no $\bm{v}_{2,2}$ to satisfy $(\bm{A}-\lambda \bm{I})\bm{v}_{2,2}=\bm{v}_{2,1}$, either, because
      \begin{equation}\nonumber%\label{eq:}%
        (\bm{A}-\lambda \bm{I})\bm{v} = \bm{v}_{2,1}
        \iff
      \begin{bmatrix}
        1 & -1 & 1 \\
        2 & -2 & 2 \\
        1 & -1 & 1 \\
      \end{bmatrix}
      \begin{bmatrix}
        v_{1} \\
        v_{2} \\
        v_{3} \\
      \end{bmatrix}
      =
      \begin{bmatrix}
        0 \\
        1 \\
        1 \\
      \end{bmatrix}
      \iff
      \begin{matrix*}[l]
      v_{1} - v_{2} + v_{3} = 0 \\
      v_{1} - v_{2} + v_{3} = 1/2 \\
      v_{1} - v_{2} + v_{3} = 1 \\
      \end{matrix*}
      \end{equation}

    \item Hence, if we use \eqref{eq:eigenvector_fail} as eigenvectors,
      we cannot find a generalized eigenvector that allows us to complete the Jordan decomposition
    \end{itemize}

  \item Here is a procedure you can use when the eigenspace of an eigenvalue $\lambda$ has multiple degrees of freedom:
    \begin{enumerate}
    \item First, find a generalized eigenvector of highest rank (rank~$2$, in this example) associated with $\lambda$,
      namely, find $\bm{v}_{1,2}$ such that
      \begin{equation}\label{eq:cond1}%
        (\bm{A}-\lambda \bm{I})^{2}\bm{v}_{1,2} = \bm{0}
        \quad\text{and}\quad
        (\bm{A}-\lambda \bm{I})\bm{v}_{1,2} \neq \bm{0}
      \end{equation}
    \item Then find a generalized eigenvector of lower rank (rank~$1$, in this example)
      by defining $\bm{v}_{1,1}$ as
      \begin{equation}\label{eq:cond2}%
        \bm{v}_{1,1} := (\bm{A}-\lambda \bm{I})\bm{v}_{1,2},
      \end{equation}
      which is, by construction, an eigenvector associated with $\lambda$
    \item Finally, find another eigenvector $\bm{v}_{2,1}$ that is linearly independent of $\bm{v}_{1,1}$
    \end{enumerate}

  \item Note that in this procedure, we find (generalized) eivenvectors in the reverse order
    \begin{itemize}
    \item previously, we first find an eigenvector $\bm{v}_{i,1}$ and then compute a generalized eigenvector $\bm{v}_{i,2}$
      by solving $(\bm{A}-\lambda \bm{I})\bm{v}_{i,2} = \bm{v}_{i,1}$ for $\bm{v}_{i,2}$
    \item here, we first find a generalized eigenvector $\bm{v}_{i,2}$ and then compute an eigenvector $\bm{v}_{i,1}$
      by solving $(\bm{A}-\lambda \bm{I})\bm{v}_{i,2} = \bm{v}_{i,1}$ for $\bm{v}_{i,1}$
    \end{itemize}

  \item For Example~1,
    first find $\bm{v}_{1,2}$ that satisfies \eqref{eq:cond1}:
    \begin{itemize}
    \item we have
      \begin{equation}\nonumber%\label{eq:}%
        (\bm{A}-\lambda \bm{I})^{2} =
        \begin{bmatrix}
          3-2 & -1 & 1 \\
          2 & 0-2 & 2 \\
          1 & -1 & 3-2 \\
        \end{bmatrix}^{2}
        =
        \begin{bmatrix}
          0 & 0 & 0 \\
          0 & 0 & 0 \\
          0 & 0 & 0 \\
        \end{bmatrix},
      \end{equation}
      meaning that
      $(\bm{A}-\lambda \bm{I})^{2}\bm{v}=\bm{0}$ for any vector $\bm{v}$
    \item thus, $\bm{v}$ is a generalized eigenvector of rank 2 if and only if
      it is not an eigenvector itself, i.e.,
      \begin{equation}\nonumber%\label{eq:}%
        (\bm{A}-\lambda \bm{I})\bm{v}\neq \bm{0}
        \iff
        v_{1} - v_{2} + v_{3} \neq 0
      \end{equation}
    \item for example, we can choose the following
      \begin{equation}\nonumber%\label{eq:}%
        \bm{v}_{1,2} :=
        \begin{bmatrix}
          1 \\ 0 \\ 0
        \end{bmatrix}
      \end{equation}
    \end{itemize}
  \item Based on this $\bm{v}_{1,2}$, we choose an eigenvector by \eqref{eq:cond2}:
    \begin{equation}\nonumber%\label{eq:}%
      \bm{v}_{1,1}
      := (\bm{A}-\lambda\bm{I})\bm{v}_{1,2}
      = 
        \begin{bmatrix}
          3-2 & -1 & 1 \\
          2 & 0-2 & 2 \\
          1 & -1 & 3-2 \\
        \end{bmatrix}
        \begin{bmatrix}
          1 \\ 0 \\ 0
        \end{bmatrix}
        =
        \begin{bmatrix}
          1 \\ 2 \\ 1
        \end{bmatrix}
      \end{equation}
    \item Finally, choose another eigenvector $\bm{v}_{2,1}$
      that is linearly independent of $\bm{v}_{1,1}$
      \begin{itemize}
      \item recall that the eigenvector associated with $\lambda=2$ must be of the form
    \begin{equation}\nonumber%\label{eq:}%
      \bm{v}
      =
      \begin{bmatrix}
        \alpha \\ \alpha + \beta \\ \beta
      \end{bmatrix}
      =
      \alpha
      \begin{bmatrix}
        1 \\ 1 \\ 0
      \end{bmatrix}
      +
      \beta
      \begin{bmatrix}
        0 \\ 1 \\ 1
      \end{bmatrix}
      \quad \forall \alpha, \beta 
    \end{equation}
  \item $\bm{v}_{1,1}$ above is the one where $(\alpha,\beta)=(1,1)$
  \item any $(\alpha,\beta)$ with $\alpha \neq \beta$ would give us an eigenvector that is linearly independent of $\bm{v}_{1,1}$
  \item for example, setting $(\alpha,\beta) =(1, 0)$ yields
    \begin{equation}\nonumber%\label{eq:}%
      \bm{v}_{2,1} : =
      \begin{bmatrix}
        1 \\
        1 \\
        0 \\
      \end{bmatrix}
    \end{equation}
  \end{itemize}
    
\item Then $\bm{A}$ should be decomposed as:
  \begin{equation}\nonumber%\label{eq:}%
    \bm{A}
    =
      \begin{bmatrix}
        \bm{v}_{1,1} & \bm{v}_{1,2} & \bm{v}_{2,1}
      \end{bmatrix}
      \underbrace{
      \begin{bmatrix}
        \lambda_{1} & 1 & 0 \\
        0 & \lambda_{1} & 0 \\
        0 & 0 & \lambda_{2} \\
      \end{bmatrix}}_{=:\bm{J}}
      \begin{bmatrix}
        \bm{v}_{1,1} & \bm{v}_{1,2} & \bm{v}_{2,1}
      \end{bmatrix}^{-1}
    \end{equation}
    where $\lambda_{1}=\lambda_{2}=\lambda = 2$

  \item In fact, 
    \begin{equation}\nonumber%\label{eq:}%
      \bm{V} :=
      \begin{bmatrix}
        \bm{v}_{1,1} & \bm{v}_{1,2} & \bm{v}_{2,1}
      \end{bmatrix}
      = 
      \begin{bmatrix}
        1 & 1 & 1 \\
        2 & 0 & 1 \\
        1 & 0 & 0 \\
      \end{bmatrix}
      \implies
      \bm{V}^{-1}
      = 
      \begin{bmatrix}
        0 & 0 & 1 \\
        1 & -1 & 1 \\
        0 & 1 & -2 \\
      \end{bmatrix}
    \end{equation}
    and
    \begin{equation}\nonumber%\label{eq:}%
      \bm{V}\bm{J}\bm{V}^{-1}
      = 
      \begin{bmatrix}
        1 & 1 & 1 \\
        2 & 0 & 1 \\
        1 & 0 & 0 \\
      \end{bmatrix}
      \begin{bmatrix}
        2 & 1 & 0 \\
        0 & 2 & 0 \\
        0 & 0 & 2 \\
      \end{bmatrix}
      \begin{bmatrix}
        0 & 0 & 1 \\
        1 & -1 & 1 \\
        0 & 1 & -2 \\
      \end{bmatrix}
      =
      \begin{bmatrix}
        3 & -1 & 1 \\
        2 & 0 & 2 \\
        1 & -1 & 3 \\
      \end{bmatrix}
      = \bm{A}
    \end{equation}

  \end{itemize}

\end{itemize}

\end{document}
