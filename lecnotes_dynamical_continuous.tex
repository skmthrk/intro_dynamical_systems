\documentclass[12pt,a4paper]{article}
\usepackage[T1]{fontenc}
\usepackage{textcomp}
\usepackage{makeidx}

\usepackage[sc, noBBpl]{mathpazo}
\usepackage{dsfont} % for indicator function
\usepackage[deluxe]{otf}
\usepackage[utf8]{inputenc}

\usepackage{amsmath}
\usepackage{amsfonts}
\usepackage{amssymb}
\usepackage{amsthm}
\usepackage{mathtools}
\usepackage{mathrsfs}

\usepackage{listings} % for programming codes

\usepackage{bm}
\usepackage[dvipdfmx]{color}
\pagestyle{plain} 
\usepackage{float}
\usepackage[dvipdfmx]{graphicx}

\usepackage[left=15mm, right=18mm, top=20mm, bottom=20mm]{geometry}

\setlength{\footskip}{40pt} 
\setlength{\abovecaptionskip}{0pt}
\setlength{\belowcaptionskip}{0pt}
\renewcommand{\baselinestretch}{1.05}

\newcommand{\argmax}{\mathop{\rm arg max}\limits}
\newcommand{\argmin}{\mathop{\rm arg min}\limits}
\newcommand{\st}{\mathop{\rm s.t.}\limits}
\newcommand{\fsize}[1]{\fontsize{#1}{#1}\selectfont}

% custom color
\usepackage[dvipdfmx]{xcolor}

% colors
\definecolor{darkred}{rgb}{.40,.00,.00}
\definecolor{navy}{rgb}{.00,.00,.40}
\definecolor{darknavy}{rgb}{.00,.00,.30}
\definecolor{lightnavy}{rgb}{.10,.10,.40}

\newcommand{\navy}{\color{navy}}
\newcommand{\deepred}{\color{darkred}}
\newcommand{\emred}[1]{\textbf{\deepred #1}}
\newcommand{\emblue}[1]{\textbf{\navy #1}}

% title and tableofcontents
\usepackage[titles]{tocloft}
\setlength{\cftbeforesecskip}{2pt}
\setlength{\cftbeforesubsecskip}{0pt}
\renewcommand{\cftsecfont}{\normalsize\mcfamily}
\renewcommand{\cftsubsecfont}{\normalsize\mcfamily}
\renewcommand{\cftsubsecfont}{\normalsize\mcfamily}
\renewcommand{\contentsname}{\large\mcfamily\bfseries \S\hspace{0.5\baselineskip}Contents\\[-\baselineskip]}
\cftsetindents{section}{10.0pt}{15.0pt}
\cftsetindents{subsection}{20.0pt}{30.0pt}
\cftpagenumbersoff{section}
\cftpagenumbersoff{subsection}
\cftpagenumbersoff{subsubsection}

\makeatletter
\def\vhrulefill#1{\leavevmode\leaders\hrule\@height#1\hfill \kern\z@}
\makeatother

\makeatletter
\def\maketitle{%
	\begin{center}\leavevmode
	\normalfont
	{\normalsize\raggedleft \@date\par}%
	\vskip -0.7\baselineskip
	\vhrulefill{0.2pt}\par
	\vskip 0.3\baselineskip
	\renewcommand{\baselinestretch}{0.80}
	{\color{darknavy}\bfseries\Large\raggedright \@title\par}%
	\renewcommand{\baselinestretch}{1.00}
	\vskip 0.2\baselineskip
	\ifx\@subtitle\@empty\else%
	{\normalsize\raggedright \@subtitle\par}%
	\fi
	\ifx\@author\@empty\else%
	\vskip 0.5\baselineskip
	{\normalsize\raggedleft \@author\par}%
	\fi
	\end{center}%
	\vskip 1.0\baselineskip
}
\def\subtitle#1{\gdef\@subtitle{#1}}
\let\@subtitle\@empty
\makeatother

% section
\newcommand{\Secindent}{0.0\baselineskip}
\newcommand{\Subsecindent}{0.0\baselineskip}
\newcommand{\lmargini}{1.3\baselineskip}
\newcommand{\lmarginii}{0.5\baselineskip}
\newcommand{\lmarginiii}{0.2\baselineskip}
\def\theenumi{\arabic{enumi}}
\def\theenumii{\alph{enumii}}
\renewcommand{\labelenumi}{\hfill\hspace{60pt}\theenumi.}
\renewcommand{\labelenumii}{\hspace{30pt}\theenumii.}
\renewcommand{\labelitemi}{\small\textbullet\hspace{2pt}}
\renewcommand{\labelitemii}{\color{gray}\small$\circ$\hspace{2pt}}
\renewcommand{\labelitemiii}{\textendash\hspace{1pt}}
\makeatletter
\renewcommand{\section}{\@startsection%
	{section}{1}{\Secindent}% name, depth, indent \z@=0pt
	{-1.0\baselineskip \@plus1.5\baselineskip \@minus0.5\baselineskip}% beforeskip (no indent if negative)
	{0.5\baselineskip \@plus0.8\baselineskip \@minus0.3\baselineskip}% afterskip (no line break if negative)
	{\reset@font\large\bfseries\scshape\color{darknavy}}% font
}
\renewcommand{\subsection}{\@startsection%
	{subsection}{2}{\Subsecindent}%
	{-0.5\baselineskip \@plus0.7\baselineskip \@minus0.4\baselineskip}%
	{0.3\baselineskip \@plus0.5\baselineskip \@minus0.2\baselineskip}%
	{\reset@font\normalsize\bfseries\scshape\color{darknavy}}% 
}
\renewcommand{\thesection}{\@arabic\c@section} % arabic, roman, Roman, alph, Alph, fnsymbol
\renewcommand{\thesubsection}{\thesection\hspace{0pt}.\hspace{0pt}\@arabic\c@subsection}
\makeatother

\makeatletter
\def\@listi{%
	\topsep = 0.2\baselineskip
	\partopsep = 0.0\baselineskip
	\parsep = 0.0\baselineskip
	\itemsep = 0.5\baselineskip
	\leftmargin = 1.5zw
	\labelsep = 0.5zw
	\itemindent = 0pt
	\labelwidth\leftmargin \advance\labelwidth-\labelsep
	\listparindent = 0zw
	\rightmargin = 0pt
	\reset@font\normalsize\normalfont
}
\let\@listI\@listi
\@listi
\def\@listii{%
	\topsep = 0.25\baselineskip
	\partopsep = 0.0\baselineskip
	\parsep = 0.0\baselineskip
	\itemsep = 0.3\baselineskip
	\leftmargin = 0.5zw
	\labelsep = .5zw
	\itemindent = 0pt
	\labelwidth\leftmargin \advance\labelwidth-\labelsep
	\listparindent = 0zw
	\rightmargin = 0pt
	\reset@font\normalsize\normalfont%
}
\def\@listiii{%
	\topsep = 0.30\baselineskip
	\partopsep = 0.0\baselineskip
	\parsep = 0.1\baselineskip
	\itemsep = 0.05\baselineskip
	\leftmargin = 1.1zw
	\labelsep = .5zw
	\itemindent = 0pt
	\labelwidth\leftmargin \advance\labelwidth-\labelsep
	\listparindent = 0zw
	\rightmargin = 0pt
	\reset@font\normalsize\normalfont%
}
\def\@listiv{%
	\topsep = 0.0\baselineskip
	\partopsep = 0.0\baselineskip
	\parsep = 0.0\baselineskip
	\itemsep = 0.0\baselineskip
	\leftmargin = 0.9zw
	\labelsep = .5zw
	\itemindent = 0pt
	\labelwidth\leftmargin \advance\labelwidth-\labelsep
	\listparindent = 0zw
	\rightmargin = 0pt
	\reset@font\normalsize\normalfont%
}
\makeatother

\newcommand{\R}{\ensuremath{\mathbb{R}}}
\newcommand{\E}{\ensuremath{\mathbb{E}}}
\newcommand{\V}{\ensuremath{\mathbb{V}}}
\newcommand{\Cov}{\ensuremath{\text{Cov}}}
\newcommand{\N}{\ensuremath{\mathbb{N}}}
\newcommand{\Z}{\ensuremath{\mathbb{Z}}}
\let\epsilon\varepsilon

\let\origitem\item
\renewcommand{\item}{\normalfont\origitem}

\usepackage[%
backref=page,
dvipdfmx,
bookmarks=true,bookmarksnumbered=true,bookmarkstype=toc,
colorlinks=true,
linkcolor=navy,
citecolor=navy,
filecolor=blue,
pagecolor=blue,
urlcolor=blue
]{hyperref}
\AtBeginDvi{\special{pdf:tounicode EUC-UCS2}}

\renewcommand*{\backref}[1]{}
\renewcommand*{\backrefalt}[4]{%
  \ifcase #1 (Not cited.)%
  \or        [#2]%
  \else      [#2]%
  \fi}
 

\title{Continuous-time models}
\subtitle{Introduction to dynamical systems~\#7}
\author{Hiroaki Sakamoto}
\date{\today}

\begin{document}
\maketitle
\tableofcontents

\section{Continuous-time linear dynamical system}

\subsection{Definitions}

\begin{itemize}

\item \textbf{Linear dynamical system}
  \begin{itemize}
  \item A continuous-time \emph{linear dynamical system} is a system of differential equations of the form
    \begin{equation}\nonumber%\label{eq:}%
      \frac{d}{dt}\bm{x}(t) = \bm{A}\bm{x}(t) + \bm{B}\bm{u}(t)
      \quad t \in \R_{+},
    \end{equation}
    where
    \begin{itemize}
    \item $\bm{x}(t)\in \R^{m}$: state vector at $t$
    \item $\bm{A}\in \R^{m\times m}$: system matrix
    \item $\bm{u}(t)\in \R^{n}$: control (input) vector at $t$
    \item $\bm{B} \in \R^{m\times n}$: diffusion matrix
    \end{itemize}
  \item Starting with some \emph{initial state} $\bm{x}(0)$, we want to know how $\bm{x}(t)$ evolves over time depending on $\bm{A}$
  \item In particular, a linear dynamical system is said to be \emph{homogeneous} if $\bm{B}=\bm{O}$, i.e.,
    \begin{equation}\nonumber%\label{eq:}%
      \frac{d}{dt}\bm{x}(t) = \bm{A}\bm{x}(t)
      \quad t \in \R_{+},
    \end{equation}
    where the behavior of $\bm{x}(t)$ is completely characterized by $\bm{A}$ and $\bm{x}(0)$
  \end{itemize}

\item \textbf{Equilibrium}
  \begin{itemize}
  \item Consider the case where the control is constant at $\bm{u}$:
    \begin{equation}\label{eq:LDS}%
      \frac{d}{dt}\bm{x}(t) = \bm{A}\bm{x}(t) + \underbrace{\bm{B}\bm{u}}_{\bm{b}},
      \quad t\in \R_{+},
    \end{equation}
  \item The homogeneous system is a special case of this with $\bm{u}=\bm{0}$
  \item We define an \emph{equilibrium point} of \eqref{eq:LDS} as $\bar{\bm{x}}$ that solves
    \begin{equation}\nonumber%\label{eq:}%
      \bm{0} = \bm{A}\bar{\bm{x}} + \bm{b},
    \end{equation}
    which obviously depends on both $\bm{A}$ and $\bm{b}$
  \end{itemize}

\item \textbf{Stability}
  \begin{itemize}

  \item An equilibrium point $\bar{\bm{x}}$ of \eqref{eq:LDS}
    is said to be \emph{asymptotically stable} if,
    starting from \textbf{any initial state} $\bm{x}(0)$,
    the state trajectory satisfies
    \begin{equation}\nonumber%\label{eq:}%
      \lim_{t\to\infty}\bm{x}(t) = \bar{\bm{x}}
    \end{equation}
  \item Notice that, for any initial state $\bm{x}(0)$ and for any constant input $\bm{b}$,
    the state trajectory $\{\bm{x}(t)\}$ satisfies
    \begin{equation}\nonumber%\label{eq:}%
      \frac{d}{dt}(\bm{x}(t) - \bar{\bm{x}}) = \bm{A}(\bm{x}(t) - \bar{\bm{x}}) \quad \forall t\in \R_{+},
    \end{equation}
    which implies that the stability of an equilibrium point is determined by $\bm{A}$ alone
  \item More generally,
    the state trajectory $\{\bm{x}(t)\}$ may or may not converge to the equilibrium point,
    depending on the initial state $\bm{x}(0)$
    
  \item We define the \emph{stable manifold} of an equilibrium point $\bar{\bm{x}}\in \R^{m}$
    as the set of initial state from which the state trajectory converges to the equilibrium point:
    \begin{equation}\nonumber%\label{eq:}%
      W(\bar{\bm{x}}) := \left\{ \bm{x}(0)\in \R^{m} \,\bigg|\, \lim_{t\to \infty}\bm{x}(t) = \bar{\bm{x}} \right\}
    \end{equation}
  \item An equilibrium point $\bar{\bm{x}}$ is asymptotically stable
    if and only if $W(\bar{\bm{x}}) = \R^{m}$
  \end{itemize}

\end{itemize}

\subsection{Examples}

\begin{itemize}

\item \textbf{Example~1}
  \begin{itemize}
  \item Consider the following homogeneous one-dimensional linear dynamical system:
    \begin{equation}\nonumber%\label{eq:}%
      \dot{x}(t) = a x(t) \quad\text{where}\quad \dot{x}(t):= \frac{d}{dt}x(t)
    \end{equation}
  \item Observe:
    \begin{itemize}
    \item the equilibrium point of the system is $\bar{x}=0$ unless $a=0$
    \item the state trajectory is given by $x(t) = e^{a t} x(0)$ because
      \begin{equation}\nonumber%\label{eq:}%
        \dot{x}(t) = a x(t)
        \implies
        \frac{d}{dt} \ln(x(t)) = a
        \implies
        \ln(x(t)/x(0)) = \int_{0}^{t}ad\tau = at
      \end{equation}
    \item $\bar{x}=0$ is asymptotically stable if and only if $a<0$
    \end{itemize}
  \end{itemize}

\item \textbf{Example~2}
  \begin{itemize}
  \item Consider the following one-dimensional linear dynamical system:
    \begin{equation}\nonumber%\label{eq:}%
      \dot{x}(t) = a x(t) + b, \quad a \neq 0
    \end{equation}
  \item Observe:
    \begin{itemize}
    \item the equilibrium point of the system is $\bar{x}= -\frac{b}{a}$
    \item the state trajectory is given by
      \begin{equation}\nonumber%\label{eq:}%
       x(t)= \bar{x} + e^{a t}(x(0)-\bar{x})  = e^{at}x(0) - \left(1 - a^{at}\right)\frac{b}{a}
     \end{equation}
     because
      \begin{equation}\nonumber%\label{eq:}%
        \dot{x}(t) = a x(t) + b
        \implies
        \frac{d}{dt}\left(x(t)-\bar{x}\right) = a (x(t)-\bar{x})
        \implies
        x(t)-\bar{x} = e^{a t}(x(0)-\bar{x})
      \end{equation}
    \item the equilibrium point is asymptotically stable if and only if $a<0$
    \end{itemize}
  \end{itemize}

  \clearpage

\item \textbf{Example~3}
  \begin{itemize}
  \item Consider the following two-dimensional linear dynamical system:
    \begin{equation}\nonumber%\label{eq:}%
      \begin{bmatrix}
        \dot{x}_{1}(t) \\
        \dot{x}_{2}(t)
      \end{bmatrix}
      =
      \begin{bmatrix}
        a_{1} & 0 \\
        0 & a_{2} \\
      \end{bmatrix}
      \begin{bmatrix}
        x_{1}(t) \\
        x_{2}(t)
      \end{bmatrix}
    \end{equation}
  \item Notice that, in this case, the evolution of
    $x_{1}(t)$ and $x_{2}(t)$ are independent:
    \begin{equation}\nonumber%\label{eq:}%
      \dot{x}_{i}(t) = a_{i}x_{i}(t) \quad i = 1, 2
    \end{equation}
  \item Observe:
    \begin{itemize}
    \item if $a_{1} \neq 0$ and $a_{2} \neq 0$,
      the unique equilibrium point of the system is
      \begin{equation}\nonumber%\label{eq:}%
        \bar{\bm{x}} =
      \begin{bmatrix}
        \bar{x}_{1} \\
        \bar{x}_{2}
      \end{bmatrix}
      = 
      \begin{bmatrix}
        0 \\
        0 \\
      \end{bmatrix}
      \end{equation}
    \item the state trajectory is given by
      \begin{equation}\nonumber%\label{eq:}%
      \begin{bmatrix}
        x_{1}(t) \\
        x_{2}(t)
      \end{bmatrix}
      = 
      \begin{bmatrix}
        e^{a_{1}t}x_{1}(0) \\
        e^{a_{2}t}x_{2}(0) \\
      \end{bmatrix}
      =
      \begin{bmatrix}
        e^{a_{1}t} & 0 \\
        0 & e^{a_{2}t} \\
      \end{bmatrix}
      \begin{bmatrix}
        x_{1}(0) \\
        x_{2}(0) \\
      \end{bmatrix}
      \quad\text{or}\quad
      \bm{x}(t) = e^{\bm{A}t}\bm{x}(0)
      \end{equation}
    \item the equilibrium point is asymptotically stable if and only if $\max\{a_{1}, a_{2}\}<0$
    \end{itemize}
  \end{itemize}

\item \textbf{Example~4}
  \begin{itemize}
  \item Consider the following $m$-dimensional linear dynamical system:
    \begin{equation}\nonumber%\label{eq:}%
      \dot{\bm{x}}(t)
      =
      \bm{A}
      \bm{x}(t),
      \quad
      \bm{A}:=
      \begin{bmatrix}
        a_{1} & 0 & \ldots & 0 \\
        0 & a_{2} & \ldots & 0 \\
        \vdots & \vdots & \ddots & \vdots \\
        0 & 0 & \ldots & a_{m} \\
      \end{bmatrix},
      \quad a_{i} \neq 0 \,\, \forall i = 1, \ldots, m
    \end{equation}

  \item Observe:
    \begin{itemize}
    \item the unique equilibrium point of the system is
      \begin{equation}\nonumber%\label{eq:}%
        \bar{\bm{x}}
        =
        \bm{0}
        =
      \begin{bmatrix}
        0 \\
        \vdots \\
        0 \\
      \end{bmatrix}
      \end{equation}
    \item the state trajectory is given by
      \begin{equation}\nonumber%\label{eq:}%
        \bm{x}(t)
        =
      \begin{bmatrix}
        e^{a_{1}t}x_{1}(0) \\
        e^{a_{2}t}x_{2}(0) \\
        \vdots \\
        e^{a_{m}t}x_{m}(0) \\
      \end{bmatrix}
      =
      \begin{bmatrix}
        e^{a_{1}t} & 0 & \ldots & 0 \\
        0 & e^{a_{2}t} & \ldots & 0 \\
        \vdots & \vdots & \ddots & \vdots \\
        0 & 0 & \ldots & e^{a_{m}t} \\
      \end{bmatrix}
      \begin{bmatrix}
        x_{1}(0) \\
        x_{2}(0) \\
        \vdots \\
        x_{m}(0) \\
      \end{bmatrix}
      \quad\text{or}\quad
      \bm{x}(t) = e^{\bm{A}t}\bm{x}(0)
      \end{equation}
    \item the equilibrium point is asymptotically stable if and only if
      \begin{equation}\nonumber%\label{eq:}%
        \max\{a_{1}, a_{2}, \ldots, a_{m}\}<0
      \end{equation}
    \end{itemize}

  \item The stable manifold is
    \begin{equation}\nonumber%\label{eq:}%
      W(\bar{\bm{x}}) =
      \left\{ \bm{x}(0) \in \R^{m}
        \,\big|\,
        \text{$x_{i}(0) = 0$ for $i$ such that $a_{i}\geq 0$}
      \right\}
    \end{equation}
  \end{itemize}

\item \textbf{Example~5}
  \begin{itemize}
  \item Consider the following $m$-dimensional linear dynamical system:
    \begin{equation}\nonumber%\label{eq:}%
      \dot{\bm{x}}(t)
      =
      \bm{A}
      \bm{x}(t) + \bm{b},
      \quad
      \bm{A}:=
      \begin{bmatrix}
        a_{1} & 0 & \ldots & 0 \\
        0 & a_{2} & \ldots & 0 \\
        \vdots & \vdots & \ddots & \vdots \\
        0 & 0 & \ldots & a_{m} \\
      \end{bmatrix}
    \end{equation}

  \item Assume $a_{i}\neq 0$ for all $i=1,\ldots, m$
    (otherwise no stable equilibrium point)

  \item Observe:
    \begin{itemize}
    \item the unique equilibrium point of the system is
      \begin{equation}\nonumber%\label{eq:}%
        \bar{\bm{x}}
        =
        -\bm{A}^{-1}\bm{b}
        =
      \begin{bmatrix}
        -\frac{b_{1}}{a_{1}} \\
        \vdots \\
        -\frac{b_{m}}{a_{m}} \\
      \end{bmatrix}
      \end{equation}
    \item the state trajectory is given by
      \begin{equation}\nonumber%\label{eq:}%
        \begin{bmatrix}
          x_{1}(t) + \frac{b_{1}}{a_{1}} \\
          x_{2}(t) + \frac{b_{1}}{a_{2}} \\
          \vdots \\
          x_{m}(t) + \frac{b_{1}}{a_{m}}\\
        \end{bmatrix}
        =
        \begin{bmatrix}
          e^{a_{1}t} \left( x_{1}(0) + \frac{b_{1}}{a_{1}}\right) \\
          e^{a_{2}t} \left( x_{2}(0) + \frac{b_{2}}{a_{m}}\right) \\
          \vdots \\
          e^{a_{m}t} \left( x_{m}(0) + \frac{b_{m}}{a_{m}}\right) \\
        \end{bmatrix}
        = 
        \begin{bmatrix}
          e^{a_{1}t} & 0 & \ldots & 0 \\
          0 & e^{a_{2}t} & \ldots & 0 \\
          \vdots & \vdots & \ddots & \vdots \\
          0 & 0 & \ldots & e^{a_{m}t} \\
        \end{bmatrix}
        \begin{bmatrix}
          x_{1}(0) + \frac{b_{1}}{a_{1}} \\
          x_{2}(0) + \frac{b_{1}}{a_{2}} \\
          \vdots \\
          x_{m}(0) + \frac{b_{1}}{a_{m}}\\
        \end{bmatrix}
      \end{equation}
      or
      \begin{equation}\nonumber%\label{eq:}%
        \bm{x}(t) - \bar{\bm{x}}
        =  e^{\bm{A}t}(\bm{x}(0)-\bar{\bm{x}})
        \quad\text{or}\quad
        \bm{x}(t)
        =\bar{\bm{x}} + e^{\bm{A}t}(\bm{x}(0)-\bar{\bm{x}})
      \end{equation}
    \item the equilibrium point is asymptotically stable if and only if
      \begin{equation}\nonumber%\label{eq:}%
        \max\{a_{1}, a_{2}, \ldots, a_{m}\}<0
      \end{equation}
    \end{itemize}

  \item The stable manifold is
    \begin{equation}\nonumber%\label{eq:}%
      W(\bar{\bm{x}}) =
      \left\{ \bm{x}(0) \in \R^{m}
        \,\bigg|\,
        \text{$x_{i}(0) = -\frac{b_{i}}{a_{i}}$ for $i$ such that $a_{i}\geq 0$}
      \right\}
    \end{equation}
  \end{itemize}

\item \textbf{Example~6}

  \begin{itemize}
  \item Consider the following two-dimensional linear dynamical system
    \begin{equation}\label{eq:2x2}%
      \begin{bmatrix}
        \dot{x}_{1}(t)\\
        \dot{x}_{2}(t)\\
      \end{bmatrix}
      =
      \underbrace{
      \begin{bmatrix}
        a_{11} & a_{12}\\
        a_{21} & a_{22}\\
      \end{bmatrix}}_{\bm{A}}
      \begin{bmatrix}
        x_{1}(t)\\
        x_{2}(t)\\
      \end{bmatrix}
    \end{equation}
  \item Suppose $\bm{A}$ is diagonalizable
    \begin{equation}\nonumber%\label{eq:}%
      \bm{A} =
      \underbrace{
      \begin{bmatrix}
        \bm{v}_{1} & \bm{v}_{2}
      \end{bmatrix}}_{\bm{V}}
    \underbrace{
      \begin{bmatrix}
        \lambda_{1} & 0 \\
        0 & \lambda_{2} \\
      \end{bmatrix}}_{\bm{\Lambda}}
      \begin{bmatrix}
        \bm{v}_{1} & \bm{v}_{2}
      \end{bmatrix}^{-1}
    \end{equation}
  \item For each $t$, define $\bm{z}(t)\in\R^{2}$ as
    \begin{equation}\label{eq:zt}%
      \bm{z}(t):=\bm{V}^{-1}\bm{x}(t)
      \quad\text{or}\quad
      \begin{bmatrix}
        z_{1}(t) \\ z_{2}(t)
      \end{bmatrix}
      :=
      \begin{bmatrix}
        \bm{v}_{1} & \bm{v}_{2}
      \end{bmatrix}^{-1}
      \begin{bmatrix}
        x_{1}(t) \\ x_{2}(t)
      \end{bmatrix},
    \end{equation}
    which means
    \begin{equation}\nonumber%\label{eq:}%
      \dot{\bm{z}}(t)
      = \bm{V}^{-1}\dot{\bm{x}}(t)
      = \bm{V}^{-1}\left(\bm{A}\bm{x}(t)\right)
      = \bm{V}^{-1}\left(\bm{V}\bm{\Lambda}\bm{V}^{-1}\bm{V}\bm{z}(t)\right)
      = \bm{\Lambda}\bm{z}(t)
    \end{equation}
    or
    \begin{equation}\nonumber%\label{eq:}%
      \begin{bmatrix}
        \dot{z}_{1}(t)\\
        \dot{z}_{2}(t)\\
      \end{bmatrix}
      =
      \begin{bmatrix}
        \lambda_{1} & 0 \\
        0 & \lambda_{2} \\
      \end{bmatrix}
      \begin{bmatrix}
        z_{1}(t)\\
        z_{2}(t)\\
      \end{bmatrix}
      =
      \begin{bmatrix}
        \lambda_{1} z_{1}(t)\\
        \lambda_{2} z_{2}(t)\\
      \end{bmatrix}
    \end{equation}
  \item Solving this system is straightforward:
    \begin{equation}\nonumber%\label{eq:}%
      \begin{bmatrix}
        z_{1}(t)\\
        z_{2}(t)\\
      \end{bmatrix}
      =
      \begin{bmatrix}
        e^{\lambda_{1} t}z_{1}(0)\\
        e^{\lambda_{2} t}z_{2}(0)\\
      \end{bmatrix}
      =
      \underbrace{
      \begin{bmatrix}
        e^{\lambda_{1} t} & 0\\
        0 & e^{\lambda_{2} t} \\
      \end{bmatrix}}_{e^{\bm{\Lambda}t}}
      \begin{bmatrix}
        z_{1}(0)\\
        z_{2}(0)\\
      \end{bmatrix}
    \end{equation}
  \item Since $\bm{x}(t)=\bm{V}\bm{z}(t)$,
    it follows that
    \begin{equation}\nonumber%\label{eq:}%
      \bm{x}(t)
      =
      \bm{V}
      \bm{z}(t)
      =
      \bm{V}
      e^{\bm{\Lambda}t}
      \bm{V}^{-1}
      \bm{V}
      \bm{z}(0)
      =
      e^{\bm{V}\bm{\Lambda}\bm{V}^{-1}t}
      \bm{x}(0)
      =
      e^{\bm{A}t}
      \bm{x}(0)
    \end{equation}
  \end{itemize}

\item \textbf{Example~7}

  \begin{itemize}
  \item Consider the same dynamical system as \eqref{eq:2x2}
  \item This time, 
    $\bm{A}$ is NOT diagonalizable
    and the Jordan decomposition yields
    \begin{equation}\nonumber%\label{eq:}%
      \bm{A} =
      \underbrace{
      \begin{bmatrix}
        \bm{v}_{1} & \bm{v}_{2}
      \end{bmatrix}}_{\bm{V}}
    \underbrace{
      \begin{bmatrix}
        \lambda & 1 \\
        0 & \lambda \\
      \end{bmatrix}}_{\bm{J}}
      \begin{bmatrix}
        \bm{v}_{1} & \bm{v}_{2}
      \end{bmatrix}^{-1}
    \end{equation}
  \item For each $t$, define $\bm{z}(t)\in\R^{2}$ as \eqref{eq:zt},
    which means
    \begin{equation}\nonumber%\label{eq:}%
      \dot{\bm{z}}(t)
      = \bm{V}^{-1}\dot{\bm{x}}(t)
      = \bm{V}^{-1}\left(\bm{A}\bm{x}(t)\right)
      = \bm{V}^{-1}\left(\bm{V}\bm{J}\bm{V}^{-1}\bm{V}\bm{z}(t)\right)
      = \bm{J}\bm{z}(t)
    \end{equation}
    or
    \begin{equation}\nonumber%\label{eq:}%
      \begin{bmatrix}
        \dot{z}_{1}(t)\\
        \dot{z}_{2}(t)\\
      \end{bmatrix}
      =
      \begin{bmatrix}
        \lambda & 1 \\
        0 & \lambda \\
      \end{bmatrix}
      \begin{bmatrix}
        z_{1}(t)\\
        z_{2}(t)\\
      \end{bmatrix}
      =
      \begin{bmatrix}
        \lambda z_{1}(t) + z_{2}(t)\\
        \lambda z_{2}(t)\\
      \end{bmatrix}
    \end{equation}
  \item First focusing on the second equation, we have
    \begin{equation}\nonumber%\label{eq:}%
      e^{-\lambda t}\dot{z}_{2}(t) - e^{-\lambda t}\lambda z_{2}(t) = 0
      \implies
      z_{2}(t) = e^{\lambda t}z_{2}(0),
    \end{equation}
    which in turn allows us to solve the first equation as
    \begin{equation}\nonumber%\label{eq:}%
      e^{-\lambda t}\dot{z}_{1}(t) - e^{-\lambda t}\lambda z_{1}(t) = z_{2}(0)
      \implies
      z_{1}(t) = e^{\lambda t}z_{1}(0) + te^{\lambda t}z_{2}(0),
    \end{equation}
    and therefore
    \begin{equation}\nonumber%\label{eq:}%
      \begin{bmatrix}
        z_{1}(t)\\
        z_{2}(t)\\
      \end{bmatrix}
      =
      \begin{bmatrix}
        e^{\lambda t}z_{1}(0) + te^{\lambda t}z_{2}(0)\\
        e^{\lambda t}z_{2}(0)\\
      \end{bmatrix}
      =
      \underbrace{
      \begin{bmatrix}
        e^{\lambda t} & te^{\lambda t} \\
        0 & e^{\lambda t} \\
      \end{bmatrix}}_{e^{\bm{J}t}}
      \begin{bmatrix}
        z_{1}(0)\\
        z_{2}(0)\\
      \end{bmatrix}
    \end{equation}
  \item Since $\bm{x}(t)=\bm{V}\bm{z}(t)$,
    it follows that
    \begin{equation}\nonumber%\label{eq:}%
      \bm{x}(t)
      =
      \bm{V}
      \bm{z}(t)
      =
      \bm{V}
      e^{\bm{J}t}
      \bm{V}^{-1}
      \bm{V}
      \bm{z}(0)
      =
      e^{\bm{V}\bm{J}\bm{V}^{-1}t}
      \bm{x}(0)
      =
      e^{\bm{A}t}
      \bm{x}(0)
    \end{equation}
  \end{itemize}

\end{itemize}

\section{Characterization}

\subsection{General method}

\begin{itemize}

\item \textbf{Trajectory and stability}
  \begin{itemize}
  \item Consider the linear dynamical system
    \begin{equation}\label{eq:LDSgeneral}%\nonumber%
      \dot{\bm{x}}(t) = \bm{A}\bm{x}(t) + \bm{b}
      \quad t \in \R_{+}
    \end{equation}
  \item Assume $\bm{A}$ does not have $0$ as its eigenvalue
    (otherwise no stable equilibrium point)
  \item Since $\lambda=0$ is not an eigenvalue of $\bm{A}$,
    we know that $\bm{A}$ is non-singular (why?), and 
    the unique equilibrium point is
    \begin{equation}\nonumber%\label{eq:}%
      \bar{\bm{x}} := -\bm{A}^{-1}\bm{b}
    \end{equation}
    and we may write
    \begin{equation}\label{eq:xt1}%
      \bm{x}(t) = \bar{\bm{x}} + e^{\bm{A}t}(\bm{x}(0)-\bar{\bm{x}}),
      \quad t \in \R_{+}
    \end{equation}
    because
    \begin{equation}\nonumber%\label{eq:}%
      \eqref{eq:LDSgeneral}
      \implies
      \frac{d}{dt}\left(\bm{x}(t) - \bar{\bm{x}}\right)
      = \bm{A}\left(\bm{x}(t) - \bar{\bm{x}}\right)
      \implies
      \frac{d}{dt}\left(e^{-\bm{A}t}\left(\bm{x}(t) - \bar{\bm{x}}\right)\right)
      = \bm{0}
    \end{equation}
  \item Hence,
    \begin{equation}\nonumber%\label{eq:}%
      \text{$\bar{\bm{x}}$ is asymptotically stable}
      \iff
      \lim_{t\to\infty}e^{\bm{A}t} = \bm{O}
      \iff
      \rho(e^{\bm{A}}) < 1
    \end{equation}
  \item Using the Jordan decomposition $\bm{A}=\bm{V}\bm{J}\bm{V}^{-1}$,
    one can rewrite \eqref{eq:xt1} as
    \begin{equation}\nonumber%\label{eq:}%
      \bm{x}(t)
      = \bar{\bm{x}}
      + \bm{V}e^{\bm{J}t}\bm{V}^{-1}(\bm{x}(0)-\bar{\bm{x}}),
      \quad t \in \R_{+}
    \end{equation}
  \item In particular, if $\bm{A}$ is diagonalizable,
    we have $\bm{J}=\bm{\Lambda}:=\text{diag}(\lambda_{1},\ldots,\lambda_{m})$ and
    \begin{equation}\nonumber%\label{eq:}%
      \bm{x}(t) = \bar{\bm{x}} +  \bm{V}e^{\bm{\Lambda}t}\bm{V}^{-1}(\bm{x}(t)-\bar{\bm{x}})
      \quad t \in \R_{+},
    \end{equation}
    in which case the stable manifold can be expressed as
    \begin{equation}\nonumber%\label{eq:}%
      W(\bar{\bm{x}}) =
      \left\{ \bm{x}(0) \in \R^{m}
        \,\bigg|\,
        \text{$\bm{e}_{i}^{\top}\bm{V}^{-1}(\bm{x}(0)-\bar{\bm{x}}) = 0$ for $i$ such that $\lambda_{i}\geq 0$}
      \right\}
    \end{equation}

  \end{itemize}

\item \textbf{Using eigenvectors as a basis}

  \begin{itemize}
  \item Let $\bm{A}\in \R^{m\times m}$ be a square matrix and
    consider the linear system of differential equations
    \begin{equation}\label{eq:DE}%
      \dot{\bm{x}}(t) = \bm{A}\bm{x}(t) \quad \forall t\in \R_{+} \ldots
    \end{equation}
  \item If $(\lambda, \bm{v})$ is an eigenpair of $\bm{A}$,
    then 
    \begin{equation}\nonumber%\label{eq:}%
      \bm{x}(t) := e^{\lambda t}\bm{v} \quad \forall t
    \end{equation}
    solves \eqref{eq:DE} with $\bm{x}(0)=\bm{v}$ because
    \begin{equation}\nonumber%\label{eq:}%
      \bm{x}(t) = e^{\lambda t}\bm{v}
      \implies
      \dot{\bm{x}}(t)
      = \lambda e^{\lambda t}\bm{v}
      = e^{\lambda t}\lambda\bm{v}
      = e^{\lambda t}\bm{A} \bm{v}
      = \bm{A} e^{\lambda t}\bm{v}
      = \bm{A} \bm{x}(t)
    \end{equation}
    
  \item If $(\lambda_{1}, \bm{v}_{1}), (\lambda_{2}, \bm{v}_{2})$ are eigenpairs of $\bm{A}$,
    then for any $z_{1}, z_{2}\in \R$,
    \begin{equation}\nonumber%\label{eq:}%
      \bm{x}(t) := z_{1}e^{\lambda_{1}t}\bm{v}_{1} + z_{2}e^{\lambda_{2}t}\bm{v}_{2} \quad \forall t
    \end{equation}
    solves \eqref{eq:DE} with $\bm{x}(0)=z_{1}\bm{v}_{1} + z_{2}\bm{v}_{2}$

  \item If $(\lambda_{1}, \bm{v}_{1}), (\lambda_{2}, \bm{v}_{2}), \ldots, (\lambda_{m}, \bm{v}_{m})$ are eigenpairs of $\bm{A}$,
    then for any $z_{1}, z_{2},\ldots, z_{m}\in \R$,
    \begin{equation}\nonumber%\label{eq:}%
      \bm{x}(t) := z_{1}e^{\lambda_{1}t}\bm{v}_{1} + z_{2}e^{\lambda_{2}t}\bm{v}_{2} + \ldots + z_{m}e^{\lambda_{m}t}\bm{v}_{m} \quad \forall t
    \end{equation}
    solves \eqref{eq:DE} with $\bm{x}(0)=z_{1}\bm{v}_{1} + z_{2}\bm{v}_{2} + \ldots + z_{m}\bm{v}_{m}$

  \item Hence, if $(\lambda_{1}, \bm{v}_{1}), (\lambda_{2}, \bm{v}_{2}), \ldots, (\lambda_{m}, \bm{v}_{m})$ are \textbf{linearly independent} eigenpairs of $\bm{A}$,
    then for any arbitrary initial state $\bm{x}(0)\in \R^{m}$,
    one can find $z_{1}, z_{2},\ldots, z_{m}\in \R$ such that
    \begin{equation}\nonumber%\label{eq:}%
      \bm{x}(0)
      = z_{1}\bm{v}_{1} + z_{2}\bm{v}_{2} + \ldots + z_{m}\bm{v}_{m}
    \end{equation}
    and the state trajectory from $\bm{x}(0)$ can be written as
    \begin{equation}\nonumber%\label{eq:}%
      \bm{x}(t)
      = z_{1}e^{\lambda_{1}t}\bm{v}_{1} + z_{2}e^{\lambda_{2}t}\bm{v}_{2} + \ldots + z_{m}e^{\lambda_{m}t}\bm{v}_{m}
      =
      \bm{V}e^{\bm{\Lambda} t}\bm{z}
      \quad \forall t
    \end{equation}
    
  \end{itemize}

\item \textbf{Long-run behavior and dominant mode}
  \begin{itemize}
  \item Consider the following $m$-dimensional linear dynamical system:
    \begin{equation}\nonumber%\label{eq:}%
      \dot{\bm{x}}(t)
      =
      \bm{A}
      \bm{x}(t) + \bm{b}
      \quad \forall t\quad
      \text{with some initial state $\bm{x}(0)\in \R^{m}$}
    \end{equation}
   
  \item Suppose that $\bm{A}\in \R^{m\times m}$ is diagonalizable
    and let
    $(\lambda_{1}, \bm{v}_{1}), (\lambda_{2}, \bm{v}_{2}), \ldots, (\lambda_{m}, \bm{v}_{m})$
    be linearly independent eigenpairs of $\bm{A}$
  \item Then, 
    there exist $z_{1}, z_{2}, \ldots, z_{m}\in \R$ such that
    \begin{equation}\nonumber%\label{eq:}%
      \bm{x}_{0} - \bar{\bm{x}}
      = z_{1}\bm{v}_{1} + z_{2}\bm{v}_{2}+ \ldots z_{m}\bm{v}_{m}
      =
      \underbrace{\begin{bmatrix}
        \bm{v}_{1} & \bm{v}_{2} & \cdots & \bm{v}_{m}
      \end{bmatrix}}_{\bm{V}}
    \underbrace{
      \begin{bmatrix}
        z_{1} \\
        z_{2} \\
        \vdots \\
        z_{m} \\
      \end{bmatrix}}_{\bm{z}}
    \end{equation}
    and thus
    \begin{equation}\nonumber%\label{eq:}%
      \bm{x}(t) - \bar{\bm{x}}
      = e^{\bm{A}t}(\bm{x}(0)-\bar{\bm{x}})
%      = \bm{A}^{t}\bm{V}\bm{z}_{0}
      =
      z_{1}e^{\lambda_{1}t}\bm{v}_{1} + z_{2}e^{\lambda_{2}t}\bm{v}_{2} + \ldots + z_{m}e^{\lambda_{m}t}\bm{v}_{m}
      \quad \forall t
    \end{equation}
  \item Let $\lambda_{i}$ be the \emph{dominant eigenvalue} of $\bm{A}$, i.e., 
    $\lambda_{i}> \lambda_{j}$ for all $j\neq i$
  \item Then, for sufficiently large $t$, 
    \begin{equation}\nonumber%\label{eq:}%
      e^{-\lambda_{i}t}(\bm{x}(t) - \bar{\bm{x}})
      = z_{1}e^{-(\lambda_{i}-\lambda_{1})}\bm{v}_{1}
      + \cdots
      + z_{i}\bm{v}_{i}
      + \cdots
      + z_{m}e^{-(\lambda_{i}-\lambda_{m})}\bm{v}_{m}
      \approx
      z_{i}\bm{v}_{i},
    \end{equation}
    or
    \begin{equation}\nonumber%\label{eq:}%
      \bm{x}(t) \approx \bar{\bm{x}} + e^{\lambda_{i}t}z_{i}\bm{v}_{i},
      \quad\text{where}\quad
      z_{i} = \bm{e}_{i}^{\top}\bm{V}^{-1}(\bm{x}(0)-\bar{\bm{x}})
    \end{equation}
  \item Observe:
    \begin{itemize}
      \item the long-term state is essentially determined by the eigenvector associated with the dominant eigenvalue of $\bm{A}$
      \item if $\rho(e^{\bm{A}})<1$, the rate at which the state converges to the equilibrium point
        is ultimately governed by the dominant eigenvalue
      \item if $\rho(e^{\bm{A}})=1$ (i.e, the dominant eigenvalue is zero),
        \begin{equation}\nonumber%\label{eq:}%
          \lim_{t\to \infty}\bm{x}(t) = \bar{\bm{x}} + z_{i}\bm{v}_{i},
        \end{equation}
        in which case $\bm{x}(t)$ neither diverges to infinity nor converges to $\bar{\bm{x}}$
        (we call such situation as \emph{marginally stable})
        and the limit depends on the initial state (through $z_{i}$)
        
    \end{itemize}
    
  \end{itemize}

\end{itemize}

\subsection{Examples}

  \begin{figure}[t]\centering%
    \includegraphics[width=480pt]%
    {figures/fig_dynamical_continuous.pdf}
    \caption{%
      Phase diagrams for
      example~1 (top left),
      example~2 (top right),
      example~4 (bottom left), and
      example~5 (bottom right)
    }
    \label{fig:dynamical_discrete}
  \end{figure} 

\begin{itemize}

\item \textbf{Example~1}
  \begin{itemize}
  \item Consider the following two-dimensional linear dynamical system:
    \begin{equation}\nonumber%\label{eq:}%
      \begin{bmatrix}
        \dot{x}_{1} \\
        \dot{x}_{2}
      \end{bmatrix}
      =
      \underbrace{
      \begin{bmatrix}
        -3/8 & -1/4 \\
        -1/16 & -3/8 \\
      \end{bmatrix}%
      }_{\bm{A}}
      \begin{bmatrix}
        x_{1}(t) \\
        x_{2}(t)
      \end{bmatrix}
    \end{equation}
    \item The state trajectory $\{\bm{x}(t)\}$ from arbitrary $\bm{x}_{0}\in \R^{2}$ is
      \begin{equation}\nonumber%\label{eq:}%
        \bm{x}(t)
        =
        e^{\bm{A}t}\bm{x}_{0},
    \end{equation}
    which is not easy to characterize

  \item So we look at the characteristic polynomial
    \begin{equation}\nonumber%\label{eq:}%
      \phi_{\bm{A}}(t) = 
      \begin{vmatrix}
        -3/8 - t & -1/4 \\
        -1/16 & -3/8 - t \\
      \end{vmatrix}
      = (-1/4-t)(-1/2-t),
    \end{equation}
    implying that
    the eigenvalues of $\bm{A}$ are $\lambda_{1}:=-1/4$ and $\lambda_{2}:=-1/2$
  \item We already see that
    the unique equilibrium point
    $\bar{\bm{x}}:=\bm{0}$ is asymptotically stable
  \item To characterize the behavior of $\{\bm{x}(t)\}$ more explicitly,
    we derive eigenvectors:
    \begin{equation}\nonumber%\label{eq:}%
      (\bm{A}-\lambda_{1}\bm{I})\bm{v}
      = \bm{0}
      \iff
      \begin{bmatrix}
        -3/8 + 1/4 & - 1/4 \\
        -1/16 & -3/8 + 1/4 \\
      \end{bmatrix}
      \begin{bmatrix}
        v_{1} \\
        v_{2} \\
      \end{bmatrix}
      =
      \begin{bmatrix}
        0 \\
        0 \\
      \end{bmatrix}
      \iff
      \bm{v} = \alpha
      \begin{bmatrix}
        1 \\
        -\frac{1}{2}
      \end{bmatrix}
      \quad \forall \alpha
    \end{equation}
    and
    \begin{equation}\nonumber%\label{eq:}%
      (\bm{A}-\lambda_{2}\bm{I})\bm{v}
      = \bm{0}
      \iff
      \begin{bmatrix}
        -3/8 + 1/2 & - 1/4 \\
        -1/16 & -3/8 + 1/2 \\
      \end{bmatrix}
      \begin{bmatrix}
        v_{1} \\
        v_{2} \\
      \end{bmatrix}
      =
      \begin{bmatrix}
        0 \\
        0 \\
      \end{bmatrix}
      \iff
      \bm{v} = \alpha
      \begin{bmatrix}
        1 \\
        \frac{1}{2}
      \end{bmatrix}
      \quad \forall \alpha
    \end{equation}
    so we choose
    \begin{equation}\nonumber%\label{eq:}%
      \bm{\Lambda}:=
      \begin{bmatrix}
       \lambda_{1} & 0\\
       0 & \lambda_{2}
      \end{bmatrix},
      \quad
      \bm{v}_{1} :=
      \begin{bmatrix}
        1 \\
        - \frac{1}{2}
      \end{bmatrix},
      \quad
      \bm{v}_{2} :=
      \begin{bmatrix}
        1 \\
        \frac{1}{2}
      \end{bmatrix}
    \end{equation}
    and
    \begin{equation}\nonumber%\label{eq:}%
      \bm{V}:=
      \begin{bmatrix}
        \bm{v}_{1} & \bm{v}_{2}
      \end{bmatrix}
      =
      \begin{bmatrix}
        1 & 1 \\
        -1/2 & 1/2 \\
      \end{bmatrix}
      \implies
      \bm{V}^{-1} =
      \begin{bmatrix}
        1/2 & -1 \\
        1/2 & 1 \\
      \end{bmatrix}
      \end{equation}
    \item Now we can express the state trajectory $\{\bm{x}(t)\}$ from arbitrary $\bm{x}_{0}\in \R^{2}$ as
      \begin{equation}\nonumber%\label{eq:}%
        \bm{x}(t) =
        e^{\bm{A}t}\bm{x}(0)
        =
        \bm{V}e^{\bm{\Lambda}t}\bm{V}^{-1}\bm{x}(0)
        =
      \begin{bmatrix}
        1 & 1 \\
        -1/2 & 1/2 \\
      \end{bmatrix}
      \begin{bmatrix}
        e^{- \frac{1}{4}t} & 0 \\
        0 & e^{- \frac{1}{2}t} \\
      \end{bmatrix}
      \begin{bmatrix}
        1/2 & -1 \\
        1/2 & 1 \\
      \end{bmatrix}
      \bm{x}(0)
      \end{equation}
    \item The trajectory converges to $\bar{\bm{x}}=\bm{0}$
      regardless of $\bm{x}(0)$
    \item Given an initial state $\bm{x}(0)$,
      defining $\bm{z}$ as
      \begin{equation}\nonumber%\label{eq:}%
        \bm{z} := \bm{V}^{-1}\bm{x}(0),
        \quad\text{or}\quad
        z_{i} := \bm{e}_{i}^{\top}\bm{V}^{-1}\bm{x}(0),
        \quad i = 1, 2
      \end{equation}
      allows us to write
    \begin{equation}\nonumber%\label{eq:}%
      \bm{x}(t)
      =
      e^{\lambda_{1}t}z_{1}\bm{v}_{1}
      +
      e^{\lambda_{2}t}z_{2}\bm{v}_{2}
      \quad \forall t
    \end{equation}
    \item Since $\lambda_{1}=-1/4$ is the dominant eigenvalue,
      the long-term behavior is characterized by
    \begin{equation}\nonumber%\label{eq:}%
      \bm{x}(t)
      \approx
      e^{\lambda_{1}t}z_{1}\bm{v}_{1}
      =
      e^{- \frac{1}{4}t}z_{1}
      \begin{bmatrix}
        1 \\ -1/2
      \end{bmatrix},
      \quad\text{where}\quad
      z_{1}:=\bm{e}_{1}^{\top}\bm{V}^{-1}\bm{x}(0)
      = \frac{1}{2}x_{1}(0) - x_{2}(0)
    \end{equation}
    for sufficiently large $t$
  \end{itemize}

\item \textbf{Example~2}
  \begin{itemize}
  \item Consider another two-dimensional linear dynamical system:
    \begin{equation}\nonumber%\label{eq:}%
      \begin{bmatrix}
        \dot{x}_{1} \\
        \dot{x}_{2}
      \end{bmatrix}
      =
      \underbrace{
      \begin{bmatrix}
        1/4 & -3/2 \\
        -3/8 & 1/4 \\
      \end{bmatrix}}_{\bm{A}}
      \begin{bmatrix}
        x_{1}(t) \\
        x_{2}(t)
      \end{bmatrix}
    \end{equation}
    \item The state trajectory $\{\bm{x}(t)\}$ from arbitrary $\bm{x}(0)\in \R^{2}$ is
      \begin{equation}\nonumber%\label{eq:}%
        \bm{x}(t)
        =
        e^{\bm{A}t}\bm{x}(0),
    \end{equation}
    which is not easy to characterize
  \item So we look at the characteristic polynomial
    \begin{equation}\nonumber%\label{eq:}%
      \phi_{\bm{A}}(t) = 
      \begin{vmatrix}
        1/4 - t & -3/2 \\
        -3/8 & 1/4 - t \\
      \end{vmatrix}
      = (1-t)(-1/2-t),
    \end{equation}
    implying that
    the eigenvalues of $\bm{A}$ are $\lambda_{1}:=1$ and $\lambda_{2}:=-1/2$
  \item Since $\lambda_{1}>0$,
    the unique equilibrium point
    $\bar{\bm{x}}:=\bm{0}$ is NOT asymptotically stable
  \item To characterize the behavior of $\{\bm{x}(t)\}$ more explicitly,
    we derive eigenvectors:
    \begin{equation}\nonumber%\label{eq:}%
      (\bm{A}-\lambda_{1}\bm{I})\bm{v}
      = \bm{0}
      \iff
      \begin{bmatrix}
        1/4 - 1 & - 3/2 \\
        -3/8 & 1/4 - 1 \\
      \end{bmatrix}
      \begin{bmatrix}
        v_{1} \\
        v_{2} \\
      \end{bmatrix}
      =
      \begin{bmatrix}
        0 \\
        0 \\
      \end{bmatrix}
      \iff
      \bm{v} = \alpha
      \begin{bmatrix}
        1 \\
        -\frac{1}{2}
      \end{bmatrix}
      \quad \forall \alpha
    \end{equation}
    and
    \begin{equation}\nonumber%\label{eq:}%
      (\bm{A}-\lambda_{2}\bm{I})\bm{v}
      = \bm{0}
      \iff
      \begin{bmatrix}
        1/4 + 1/2 & - 3/2 \\
        -3/8 & 1/4 + 1/2 \\
      \end{bmatrix}
      \begin{bmatrix}
        v_{1} \\
        v_{2} \\
      \end{bmatrix}
      =
      \begin{bmatrix}
        0 \\
        0 \\
      \end{bmatrix}
      \iff
      \bm{v} = \alpha
      \begin{bmatrix}
        1 \\
        \frac{1}{2}
      \end{bmatrix}
      \quad \forall \alpha
    \end{equation}
    so we choose
    \begin{equation}\nonumber%\label{eq:}%
      \bm{\Lambda}:=
      \begin{bmatrix}
       \lambda_{1} & 0\\
       0 & \lambda_{2}
      \end{bmatrix},
      \quad
      \bm{v}_{1} :=
      \begin{bmatrix}
        1 \\
        - \frac{1}{2}
      \end{bmatrix},
      \quad
      \bm{v}_{2} :=
      \begin{bmatrix}
        1 \\
        \frac{1}{2}
      \end{bmatrix}
    \end{equation}
    and
    \begin{equation}\nonumber%\label{eq:}%
      \bm{V}:=
      \begin{bmatrix}
        \bm{v}_{1} & \bm{v}_{2}
      \end{bmatrix}
      =
      \begin{bmatrix}
        1 & 1 \\
        -1/2 & 1/2 \\
      \end{bmatrix}
      \implies
      \bm{V}^{-1} =
      \begin{bmatrix}
        1/2 & -1 \\
        1/2 & 1 \\
      \end{bmatrix}
      \end{equation}
    \item Now we can express the state trajectory $\{\bm{x}(t)\}$ from arbitrary $\bm{x}(0)\in \R^{2}$ as
      \begin{equation}\nonumber%\label{eq:}%
        \bm{x}(t) =
        e^{\bm{A}t}\bm{x}(0)
        =
        \bm{V}e^{\bm{\Lambda}t}\bm{V}^{-1}\bm{x}(0)
        =
      \begin{bmatrix}
        1 & 1 \\
        -1/2 & 1/2 \\
      \end{bmatrix}
      \begin{bmatrix}
        e^{t} & 0 \\
        0 & e^{- \frac{1}{2}t} \\
      \end{bmatrix}
      \begin{bmatrix}
        1/2 & -1 \\
        1/2 & 1 \\
      \end{bmatrix}
      \bm{x}(0)
      \end{equation}
    \item
      The trajectory converges to $\bar{\bm{x}}=\bm{0}$
      only if we choose $\bm{x}(0)$
      to nullify the impact of $e^{t}$
    \item More specifically, we need to choose $\bm{x}(0)$
      in such a way that
      \begin{equation}\nonumber%\label{eq:}%
      \begin{bmatrix}
        1/2 & -1 \\
        1/2 & 1 \\
      \end{bmatrix}
      \bm{x}(0)
      = 
      \begin{bmatrix}
        0 \\
        \ldots \\
      \end{bmatrix}
      \iff
      \bm{e}_{1}^{\top}
      \underbrace{\begin{bmatrix}
        1/2 & -1 \\
        1/2 & 1 \\
      \end{bmatrix}}_{\bm{V}^{-1}}
      \bm{x}(0) = 0
      \iff
      \frac{1}{2}x_{1}(0) - x_{2}(0) = 0
      \end{equation}
    \item Hence, the stable manifold is
      \begin{equation}\nonumber%\label{eq:}%
        W(\bar{\bm{x}}) = \left\{\bm{x}(0)=(x_{1}(0),x_{2}(0))\in \R^{2}\,\bigg|\, \frac{1}{2}x_{1}(0) - x_{2}(0) = 0\right\}
      \end{equation}
    \item Given an initial state $\bm{x}(0)$,
      defining $\bm{z}$ as
      \begin{equation}\nonumber%\label{eq:}%
        \bm{z} := \bm{V}^{-1}\bm{x}(0),
        \quad\text{or}\quad
        z_{i} := \bm{e}_{i}^{\top}\bm{V}^{-1}\bm{x}(0),
        \quad i = 1, 2
      \end{equation}
      allows us to write
    \begin{equation}\nonumber%\label{eq:}%
      \bm{x}(t)
      =
      e^{\lambda_{1}t}z_{1}\bm{v}_{1}
      +
      e^{\lambda_{2}t}z_{2}\bm{v}_{2}
      \quad \forall t
    \end{equation}
    \item Since $\lambda_{1}$ is the dominant eigenvalue,
      the long-term behavior is characterized by
    \begin{equation}\nonumber%\label{eq:}%
      \bm{x}(t)
      \approx
      e^{\lambda_{1}t}z_{1}\bm{v}_{1}
      =
      e^{t}z_{1}
      \begin{bmatrix}
        1 \\ -1/2
      \end{bmatrix},
      \quad\text{where}\quad
      z_{1}:=\bm{e}_{1}^{\top}\bm{V}^{-1}\bm{x}(0)
      = \frac{1}{2}x_{1}(0) - x_{2}(0)
    \end{equation}
    for sufficiently large $t$
  \end{itemize}

\item \textbf{Example~3}
  \begin{itemize}
  \item Consider a slightly modified version of the example above:
    \begin{equation}\nonumber%\label{eq:}%
      \begin{bmatrix}
        \dot{x}_{1} \\
        \dot{x}_{2}
      \end{bmatrix}
      =
      \underbrace{
      \begin{bmatrix}
        1/4 & -3/2 \\
        -3/8 & 1/4 \\
      \end{bmatrix}}_{\bm{A}}
      \begin{bmatrix}
        x_{1}(t) \\
        x_{2}(t)
      \end{bmatrix}
      +
      \underbrace{
      \begin{bmatrix}
        2 \\
        1
      \end{bmatrix}}_{\bm{b}}
    \end{equation}
  \item Since $0$ is not an eigenvalue of $\bm{A}$,
    we know that $\bm{A}$ is non-singular and
    \begin{equation}\nonumber%\label{eq:}%
      \bm{A}^{-1}
      = 
      \begin{bmatrix}
        -1/2 & -3 \\
        -3/4 & -1/2 \\
      \end{bmatrix}
    \end{equation}
  \item The unique equilibrium point is
    \begin{equation}\nonumber%\label{eq:}%
      \bar{\bm{x}} =
      -\bm{A}^{-1}\bm{b} =
      -
      \begin{bmatrix}
        -1/2 & -3 \\
        -3/4 & -1/2 \\
      \end{bmatrix}
      \begin{bmatrix}
        2 \\
        1
      \end{bmatrix}
      =
      \begin{bmatrix}
        4 \\
        2
      \end{bmatrix}
    \end{equation}
  \item The state trajectory $\{\bm{x}(t)\}$ from arbitrary $\bm{x}(0)\in \R^{2}$ is
    \begin{align}
      \bm{x}(t)
      & = \bar{\bm{x}} + e^{\bm{A}t}(\bm{x}(0)-\bar{\bm{x}})
        =
        \bar{\bm{x}}
        +
        \bm{V}e^{\bm{\Lambda}t}\bm{V}^{-1}(\bm{x}(0)-\bar{\bm{x}})        
        \nonumber \\
        & 
        =
        \begin{bmatrix} 4 \\ 2 \end{bmatrix}
        +
      \begin{bmatrix}
        1 & 1 \\
        -1/2 & 1/2 \\
      \end{bmatrix}
      \begin{bmatrix}
        e^{t} & 0 \\
        0 & e^{- \frac{1}{2}t} \\
      \end{bmatrix}
      \begin{bmatrix}
        1/2 & -1 \\
        1/2 & 1 \\
      \end{bmatrix}
      \left(\bm{x}(0) - \begin{bmatrix} 4 \\ 2 \end{bmatrix}\right)
    \nonumber%\label{eq:}%
    \end{align}
    \item
      The trajectory converges to $\bar{\bm{x}}=(4, 2)^{\top}$
      only if we choose $\bm{x}(0)$
      to nullify the impact of $e^{t}$
    \item More specifically, we need to choose $\bm{x}(0)$
      in such a way that
      \begin{equation}\nonumber%\label{eq:}%
      \begin{bmatrix}
        1/2 & -1 \\
        1/2 & 1 \\
      \end{bmatrix}
      (\bm{x}(0)-\bar{\bm{x}})
      = 
      \begin{bmatrix}
        0 \\
        \ldots \\
      \end{bmatrix}
      \iff
      \frac{1}{2}(x_{1}(0)-4) - (x_{2}(0)-2) = 0
      \end{equation}
    \item Hence, the stable manifold is
      \begin{equation}\nonumber%\label{eq:}%
        W(\bar{\bm{x}}) = \left\{\bm{x}(0)=(x_{1}(0),x_{2}(0))\in \R^{2}\,\bigg|\, \frac{1}{2}(x_{1}(0)-4) - (x_{2}(0)-2) = 0 \right\}
      \end{equation}
    \item Given an initial state $\bm{x}(0)$,
      defining $\bm{z}$ as
      \begin{equation}\nonumber%\label{eq:}%
        \bm{z} := \bm{V}^{-1}(\bm{x}(0)-\bar{\bm{x}}),
        \quad\text{or}\quad
        z_{i} := \bm{e}_{i}^{\top}\bm{V}^{-1}(\bm{x}(0)-\bar{\bm{x}}),
        \quad i = 1, 2
      \end{equation}
      allows us to write
    \begin{equation}\nonumber%\label{eq:}%
      \bm{x}(t)
      =
      \bar{\bm{x}} + 
      e^{\lambda_{1}t}z_{1}\bm{v}_{1}
      +
      e^{\lambda_{2}t}z_{2}\bm{v}_{2}
      \quad \forall t
    \end{equation}
    \item Since $\lambda_{1}=1$ is the dominant eigenvalue,
      the long-term behavior is characterized by
    \begin{equation}\nonumber%\label{eq:}%
      \bm{x}(t)
      \approx
      \bar{\bm{x}} +
      e^{\lambda_{1}t}z_{1}\bm{v}_{1}
      =
      e^{t}z_{1}
      \begin{bmatrix}
        1 \\ -1/2
      \end{bmatrix},
      \quad\text{where}\quad
      z_{1}:=\bm{e}_{1}^{\top}\bm{V}^{-1}(\bm{x}(0)-\bar{\bm{x}})
      = \frac{1}{2}(x_{1}(0)-4) - (x_{2}(0)-2)
    \end{equation}
    for sufficiently large $t$
  \item The phase diagram looks exactly the same as in example~2,
    except that now the center of the graph is replaced by $\bar{\bm{x}}=(4,2)^{\top}$
  \end{itemize}

\clearpage
\item \textbf{Example~4}
  \begin{itemize}
  \item Consider the following two-dimensional linear dynamical system:
    \begin{equation}\nonumber%\label{eq:}%
      \begin{bmatrix}
        \dot{x}_{1}(t) \\
        \dot{x}_{2}(t)
      \end{bmatrix}
      =
      \underbrace{
      \begin{bmatrix}
        -1/4 & -1/2 \\
        -1/8 & -1/4 \\
      \end{bmatrix}}_{\bm{A}}
      \begin{bmatrix}
        x_{1}(t) \\
        x_{2}(t)
      \end{bmatrix}
    \end{equation}
  \item The characteristic polynomial of $\bm{A}$ is
    \begin{equation}\nonumber%\label{eq:}%
      \phi_{\bm{A}}(t) = 
      \begin{vmatrix}
        -1/4 - t & -1/2 \\
        -1/8 & -1/4 - t \\
      \end{vmatrix}
      = -t(-1/2-t),
    \end{equation}
    implying that
    the eigenvalues of $\bm{A}$ are $\lambda_{1}:=0$ and $\lambda_{2}:=-1/2$

  \item Note that $\bm{A}$ is singular
    because its eigenvalues include $0$:
    \begin{itemize}
    \item obviously, $\bm{0}$ is \emph{an} equilibrium point of the system
    \item there are many other equilibrium points, and in fact,
      any eigenvector associated with the zero eigenvalue is an equilibrium point because
      \begin{equation}\nonumber%\label{eq:}%
        \bm{A}(\alpha\bm{v}_{1}) = \lambda_{1}(\alpha\bm{v}_{1}) = \bm{0}
        \quad \forall \alpha\in \R
      \end{equation}
    \item none of these equilibrium points is asymptotically stable
    \end{itemize}
    
  \item To characterize the behavior of $\{\bm{x}(t)\}$ more explicitly,
    we derive eigenvectors:
    \begin{equation}\nonumber%\label{eq:}%
      (\bm{A}-\lambda_{1}\bm{I})\bm{v}
      = \bm{0}
      \iff
      \begin{bmatrix}
        -1/4 - 0 & - 1/2 \\
        -1/8 & -1/4 - 0 \\
      \end{bmatrix}
      \begin{bmatrix}
        v_{1} \\
        v_{2} \\
      \end{bmatrix}
      =
      \begin{bmatrix}
        0 \\
        0 \\
      \end{bmatrix}
      \iff
      \bm{v} = \alpha
      \begin{bmatrix}
        1 \\
        -\frac{1}{2}
      \end{bmatrix}
      \quad \forall \alpha
    \end{equation}
    and
    \begin{equation}\nonumber%\label{eq:}%
      (\bm{A}-\lambda_{2}\bm{I})\bm{v}
      = \bm{0}
      \iff
      \begin{bmatrix}
        -1/4 + 1/2 & - 1/2 \\
        -1/8 & -1/4 + 1/2 \\
      \end{bmatrix}
      \begin{bmatrix}
        v_{1} \\
        v_{2} \\
      \end{bmatrix}
      =
      \begin{bmatrix}
        0 \\
        0 \\
      \end{bmatrix}
      \iff
      \bm{v} = \alpha
      \begin{bmatrix}
        1 \\
        \frac{1}{2}
      \end{bmatrix}
      \quad \forall \alpha
    \end{equation}
    so we choose
    \begin{equation}\nonumber%\label{eq:}%
      \bm{\Lambda}:=
      \begin{bmatrix}
       \lambda_{1} & 0\\
       0 & \lambda_{2}
      \end{bmatrix},
      \quad
      \bm{v}_{1} :=
      \begin{bmatrix}
        1 \\
        - \frac{1}{2}
      \end{bmatrix},
      \quad
      \bm{v}_{2} :=
      \begin{bmatrix}
        1 \\
        \frac{1}{2}
      \end{bmatrix}
    \end{equation}
    and
    \begin{equation}\nonumber%\label{eq:}%
      \bm{V}:=
      \begin{bmatrix}
        \bm{v}_{1} & \bm{v}_{2}
      \end{bmatrix}
      =
      \begin{bmatrix}
        1 & 1 \\
        -1/2 & 1/2 \\
      \end{bmatrix}
      \implies
      \bm{V}^{-1} =
      \begin{bmatrix}
        1/2 & -1 \\
        1/2 & 1 \\
      \end{bmatrix}
      \end{equation}
    \item Now we can express the state trajectory $\{\bm{x}(t)\}$ from arbitrary $\bm{x}(0)\in \R^{2}$ as
      \begin{equation}\nonumber%\label{eq:}%
        \bm{x}(t) =
        e^{\bm{A}t}\bm{x}(0)
        =
        \bm{V}e^{\bm{\Lambda}t}\bm{V}^{-1}\bm{x}(0)
        =
      \begin{bmatrix}
        1 & 1 \\
        -1/2 & 1/2 \\
      \end{bmatrix}
      \begin{bmatrix}
        1 & 0 \\
        0 & e^{-\frac{1}{2}t} \\
      \end{bmatrix}
      \begin{bmatrix}
        1/2 & -1 \\
        1/2 & 1 \\
      \end{bmatrix}
      \bm{x}(0)
      \end{equation}
    \item Given an initial state $\bm{x}(0)$,
      defining $\bm{z}$ as
      \begin{equation}\nonumber%\label{eq:}%
        \bm{z} := \bm{V}^{-1}\bm{x}(0),
        \quad\text{or}\quad
        z_{i} := \bm{e}_{i}^{\top}\bm{V}^{-1}\bm{x}(0),
        \quad i = 1, 2
      \end{equation}
      allows us to write
    \begin{equation}\nonumber%\label{eq:}%
      \bm{x}(t)
      =
      e^{\lambda_{1}t}z_{1}\bm{v}_{1}
      +
      e^{\lambda_{2}t}z_{2}\bm{v}_{2}
      =
      z_{1}\bm{v}_{1}
      +
      e^{-\frac{1}{2}t}z_{2}\bm{v}_{2}
      \to
      z_{1}\bm{v}_{1}
      =
      \left(\frac{1}{2}x_{1}(0) - x_{2}(0)\right)\bm{v}_{1}
      \quad t \to \infty
    \end{equation}
    meaning that the state trajectory
    \begin{itemize}
    \item moves in parallel with $\bm{v}_{2}$
    \item converges to a particular point on the set $\{\bm{x}\in \R^{2}\,|\, \bm{x}=\alpha \bm{v}_{1}, \alpha \in \R\}$
    \end{itemize}
    
  \end{itemize}

\clearpage
\item \textbf{Example~5}
  \begin{itemize}
  \item Consider the following two-dimensional linear dynamical system:
    \begin{equation}\nonumber%\label{eq:}%
      \begin{bmatrix}
        \dot{x}_{1} \\
        \dot{x}_{2}
      \end{bmatrix}
      =
      \underbrace{
      \begin{bmatrix}
        2 & -2 \\
        -1/2 & 2 \\
      \end{bmatrix}}_{\bm{A}}
      \begin{bmatrix}
        x_{1}(t) \\
        x_{2}(t)
      \end{bmatrix}
    \end{equation}
  \item The characteristic polynomial of $\bm{A}$ is
    \begin{equation}\nonumber%\label{eq:}%
      \phi_{\bm{A}}(t) = 
      \begin{vmatrix}
        2 - t & -2 \\
        -1/2 & 2 - t \\
      \end{vmatrix}
      = (3-t)(1-t),
    \end{equation}
    implying that
    the eigenvalues of $\bm{A}$ are $\lambda_{1}:=3$ and $\lambda_{2}:=1$

  \item The unique equilibrium point of the system is $\bar{\bm{x}}=\bm{0}$,
    which is not asymptotically stable because $\rho(e^{\bm{A}})\geq 1$

  \item We can express the state trajectory $\{\bm{x}(t)\}$ from arbitrary $\bm{x}(0)\in \R^{2}$ as
      \begin{equation}\nonumber%\label{eq:}%
        \bm{x}(t) =
        e^{\bm{A}t}\bm{x}(0)
        =
        \bm{V}e^{\bm{\Lambda}t}\bm{V}^{-1}\bm{x}(0)
        =
      \begin{bmatrix}
        1 & 1 \\
        -1/2 & 1/2 \\
      \end{bmatrix}
      \begin{bmatrix}
        e^{3t} & 0 \\
        0 & e^{t} \\
      \end{bmatrix}
      \begin{bmatrix}
        1/2 & -1 \\
        1/2 & 1 \\
      \end{bmatrix}
      \bm{x}(0)
    \end{equation}
  \item Obviously,
    \begin{equation}\nonumber%\label{eq:}%
      \lim_{t\to \infty}\bm{x}(t) =
      \begin{bmatrix}
        1 & 1 \\
        -1/2 & 1/2 \\
      \end{bmatrix}
      \begin{bmatrix}
        \lim_{t\to\infty}e^{3t} & 0 \\
        0 & \lim_{t\to\infty}e^{t} \\
      \end{bmatrix}
      \begin{bmatrix}
        1/2 & -1 \\
        1/2 & 1 \\
      \end{bmatrix}
      \bm{x}(0)
      =
      \begin{bmatrix}
        \infty \\
        \infty \\
      \end{bmatrix}
    \end{equation}
    unless $\bm{x}(0) = \bar{\bm{x}}=\bm{0}$
    \item Given an initial state $\bm{x}(0)$,
      defining $\bm{z}$ as
      \begin{equation}\nonumber%\label{eq:}%
        \bm{z} := \bm{V}^{-1}\bm{x}(0),
        \quad\text{or}\quad
        z_{i} := \bm{e}_{i}^{\top}\bm{V}^{-1}\bm{x}(0),
        \quad i = 1, 2
      \end{equation}
      allows us to write
    \begin{equation}\nonumber%\label{eq:}%
      \bm{x}(t)
      =
      e^{\lambda_{1}t}z_{1}\bm{v}_{1}
      +
      e^{\lambda_{2}t}z_{2}\bm{v}_{2}
      \quad \forall t
    \end{equation}
    \item Since $\lambda_{1}=3$ is the dominant eigenvalue,
      the long-term behavior is characterized by
    \begin{equation}\nonumber%\label{eq:}%
      \bm{x}(t)
      \approx
      e^{\lambda_{1}t}z_{1}\bm{v}_{1}
      =
      e^{3t}z_{1}
      \begin{bmatrix}
        1 \\ -1/2
      \end{bmatrix},
      \quad\text{where}\quad
      z_{1}:=\bm{e}_{1}^{\top}\bm{V}^{-1}\bm{x}(0)
      = \frac{1}{2}x_{1}(0) - x_{2}(0)
    \end{equation}
    for sufficiently large $t$
  \end{itemize}

\end{itemize}

\section{Discretization of continuous-time systems}

\begin{itemize} 
\item \textbf{General solution}
  \begin{itemize}
  \item Consider a general linear dynamical system
    \begin{equation}\label{eq:model_general}%
      \dot{\bm{x}}(t) = \bm{A}\bm{x}(t) + \bm{B}\bm{u}(t)
      \quad t \in \R_{+}
    \end{equation}
  \item Note that
    \begin{align}
      \eqref{eq:model_general}
        & \implies \frac{d}{dt}\left(e^{-\bm{A}t}\bm{x}(t)\right) = e^{-\bm{A}t}\dot{\bm{x}}(t) - \bm{A}e^{-\bm{A}t}\bm{x}(t) = e^{-\bm{A}t}\bm{B}\bm{u}(t) \quad t \in \R_{+} \label{eq:integration} \\
        & \implies e^{-\bm{A}t}\bm{x}(t) - e^{-\bm{A}0}\bm{x}(0) = \int_{0}^{t}e^{-\bm{A}\tau}\bm{B}\bm{u}(\tau)d\tau \quad t \in \R_{+}
    \nonumber%\label{eq:}%
    \end{align}
    and therefore, the state trajectory can be written as
    \begin{equation}\label{eq:trajectory_general}%
%      \bm{x}(t) = e^{\bm{A}t}\bm{x}(0) + e^{\bm{A}t}\int_{0}^{t}e^{-\bm{A}\tau}\bm{B}\bm{u}(\tau)d\tau
      \bm{x}(t) = e^{\bm{A}t}\bm{x}(0) + \int_{0}^{t}e^{\bm{A}(t-\tau)}\bm{B}\bm{u}(\tau)d\tau
      \quad t \in \R
    \end{equation}
  \end{itemize}

\item \textbf{Discretization}
  \begin{itemize}
  \item In some cases (like when observations can only be made at discrete points in time),
    one might want to convert the continuous-time model into the corresponding discrete-time model
  \item Note that \eqref{eq:integration} implies
    that for all $t, \Delta t\in \R$
    \begin{equation}\nonumber%
      e^{-\bm{A}(t+\Delta t)}\bm{x}(t+\Delta t) - e^{-\bm{A}t}\bm{x}(t)
      = \int_{t}^{t+\Delta t}e^{-\bm{A}\tau}\bm{B}\bm{u}(\tau)d\tau
    \end{equation}
    and thus
    \begin{equation}\label{eq:trajectory_deltat}%
      \bm{x}(t+\Delta t)
      = e^{\bm{A}\Delta t}\bm{x}(t)
      + \int_{t}^{t+\Delta t}e^{\bm{A}(t+\Delta t -\tau)}\bm{B}\bm{u}(\tau)d\tau
    \end{equation}
  \item In particular, choosing $\Delta t = 1$ yields
    \begin{equation}\label{eq:trajectory_discrete0}%
      \bm{x}(t+1)
      = e^{\bm{A}}\bm{x}(t)
      + \int_{t}^{t+1}e^{\bm{A}(t+1-\tau)}\bm{B}\bm{u}(\tau)d\tau
      \quad \forall t = 0, 1, 2, \ldots,
    \end{equation}
  \item Let us consider the situation where one can only change the value of $u(t)$ at discrete points in time $t=0, 1, 2, \ldots$:
    \begin{equation}\nonumber%\label{eq:}%
      u(t) =
      \begin{cases}
      u_{0} \quad \forall t \in [0, 1) \\
      u_{1} \quad \forall t \in [1, 2) \\
      u_{2} \quad \forall t \in [2, 3) \\
      \vdots
      \end{cases}
    \end{equation}
    then
    \begin{equation}\nonumber%\label{eq:}%
      \int_{t}^{t+1}e^{\bm{A}(t+1-\tau)}\bm{B}\bm{u}(\tau)d\tau
      = 
      \int_{t}^{t+1}e^{\bm{A}(t+1-\tau)}d\tau\bm{B}\bm{u}_{t}
      = 
      \int_{0}^{1}e^{\bm{A}\tau}d\tau\bm{B}\bm{u}_{t}
      = 
      \bm{A}^{-1}(e^{\bm{A}}-\bm{I})\bm{B}\bm{u}_{t},
    \end{equation}
    and therefore,
    the state trajectory satisfies
    the following difference equations
    \begin{equation}\label{eq:trajectory_discrete}%
      \bm{x}(t+1)
      = \bm{A}_{d}\bm{x}(t)
      + \bm{B}_{d}\bm{u}_{t}
      \quad \forall t = 0, 1, 2, \ldots,
    \end{equation}
    where
    \begin{equation}\nonumber%\label{eq:}%
      \bm{A}_{d} := e^{\bm{A}},
      \quad
      \bm{B}_{d} := \bm{A}^{-1}(e^{\bm{A}}-\bm{I})\bm{B}
    \end{equation}
  \item \eqref{eq:trajectory_discrete}
    is a discretized version of the continuous model~\eqref{eq:model_general}
  \item Conversely, if a discrete-time dynamical system of the form
    \begin{equation}\nonumber%\label{eq:}%
      \bm{x}_{t+1} = \bm{A}\bm{x}_{t} + \bm{B}\bm{u}_{t} \quad \forall t = 0, 1, 2, \ldots
    \end{equation}
    is given,
    one can transform it into a corresponding continuous-time model as
    \begin{equation}\nonumber%\label{eq:}%
      \dot{\bm{x}}(t) = \bm{A}_{c}\bm{x}(t) + \bm{B}_{c}\bm{u}(t) \quad \forall t \in \R_{+},
    \end{equation}
    where
    \begin{equation}\nonumber%\label{eq:}%
      \bm{A}_{c} := \ln(\bm{A}),
      \quad
      \bm{B}_{c} = (\bm{A}-\bm{I})^{-1}\ln(\bm{A})\bm{B}
    \end{equation}
    
  \end{itemize}

\end{itemize}


\end{document}
