\documentclass[12pt,a4paper]{article} 
\usepackage[T1]{fontenc}
\usepackage{textcomp}
\usepackage{makeidx}

\usepackage[sc, noBBpl]{mathpazo}
\usepackage{dsfont} % for indicator function
\usepackage[deluxe]{otf}
\usepackage[utf8]{inputenc}

\usepackage{amsmath}
\usepackage{amsfonts}
\usepackage{amssymb}
\usepackage{amsthm}
\usepackage{mathtools}
\usepackage{mathrsfs}

\usepackage{listings} % for programming codes

\usepackage{bm}
\usepackage[dvipdfmx]{color}
\pagestyle{plain} 
\usepackage{float}
\usepackage[dvipdfmx]{graphicx}

\usepackage[left=15mm, right=18mm, top=20mm, bottom=20mm]{geometry}

\setlength{\footskip}{40pt} 
\setlength{\abovecaptionskip}{0pt}
\setlength{\belowcaptionskip}{0pt}
\renewcommand{\baselinestretch}{1.05}

\newcommand{\argmax}{\mathop{\rm arg max}\limits}
\newcommand{\argmin}{\mathop{\rm arg min}\limits}
\newcommand{\st}{\mathop{\rm s.t.}\limits}
\newcommand{\fsize}[1]{\fontsize{#1}{#1}\selectfont}

% custom color
\usepackage[dvipdfmx]{xcolor}

% colors
\definecolor{darkred}{rgb}{.40,.00,.00}
\definecolor{navy}{rgb}{.00,.00,.40}
\definecolor{darknavy}{rgb}{.00,.00,.30}
\definecolor{lightnavy}{rgb}{.10,.10,.40}

\newcommand{\navy}{\color{navy}}
\newcommand{\deepred}{\color{darkred}}
\newcommand{\emred}[1]{\textbf{\deepred #1}}
\newcommand{\emblue}[1]{\textbf{\navy #1}}

% title and tableofcontents
\usepackage[titles]{tocloft}
\setlength{\cftbeforesecskip}{2pt}
\setlength{\cftbeforesubsecskip}{0pt}
\renewcommand{\cftsecfont}{\normalsize\mcfamily}
\renewcommand{\cftsubsecfont}{\normalsize\mcfamily}
\renewcommand{\cftsubsecfont}{\normalsize\mcfamily}
\renewcommand{\contentsname}{\large\mcfamily\bfseries \S\hspace{0.5\baselineskip}Contents\\[-\baselineskip]}
\cftsetindents{section}{10.0pt}{15.0pt}
\cftsetindents{subsection}{20.0pt}{30.0pt}
\cftpagenumbersoff{section}
\cftpagenumbersoff{subsection}
\cftpagenumbersoff{subsubsection}

\makeatletter
\def\vhrulefill#1{\leavevmode\leaders\hrule\@height#1\hfill \kern\z@}
\makeatother

\makeatletter
\def\maketitle{%
	\begin{center}\leavevmode
	\normalfont
	{\normalsize\raggedleft \@date\par}%
	\vskip -0.7\baselineskip
	\vhrulefill{0.2pt}\par
	\vskip 0.3\baselineskip
	\renewcommand{\baselinestretch}{0.80}
	{\color{darknavy}\bfseries\Large\raggedright \@title\par}%
	\renewcommand{\baselinestretch}{1.00}
	\vskip 0.2\baselineskip
	\ifx\@subtitle\@empty\else%
	{\normalsize\raggedright \@subtitle\par}%
	\fi
	\ifx\@author\@empty\else%
	\vskip 0.5\baselineskip
	{\normalsize\raggedleft \@author\par}%
	\fi
	\end{center}%
	\vskip 1.0\baselineskip
}
\def\subtitle#1{\gdef\@subtitle{#1}}
\let\@subtitle\@empty
\makeatother

% section
\newcommand{\Secindent}{0.0\baselineskip}
\newcommand{\Subsecindent}{0.0\baselineskip}
\newcommand{\lmargini}{1.3\baselineskip}
\newcommand{\lmarginii}{0.5\baselineskip}
\newcommand{\lmarginiii}{0.2\baselineskip}
\def\theenumi{\arabic{enumi}}
\def\theenumii{\alph{enumii}}
\renewcommand{\labelenumi}{\hfill\hspace{60pt}\theenumi.}
\renewcommand{\labelenumii}{\hspace{30pt}\theenumii.}
\renewcommand{\labelitemi}{\small\textbullet\hspace{2pt}}
\renewcommand{\labelitemii}{\color{gray}\small$\circ$\hspace{2pt}}
\renewcommand{\labelitemiii}{\textendash\hspace{1pt}}
\makeatletter
\renewcommand{\section}{\@startsection%
	{section}{1}{\Secindent}% name, depth, indent \z@=0pt
	{-1.0\baselineskip \@plus1.5\baselineskip \@minus0.5\baselineskip}% beforeskip (no indent if negative)
	{0.5\baselineskip \@plus0.8\baselineskip \@minus0.3\baselineskip}% afterskip (no line break if negative)
	{\reset@font\large\bfseries\scshape\color{darknavy}}% font
}
\renewcommand{\subsection}{\@startsection%
	{subsection}{2}{\Subsecindent}%
	{-0.5\baselineskip \@plus0.7\baselineskip \@minus0.4\baselineskip}%
	{0.3\baselineskip \@plus0.5\baselineskip \@minus0.2\baselineskip}%
	{\reset@font\normalsize\bfseries\scshape\color{darknavy}}% 
}
\renewcommand{\thesection}{\@arabic\c@section} % arabic, roman, Roman, alph, Alph, fnsymbol
\renewcommand{\thesubsection}{\thesection\hspace{0pt}.\hspace{0pt}\@arabic\c@subsection}
\makeatother

\makeatletter
\def\@listi{%
	\topsep = 0.2\baselineskip
	\partopsep = 0.0\baselineskip
	\parsep = 0.0\baselineskip
	\itemsep = 0.5\baselineskip
	\leftmargin = 1.5zw
	\labelsep = 0.5zw
	\itemindent = 0pt
	\labelwidth\leftmargin \advance\labelwidth-\labelsep
	\listparindent = 0zw
	\rightmargin = 0pt
	\reset@font\normalsize\normalfont
}
\let\@listI\@listi
\@listi
\def\@listii{%
	\topsep = 0.25\baselineskip
	\partopsep = 0.0\baselineskip
	\parsep = 0.0\baselineskip
	\itemsep = 0.3\baselineskip
	\leftmargin = 0.5zw
	\labelsep = .5zw
	\itemindent = 0pt
	\labelwidth\leftmargin \advance\labelwidth-\labelsep
	\listparindent = 0zw
	\rightmargin = 0pt
	\reset@font\normalsize\normalfont%
}
\def\@listiii{%
	\topsep = 0.30\baselineskip
	\partopsep = 0.0\baselineskip
	\parsep = 0.1\baselineskip
	\itemsep = 0.05\baselineskip
	\leftmargin = 1.1zw
	\labelsep = .5zw
	\itemindent = 0pt
	\labelwidth\leftmargin \advance\labelwidth-\labelsep
	\listparindent = 0zw
	\rightmargin = 0pt
	\reset@font\normalsize\normalfont%
}
\def\@listiv{%
	\topsep = 0.0\baselineskip
	\partopsep = 0.0\baselineskip
	\parsep = 0.0\baselineskip
	\itemsep = 0.0\baselineskip
	\leftmargin = 0.9zw
	\labelsep = .5zw
	\itemindent = 0pt
	\labelwidth\leftmargin \advance\labelwidth-\labelsep
	\listparindent = 0zw
	\rightmargin = 0pt
	\reset@font\normalsize\normalfont%
}
\makeatother

\newcommand{\R}{\ensuremath{\mathbb{R}}}
\newcommand{\E}{\ensuremath{\mathbb{E}}}
\newcommand{\V}{\ensuremath{\mathbb{V}}}
\newcommand{\Cov}{\ensuremath{\text{Cov}}}
\newcommand{\N}{\ensuremath{\mathbb{N}}}
\newcommand{\Z}{\ensuremath{\mathbb{Z}}}
\let\epsilon\varepsilon

\let\origitem\item
\renewcommand{\item}{\normalfont\origitem}

\usepackage[%
backref=page,
dvipdfmx,
bookmarks=true,bookmarksnumbered=true,bookmarkstype=toc,
colorlinks=true,
linkcolor=navy,
citecolor=navy,
filecolor=blue,
pagecolor=blue,
urlcolor=blue
]{hyperref}
\AtBeginDvi{\special{pdf:tounicode EUC-UCS2}}

\renewcommand*{\backref}[1]{}
\renewcommand*{\backrefalt}[4]{%
  \ifcase #1 (Not cited.)%
  \or        [#2]%
  \else      [#2]%
  \fi}
 

\title{Matrix exponential}
\subtitle{Introduction to dynamical systems~\#5}
\author{Hiroaki Sakamoto}
\date{\today}

\begin{document}
\maketitle
\tableofcontents

\section{Exponential function}

\subsection{Real exponential function}

\begin{itemize}
\item \textbf{Definition}
  \begin{itemize}
  \item Recall that the real exponential function $e^{a}$ can be defined as
    \begin{equation}\nonumber%\label{eq:}%
      e^{a} := \sum_{k=0}^{\infty} \frac{1}{k!}a^{k} = 1 + a + \frac{1}{2!}a^{2} + \frac{1}{3!}a^{3} + \ldots,
    \end{equation}
    which is well defined for every $a\in \R$ (why?)
  \end{itemize}

\item \textbf{Properties}
  \begin{itemize}
  \item $e^{0} = 1$
  \item $e^{a} >0$
  \item $e^{a+b} = e^{a}e^{b}$
  \item $(e^{a})^{k} = e^{ka}$
  \item $\frac{d}{dt}e^{at} = ae^{at}$
  \item $\lim_{t\to\infty}e^{at} = 0$ if and only if $a<0$
  \end{itemize}

\end{itemize}

\subsection{Matrix exponential function}

\begin{itemize}
\item \textbf{Definition}
  \begin{itemize}
  \item Similar to the real exponential function, we define the matrix exponential function as
    \begin{equation}\nonumber%\label{eq:}%
      e^{\bm{A}} := 
      \sum_{k=0}^{\infty} \frac{1}{k!}\bm{A}^{k}
      = \bm{I} + \bm{A} + \frac{1}{2!}\bm{A}^{2} + \frac{1}{3!}\bm{A}^{3} + \ldots,
    \end{equation}
    which is also well defined for every square matrix $\bm{A}$ (why?)
  \item Given $\bm{A}$, if there exists a matrix $\bm{B}$ such that
    \begin{equation}\nonumber%\label{eq:}%
      e^{\bm{B}} = \bm{A},
    \end{equation}
    we call it the logarithm of $\bm{A}$ and write $\ln(\bm{A}):=\bm{B}$
  \end{itemize}

  \clearpage
\item \textbf{Properties}
  \begin{enumerate}
  \item $e^{\bm{O}} = \bm{I}$ where $\bm{O}$ is the zero matrix
  \item $e^{\bm{A}^{\top}} = \left(e^{\bm{A}}\right)^{\top}$
  \item If $\bm{A}$ and $\bm{B}$ commute\footnote{%
        We say that two matrices
        $\bm{A}, \bm{B}$ commute if $\bm{A}\bm{B} = \bm{B}\bm{A}$.
        Obviously, $\bm{A}^{k}$ and $\bm{A}^{l}$
        commute for any $k, l\in \N$.  Also, if $\bm{A}$ and $\bm{B}$ commute,
        so do $(\bm{A}+\bm{B})$ and $\bm{A}$.
        Moreover, if $\bm{A}$ and $\bm{B}$ commute, so do $\bm{A}$ and $\bm{B}^{k}$ for any $k$.
      },
      then $\bm{B}e^{\bm{A}t} = e^{\bm{A}t}\bm{B}$ for any $t\in\R$
    \item If $\bm{A}$ is non-singular, then $e^{\bm{A}\bm{B}\bm{A}^{-1}} = \bm{A}e^{\bm{B}}\bm{A}^{-1}$
    \item For any $s, t \in \R$, we have $e^{\bm{A}s}e^{\bm{A}t} = e^{\bm{A}(s+t)}$
    \item For any $\bm{A}$, $e^{\bm{A}}$ is non-singular and $(e^{\bm{A}})^{-1}=e^{-\bm{A}}$
    \item $\frac{d}{dt}e^{\bm{A}t} = \bm{A}e^{\bm{A}t}$
    \item If $\bm{A}$ and $\bm{B}$ commute, then $e^{\bm{A}}e^{\bm{B}} = e^{\bm{A}+\bm{B}}$
    \item For any $k\in \N$, $(e^{\bm{A}})^{k} = e^{\bm{A}k}$
    \item If $\bm{A}$ and $\bm{B}$ commute, so do $e^{\bm{A}}$ and $e^{\bm{B}}$
    \item If $(\lambda, \bm{v})$ is an eigenpair of $\bm{A}$, then
      $(e^{\lambda t}, \bm{v})$ is an eigenpair of $e^{\bm{A}t}$ for all $t\in \R$
    \item $\lim_{t\to\infty}e^{\bm{A}t}=\bm{O}$ if and only if the eigenvalues of $\bm{A}$ are all negative
    \item If $\bm{A}$ is a diagonal matrix, so is $e^{\bm{A}}$ and
      \begin{equation}\nonumber%\label{eq:}%
        \bm{A}
        =
        \begin{bmatrix}
          a_{1} & 0 & \cdots & 0 \\
          0 & a_{2} & \cdots & 0 \\
          \vdots & \vdots & \ddots & \vdots \\
          0 & 0 & \cdots & a_{m} \\
        \end{bmatrix}
        \implies
        e^{\bm{A}} = 
        \begin{bmatrix}
          e^{a_{1}} & 0 & \cdots & 0 \\
          0 & e^{a_{2}} & \cdots & 0 \\
          \vdots & \vdots & \ddots & \vdots \\
          0 & 0 & \cdots & e^{a_{m}} \\
        \end{bmatrix}
      \end{equation}
    \item If $\bm{J}_{m}(\lambda)$ is a Jordan block of size $m$, then
      \begin{equation}\nonumber%\label{eq:}%
        e^{\bm{J}_{m}(\lambda)t}
        =
        \begin{bmatrix}
          e^{\lambda t} & \frac{t}{1!}e^{\lambda t} & \frac{t^{2}}{2!}e^{\lambda t} & \cdots & \frac{t^{m-1}}{(m-1)!}e^{\lambda t} \\
          0 & e^{\lambda t} & \frac{t}{1!}e^{\lambda t} & \cdots & \frac{t^{m-2}}{(m-2)!}e^{\lambda t} \\
          0 & 0 & e^{\lambda t} & \ddots & \vdots \\
          \vdots & \vdots & \vdots & \ddots & \vdots \\
          0 & 0 & 0  & 0 & e^{\lambda t} \\
        \end{bmatrix}
        \quad \forall t \in \R
      \end{equation}
    \item For a Jordan matrix $\bm{J}\in \R^{n\times n}$,
    \begin{equation}\nonumber%\label{eq:}%
      \bm{J} =
      \begin{bmatrix}
        \bm{J}_{n_{1}}(\lambda_{1}) & \bm{O} & \cdots & \bm{O} \\
        \bm{O} & \bm{J}_{n_{2}}(\lambda_{2}) & \cdots & \bm{O} \\
        \vdots & \vdots & \ddots & \vdots \\
        \bm{O} & \bm{O} & \cdots & \bm{J}_{n_{d}}(\lambda_{d}) \\
      \end{bmatrix}
      \implies
      e^{\bm{J}t}
      =
      \begin{bmatrix}
        e^{\bm{J}_{n_{1}}(\lambda_{1})t} & \bm{O} & \cdots & \bm{O} \\
        \bm{O} & e^{\bm{J}_{n_{2}}(\lambda_{2})t} & \cdots & \bm{O} \\
        \vdots & \vdots & \ddots & \vdots \\
        \bm{O} & \bm{O} & \cdots & e^{\bm{J}_{n_{d}}(\lambda_{d})t} \\
      \end{bmatrix}
      \quad \forall t \in \R
    \end{equation}

    \item If $\mu$ is an eigenvalue of $e^{\bm{A}}$,
      then $\mu>0$ and $\lambda:=\ln(\mu)$ is an eigenvalue of $\bm{A}$

    \item The eigenvalues of $\bm{A}$ are all negative if and only if $\rho(e^{\bm{A}})< 1$

    \end{enumerate}

    \clearpage

\item \textbf{Proof}
  \begin{enumerate}
    \item $e^{\bm{O}} = \bm{I}$ where $\bm{O}$ is the zero matrix (by definition)
    \item $e^{\bm{A}^{\top}} = \left(e^{\bm{A}}\right)^{\top}$ because
      \begin{align}
        e^{\bm{A}^{\top}}
        & := I + \bm{A}^{\top} + \frac{1}{2!}(\bm{A}^{\top})^{2} + \frac{1}{3!}(\bm{A}^{\top})^{3} + \cdots  \nonumber \\
        & = I^{\top} + \bm{A}^{\top} + \frac{1}{2!}(\bm{A}^{2})^{\top} + \frac{1}{3!}(\bm{A}^{3})^{\top} + \cdots \nonumber \\
        & = \left(I + \bm{A} + \frac{1}{2!}\bm{A}^{2} + \frac{1}{3!}\bm{A}^{3} + \cdots\right)^{\top} = (e^{\bm{A}})^{\top}
          \nonumber%\label{eq:}%
      \end{align}
    \item If $\bm{A}$ and $\bm{B}$ commute,
      then $\bm{B}e^{\bm{A}t} = e^{\bm{A}t}\bm{B}$ for any $t\in\R$ because
      \begin{align}
       \bm{B}e^{\bm{A}t} 
       & = \bm{B} + \bm{B}\bm{A}t + \frac{t^{2}}{2!}\bm{B}\bm{A}^{2} + \frac{t^{3}}{3!}\bm{B}\bm{A}^{3} + \cdots \nonumber \\
       & = \bm{B} + \bm{A}\bm{B}t + \frac{t^{2}}{2!}\bm{A}^{2}\bm{B} + \frac{t^{3}}{3!}\bm{A}^{3}\bm{B} + \cdots = e^{\bm{A}t}\bm{B}
         \nonumber%\label{eq:}%
      \end{align}
    \item If $\bm{A}$ is non-singular, then $e^{\bm{A}\bm{B}\bm{A}^{-1}} = \bm{A}e^{\bm{B}}\bm{A}^{-1}$
      because
      \begin{align}
        e^{\bm{A}\bm{B}\bm{A}^{-1}}
        & := \bm{I} + \bm{A}\bm{B}\bm{A}^{-1} + \frac{1}{2!}(\bm{A}\bm{B}\bm{A}^{-1})^{2} + \frac{1}{3!}(\bm{A}\bm{B}\bm{A}^{-1})^{3} + \cdots  \nonumber \\
        & = \bm{A}\bm{A}^{-1} + \bm{A}\bm{B}\bm{A}^{-1} + \frac{1}{2!}\bm{A}\bm{B}^{2}\bm{A}^{-1} + \frac{1}{3!}\bm{A}\bm{B}^{3}\bm{A}^{-1} + \cdots  \nonumber \\
        & = \bm{A} \left(\bm{I} + \bm{B} + \frac{1}{2!}\bm{B}^{2} + \frac{1}{3!}\bm{B}^{3} + \cdots \right)\bm{A}^{-1} = \bm{A}e^{\bm{B}}\bm{A}^{-1}
          \nonumber%\label{eq:}%
      \end{align}
    \item For any $s, t \in \R$, we have $e^{\bm{A}s}e^{\bm{A}t} = e^{\bm{A}(s+t)}$
      because
      \begin{align}
        e^{\bm{A}s}e^{\bm{A}t}
        & 
          = \left(\sum_{j=0}^{\infty} \frac{\bm{A}^{j}s^{j}}{j!}\right)\left(\sum_{k=0}^{\infty}\frac{\bm{A}^{k}t^{k}}{k!}\right)
          = \sum_{j=0}^{\infty}\sum_{k=0}^{\infty}\frac{\bm{A}^{j}s^{j}}{j!}\frac{\bm{A}^{k}t^{k}}{k!} \nonumber \\
        &
          = \sum_{j=0}^{\infty}\sum_{l=j}^{\infty}\frac{\bm{A}^{j}s^{j}}{j!}\frac{\bm{A}^{l-j}t^{l-j}}{(l-j)!}
          = \sum_{l=0}^{\infty}\sum_{j=0}^{l}\frac{\bm{A}^{j}s^{j}}{j!}\frac{\bm{A}^{l-j}t^{l-j}}{(l-j)!} \nonumber \\
        &
          = \sum_{l=0}^{\infty}\frac{\bm{A}^{l}}{l!}\underbrace{\sum_{j=0}^{l}\frac{l!}{j!(l-j)!}s^{j}t^{l-j}}_{=(s+t)^{l} \text{ binomial theorem}}
          = \sum_{j=0}^{\infty}\frac{(\bm{A}(s+t))^{l}}{l!}
          = e^{\bm{A}(s+t)}
          \nonumber%\label{eq:}%
      \end{align}
      where the fourth equality uses the fact that
      \begin{align}
        \sum_{j=0}^{\infty}\sum_{l=j}^{\infty}x_{j,l}
          & = x_{0,0} + x_{0,1} + x_{0,2} + x_{0,3} + \cdots  \nonumber \\[-13pt]
          & \qquad\quad\hspace{-1pt} + x_{1,1} + x_{1,2} + x_{1,3} + x_{1,4} + \cdots  \nonumber \\
          & \qquad\qquad\quad\hspace{8pt} + x_{2,2} + x_{2,3} + x_{2,4} + x_{2,5} + \cdots  \nonumber \\
          & = \sum_{j=0}^{0}x_{j,0} + \sum_{j=0}^{1}x_{j,1} + \sum_{j=0}^{2}x_{j,2} + \sum_{j=0}^{3}x_{j,3} + \cdots = \sum_{l=0}^{\infty}\sum_{j=0}^{l}x_{j,l} %\nonumber \\
      \nonumber%\label{eq:}%
      \end{align}
    \item $e^{\bm{A}}$ is invertible for any $\bm{A}$
      and $(e^{\bm{A}})^{-1}=e^{-\bm{A}}$ because
      \begin{equation}\nonumber%\label{eq:}%
        e^{\bm{A}}e^{-\bm{A}} = e^{\bm{A}\times 1}e^{\bm{A}\times -1} = e^{\bm{A}\times(1-1)} = e^{\bm{A}\times 0} = e^{\bm{O}} = \bm{I}
      \end{equation}
    \item $\frac{d}{dt}e^{\bm{A}t} = \bm{A}e^{\bm{A}t}$
      because defining $\bm{F}:\R\to\R^{n\times n}$ by $\bm{F}(t):=e^{\bm{A}t}$ gives
      \begin{align}
        \frac{d \bm{F}(t)}{dt}
        & := \lim_{h\to 0}\frac{\bm{F}(t+h)-\bm{F}(t)}{h}
        = \lim_{h\to 0}\frac{e^{\bm{A}(t+h)} - e^{\bm{A}t}}{h}
        = \lim_{h\to 0}\frac{e^{\bm{A}t}e^{\bm{A}h} - e^{\bm{A}t}}{h}
        = e^{\bm{A}t}\lim_{h\to 0}\frac{e^{\bm{A}h} - I}{h}  \nonumber \\
        & = e^{\bm{A}t}\lim_{h\to 0}\frac{\frac{1}{1!}\bm{A}h + \frac{1}{2!}\bm{A}^{2}h^{2} + \frac{1}{3!}\bm{A}^{3}h^{3}+\cdots }{h}  \nonumber \\
        & = e^{\bm{A}t}\lim_{h\to 0}\left(\frac{1}{1!}\bm{A} + \frac{1}{2!}\bm{A}^{2}h + \frac{1}{3!}\bm{A}^{3}h^{2} + \cdots \right) = e^{\bm{A}t}\bm{A} = \bm{A} e^{\bm{A}t}
        \nonumber%\label{eq:}%
      \end{align}
    \item If $\bm{A}\in \R^{n\times n}$ and $\bm{B}\in \R^{n\times n}$ commute, then $e^{\bm{A}}e^{\bm{B}} = e^{\bm{A}+\bm{B}}$ because
      defining $\bm{F}:\R \to \R^{n\times n}$ by
      \begin{equation}\nonumber%\label{eq:}%
        \bm{F}(t) := e^{(\bm{A}+\bm{B})t}e^{-\bm{B}t}e^{-\bm{A}t} \quad \forall t \in \R
      \end{equation}
      yields
      \begin{align}
        \frac{d \bm{F}(t)}{dt}
          & = (\bm{A}+\bm{B})e^{(\bm{A}+\bm{B})t}e^{-\bm{A}t}e^{-\bm{B}t} + e^{(\bm{A}+\bm{B})t}e^{-\bm{A}t}(-\bm{B})e^{-\bm{B}t} + e^{(\bm{A}+\bm{B})t}(-\bm{A})e^{-\bm{A}t}e^{-\bm{B}t} \nonumber \\
          & = (\bm{A}+\bm{B})e^{(\bm{A}+\bm{B})t}e^{-\bm{A}t}e^{-\bm{B}t} -\bm{B}e^{(\bm{A}+\bm{B})t}e^{-\bm{A}t}e^{-\bm{B}t} -\bm{A}e^{(\bm{A}+\bm{B})t}e^{-\bm{A}t}e^{-\bm{B}t} \nonumber \\
          & = (\bm{A}+\bm{B}-\bm{B}-\bm{A})e^{(\bm{A}+\bm{B})t}e^{-\bm{A}t}e^{-\bm{B}t} = \bm{O},
      \nonumber%\label{eq:}%
      \end{align}
      which means $\bm{F}(t)$ is independent of $t$ and in particular
      \begin{equation}\nonumber%\label{eq:}%
        e^{\bm{A}+\bm{B}}e^{-\bm{B}}e^{-\bm{A}} = \bm{F}(1) = \bm{F}(0) = e^{(\bm{A}+\bm{B})0}e^{-\bm{B}0}e^{-\bm{A}0} = \bm{I}.
      \end{equation}
    \item For any $k\in \N$, $(e^{\bm{A}})^{k} = e^{\bm{A}k}$ because
      $\bm{A}$ and $\bm{A}$ commute and thus
      \begin{equation}\nonumber%\label{eq:}%
        (e^{\bm{A}})^{2} = e^{\bm{A}}e^{\bm{A}} = e^{\bm{A}+\bm{A}} = e^{\bm{A}2}
      \end{equation}
      
    \item If $\bm{A}$ and $\bm{B}$ commute, so do $e^{\bm{A}}$ and $e^{\bm{B}}$ because
      \begin{equation}\nonumber%\label{eq:}%
        e^{\bm{A}}e^{\bm{B}} = e^{\bm{A}+\bm{B}} = e^{\bm{B}+\bm{A}} = e^{\bm{B}}e^{\bm{A}}
      \end{equation}

    \item If $(\lambda, \bm{v})$ is an eigenpair of $\bm{A}$, then
      $e^{\bm{A}t}\bm{v} = e^{\lambda t}\bm{v}$ because
      \begin{equation}\nonumber%\label{eq:}%
        \bm{A}^{k}\bm{v}
        = \lambda \bm{A}^{k-1}\bm{v}
        = \lambda^{2} \bm{A}^{k-2}\bm{v}
        = \lambda^{k}\bm{v}
        \quad \forall k \in \N
      \end{equation}
      and therefore
      \begin{align}
        e^{\bm{A}t}\bm{v}
          & = \left(I + \frac{1}{1!}\bm{A}t + \frac{1}{2!}\bm{A}^{2}t^{2} + \frac{1}{3!}\bm{A}^{3}t^{3} + \cdots \right)\bm{v}  \nonumber \\
          & = \left(\bm{v} + \frac{1}{1!}\bm{A}\bm{v}t + \frac{1}{2!}\bm{A}^{2}\bm{v}t^{2} + \frac{1}{3!}\bm{A}^{3}\bm{v}t^{3} + \cdots \right)  \nonumber \\
          & = \left(\bm{v} + \frac{1}{1!}\lambda \bm{v} t + \frac{1}{2!}\lambda^{2}\bm{v}t^{2} + \frac{1}{3!}\lambda^{3}\bm{v}t^{3} + \cdots \right)  \nonumber \\
          & = \left(1 + \frac{1}{1!}(\lambda t) + \frac{1}{2!}(\lambda t)^{2} + \frac{1}{3!}(\lambda t)^{3} + \cdots \right)\bm{v} = e^{\lambda t}\bm{v}
      \nonumber%\label{eq:}%
      \end{align}

    \item $\lim_{t\to\infty}e^{\bm{A}t}=\bm{O}$ if and only if the eigenvalues of $\bm{A}$ are all negative because
      \begin{equation}\nonumber%\label{eq:}%
        \lim_{k\to\infty} (e^{\bm{A}})^{k}
        = \bm{O}
        \iff
        \rho(e^{\bm{A}})<1
        \iff
        \max\{e^{\lambda_{1}},\ldots, e^{\lambda_{n}}\} <1
        \iff
        \max\{\lambda_{1},\ldots, \lambda_{n}\} < 0
      \end{equation}

    \item If $\bm{A}$ is a diagonal matrix, so is $e^{\bm{A}}$ and
      \begin{equation}\nonumber%\label{eq:}%
        \bm{A}
        =
        \begin{bmatrix}
          a_{1} & 0 & \cdots & 0 \\
          0 & a_{2} & \cdots & 0 \\
          \vdots & \vdots & \ddots & \vdots \\
          0 & 0 & \cdots & a_{m} \\
        \end{bmatrix}
        \implies
        e^{\bm{A}} = 
        \begin{bmatrix}
          e^{a_{1}} & 0 & \cdots & 0 \\
          0 & e^{a_{2}} & \cdots & 0 \\
          \vdots & \vdots & \ddots & \vdots \\
          0 & 0 & \cdots & e^{a_{m}} \\
        \end{bmatrix}
      \end{equation}
      because
      \begin{equation}\nonumber%\label{eq:}%
        e^{\bm{A}} =
        \sum_{k=0}^{\infty} \frac{1}{k!}\bm{A}^{k}
        =
        \begin{bmatrix}
          \sum_{k=0}^{\infty} \frac{1}{k!}a_{1}^{k} & 0 & \cdots & 0 \\
          0 & \sum_{k=0}^{\infty} \frac{1}{k!}a_{2}^{k} & \cdots & 0 \\
          \vdots & \vdots & \ddots & \vdots \\
          0 & 0 & \cdots & \sum_{k=0}^{\infty} \frac{1}{k!}a_{m}^{k} \\
        \end{bmatrix}
        =
        \begin{bmatrix}
          e^{a_{1}} & 0 & \cdots & 0 \\
          0 & e^{a_{2}} & \cdots & 0 \\
          \vdots & \vdots & \ddots & \vdots \\
          0 & 0 & \cdots & e^{a_{m}} \\
        \end{bmatrix}
      \end{equation}

    \item If $\bm{J}_{m}(\lambda)$ is a Jordan block of size $m$, then
      \begin{equation}\nonumber%\label{eq:}%
        e^{\bm{J}_{m}(\lambda)t}
        =
        \begin{bmatrix}
          e^{\lambda t} & \frac{t}{1!}e^{\lambda t} & \frac{t^{2}}{2!}e^{\lambda t} & \cdots & \frac{t^{m-1}}{(m-1)!}e^{\lambda t} \\
          0 & e^{\lambda t} & \frac{t}{1!}e^{\lambda t} & \cdots & \frac{t^{m-2}}{(m-2)!}e^{\lambda t} \\
          0 & 0 & e^{\lambda t} & \ddots & \vdots \\
          \vdots & \vdots & \vdots & \ddots & \vdots \\
          0 & 0 & 0  & 0 & e^{\lambda t} \\
        \end{bmatrix}
        \quad \forall t \in \R
      \end{equation}
      because
    \begin{equation}\nonumber%\label{eq:Jnit}%
      (\bm{J}_{m}(\lambda)t)^{k}
      =
      t^{k}
      \begin{bmatrix}
        c_{0}(k) & c_{1}(k) & c_{2}(k) & \cdots & c_{m-1}(k) \\
        0 & c_{0}(k) & c_{1}(k) & \cdots & c_{m-2}(k) \\
        0 & 0 & c_{0}(k) & \ddots & \vdots \\
        \vdots & \vdots & \vdots & \ddots & \vdots \\
        0 & 0 & 0  & 0 & c_{0}(k) \\
      \end{bmatrix},
      \quad
      c_{l}(k) :=
      \begin{cases}
        \frac{k!}{l!(k-l)!}\lambda^{k-l} & k\geq l \\
        0 & k < l \\
      \end{cases}
    \end{equation}
    and
    \begin{equation}\nonumber%\label{eq:}%
      e^{\bm{J}_{m}(\lambda)t}
      := \sum_{k=0}^{\infty}\frac{1}{k!}(\bm{J}_{m}(\lambda)t)^{k}
      =
      \begin{bmatrix}
        \sum_{k=0}^{\infty}\frac{t^{k}}{k!}c_{0}(k) & \sum_{k=0}^{\infty}\frac{t^{k}}{k!}c_{1}(k) & \sum_{k=0}^{\infty}\frac{t^{k}}{k!}c_{2}(k) & \cdots & \sum_{k=0}^{\infty}\frac{t^{k}}{k!}c_{m-1}(k) \\
        0 & \sum_{k=0}^{\infty}\frac{t^{k}}{k!}c_{0}(k) & \sum_{k=0}^{\infty}\frac{t^{k}}{k!}c_{1}(k) & \cdots & \sum_{k=0}^{\infty}\frac{t^{k}}{k!}c_{m-2}(k) \\
        0 & 0 & \sum_{k=0}^{\infty}\frac{t^{k}}{k!}c_{0}(k) & \ddots & \vdots \\
        \vdots & \vdots & \vdots & \ddots & \vdots \\
        0 & 0 & 0  & 0 & \sum_{k=0}^{\infty}\frac{t^{k}}{k!}c_{0}(k) \\
      \end{bmatrix}
    \end{equation}
    where
    \begin{equation}\nonumber%\label{eq:}%
      \sum_{k=0}^{\infty}\frac{t^{k}}{k!}c_{l}(k)
      = 
      \sum_{k=l}^{\infty}\frac{t^{k}}{k!}c_{l}(k)
      = 
      \sum_{k=l}^{\infty}\frac{t^{l}t^{k-l}}{k!}\frac{k!}{l!(t-l)!}\lambda_{i}^{k-l}
      = 
      \frac{t^{l}}{l!}\sum_{k=l}^{\infty}\frac{1}{(k-l)!}(\lambda_{i}t)^{k-l}
      = 
      \frac{t^{l}}{l!}e^{\lambda_{i}t}
    \end{equation}

    \item For a Jordan matrix $\bm{J}\in \R^{n\times n}$,
    \begin{equation}\nonumber%\label{eq:}%
      \bm{J} =
      \begin{bmatrix}
        \bm{J}_{n_{1}}(\lambda_{1}) & \bm{O} & \cdots & \bm{O} \\
        \bm{O} & \bm{J}_{n_{2}}(\lambda_{2}) & \cdots & \bm{O} \\
        \vdots & \vdots & \ddots & \vdots \\
        \bm{O} & \bm{O} & \cdots & \bm{J}_{n_{d}}(\lambda_{d}) \\
      \end{bmatrix}
      \implies
      e^{\bm{J}t}
      =
      \begin{bmatrix}
        e^{\bm{J}_{n_{1}}(\lambda_{1})t} & \bm{O} & \cdots & \bm{O} \\
        \bm{O} & e^{\bm{J}_{n_{2}}(\lambda_{2})t} & \cdots & \bm{O} \\
        \vdots & \vdots & \ddots & \vdots \\
        \bm{O} & \bm{O} & \cdots & e^{\bm{J}_{n_{d}}(\lambda_{d})t} \\
      \end{bmatrix}
    \end{equation}
    because
    \begin{equation}\nonumber%\label{eq:}%
      e^{\bm{J}t}
      := \sum_{k=0}^{\infty}\frac{t^{k}}{k!}\bm{J}^{k}
      =      
      \begin{bmatrix}
        \sum_{k=0}^{\infty}\frac{1}{k!}(\bm{J}_{n_{1}}(\lambda_{1})t)^{k} & \bm{O} & \cdots & \bm{O} \\
        \bm{O} & \sum_{k=0}^{\infty}\frac{1}{k!}(\bm{J}_{n_{2}}(\lambda_{2})t)^{k} & \cdots & \bm{O} \\
        \vdots & \vdots & \ddots & \vdots \\
        \bm{O} & \bm{O} & \cdots & \sum_{k=0}^{\infty}\frac{1}{k!}(\bm{J}_{n_{d}}(\lambda_{d})t)^{k} \\
      \end{bmatrix}
    \end{equation}
    
  \item Decompose $\bm{A}$ as $\bm{V}\bm{J}\bm{V}^{-1}$
    where $\bm{J}$ is the Jordan normal form of $\bm{A}$.
    Since $e^{\bm{A}} = \bm{V}e^{\bm{J}}\bm{V}^{-1}$,
    we know that $e^{\bm{A}}$ and $e^{\bm{J}}$ are similar.
    Because $e^{\bm{J}}$ is a triangle matrix,
    it follows that eigenvalues of $e^{\bm{A}}$ are diagonal elements of $e^{\bm{J}}$,
    which is $e^{\lambda_{1}}, e^{\lambda_{2}}, \ldots, e^{\lambda_{n}}$
    where $\lambda_{1}, \lambda_{2}, \ldots, \lambda_{n}$ are eigenvalues of $\bm{A}$.
    Therefore, if $\mu$ is an eigenvalue of $e^{\bm{A}}$,
    there must exist an eigenvalue $\lambda$ of $\bm{A}$ such that $\mu = e^{\lambda}>0$.

  \item Immediate from the results (11 and 16) above

  \end{enumerate}
\end{itemize}

\subsection{Examples}

\begin{itemize}

\item \textbf{Example~1}
  \begin{itemize}
  \item Consider a square matrix
    \begin{equation}\nonumber%\label{eq:}%
      \bm{A} :=
      \begin{bmatrix}
        1 & 0 & 0 \\
        6 & -2 & -6 \\
        -2 & 1 & 3 \\
      \end{bmatrix}
    \end{equation}
  \item Let us say we want to compute the exponential of $\bm{A}$:
    \begin{equation}\nonumber%\label{eq:}%
      e^{\bm{A}} := \bm{I} + \bm{A} + \frac{1}{2!}\bm{A}^{2} + \frac{1}{3!}\bm{A}^{3} + \ldots,
    \end{equation}
    the right-hand side of which is not easy to deal with
  \item Notice that $\bm{A}$ is decomposed as
    \begin{equation}\nonumber%\label{eq:}%
      \begin{bmatrix}
        1 & 0 & 0 \\
        6 & -2 & -6 \\
        -2 & 1 & 3 \\
      \end{bmatrix}
      =
      \underbrace{
      \begin{bmatrix}
        0 & 1 & 1 \\
        -3 & 2 & 0\\
        1 & 0 & 1 \\
      \end{bmatrix}}_{\bm{V}}
    \underbrace{
      \begin{bmatrix}
        0 & 0 & 0 \\
        0 & 1 & 0 \\
        0 & 0 & 1 \\
      \end{bmatrix}}_{\bm{\Lambda}}
    \underbrace{
      \begin{bmatrix}
        2 & -1 & -2 \\
        3 & -1 & -3 \\
        -2 & 1 & 3 \\
      \end{bmatrix}}_{\bm{V}^{-1}}
    \end{equation}
    so
    \begin{equation}
      e^{\bm{A}}
         = e^{\bm{V}\bm{\Lambda}\bm{V}^{-1}}
         = \bm{V}e^{\bm{\Lambda}}\bm{V}^{-1}
         = 
      \begin{bmatrix}
        0 & 1 & 1 \\
        -3 & 2 & 0\\
        1 & 0 & 1 \\
      \end{bmatrix}
      \begin{bmatrix}
        1 & 0 & 0 \\
        0 & e & 0 \\
        0 & 0 & e \\
      \end{bmatrix}
      \begin{bmatrix}
        2 & -1 & -2 \\
        3 & -1 & -3 \\
        -2 & 1 & 3 \\
      \end{bmatrix}
    =
    \begin{bmatrix}
      e & 0 & 0 \\
      -6+6e & 3-2e & 6-6e \\
      2-2e & -1+e & -2+3e \\
    \end{bmatrix}
    \nonumber%\label{eq:}%
  \end{equation}

  \end{itemize}

\item \textbf{Example~2}
  \begin{itemize}
  \item Consider a square matrix
    \begin{equation}\nonumber%\label{eq:}%
      \bm{A} :=
      \begin{bmatrix}
        4 & 1 \\
        -1 & 2 \\
      \end{bmatrix}
    \end{equation}
  \item Let us say we want to compute $e^{\bm{A}t}$ for each real $t\in \R$
  \item Notice that $\bm{A}$ is decomposed as
    \begin{equation}\nonumber%\label{eq:}%
      \begin{bmatrix}
        4 & 1 \\
        -1 & 2 \\
      \end{bmatrix}
      =
      \underbrace{
      \begin{bmatrix}
        1 & 1 \\
        -1 & 0 \\
      \end{bmatrix}}_{\bm{V}}
    \underbrace{
      \begin{bmatrix}
        3 & 1 \\
        0 & 3 \\
      \end{bmatrix}}_{\bm{J}}
    \underbrace{
      \begin{bmatrix}
        0 & -1 \\
        1 & 1 \\
      \end{bmatrix}}_{\bm{V}^{-1}}
    \end{equation}
    so
    \begin{equation}
      e^{\bm{A}t}
         = \bm{V}e^{\bm{J}t}\bm{V}^{-1}
         =
      \begin{bmatrix}
        1 & 1 \\
        -1 & 0 \\
      \end{bmatrix}
      \begin{bmatrix}
        e^{3t} & te^{3t} \\
        0 & e^{3t} \\
      \end{bmatrix}
      \begin{bmatrix}
        0 & -1 \\
        1 & 1 \\
      \end{bmatrix}
      =
      \begin{bmatrix}
        te^{3t} + e^{3t} & te^{3t} \\
        -te^{3t} & e^{3t} -te^{3t} \\
      \end{bmatrix}
    \nonumber%\label{eq:}%
  \end{equation}

  \end{itemize}
\end{itemize}

\end{document}
