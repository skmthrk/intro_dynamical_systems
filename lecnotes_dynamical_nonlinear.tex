\documentclass[12pt,a4paper]{article} 
\usepackage[T1]{fontenc}
\usepackage{textcomp}
\usepackage{makeidx}

\usepackage[sc, noBBpl]{mathpazo}
\usepackage{dsfont} % for indicator function
\usepackage[deluxe]{otf}
\usepackage[utf8]{inputenc}

\usepackage{amsmath}
\usepackage{amsfonts}
\usepackage{amssymb}
\usepackage{amsthm}
\usepackage{mathtools}
\usepackage{mathrsfs}

\usepackage{listings} % for programming codes

\usepackage{bm}
\usepackage[dvipdfmx]{color}
\pagestyle{plain} 
\usepackage{float}
\usepackage[dvipdfmx]{graphicx}

\usepackage[left=15mm, right=18mm, top=20mm, bottom=20mm]{geometry}

\setlength{\footskip}{40pt} 
\setlength{\abovecaptionskip}{0pt}
\setlength{\belowcaptionskip}{0pt}
\renewcommand{\baselinestretch}{1.05}

\newcommand{\argmax}{\mathop{\rm arg max}\limits}
\newcommand{\argmin}{\mathop{\rm arg min}\limits}
\newcommand{\st}{\mathop{\rm s.t.}\limits}
\newcommand{\fsize}[1]{\fontsize{#1}{#1}\selectfont}

% custom color
\usepackage[dvipdfmx]{xcolor}

% colors
\definecolor{darkred}{rgb}{.40,.00,.00}
\definecolor{navy}{rgb}{.00,.00,.40}
\definecolor{darknavy}{rgb}{.00,.00,.30}
\definecolor{lightnavy}{rgb}{.10,.10,.40}

\newcommand{\navy}{\color{navy}}
\newcommand{\deepred}{\color{darkred}}
\newcommand{\emred}[1]{\textbf{\deepred #1}}
\newcommand{\emblue}[1]{\textbf{\navy #1}}

% title and tableofcontents
\usepackage[titles]{tocloft}
\setlength{\cftbeforesecskip}{2pt}
\setlength{\cftbeforesubsecskip}{0pt}
\renewcommand{\cftsecfont}{\normalsize\mcfamily}
\renewcommand{\cftsubsecfont}{\normalsize\mcfamily}
\renewcommand{\cftsubsecfont}{\normalsize\mcfamily}
\renewcommand{\contentsname}{\large\mcfamily\bfseries \S\hspace{0.5\baselineskip}Contents\\[-\baselineskip]}
\cftsetindents{section}{10.0pt}{15.0pt}
\cftsetindents{subsection}{20.0pt}{30.0pt}
\cftpagenumbersoff{section}
\cftpagenumbersoff{subsection}
\cftpagenumbersoff{subsubsection}

\makeatletter
\def\vhrulefill#1{\leavevmode\leaders\hrule\@height#1\hfill \kern\z@}
\makeatother

\makeatletter
\def\maketitle{%
	\begin{center}\leavevmode
	\normalfont
	{\normalsize\raggedleft \@date\par}%
	\vskip -0.7\baselineskip
	\vhrulefill{0.2pt}\par
	\vskip 0.3\baselineskip
	\renewcommand{\baselinestretch}{0.80}
	{\color{darknavy}\bfseries\Large\raggedright \@title\par}%
	\renewcommand{\baselinestretch}{1.00}
	\vskip 0.2\baselineskip
	\ifx\@subtitle\@empty\else%
	{\normalsize\raggedright \@subtitle\par}%
	\fi
	\ifx\@author\@empty\else%
	\vskip 0.5\baselineskip
	{\normalsize\raggedleft \@author\par}%
	\fi
	\end{center}%
	\vskip 1.0\baselineskip
}
\def\subtitle#1{\gdef\@subtitle{#1}}
\let\@subtitle\@empty
\makeatother

% section
\newcommand{\Secindent}{0.0\baselineskip}
\newcommand{\Subsecindent}{0.0\baselineskip}
\newcommand{\lmargini}{1.3\baselineskip}
\newcommand{\lmarginii}{0.5\baselineskip}
\newcommand{\lmarginiii}{0.2\baselineskip}
\def\theenumi{\arabic{enumi}}
\def\theenumii{\alph{enumii}}
\renewcommand{\labelenumi}{\hfill\hspace{60pt}\theenumi.}
\renewcommand{\labelenumii}{\hspace{30pt}\theenumii.}
\renewcommand{\labelitemi}{\small\textbullet\hspace{2pt}}
\renewcommand{\labelitemii}{\color{gray}\small$\circ$\hspace{2pt}}
\renewcommand{\labelitemiii}{\textendash\hspace{1pt}}
\makeatletter
\renewcommand{\section}{\@startsection%
	{section}{1}{\Secindent}% name, depth, indent \z@=0pt
	{-1.0\baselineskip \@plus1.5\baselineskip \@minus0.5\baselineskip}% beforeskip (no indent if negative)
	{0.5\baselineskip \@plus0.8\baselineskip \@minus0.3\baselineskip}% afterskip (no line break if negative)
	{\reset@font\large\bfseries\scshape\color{darknavy}}% font
}
\renewcommand{\subsection}{\@startsection%
	{subsection}{2}{\Subsecindent}%
	{-0.5\baselineskip \@plus0.7\baselineskip \@minus0.4\baselineskip}%
	{0.3\baselineskip \@plus0.5\baselineskip \@minus0.2\baselineskip}%
	{\reset@font\normalsize\bfseries\scshape\color{darknavy}}% 
}
\renewcommand{\thesection}{\@arabic\c@section} % arabic, roman, Roman, alph, Alph, fnsymbol
\renewcommand{\thesubsection}{\thesection\hspace{0pt}.\hspace{0pt}\@arabic\c@subsection}
\makeatother

\makeatletter
\def\@listi{%
	\topsep = 0.2\baselineskip
	\partopsep = 0.0\baselineskip
	\parsep = 0.0\baselineskip
	\itemsep = 0.5\baselineskip
	\leftmargin = 1.5zw
	\labelsep = 0.5zw
	\itemindent = 0pt
	\labelwidth\leftmargin \advance\labelwidth-\labelsep
	\listparindent = 0zw
	\rightmargin = 0pt
	\reset@font\normalsize\normalfont
}
\let\@listI\@listi
\@listi
\def\@listii{%
	\topsep = 0.25\baselineskip
	\partopsep = 0.0\baselineskip
	\parsep = 0.0\baselineskip
	\itemsep = 0.3\baselineskip
	\leftmargin = 0.5zw
	\labelsep = .5zw
	\itemindent = 0pt
	\labelwidth\leftmargin \advance\labelwidth-\labelsep
	\listparindent = 0zw
	\rightmargin = 0pt
	\reset@font\normalsize\normalfont%
}
\def\@listiii{%
	\topsep = 0.30\baselineskip
	\partopsep = 0.0\baselineskip
	\parsep = 0.1\baselineskip
	\itemsep = 0.05\baselineskip
	\leftmargin = 1.1zw
	\labelsep = .5zw
	\itemindent = 0pt
	\labelwidth\leftmargin \advance\labelwidth-\labelsep
	\listparindent = 0zw
	\rightmargin = 0pt
	\reset@font\normalsize\normalfont%
}
\def\@listiv{%
	\topsep = 0.0\baselineskip
	\partopsep = 0.0\baselineskip
	\parsep = 0.0\baselineskip
	\itemsep = 0.0\baselineskip
	\leftmargin = 0.9zw
	\labelsep = .5zw
	\itemindent = 0pt
	\labelwidth\leftmargin \advance\labelwidth-\labelsep
	\listparindent = 0zw
	\rightmargin = 0pt
	\reset@font\normalsize\normalfont%
}
\makeatother

\newcommand{\R}{\ensuremath{\mathbb{R}}}
\newcommand{\E}{\ensuremath{\mathbb{E}}}
\newcommand{\V}{\ensuremath{\mathbb{V}}}
\newcommand{\Cov}{\ensuremath{\text{Cov}}}
\newcommand{\N}{\ensuremath{\mathbb{N}}}
\newcommand{\Z}{\ensuremath{\mathbb{Z}}}
\let\epsilon\varepsilon

\let\origitem\item
\renewcommand{\item}{\normalfont\origitem}

\usepackage[%
backref=page,
dvipdfmx,
bookmarks=true,bookmarksnumbered=true,bookmarkstype=toc,
colorlinks=true,
linkcolor=navy,
citecolor=navy,
filecolor=blue,
pagecolor=blue,
urlcolor=blue
]{hyperref}
\AtBeginDvi{\special{pdf:tounicode EUC-UCS2}}

\renewcommand*{\backref}[1]{}
\renewcommand*{\backrefalt}[4]{%
  \ifcase #1 (Not cited.)%
  \or        [#2]%
  \else      [#2]%
  \fi}
 

\title{Nonlinear models}
\subtitle{Introduction to dynamical systems~\#8}
\author{Hiroaki Sakamoto}
\date{\today}

\begin{document}
\maketitle
\tableofcontents

\section{Nonlinear dynamical system}

\subsection{Continuous-time model}

\begin{itemize}

\item \textbf{General Dynamical System}
  \begin{itemize}
  \item A (time-invariant) continuous-time \emph{dynamical system} is a system of differential equations of the form
    \begin{equation}\nonumber%\label{eq:}%
      \frac{d}{dt}\bm{x}(t) = \bm{f}(\bm{x}(t),\bm{u}(t)) \quad t \in \R_{+},
    \end{equation} where
    \begin{itemize}
    \item $\bm{x}(t)\in \R^{m}$: state vector at time $t$
    \item $\bm{u}(t)\in \R^{n}$: control (input) vector at time $t$
    \item $\bm{f} \in \R^{m}\times \R^{n} \to \R^{m}$: vector-valued function
    \end{itemize}
  \item A linear dynamical system is a special case where
    \begin{equation}\nonumber%\label{eq:}%
      \bm{f}(\bm{x},\bm{u}) =\bm{A}\bm{x} + \bm{B}\bm{u}
    \end{equation}
  \item Starting with some \emph{initial state} $\bm{x}(0)$, we want to know how $\bm{x}(t)$ evolves over time depending on $\bm{f}$ and $\bm{u}$.
  \end{itemize}
\item \textbf{Equilibrium}
  \begin{itemize}
  \item Consider the case where the control input remains constant at $\bar{\bm{u}}$:
    \begin{equation}\label{eq:DS}% 
      \frac{d}{dt}\bm{x}(t) = \bm{f}(\bm{x}(t),\bar{\bm{u}}) \quad t\in \R_{+}
    \end{equation}
  \item An \emph{equilibrium point} $\bar{\bm{x}}$ of \eqref{eq:DS} is
    defined as a solution to
    \begin{equation}\nonumber%
      \bm{0} =
      \bm{f}(\bar{\bm{x}},\bar{\bm{u}}),
    \end{equation} which depends on both the system dynamics $\bm{f}$ and the control input $\bar{\bm{u}}$
  \end{itemize}

\item \textbf{Stability}
  \begin{itemize}
  \item An equilibrium point $\bar{\bm{x}}$ of \eqref{eq:DS}
    is said to be \emph{stable} if
    for any $\epsilon >0$,
    there exists $\delta>0$ such that
    \begin{equation}\nonumber%\label{eq:}%
      \lVert \bm{x}(0) - \bar{\bm{x}} \rVert < \delta
      \implies 
      \lVert \bm{x}(t) - \bar{\bm{x}} \rVert < \epsilon \quad \forall t> 0,
    \end{equation}
    namely, the state trajectory stays arbitrarily close to the equilibrium point
    as long as the initial state is close enough to the equilibrium point
  \item An equilibrium point $\bar{\bm{x}}$ of \eqref{eq:DS}
    is said to be \emph{asymptotically stable} if
    a) it is stable (as define above), and
    b) there exists $\bar{\delta}>0$ such that
    \begin{equation}\nonumber%\label{eq:}%
      \lVert \bm{x}(0) - \bar{\bm{x}} \rVert < \bar{\delta}
      \implies 
      \lim_{t \to \infty}\bm{x}(t) =  \bar{\bm{x}},
    \end{equation}
    namely, the state trajectory actually converges to the equilibrium point
    as long as the initial state is close enough to the equilibrium point
  \end{itemize}

\item \textbf{Example~1}
  \begin{itemize}
  \item Consider the following one-dimensional dynamical system:
    \begin{equation}\label{eq:Bernoulli}%
      \dot{x}(t) = ax(t) + b(x(t))^{\gamma},
      \quad ab < 0, \, \gamma\in \R\setminus\{1\},
    \end{equation}
    which generalizes the linear dynamical system $\dot{x}(t)= ax(t) + b$ where $\gamma=0$
  \item Observe:
    \begin{itemize}
    \item For $\gamma = 0$ (linear case),
      the system has a unique equilibrium point:
      \begin{equation}\nonumber%\label{eq:}%
        a\bar{x} + b = 0
        \iff
        \bar{x} = - \frac{b}{a}
      \end{equation}
    \item For $\gamma \neq 0$, the system has two (trivial and non-trivial) equilibrium points:
      \begin{equation}\nonumber%\label{eq:}%
        a\bar{x} + b\bar{x}^{\gamma} = 0
        \iff
        \bar{x} = 0
        \text{ or }
        \bar{x} = \left(- \frac{b}{a}\right)^{\frac{1}{1-\gamma}}
      \end{equation}
    \item Since \eqref{eq:Bernoulli} is a Bernoulli differential equation,\footnote{
        Notice that \eqref{eq:Bernoulli} may be rewritten as
        \begin{equation}\nonumber%\label{eq:}%
          (x(t))^{-\gamma}\dot{x}(t) = a(x(t))^{1-\gamma} + b
          \quad\implies\quad
          \frac{d}{dt}(x(t))^{1-\gamma} = (1-\gamma)a (x(t))^{1-\gamma} + (1-\gamma)b
          \quad \forall t,
        \end{equation}
        which in turn implies
        \begin{equation}\nonumber%\label{eq:}%
          \frac{d}{dt}\left\{ (x(t))^{1-\gamma} - (\bar{x})^{1-\gamma} \right\}
          = (1-\gamma)a\left((x(t))^{1-\gamma} - (\bar{x})^{1-\gamma}\right)
          \quad \forall t,
          \quad \bar{x}:= \left(- \frac{b}{a}\right)^{\frac{1}{1-\gamma}}
        \end{equation}
        or
        \begin{equation}\nonumber%\label{eq:}%
          \frac{d}{dt} \ln \left((x(t))^{1-\gamma} - (\bar{x})^{1-\gamma}\right)
          = (1-\gamma)a
          \quad \forall t.
        \end{equation}
        Therefore, integrating both sides over $[0,t]$ yields
        \begin{equation}\nonumber%\label{eq:}%
          \ln \left(\frac{(x(t))^{1-\gamma} - (\bar{x})^{1-\gamma}}{(x(0))^{1-\gamma} - (\bar{x})^{1-\gamma}}\right)
          = (1-\gamma)at
          \quad\implies\quad
          (x(t))^{1-\gamma} - (\bar{x})^{1-\gamma}
          = \left((x(0))^{1-\gamma} - (\bar{x})^{1-\gamma}\right)e^{(1-\gamma)at},
        \end{equation}
        which gives \eqref{eq:Bernoulli_solution}.
      }
      the exact solution can be derived as
      \begin{equation}\label{eq:Bernoulli_solution}%
        x(t) = \left(\left((x(0))^{1-\gamma} + \frac{b}{a}\right)e^{(1-\gamma)at} - \frac{b}{a}\right)^{\frac{1}{1-\gamma}}
      \end{equation}
    \item Trivial equilibrium, $\bar{x}=0$, is asymptotically stable if and only if $\gamma >1$ and $a<0$
    \item Non-trivial equilibrium, $\bar{x}=(-b/a)^{\frac{1}{1-\gamma}}$, is asymptotically stable if and only if $(1-\gamma)a<0$
    \end{itemize}
  \end{itemize}

\clearpage

\item \textbf{Example~2}
  \begin{itemize}
  \item Consider the following two-dimensional dynamical system:
    \begin{equation}\label{eq:ex2}%
      \begin{bmatrix}
        \dot{x}_{1}(t) \\
        \dot{x}_{2}(t) \\
      \end{bmatrix}
      =
      \begin{bmatrix}
      a_{1}x_{1}(t) - b_{1}(x_{1}(t)+x_{2}(t))x_{1}(t) \\
      a_{2}x_{2}(t) - b_{2}(x_{1}(t)+x_{2}(t))x_{2}(t) \\
      \end{bmatrix},
      \quad
      a_{i}, b_{i}>0,
      \quad
      \frac{a_{1}}{b_{1}} \neq \frac{a_{2}}{b_{2}}
    \end{equation}
  \item Observe:
    \begin{itemize}
    \item The system has a trivial equilibrium point, $\bar{x}_{1}=\bar{x}_{2} = 0$
    \item There is no equilibrium point where both $\bar{x}_{1}$ and $\bar{x}_{2}$ are non-zero,\footnote{
        If $(\bar{x}_{1}, \bar{x}_{2})$ is an equilibrium point with $\bar{x}_{1}\neq 0$ and $x_{2}\neq 0$,
        then \eqref{eq:ex2} implies
        \begin{equation}\nonumber%\label{eq:}%
          a_{1} + b_{1}(\bar{x}_{1}+\bar{x}_{2})
          = \frac{\dot{\bar{x}}_{1}}{\bar{x}_{1}}
          = 0
          = \frac{\dot{\bar{x}}_{2}}{\bar{x}_{2}}
          =
          a_{2} + b_{2}(\bar{x}_{1}+\bar{x}_{2}),
        \end{equation}
        which is only possible when $\frac{a_{1}}{b_{1}} = \frac{a_{2}}{b_{2}}$, a contradiction.
      } indicating that at least one of them must be zero
    \item It follows that there are only two non-trivial equilibrium points:
      \begin{equation}\nonumber%\label{eq:}%
        \begin{bmatrix}
          \bar{x}_{1} \\
          \bar{x}_{2} \\
        \end{bmatrix}
        = 
        \begin{bmatrix}
          a_{1}/b_{1} \\
          0 \\
        \end{bmatrix}
        \quad\text{and}\quad
        \begin{bmatrix}
          \bar{x}_{1} \\
          \bar{x}_{2} \\
        \end{bmatrix}
        = 
        \begin{bmatrix}
          0 \\
          a_{2}/b_{2} \\
        \end{bmatrix}
      \end{equation}
    \item Are these equilibrium points stable? If so, in what condition?
    \end{itemize}
  \end{itemize}

\end{itemize}

\subsection{Discrete-time model}

\begin{itemize}

\item \textbf{General dynamical system}
  \begin{itemize}
  \item A (time-invariant) discrete-time \emph{dynamical system} is a system of difference equations of the form
    \begin{equation}\nonumber%\label{eq:}%
      \bm{x}_{t+1} = \bm{f}(\bm{x}_{t},\bm{u}_{t})
      \quad t = 0, 1, 2, \ldots,
    \end{equation}
    where
    \begin{itemize}
    \item $\bm{x}_{t}\in \R^{m}$: state vector at $t$
    \item $\bm{u}_{t}\in \R^{n}$: control (input) vector at $t$
    \item $\bm{f} \in \R^{m}\times \R^{n} \to \R^{m}$: vector-valued function
    \end{itemize}
  \item A linear dynamical system is a special case where
    \begin{equation}\nonumber%\label{eq:}%
      \bm{f}(\bm{x},\bm{u}) = \bm{A}\bm{x} + \bm{B}\bm{u}
    \end{equation}
  \item Starting with some \emph{initial state} $\bm{x}_{0}$,
    we want to know how $\bm{x}_{t}$ evolves over time depending on $\bm{f}$ and $\bm{u}$
  \end{itemize}

\item \textbf{Equilibrium}
  \begin{itemize}
  \item Consider the case where the control is constant at $\bar{\bm{u}}$:
    \begin{equation}\label{eq:DS_discrete}%
      \bm{x}_{t+1} = \bm{f}(\bm{x}_{t},\bar{\bm{u}})
      \quad t = 0, 1, 2, \ldots,
    \end{equation}
  \item We define an \emph{equilibrium point} of \eqref{eq:DS_discrete} as $\bar{\bm{x}}$ that solves
    \begin{equation}\nonumber%\label{eq:}%
      \bar{\bm{x}} = \bm{f}(\bar{\bm{x}},\bar{\bm{u}})
    \end{equation}
    which obviously depends on both $\bm{f}$ and $\bar{\bm{u}}$
  \end{itemize}

\item \textbf{Stability}
  \begin{itemize}

  \item An equilibrium point $\bar{\bm{x}}$ of \eqref{eq:DS_discrete}
    is said to be \emph{stable} if
    for any $\epsilon >0$,
    there exists $\delta>0$ such that
    \begin{equation}\nonumber%\label{eq:}%
      \lVert \bm{x}_{0} - \bar{\bm{x}} \rVert < \delta
      \implies 
      \lVert \bm{x}_{t} - \bar{\bm{x}} \rVert < \epsilon \quad \forall t = 1, 2, \ldots
    \end{equation}
  \item An equilibrium point $\bar{\bm{x}}$ of \eqref{eq:DS_discrete}
    is said to be \emph{asymptotically stable} if
    a) it is stable, and b)
    there exists $\bar{\delta}>0$ such that
    \begin{equation}\nonumber%\label{eq:}%
      \lVert \bm{x}_{0} - \bar{\bm{x}} \rVert < \bar{\delta}
      \implies 
      \lim_{t \to \infty}\bm{x}_{t} =  \bar{\bm{x}},
    \end{equation}
  \end{itemize}

\item \textbf{Example~3}
  \begin{itemize}
  \item Consider the following one-dimensional dynamical system:
    \begin{equation}\label{eq:ex3}%
      x_{t+1} = \frac{ax_{t}+b}{cx_{t}+1},
      \quad a \neq 1,
    \end{equation}
    which generalizes the linear dynamical system $x_{t+1} = ax_{t} + b$ where $c=0$

  \item Observe:
    \begin{itemize}
    \item If $c = 0$, the system has a unique equilibrium point:
      \begin{equation}\nonumber%\label{eq:}%
        \bar{x} = a\bar{x} + b
        \iff
        \bar{x} = \frac{b}{1-a}
      \end{equation}
    \item If $c \neq 0$, the system has two equilibrium points:
      \begin{equation}\nonumber%\label{eq:}%
        \bar{x} = \frac{a\bar{x}+b}{c\bar{x}+1}
        \iff
        \bar{x} =
        \begin{cases}
          - \frac{1}{2}\frac{1-a}{c} + \sqrt{\left(\frac{1}{2}\frac{1-a}{c}\right)^{2} + \frac{b}{c}} =:\bar{x}_{+} > 0  \\
          - \frac{1}{2}\frac{1-a}{c} - \sqrt{\left(\frac{1}{2}\frac{1-a}{c}\right)^{2} + \frac{b}{c}} =:\bar{x}_{-} < 0
        \end{cases}
      \end{equation}
    \end{itemize}
  \item One may rewrite \eqref{eq:ex3} as
    \begin{equation}\nonumber%\label{eq:}%
      \frac{1}{x_{t+1}-\bar{x}} = \underbrace{\frac{1+c\bar{x}}{a-c\bar{x}}}_{=:\alpha}\frac{1}{x_{t}-\bar{x}} + \frac{c}{a-c\bar{x}},
      \quad \forall \bar{x} \in \{\bar{x}_{+}, \bar{x}_{-}\},
    \end{equation}
    which implies that the state trajectory satisfies
    \begin{equation}\nonumber%\label{eq:}%
      \frac{1}{x_{t}-\bar{x}} = \alpha^{t}\frac{1}{x_{0}-\bar{x}} + \frac{1-\alpha^{t}}{1-\alpha}\frac{c}{a-c\bar{x}},
    \end{equation}
    or
    \begin{equation}\label{eq:xt1}%
      x_{t} = \bar{x} + \left(\frac{1}{x_{0}-\bar{x}}\alpha^{t} + \frac{1-\alpha^{t}}{1-\alpha} \frac{c}{a-c\bar{x}} \right)^{-1},
      \quad \bar{x} \in \{\bar{x}_{+}, \bar{x}_{-}\}
    \end{equation}
    where
    \begin{equation}\nonumber%\label{eq:}%
      \alpha
      = \frac{1+c\bar{x}}{a-c\bar{x}}
      =
      \begin{cases}
        \frac{1+c\bar{x}_{+}}{a-c\bar{x}_{+}}
        =
        \frac{\frac{1}{2}\frac{1+a}{c} + \sqrt{\left(\frac{1}{2}\frac{1-a}{c}\right)^{2} + \frac{b}{c}}}{\frac{1}{2}\frac{1+a}{c} - \sqrt{\left(\frac{1}{2}\frac{1-a}{c}\right)^{2} + \frac{b}{c}}} & \text{if $\bar{x}=\bar{x}_{+}$}\\
        \frac{1+c\bar{x}_{-}}{a-c\bar{x}_{-}} = 
        \frac{\frac{1}{2}\frac{1+a}{c} - \sqrt{\left(\frac{1}{2}\frac{1-a}{c}\right)^{2} + \frac{b}{c}}}{\frac{1}{2}\frac{1+a}{c} + \sqrt{\left(\frac{1}{2}\frac{1-a}{c}\right)^{2} + \frac{b}{c}}} & \text{if $\bar{x}=\bar{x}_{-}$}\\
      \end{cases}
    \end{equation}
  \item It follows from \eqref{eq:xt1} that an equilibrium point $\bar{x}$ is asymptotically stable if and only if
    \begin{equation}\nonumber%\label{eq:}%
      \lim_{t\to\infty} \left(\frac{1}{x_{0}-\bar{x}}\alpha^{t} + \frac{1-\alpha^{t}}{1-\alpha} \frac{c}{a-c\bar{x}}\right) = \pm \infty
      \text{ if $x_{0}$ is close enough to $\bar{x}$}
      \iff
      |\alpha|>1,
    \end{equation}
    from which we conclude that
    $\bar{x}_{+}$ is asymptotically stable
    whereas
    $\bar{x}_{-}$ is not
  \end{itemize}

\item \textbf{Example~4}
  \begin{itemize}
  \item Consider the following two-dimensional dynamical system:
    \begin{equation}\label{eq:ex4}%
      \begin{bmatrix}
        x_{1,t+1} \\
        x_{2,t+1} \\
      \end{bmatrix}
      =
      \begin{bmatrix}
      a_{1} x_{1,t} + b_{1}x_{2,t}^{2} \\
      b_{2}x_{1,t} + a_{2} x_{2,t} \\
      \end{bmatrix},
      \quad a_{1},a_{2}\neq 1,
      \quad b_{1},b_{2}\neq 0
    \end{equation}
  \item Observe:
    \begin{itemize}
    \item There are two equilibrium points:
      \begin{equation}\nonumber%\label{eq:}%
      \begin{bmatrix}
        \bar{x}_{1} \\
        \bar{x}_{2} \\
      \end{bmatrix}
      =
      \begin{bmatrix}
        a_{1}\bar{x}_{1} + b_{1}\bar{x}_{2}^{2} \\
        b_{2}\bar{x}_{1} + a_{2}\bar{x}_{2} \\
      \end{bmatrix}
      \iff  
      \begin{bmatrix}
        \bar{x}_{1} \\
        \bar{x}_{2} \\
      \end{bmatrix}
      = 
      \begin{bmatrix}
        0 \\
        0 \\
      \end{bmatrix}
      \,\text{ or }\,
      \begin{bmatrix}
        \bar{x}_{1} \\
        \bar{x}_{2} \\
      \end{bmatrix}
      = 
      \begin{bmatrix}
        \frac{1-a_{1}}{b_{1}}\left(\frac{1-a_{2}}{b_{2}}\right)^{2} \\
        \frac{1-a_{1}}{b_{1}}\frac{1-a_{2}}{b_{2}} \\
      \end{bmatrix}
      \end{equation}
    \item Are these equilibrium points stable? If so, in what condition?
    \end{itemize}
  \end{itemize}

\end{itemize}

\section{Linearization}

\subsection{Continuous-time model}

\begin{itemize}
\item \textbf{Linearized dynamical system}
  \begin{itemize}
  \item Consider the following dynamical system
    \begin{equation}\label{eq:nlds_continuous}%
      \dot{\bm{x}}(t) = \bm{f}(\bm{x}(t),\bm{u}(t))
      \quad\text{or}\quad
      \begin{bmatrix}
        \dot{x}_{1}(t) \\
        \dot{x}_{2}(t) \\
        \vdots \\
        \dot{x}_{m}(t) \\
      \end{bmatrix}
      = 
      \begin{bmatrix}
        f_{1}(\bm{x}(t),\bm{u}(t)) \\
        f_{2}(\bm{x}(t),\bm{u}(t)) \\
        \vdots \\
        f_{m}(\bm{x}(t),\bm{u}(t)) \\
      \end{bmatrix}
    \end{equation}
  \item Let $(\bar{\bm{x}}, \bar{\bm{u}})$ be \emph{an} equilibrium point of the system
    \begin{equation}\nonumber%\label{eq:}%
      \bm{f}(\bar{\bm{x}},\bar{\bm{u}}) = \bm{0}
    \end{equation}
  \item If $(\bm{x}(t),\bm{u}(t))$ is close to $(\bar{\bm{x}},\bar{\bm{u}})$,
    \begin{equation}
      \bm{f}(\bm{x}(t),\bm{u}(t))
      \approx
        \underbrace{\bm{f}(\bar{\bm{x}},\bar{\bm{u}})}_{=\bm{0}}
        +
        \frac{d\bm{f}(\bar{\bm{x}},\bar{\bm{u}})}{d\bm{x}}(\bm{x}(t)-\bar{\bm{x}})
        +
        \frac{d\bm{f}(\bar{\bm{x}},\bar{\bm{u}})}{d\bm{u}}(\bm{u}(t)-\bar{\bm{u}})
    \nonumber%\label{eq:}%
    \end{equation}
    where
    \begin{equation}\nonumber%\label{eq:}%
      \frac{d\bm{f}(\bar{\bm{x}},\bar{\bm{u}})}{d\bm{x}}
      =
      \begin{bmatrix}
        \frac{\partial f_{1}(\bar{\bm{x}},\bar{\bm{u}})}{\partial x_{1}} & \frac{\partial f_{1}(\bar{\bm{x}},\bar{\bm{u}})}{\partial x_{2}}  & \ldots & \frac{\partial f_{1}(\bar{\bm{x}},\bar{\bm{u}})}{\partial x_{m}} \\
        \frac{\partial f_{2}(\bar{\bm{x}},\bar{\bm{u}})}{\partial x_{1}} & \frac{\partial f_{2}(\bar{\bm{x}},\bar{\bm{u}})}{\partial x_{2}}  & \ldots & \frac{\partial f_{2}(\bar{\bm{x}},\bar{\bm{u}})}{\partial x_{m}} \\
        \vdots & \vdots  & \ddots & \vdots \\
        \frac{\partial f_{m}(\bar{\bm{x}},\bar{\bm{u}})}{\partial x_{1}} & \frac{\partial f_{m}(\bar{\bm{x}},\bar{\bm{u}})}{\partial x_{2}}  & \ldots & \frac{\partial f_{m}(\bar{\bm{x}},\bar{\bm{u}})}{\partial x_{m}} \\
      \end{bmatrix},
      \quad
      \frac{d\bm{f}(\bar{\bm{x}},\bar{\bm{u}})}{d\bm{u}}
      =
      \begin{bmatrix}
        \frac{\partial f_{1}(\bar{\bm{x}},\bar{\bm{u}})}{\partial u_{1}} & \frac{\partial f_{1}(\bar{\bm{x}},\bar{\bm{u}})}{\partial x_{2}}  & \ldots & \frac{\partial f_{1}(\bar{\bm{x}},\bar{\bm{u}})}{\partial u_{n}} \\
        \frac{\partial f_{2}(\bar{\bm{x}},\bar{\bm{u}})}{\partial u_{1}} & \frac{\partial f_{2}(\bar{\bm{x}},\bar{\bm{u}})}{\partial x_{2}}  & \ldots & \frac{\partial f_{2}(\bar{\bm{x}},\bar{\bm{u}})}{\partial u_{n}} \\
        \vdots & \vdots  & \ddots & \vdots \\
        \frac{\partial f_{m}(\bar{\bm{x}},\bar{\bm{u}})}{\partial u_{1}} & \frac{\partial f_{m}(\bar{\bm{x}},\bar{\bm{u}})}{\partial x_{2}}  & \ldots & \frac{\partial f_{m}(\bar{\bm{x}},\bar{\bm{u}})}{\partial u_{n}} \\
      \end{bmatrix}
    \end{equation}
  \item Hence, in a neighborhood of $(\bar{\bm{x}},\bar{\bm{u}})$,
    dynamical system~\eqref{eq:nlds_continuous} can be approximated by the following \emph{linear} system
    \begin{equation}\nonumber%\label{eq:}%
      \frac{d}{dt}\left(\bm{x}(t)-\bar{\bm{x}}\right)
      =
      \bm{A}(\bm{x}(t)-\bar{\bm{x}})
      +
      \bm{B}(\bm{u}(t)-\bar{\bm{u}}),
      \quad\text{where}\quad
      \bm{A} := \frac{d\bm{f}(\bar{\bm{x}},\bar{\bm{u}})}{d\bm{x}},
      \quad
      \bm{B} := \frac{d\bm{f}(\bar{\bm{x}},\bar{\bm{u}})}{d\bm{u}}
    \end{equation}
  \item Stability of each equilibrium point
    and the state trajectory around it can be characterized
    based on the system matrix $\bm{A}$
  \item Since $\bm{A}$ depends on $\bar{\bm{x}}$,
    we need to use different $\bm{A}$ for analyzing stability of different equilibrium points $\bar{\bm{x}}$
  \end{itemize}
  \clearpage

\item \textbf{Example~1}
  \begin{itemize}
  \item Let us revisit Example~1 and consider the following one-dimensional dynamical system:
    \begin{equation}\nonumber%\label{eq:Bernoulli_rev}%
      \dot{x}(t) = \underbrace{ax(t) + b(x(t))^{\gamma}}_{f(x(t))},
      \quad ab < 0, \, \gamma\in \R\setminus\{1\}
    \end{equation}
  \item Linearly approximating the system around an equilibrium point $\bar{x}$ yields
    \begin{equation}\nonumber%\label{eq:}%
      \frac{d}{dt}(\bar{x}(t)-\bar{x})
      = \underbrace{\left(a + \gamma b(\bar{x})^{\gamma-1}\right)}_{f'(\bar{x})}(x(t)-\bar{x}),
    \end{equation}
    which indicates that the equilibrium point is asymptotically stable
    if and only if
    \begin{equation}\nonumber%\label{eq:}%
      f'(\bar{x}) = a + \gamma b(\bar{x})^{\gamma-1}< 0
    \end{equation}
  \item Assume $\gamma \neq 0$ so that 
    the system has two equilibrium points:
    \begin{equation}\nonumber%\label{eq:}%
      \bar{x} = 0
      \quad
      \text{ and }
      \quad
      \bar{x} = \left(- \frac{b}{a}\right)^{\frac{1}{1-\gamma}}
    \end{equation}
  \item Observe that
    \begin{itemize}
    \item $\bar{x}=0$ is asymptotically stable if and only if
      \begin{equation}\nonumber%\label{eq:}%
        a + \gamma b(\bar{x})^{\gamma-1}\big|_{\bar{x}=0} < 0
        \iff a < 0 \text{ and } \gamma > 1
      \end{equation}
    \item $\bar{x}=(-b/a)^{\frac{1}{1-\gamma}}$ is asymptotically stable if and only if
      \begin{equation}\nonumber%\label{eq:}%
        a + \gamma b \left(\bar{x}\right)^{\gamma-1}\bigg|_{\bar{x}=(-b/a)^{\frac{1}{1-\gamma}}} < 0
        \iff
        (1 - \gamma)a < 0
      \end{equation}
    \end{itemize}
  \item The linearized model allows us to reach the same conclusion
    without explicitly solving for the state trajectory
    
  \end{itemize}

\item \textbf{Example~2}
  \begin{itemize}
  \item Let us revisit Example~2 and consider the following two-dimensional dynamical system:
    \begin{equation}\nonumber%\label{eq:ex2_rev}%
      \begin{bmatrix}
        \dot{x}_{1}(t) \\
        \dot{x}_{2}(t) \\
      \end{bmatrix}
      =
      \underbrace{
      \begin{bmatrix}
      a_{1}x_{1}(t) - b_{1}(x_{1}(t)+x_{2}(t))x_{1}(t) \\
      a_{2}x_{2}(t) - b_{2}(x_{1}(t)+x_{2}(t))x_{2}(t) \\
      \end{bmatrix}}_{\bm{f}(\bm{x}(t))},
      \quad
      a_{i}, b_{i}>0,
      \quad
      \frac{a_{1}}{b_{1}} \neq \frac{a_{2}}{b_{2}},
    \end{equation}
    which we know has three equilibrium points:
    \begin{equation}\nonumber%\label{eq:}%
      \begin{bmatrix}
        \bar{x}_{1} \\
        \bar{x}_{2} \\
      \end{bmatrix}
      =
      \begin{bmatrix}
        0 \\
        0 \\
      \end{bmatrix},
      \quad
        \begin{bmatrix}
          \bar{x}_{1} \\
          \bar{x}_{2} \\
        \end{bmatrix}
        = 
        \begin{bmatrix}
          a_{1}/b_{1} \\
          0 \\
        \end{bmatrix},
        \quad\text{and}\quad
        \begin{bmatrix}
          \bar{x}_{1} \\
          \bar{x}_{2} \\
        \end{bmatrix}
        = 
        \begin{bmatrix}
          0 \\
          a_{2}/b_{2} \\
        \end{bmatrix}
    \end{equation}
  \item Linearly approximating the system around an equilibrium point $\bar{\bm{x}}$ yields
    \begin{equation}\nonumber%\label{eq:}%
      \begin{bmatrix}
        \dot{x}_{1}(t) \\
        \dot{x}_{2}(t) \\
      \end{bmatrix}
      = \underbrace{
        \begin{bmatrix}
          a_{1} - 2b_{1}\bar{x}_{1} - b_{1}\bar{x}_{2} & -b_{1}\bar{x}_{1} \\
          -b_{2}\bar{x}_{2} & a_{2} - 2b_{2}\bar{x}_{2} - b_{2}\bar{x}_{1} \\
        \end{bmatrix}
      }_{\frac{d\bm{f}(\bar{\bm{x}})}{d\bm{x}}}
      \begin{bmatrix}
        x_{1}(t)-\bar{x}_{1} \\
        x_{2}(t)-\bar{x}_{2} \\
      \end{bmatrix}
    \end{equation}
    which indicates that the equilibrium point is asymptotically stable
    if and only if
    \begin{equation}\nonumber%\label{eq:}%
      \rho \left(e^{\frac{d\bm{f}(\bar{\bm{x}})}{d\bm{x}}}\right) < 1
      \iff
      \text{eigenvalues of $\frac{d\bm{f}(\bar{\bm{x}})}{d\bm{x}}$ are all negative}
    \end{equation}

  \item For $\bar{\bm{x}}=(0, 0)^{\top}$, we have
    \begin{equation}\nonumber%\label{eq:}%
      \frac{d\bm{f}(\bar{\bm{x}})}{d\bm{x}}\bigg|_{
        \bar{\bm{x}}=
        \begin{bmatrix}
          0 \\
          0 \\
        \end{bmatrix}
      }
      =
      \begin{bmatrix}
        a_{1} & 0 \\
        0 & a_{2} \\
      \end{bmatrix}
    \end{equation}
    and therefore
    the equilibrium point is not stable because both of the eigenvalues are positive ($a_{1}>0$ and $a_{2}>0$)
    
  \item For $\bar{\bm{x}}=(a_{1}/b_{1}, 0)^{\top}$, we have
    \begin{equation}\nonumber%\label{eq:}%
      \frac{d\bm{f}(\bar{\bm{x}})}{d\bm{x}}\bigg|_{
        \bar{\bm{x}}=
        \begin{bmatrix}
          a_{1}/b_{1} \\
          0 \\
        \end{bmatrix}
      }
      =
      \begin{bmatrix}
        - a_{1} & -a_{1} \\
        0 & -\left(\frac{a_{1}}{b_{1}}-\frac{a_{2}}{b_{2}}\right)b_{2} \\
      \end{bmatrix}
    \end{equation}
    and therefore
    the equilibrium point is asymptotically stable if and only if $\frac{a_{1}}{b_{1}} > \frac{a_{2}}{b_{2}}$
    
  \item For $\bar{\bm{x}}=(0, a_{2}/b_{2})^{\top}$, we have
    \begin{equation}\nonumber%\label{eq:}%
      \frac{d\bm{f}(\bar{\bm{x}})}{d\bm{x}}\bigg|_{
        \bar{\bm{x}}=
        \begin{bmatrix}
          0 \\
          a_{2}/b_{2} \\
        \end{bmatrix}
      }
      =
      \begin{bmatrix}
        -\left(\frac{a_{2}}{b_{2}}-\frac{a_{1}}{b_{1}}\right)b_{1} & 0 \\
        -a_{2} & - a_{2} \\
      \end{bmatrix}
    \end{equation}
    and therefore
    the equilibrium point is asymptotically stable if and only if $\frac{a_{2}}{b_{2}} > \frac{a_{1}}{b_{1}}$
    
  \end{itemize}

\end{itemize}

\subsection{Discrete-time model}

\begin{itemize}
\item \textbf{Linearized dynamical system}
  \begin{itemize}
  \item Consider the following dynamical system
    \begin{equation}\label{eq:nlds_discrete}%
      \bm{x}_{t+1} = \bm{f}(\bm{x}_{t},\bm{u}_{t})
      \quad\text{or}\quad
      \begin{bmatrix}
        x_{1,t+1} \\
        x_{2,t+1} \\
        \vdots \\
        x_{m,t+1} \\
      \end{bmatrix}
      = 
      \begin{bmatrix}
        f_{1}(\bm{x}_{t},\bm{u}_{t}) \\
        f_{2}(\bm{x}_{t},\bm{u}_{t}) \\
        \vdots \\
        f_{m}(\bm{x}_{t},\bm{u}_{t}) \\
      \end{bmatrix}
    \end{equation}
  \item Let $(\bar{\bm{x}},\bar{\bm{u}})$ be \emph{an} equilibrium point of the system:
    $\bm{f}(\bar{\bm{x}},\bar{\bm{u}}) = \bar{\bm{x}}$
  \item If $(\bm{x}_{t},\bm{u}_{t})$ is close to $(\bar{\bm{x}},\bar{\bm{u}})$,
    \begin{equation}
      \bm{f}(\bm{x}_{t},\bm{u}_{t})
      \approx
        \underbrace{\bm{f}(\bar{\bm{x}},\bar{\bm{u}})}_{=\bar{\bm{x}}}
        +
        \frac{d\bm{f}(\bar{\bm{x}},\bar{\bm{u}})}{d\bm{x}}(\bm{x}_{t}-\bar{\bm{x}})
        +
        \frac{d\bm{f}(\bar{\bm{x}},\bar{\bm{u}})}{d\bm{u}}(\bm{u}_{t}-\bar{\bm{u}})
    \nonumber%\label{eq:}%
    \end{equation}
    where
    \begin{equation}\nonumber%\label{eq:}%
      \frac{d\bm{f}(\bar{\bm{x}},\bar{\bm{u}})}{d\bm{x}}
      =
      \begin{bmatrix}
        \frac{\partial f_{1}(\bar{\bm{x}},\bar{\bm{u}})}{\partial x_{1}} & \frac{\partial f_{1}(\bar{\bm{x}},\bar{\bm{u}})}{\partial x_{2}}  & \ldots & \frac{\partial f_{1}(\bar{\bm{x}},\bar{\bm{u}})}{\partial x_{m}} \\
        \frac{\partial f_{2}(\bar{\bm{x}},\bar{\bm{u}})}{\partial x_{1}} & \frac{\partial f_{2}(\bar{\bm{x}},\bar{\bm{u}})}{\partial x_{2}}  & \ldots & \frac{\partial f_{2}(\bar{\bm{x}},\bar{\bm{u}})}{\partial x_{m}} \\
        \vdots & \vdots  & \ddots & \vdots \\
        \frac{\partial f_{m}(\bar{\bm{x}},\bar{\bm{u}})}{\partial x_{1}} & \frac{\partial f_{m}(\bar{\bm{x}},\bar{\bm{u}})}{\partial x_{2}}  & \ldots & \frac{\partial f_{m}(\bar{\bm{x}},\bar{\bm{u}})}{\partial x_{m}} \\
      \end{bmatrix},
      \quad
      \frac{d\bm{f}(\bar{\bm{x}},\bar{\bm{u}})}{d\bm{u}}
      =
      \begin{bmatrix}
        \frac{\partial f_{1}(\bar{\bm{x}},\bar{\bm{u}})}{\partial u_{1}} & \frac{\partial f_{1}(\bar{\bm{x}},\bar{\bm{u}})}{\partial x_{2}}  & \ldots & \frac{\partial f_{1}(\bar{\bm{x}},\bar{\bm{u}})}{\partial u_{n}} \\
        \frac{\partial f_{2}(\bar{\bm{x}},\bar{\bm{u}})}{\partial u_{1}} & \frac{\partial f_{2}(\bar{\bm{x}},\bar{\bm{u}})}{\partial x_{2}}  & \ldots & \frac{\partial f_{2}(\bar{\bm{x}},\bar{\bm{u}})}{\partial u_{n}} \\
        \vdots & \vdots  & \ddots & \vdots \\
        \frac{\partial f_{m}(\bar{\bm{x}},\bar{\bm{u}})}{\partial u_{1}} & \frac{\partial f_{m}(\bar{\bm{x}},\bar{\bm{u}})}{\partial x_{2}}  & \ldots & \frac{\partial f_{m}(\bar{\bm{x}},\bar{\bm{u}})}{\partial u_{n}} \\
      \end{bmatrix}
    \end{equation}
  \item Hence, in a neighborhood of $(\bar{\bm{x}},\bar{\bm{u}})$,
    dynamical system~\eqref{eq:nlds_discrete} can be approximated by the following linear system
    \begin{equation}\nonumber%\label{eq:}%
      \bm{x}_{t+1} - \bar{\bm{x}}
      =
      \bm{A}(\bm{x}_{t}-\bar{\bm{x}})
      +
      \bm{B}(\bm{u}_{t}-\bar{\bm{u}}),
      \quad\text{where}\quad
      \bm{A} := \frac{d\bm{f}(\bar{\bm{x}},\bar{\bm{u}})}{d\bm{x}},
      \quad
      \bm{B} := \frac{d\bm{f}(\bar{\bm{x}},\bar{\bm{u}})}{d\bm{u}}
    \end{equation}
  \item Stability of each equilibrium point
    and the state trajectory around it can be characterized
    based on the system matrix $\bm{A}$
  \item Note that we need to use different $\bm{A}$ for different equilibrium point
  \end{itemize}

\clearpage
\item \textbf{Example~3}
  \begin{itemize}
  \item Let us revisit Example~3 and consider the following one-dimensional dynamical system:
    \begin{equation}\nonumber%\label{eq:ex3_rev}%
      x_{t+1} = \underbrace{\frac{ax_{t}+b}{cx_{t}+1}}_{f(x_{t})},
      \quad a \neq 1, \quad c\neq 0,
    \end{equation}
    which we know has two equilibrium points:
    \begin{equation}\nonumber%\label{eq:}%
      \bar{x} = \frac{a\bar{x}+b}{c\bar{x}+1}
      \iff
      \bar{x} =
      \begin{cases}
        - \frac{1}{2}\frac{1-a}{c} + \sqrt{\left(\frac{1}{2}\frac{1-a}{c}\right)^{2} + \frac{b}{c}} =:\bar{x}_{+} > 0  \\
        - \frac{1}{2}\frac{1-a}{c} - \sqrt{\left(\frac{1}{2}\frac{1-a}{c}\right)^{2} + \frac{b}{c}} =:\bar{x}_{-} < 0
      \end{cases}
    \end{equation}
  \item Linearly approximating the system around an equilibrium point $\bar{x}$ yields
    \begin{equation}\nonumber%\label{eq:}%
      x_{t+1} - \bar{x} =
      \underbrace{
        \frac{a-c\bar{x}}{c\bar{x}+1}
        }_{f'(\bar{x})}
        (x_{t}-\bar{x}),
      \end{equation}
      which indicates that the equilibrium point is asymptotically stable if and only if
      \begin{equation}\nonumber%\label{eq:}%
        \left|f'(\bar{x})\right| < 1
        \iff
        \left|\frac{a-c\bar{x}}{c\bar{x}+1}\right| < 1
        \iff
        \left|\frac{1+c\bar{x}}{a-c\bar{x}}\right| > 1,
      \end{equation}
    \item Since
      \begin{equation}\nonumber%\label{eq:}%
        \left|\frac{1+c\bar{x}}{a-c\bar{x}}\right|
        =
        \begin{cases}
          \left|\frac{\frac{1}{2}\frac{1+a}{c} + \sqrt{\left(\frac{1}{2}\frac{1-a}{c}\right)^{2} +
          \frac{b}{c}}}{\frac{1}{2}\frac{1+a}{c} - \sqrt{\left(\frac{1}{2}\frac{1-a}{c}\right)^{2} +
          \frac{b}{c}}}\right| > 1 & \text{if $\bar{x}=\bar{x}_{+}$}\\[15pt] 
          \left|\frac{\frac{1}{2}\frac{1+a}{c} - \sqrt{\left(\frac{1}{2}\frac{1-a}{c}\right)^{2} +
          \frac{b}{c}}}{\frac{1}{2}\frac{1+a}{c} + \sqrt{\left(\frac{1}{2}\frac{1-a}{c}\right)^{2} +
          \frac{b}{c}}}\right| < 1 & \text{if $\bar{x}=\bar{x}_{-}$}\\
        \end{cases}
      \end{equation}
      we conclude that
      $\bar{x}_{+}$ is asymptotically stable
      whereas
      $\bar{x}_{-}$ is not stable
      
  \item The linearized model allows us to reach the same conclusion
    without explicitly solving for the state trajectory
  \end{itemize}

\item \textbf{Example~4}
  \begin{itemize}
  \item Let us revisit Example~4 and consider the following two-dimensional dynamical system:
    \begin{equation}\nonumber%\label{eq:ex4}%
      \begin{bmatrix}
        x_{1,t+1} \\
        x_{2,t+1} \\
      \end{bmatrix}
      =
      \underbrace{
      \begin{bmatrix}
      a_{1} x_{1,t} + b_{1}x_{2,t}^{2} \\
      b_{2}x_{1,t} + a_{2} x_{2,t} \\
      \end{bmatrix}}_{\bm{f}(\bm{x}_{t})},
      \quad a_{1},a_{2}\neq 1,
      \quad b_{1},b_{2}\neq 0,
    \end{equation}
    which we know has two equilibrium points
    \begin{equation}\nonumber%\label{eq:}%
      \begin{bmatrix}
        \bar{x}_{1} \\
        \bar{x}_{2} \\
      \end{bmatrix}
      = 
      \begin{bmatrix}
        0 \\
        0 \\
      \end{bmatrix}
      \,\text{ or }\,
      \begin{bmatrix}
        \bar{x}_{1} \\
        \bar{x}_{2} \\
      \end{bmatrix}
      = 
      \begin{bmatrix}
        \frac{1-a_{1}}{b_{1}}\left(\frac{1-a_{2}}{b_{2}}\right)^{2} \\
        \frac{1-a_{1}}{b_{1}}\frac{1-a_{2}}{b_{2}} \\
      \end{bmatrix}
    \end{equation}
  \item We are not able to explicitly solve for the state trajectory here
    so we turn to linearization
  \item To fix context, assume $|a_{1}|<1$ and $|a_{2}|<1$
  \item Linearly approximating the system around an equilibrium point $\bar{\bm{x}}$ yields
    \begin{equation}\nonumber%\label{eq:ex4}%
      \begin{bmatrix}
        x_{1,t+1} - \bar{x}_{1} \\
        x_{2,t+1} - \bar{x}_{2} \\
      \end{bmatrix}
      =
      \underbrace{
      \begin{bmatrix}
      a_{1} & 2b_{1}\bar{x}_{2} \\
      b_{2} & a_{2} \\
      \end{bmatrix}}_{\frac{d\bm{f}(\bar{\bm{x}})}{d\bm{x}}}
      \begin{bmatrix}
      x_{1,t} - \bar{x}_{1} \\
      x_{2,t} - \bar{x}_{2} \\
      \end{bmatrix},
    \end{equation}
    which indicates that the equilibrium point is asymptotically stable
    if and only if
    \begin{equation}\nonumber%\label{eq:}%
      \rho \left(\frac{d\bm{f}(\bar{\bm{x}})}{d\bm{x}}\right) < 1
      \iff
      \text{eigenvalues of $\frac{d\bm{f}(\bar{\bm{x}})}{d\bm{x}}$ are within unit circle}
    \end{equation}

  \item For $\bar{\bm{x}}=(0, 0)^{\top}$, we have
    \begin{equation}\nonumber%\label{eq:}%
      \frac{d\bm{f}(\bar{\bm{x}})}{d\bm{x}}\bigg|_{
        \bar{\bm{x}}=
        \begin{bmatrix}
          0 \\
          0 \\
        \end{bmatrix}
      }
      =
      \begin{bmatrix}
        a_{1} & 0 \\
        b_{2} & a_{2} \\
      \end{bmatrix}
    \end{equation}
    and therefore
    the equilibrium point is asymptotically stable (since $|a_{1}|<1$ and $|a_{2}|<1$)

  \item For $\bar{\bm{x}}=\left(\frac{1-a_{1}}{b_{1}}\left(\frac{1-a_{2}}{b_{2}}\right)^{2}, \frac{1-a_{1}}{b_{1}}\frac{1-a_{2}}{b_{2}}\right)^{\top}$, we have
    \begin{equation}\nonumber%\label{eq:}%
      \frac{d\bm{f}(\bar{\bm{x}})}{d\bm{x}}\Bigg|_{
        \bar{\bm{x}}=
        \begin{bmatrix}
          \frac{1-a_{1}}{b_{1}}\left(\frac{1-a_{2}}{b_{2}}\right)^{2} \\
          \frac{1-a_{1}}{b_{1}}\frac{1-a_{2}}{b_{2}} \\
        \end{bmatrix}
      }
      =
      \begin{bmatrix}
        a_{1} & 2\frac{(1-a_{1})(1-a_{2})}{b_{2}} \\
        b_{2} & a_{2} \\
      \end{bmatrix}
    \end{equation}
    and observe
    \begin{itemize}
    \item The characteristic polynomial is
      \begin{equation}\nonumber%\label{eq:}%
        \phi_{\frac{d\bm{f}(\bar{\bm{x}})}{d\bm{x}}}(t)
        = (a_{1}-t)(a_{2}-t) - 2(1-a_{1})(1-a_{2})
      \end{equation}
      and in particular
      \begin{equation}\nonumber%\label{eq:}%
        \phi_{\frac{d\bm{f}(\bar{\bm{x}})}{d\bm{x}}}(1)
        = -(1-a_{1})(1-a_{2}) < 0
      \end{equation}
    \item Hence, one of the eigenvalue is strictly greater than $1$,
      implying that the equilibrium point is not stable
    \end{itemize}

  \end{itemize}

\end{itemize}

\end{document}
