\documentclass[12pt,a4paper]{article} 
\usepackage[T1]{fontenc}
\usepackage{textcomp}
\usepackage{makeidx}

\usepackage[sc, noBBpl]{mathpazo}
\usepackage{dsfont} % for indicator function
\usepackage[deluxe]{otf}
\usepackage[utf8]{inputenc}

\usepackage{amsmath}
\usepackage{amsfonts}
\usepackage{amssymb}
\usepackage{amsthm}
\usepackage{mathtools}
\usepackage{mathrsfs}

\usepackage{listings} % for programming codes

\usepackage{bm}
\usepackage[dvipdfmx]{color}
\pagestyle{plain} 
\usepackage{float}
\usepackage[dvipdfmx]{graphicx}

\usepackage[left=15mm, right=18mm, top=20mm, bottom=20mm]{geometry}

\setlength{\footskip}{40pt} 
\setlength{\abovecaptionskip}{0pt}
\setlength{\belowcaptionskip}{0pt}
\renewcommand{\baselinestretch}{1.05}

\newcommand{\argmax}{\mathop{\rm arg max}\limits}
\newcommand{\argmin}{\mathop{\rm arg min}\limits}
\newcommand{\st}{\mathop{\rm s.t.}\limits}
\newcommand{\fsize}[1]{\fontsize{#1}{#1}\selectfont}

% custom color
\usepackage[dvipdfmx]{xcolor}

% colors
\definecolor{darkred}{rgb}{.40,.00,.00}
\definecolor{navy}{rgb}{.00,.00,.40}
\definecolor{darknavy}{rgb}{.00,.00,.30}
\definecolor{lightnavy}{rgb}{.10,.10,.40}

\newcommand{\navy}{\color{navy}}
\newcommand{\deepred}{\color{darkred}}
\newcommand{\emred}[1]{\textbf{\deepred #1}}
\newcommand{\emblue}[1]{\textbf{\navy #1}}

% title and tableofcontents
\usepackage[titles]{tocloft}
\setlength{\cftbeforesecskip}{2pt}
\setlength{\cftbeforesubsecskip}{0pt}
\renewcommand{\cftsecfont}{\normalsize\mcfamily}
\renewcommand{\cftsubsecfont}{\normalsize\mcfamily}
\renewcommand{\cftsubsecfont}{\normalsize\mcfamily}
\renewcommand{\contentsname}{\large\mcfamily\bfseries \S\hspace{0.5\baselineskip}Contents\\[-\baselineskip]}
\cftsetindents{section}{10.0pt}{15.0pt}
\cftsetindents{subsection}{20.0pt}{30.0pt}
\cftpagenumbersoff{section}
\cftpagenumbersoff{subsection}
\cftpagenumbersoff{subsubsection}

\makeatletter
\def\vhrulefill#1{\leavevmode\leaders\hrule\@height#1\hfill \kern\z@}
\makeatother

\makeatletter
\def\maketitle{%
	\begin{center}\leavevmode
	\normalfont
	{\normalsize\raggedleft \@date\par}%
	\vskip -0.7\baselineskip
	\vhrulefill{0.2pt}\par
	\vskip 0.3\baselineskip
	\renewcommand{\baselinestretch}{0.80}
	{\color{darknavy}\bfseries\Large\raggedright \@title\par}%
	\renewcommand{\baselinestretch}{1.00}
	\vskip 0.2\baselineskip
	\ifx\@subtitle\@empty\else%
	{\normalsize\raggedright \@subtitle\par}%
	\fi
	\ifx\@author\@empty\else%
	\vskip 0.5\baselineskip
	{\normalsize\raggedleft \@author\par}%
	\fi
	\end{center}%
	\vskip 1.0\baselineskip
}
\def\subtitle#1{\gdef\@subtitle{#1}}
\let\@subtitle\@empty
\makeatother

% section
\newcommand{\Secindent}{0.0\baselineskip}
\newcommand{\Subsecindent}{0.0\baselineskip}
\newcommand{\lmargini}{1.3\baselineskip}
\newcommand{\lmarginii}{0.5\baselineskip}
\newcommand{\lmarginiii}{0.2\baselineskip}
\def\theenumi{\arabic{enumi}}
\def\theenumii{\alph{enumii}}
\renewcommand{\labelenumi}{\hfill\hspace{60pt}\theenumi.}
\renewcommand{\labelenumii}{\hspace{30pt}\theenumii.}
\renewcommand{\labelitemi}{\small\textbullet\hspace{2pt}}
\renewcommand{\labelitemii}{\color{gray}\small$\circ$\hspace{2pt}}
\renewcommand{\labelitemiii}{\textendash\hspace{1pt}}
\makeatletter
\renewcommand{\section}{\@startsection%
	{section}{1}{\Secindent}% name, depth, indent \z@=0pt
	{-1.0\baselineskip \@plus1.5\baselineskip \@minus0.5\baselineskip}% beforeskip (no indent if negative)
	{0.5\baselineskip \@plus0.8\baselineskip \@minus0.3\baselineskip}% afterskip (no line break if negative)
	{\reset@font\large\bfseries\scshape\color{darknavy}}% font
}
\renewcommand{\subsection}{\@startsection%
	{subsection}{2}{\Subsecindent}%
	{-0.5\baselineskip \@plus0.7\baselineskip \@minus0.4\baselineskip}%
	{0.3\baselineskip \@plus0.5\baselineskip \@minus0.2\baselineskip}%
	{\reset@font\normalsize\bfseries\scshape\color{darknavy}}% 
}
\renewcommand{\thesection}{\@arabic\c@section} % arabic, roman, Roman, alph, Alph, fnsymbol
\renewcommand{\thesubsection}{\thesection\hspace{0pt}.\hspace{0pt}\@arabic\c@subsection}
\makeatother

\makeatletter
\def\@listi{%
	\topsep = 0.2\baselineskip
	\partopsep = 0.0\baselineskip
	\parsep = 0.0\baselineskip
	\itemsep = 0.5\baselineskip
	\leftmargin = 1.5zw
	\labelsep = 0.5zw
	\itemindent = 0pt
	\labelwidth\leftmargin \advance\labelwidth-\labelsep
	\listparindent = 0zw
	\rightmargin = 0pt
	\reset@font\normalsize\normalfont
}
\let\@listI\@listi
\@listi
\def\@listii{%
	\topsep = 0.25\baselineskip
	\partopsep = 0.0\baselineskip
	\parsep = 0.0\baselineskip
	\itemsep = 0.3\baselineskip
	\leftmargin = 0.5zw
	\labelsep = .5zw
	\itemindent = 0pt
	\labelwidth\leftmargin \advance\labelwidth-\labelsep
	\listparindent = 0zw
	\rightmargin = 0pt
	\reset@font\normalsize\normalfont%
}
\def\@listiii{%
	\topsep = 0.30\baselineskip
	\partopsep = 0.0\baselineskip
	\parsep = 0.1\baselineskip
	\itemsep = 0.05\baselineskip
	\leftmargin = 1.1zw
	\labelsep = .5zw
	\itemindent = 0pt
	\labelwidth\leftmargin \advance\labelwidth-\labelsep
	\listparindent = 0zw
	\rightmargin = 0pt
	\reset@font\normalsize\normalfont%
}
\def\@listiv{%
	\topsep = 0.0\baselineskip
	\partopsep = 0.0\baselineskip
	\parsep = 0.0\baselineskip
	\itemsep = 0.0\baselineskip
	\leftmargin = 0.9zw
	\labelsep = .5zw
	\itemindent = 0pt
	\labelwidth\leftmargin \advance\labelwidth-\labelsep
	\listparindent = 0zw
	\rightmargin = 0pt
	\reset@font\normalsize\normalfont%
}
\makeatother

\newcommand{\R}{\ensuremath{\mathbb{R}}}
\newcommand{\E}{\ensuremath{\mathbb{E}}}
\newcommand{\V}{\ensuremath{\mathbb{V}}}
\newcommand{\Cov}{\ensuremath{\text{Cov}}}
\newcommand{\N}{\ensuremath{\mathbb{N}}}
\newcommand{\Z}{\ensuremath{\mathbb{Z}}}
\let\epsilon\varepsilon

\let\origitem\item
\renewcommand{\item}{\normalfont\origitem}

\usepackage[%
backref=page,
dvipdfmx,
bookmarks=true,bookmarksnumbered=true,bookmarkstype=toc,
colorlinks=true,
linkcolor=navy,
citecolor=navy,
filecolor=blue,
pagecolor=blue,
urlcolor=blue
]{hyperref}
\AtBeginDvi{\special{pdf:tounicode EUC-UCS2}}

\renewcommand*{\backref}[1]{}
\renewcommand*{\backrefalt}[4]{%
  \ifcase #1 (Not cited.)%
  \or        [#2]%
  \else      [#2]%
  \fi}
 

\title{Eigenvalues and eigenvectors}
\subtitle{Introduction to dynamical systems~\#2}
\author{Hiroaki Sakamoto}
\date{\today}

\DeclareMathOperator{\vect}{vec}
 
\begin{document}
\maketitle
\tableofcontents

\section{Eigenvalues and eigenvectors}

\subsection{Definition}

\begin{itemize}

\item \textbf{Definition of eigenvalues and eigenvectors}
  \begin{itemize}
  \item Let $\bm{A}\in \R^{n\times n}$ be a square matrix.
  \item If a real value $\lambda\in \R$
    and a non-zero vector $\bm{v}\in\R^{n}\setminus\{\bm{0}\}$ jointly satisfy
    \begin{equation}\nonumber%\label{eq:}%
      \bm{A}\bm{v} = \lambda \bm{v},
    \end{equation}
    we say that $\lambda$ is an \emph{eigenvalue} of $\bm{A}$
    and $\bm{v}$ is an \emph{eigenvector} of $\bm{A}$ associated with $\lambda$
    (and we also call $(\lambda, \bm{v})$ an \emph{eigenpair} of $\bm{A}$)
  \item Notice:
    \begin{itemize}
    \item If $\bm{v}_{1}$ and $\bm{v}_{2}$ are both eigenvectors associated with the same eigenvalue $\lambda$ of $\bm{A}$,
      then so is their linear combination $\alpha_{1}\bm{v}_{2}+ \alpha_{2}\bm{v}_{2}$ because
      \begin{equation}\nonumber%\label{eq:}%
        \bm{A}\left(\alpha_{1}\bm{v}_{1}+ \alpha_{2}\bm{v}_{2}\right)
        = \left(\alpha_{1}\bm{A}\bm{v}_{1}+ \alpha_{2}\bm{A}\bm{v}_{2}\right)
        = \left(\alpha_{1}\lambda\bm{v}_{1}+ \alpha_{2}\lambda\bm{v}_{2}\right)
        = \lambda \left(\alpha_{1}\bm{v}_{1}+ \alpha_{2}\bm{v}_{2}\right)
      \end{equation}
    \item If $(\lambda, \bm{v})$ is an eigenpair of $\bm{A}$,
      then $(\lambda^{k}, \bm{v})$ is an eigenpair of $\bm{A}^{k}$ for any $k\in \N$
      because
      $\bm{A}^{k}\bm{v} = \bm{A}^{k-1}\bm{A}\bm{v}  = \lambda\bm{A}^{k-1}\bm{v} = \lambda^{2}\bm{A}^{k-2}\bm{v} = \cdots = \lambda^{k}\bm{A}^{k-k}\bm{v} = \lambda^{k}\bm{v}$
    \item Provided that $\bm{A}$ is non-singular (which is the case if and only if $0$ is not an eigenvalue of $\bm{A}$; See below),
      \begin{equation}\nonumber%\label{eq:}%
        \text{$(\lambda,\bm{v})$ is an eigenpair of $\bm{A}$}
        \iff
        \text{$(\lambda^{-1}, \bm{v})$ is an eigenpair of $\bm{A}^{-1}$}
      \end{equation}

  \item If $\bm{A}\in \R^{m\times m}$ has eigenpairs $(\lambda_{1},\bm{v}_{1}),(\lambda_{2},\bm{v}_{2}),\ldots,(\lambda_{m},\bm{v}_{m})$
      and $\bm{B}\in \R^{p\times p}$ has eigenpairs $(\mu_{1},\bm{u}_{1}),(\mu_{2},\bm{u}_{2}),\ldots,(\mu_{p},\bm{u}_{p})$,
      then, for any $i=1,2, \ldots, m$ and $j=1,2, \ldots, p$,
      \begin{equation}\nonumber%\label{eq:}%
        (\bm{A}\otimes \bm{B}) (\bm{v}_{i}\otimes \bm{u}_{j})
        =
        (\bm{A}\bm{v}_{i}\otimes \bm{B}\bm{u}_{j})
        =
        (\lambda_{i}\bm{v}_{i}\otimes \mu_{j}\bm{u}_{j})
        =
        \lambda_{i}\mu_{j} (\bm{v}_{i}\otimes \bm{u}_{j}),
      \end{equation}
      meaning that $(\lambda_{i}\mu_{j}, \bm{v}_{i}\otimes \bm{u}_{j})$ is an eigenpair of $\bm{A}\otimes \bm{B}$,
      which implies that $\bm{A}\otimes \bm{B}\in \R^{mp\times mp}$ has the following $mp$ eigenvalues
      \begin{equation}\nonumber%\label{eq:}%
        \lambda_{i}\mu_{j} \quad \forall i = 1, 2, \ldots, m, \quad \forall j= 1,2, \ldots, p
      \end{equation}
    \end{itemize}
  \end{itemize}

\clearpage
\item \textbf{Example}
  \begin{itemize}
  \item Consider a square matrix
    \begin{equation}\nonumber%\label{eq:}%
      \bm{A} :=
      \begin{bmatrix}
        5/2 & -1 \\
        1 & 0
      \end{bmatrix}
    \end{equation}
  \item Then
    $\lambda_{1}:=2$ is an eigenvalue of $\bm{A}$
    and $\bm{v}_{1}:=(2, 1)^{\top}$ is an eigenvector of $\bm{A}$ associated with $\lambda_{1}$
    because
    \begin{equation}\nonumber%\label{eq:}%
      \bm{A}
      \bm{v}_{1}
      = 
      \begin{bmatrix}
        5/2 & -1 \\
        1 & 0
      \end{bmatrix}
      \begin{bmatrix}
        2 \\
        1
      \end{bmatrix}
      = 
      \begin{bmatrix}
        4 \\
        2
      \end{bmatrix}
      = 2
      \begin{bmatrix}
        2 \\
        1
      \end{bmatrix}
      = \lambda_{1}\bm{v}_{1}
    \end{equation}

  \item Also,
    $\lambda_{2}:=1/2$ is another eigenvalue of $\bm{A}$
    and $\bm{v}_{2}:=(1, 2)^{\top}$ is an eigenvector of $\bm{A}$ associated with $\lambda_{2}$
    because
    \begin{equation}\nonumber%\label{eq:}%
      \bm{A}
      \bm{v}_{2}
      = 
      \begin{bmatrix}
        5/2 & -1 \\
        1 & 0
      \end{bmatrix}
      \begin{bmatrix}
        1 \\
        2
      \end{bmatrix}
      = 
      \begin{bmatrix}
        1/2 \\
        1
      \end{bmatrix}
      = \frac{1}{2}
      \begin{bmatrix}
        1 \\
        2
      \end{bmatrix}
      = \lambda_{2}\bm{v}_{2}
    \end{equation}
  \end{itemize}

\end{itemize}

\subsection{Characteristic polynomials}

\begin{itemize}

\item \textbf{Definition of characteristic polynomials}
  \begin{itemize}
  \item Let $\bm{A}\in \R^{n\times n}$ be a square matrix.
  \item Define a function $\phi_{\bm{A}}:\R \to \R$ by
    \begin{equation}\nonumber%\label{eq:}%
      \phi_{\bm{A}}(t) := |\bm{A}-t\bm{I}| \quad \forall t\in \R,
    \end{equation}
    where $\bm{I}\in \R^{n\times n}$ is the identity matrix.
    We call $\phi_{\bm{A}}$ the \emph{characteristic polynomial} of $\bm{A}$
  \end{itemize}

\item \textbf{Useful results}

  \begin{itemize}
  \item Note that $\lambda$ is an eigenvalue of $\bm{A}$
    if and only if $\phi_{\bm{A}}(\lambda)=0$ because:
    \begin{itemize}
    \item $|\bm{A}-\lambda \bm{I}| = 0$ iff $(\bm{A}-\lambda\bm{I})$ is singular (i.e., not invertible)
    \item $(\bm{A}-\lambda\bm{I})$ is singular iff column vectors of $(\bm{A}-\lambda\bm{I})$ are linearly dependent
    \item column vectors of $(\bm{A}-\lambda\bm{I})$ are linearly dependent iff $(\bm{A}-\lambda\bm{I})\bm{v}=\bm{0}$ for some $\bm{v}\neq \bm{0}$
    \end{itemize}
  \item In general, $\bm{A}\in \R^{n\times n}$ has $m\leq n$ distinct eigenvalues $\lambda_{1},\ldots, \lambda_{m}$ if and only if
    \begin{equation}\nonumber%\label{eq:}%
      \phi_{\bm{A}}(t) = (\lambda_{1}-t)^{k_{1}}(\lambda_{2}-t)^{k_{2}}\cdots(\lambda_{m}-t)^{k_{m}},
    \end{equation}
    where $k_{i}\in \N$ is called the \emph{algebraic multiplicity} of $\lambda_{i}$,
    satisfying $k_{1}+k_{2}+\ldots+k_{m}=n$
  \item An immediate consequence:
    \begin{equation}\nonumber%\label{eq:}%
      \text{$\bm{A}$ is singular}
      \iff
      |\bm{A}| = 0
      \iff \phi_{\bm{A}}(0) = 0
      \iff \text{$0$ is an eigenvalue of $\bm{A}$}
    \end{equation}
  \item If $\bm{A}$ is a triangle matrix,
    the eigenvalues of $\bm{A}$ are the diagonal elements of $\bm{A}$ because
    \begin{equation}\nonumber%\label{eq:}%
      \bm{A} =
      \begin{bmatrix}
        a_{11} & a_{12} & \cdots & a_{1n} \\
        0 & a_{22} & \cdots & a_{2n} \\
        \vdots & \vdots & \ddots & \vdots \\
        0  & 0 & \cdots & a_{nn} \\
      \end{bmatrix}
      \implies
      \phi_{\bm{A}}(t)
      = (a_{11}-t)(a_{22}-t)\cdots (a_{nn}-t)
    \end{equation}
  \item The determinant of $\bm{A}$ equals the product of all eigenvalues of $\bm{A}$ (including duplicates):
    \begin{equation}\nonumber%\label{eq:}%
      |\bm{A}| = |\bm{A}-0\bm{I}|
      = \phi_{\bm{A}}(0)
      = (\lambda_{1}-0)^{k_{1}}(\lambda_{2}-0)^{k_{2}}\cdots(\lambda_{m}-0)^{k_{m}}
      = \lambda_{1}^{k_{1}}\lambda_{2}^{k_{2}}\cdots\lambda_{m}^{k_{m}}
    \end{equation}
    
  \end{itemize}

\item \textbf{Examples}
  \begin{itemize}
  \item Consider a square matrix
    \begin{equation}\nonumber%\label{eq:}%
      \bm{A} :=
      \begin{bmatrix}
        5/2 & -1 \\
        1 & 0
      \end{bmatrix},
    \end{equation}
    whose characteristic polynomial is
    \begin{equation}\nonumber%\label{eq:}%
      \phi_{\bm{A}}(t) :=
      \left|
      \begin{bmatrix}
        5/2 & -1 \\
        1 & 0
      \end{bmatrix}
      - t
      \begin{bmatrix}
        1 & 0 \\
        0 & 1
      \end{bmatrix}
      \right|
      = 
      \begin{vmatrix}
        5/2 - t & -1 \\
        1 & - t
      \end{vmatrix}
      = - \left(\frac{5}{2} - t\right)t + 1
      = (t - 2) \left(t - \frac{1}{2}\right),
    \end{equation}
    meaning that
    $\lambda_{1}:=2$ and $\lambda_{1}:=1/2$
    are two eigenvalues of $\bm{A}$
    and their algebraic multiplicity is $1$

  \item Consider another square matrix
    \begin{equation}\nonumber%\label{eq:}%
      \bm{A} :=
      \begin{bmatrix}
        5 & 4 \\
        -1 & 1 \\
      \end{bmatrix},
    \end{equation}
    whose characteristic polynomial is
    \begin{equation}\nonumber%\label{eq:}%
      \phi_{\bm{A}}(t) =
      \begin{vmatrix}
        5-t & 4 \\
        -1 & 1-t \\
      \end{vmatrix}
      =
      (t-3)^{2},
    \end{equation}
    which means that
    $\lambda_{1}:=3$
    is the unique eigenvalue of $\bm{A}$
    and its algebraic multiplicity is $2$

  \end{itemize}

\end{itemize}

%\clearpage
\subsection{Solving for eigenvectors}

\begin{itemize}

\item \textbf{Procedure}
  \begin{enumerate}
  \item Given a square matrix $\bm{A}\in \R^{n\times n}$,
    find eigenvalues $\lambda_{1},\lambda_{2},\ldots, \lambda_{n}$
    of $\bm{A}$
    by solving $\phi_{\bm{A}}(\lambda)=0$ for $\lambda$
  \item For each $\lambda_{i}$,
    solve the linear system of equations $\bm{A}\bm{v} = \lambda_{i}\bm{v}$
    for $\bm{v}\in\R^{n}\setminus\{\bm{0}\}$,
    i.e.,
    \begin{equation}\nonumber%\label{eq:}%
      \left(\bm{A} - \lambda_{i}\bm{I}\right)\bm{v} = \bm{0}
      \iff
      \bm{v} = \ldots,
    \end{equation}
    which is an eigenvector of $\bm{A}$ associated with $\lambda_{i}$
  \end{enumerate}

\item \textbf{Example}
  \begin{itemize}
  \item Consider a square matrix
    \begin{equation}\nonumber%\label{eq:}%
      \bm{A} :=
      \begin{bmatrix}
        5/2 & -1 \\
        1 & 0
      \end{bmatrix}
    \end{equation}
  \item We know that
    \begin{equation}\nonumber%\label{eq:}%
      \phi_{\bm{A}}(\lambda) = 0
      \iff
      \lambda = 2, \frac{1}{2}
    \end{equation}
    so let $\lambda_{1}:=2$, $\lambda_{2}:=1/2$
  \item Solve $(\bm{A}-\lambda_{1}\bm{I})\bm{v}=\bm{0}$ for $\bm{v}$, i.e.,
    \begin{equation}\nonumber%\label{eq:}%
      \begin{bmatrix}
        5/2-2 & -1 \\
        1 & - 2
      \end{bmatrix}
      \begin{bmatrix}
        v_{1} \\
        v_{2} \\
      \end{bmatrix}
      = 
      \begin{bmatrix}
        0 \\
        0 \\
      \end{bmatrix}
      \iff
      v_{1} = 2v_{2}
      \iff
      \begin{bmatrix}
        v_{1} \\
        v_{2} \\
      \end{bmatrix}
      =
      \alpha
      \begin{bmatrix}
        2\\
        1
      \end{bmatrix}
      \quad \forall \alpha \in \R,
    \end{equation}
    meaning that $\bm{v}_{1}:= (2, 1)^{\top}$
    is an eigenvector of $\bm{A}$ associated with $\lambda_{1}=2$
  \item Similarly, solve $(\bm{A}-\lambda_{2}\bm{I})\bm{v}=\bm{0}$ for $\bm{v}$, i.e.,
    \begin{equation}\nonumber%\label{eq:}%
      \begin{bmatrix}
        5/2-1/2 & -1 \\
        1 & - 1/2
      \end{bmatrix}
      \begin{bmatrix}
        v_{1} \\
        v_{2} \\
      \end{bmatrix}
      = 
      \begin{bmatrix}
        0 \\
        0 \\
      \end{bmatrix}
      \iff
      v_{1} = \frac{1}{2}v_{2}
      \iff
      \begin{bmatrix}
        v_{1} \\
        v_{2} \\
      \end{bmatrix}
      =
      \alpha
      \begin{bmatrix}
        1\\
        2
      \end{bmatrix}
      \quad \forall \alpha \in \R,
    \end{equation}
    meaning that $\bm{v}_{2}:= (1,2)^{\top}$
    is an eigenvector of $\bm{A}$ associated with $\lambda_{2}=1/2$

  \end{itemize}

\end{itemize}

\section{Diagonalization}

\subsection{Definition}

\begin{itemize}
\item \textbf{Definition of diagonalizable matrices}
  \begin{itemize}
  \item A square matrix $\bm{A}\in \R^{n\times n}$ is said to be \emph{diagonalizable}
    if there exists a nonsingular (i.e., invertible) matrix $\bm{V}\in \R^{n\times n}$
    and a diagonal matrix $\bm{\Lambda}\in \R^{n\times n}$ such that
    \begin{equation}\nonumber%\label{eq:}%
      \bm{A} = \bm{V}\bm{\Lambda}\bm{V}^{-1}
    \end{equation}
  \item $\bm{\Lambda}$ is the `simplest' matrix that is \emph{similar} to $\bm{A}$
  \end{itemize}
\item \textbf{Remark}
  \begin{itemize}
  \item Let $\lambda_{1},\lambda_{2},\ldots, \lambda_{n}$ be (arbitrarily chosen) $n$ eigenvalues of $\bm{A}$ 
    and let $\bm{v}_{1}, \bm{v}_{2}, \ldots, \bm{v}_{n}$ be the associated eigenvectors
  \item Let $\bm{\Lambda}:=\text{diag}(\lambda_{1},\ldots, \lambda_{n})\in \R^{n\times n}$
    and $\bm{V}:=[\bm{v}_{1} \ldots \bm{v}_{n}]\in \R^{n\times n}$
  \item Since $\bm{A}\bm{v}_{i} = \lambda_{i}\bm{v}_{i}$ for $i=1,2,\ldots, n$, we have
    \begin{equation}\nonumber%\label{eq:}%
      \bm{A}
      \begin{bmatrix}
        \bm{v}_{1} & \bm{v}_{2} & \ldots & \bm{v}_{n}
      \end{bmatrix}
      =
      \begin{bmatrix}
        \bm{A}\bm{v}_{1} & \bm{A}\bm{v}_{2} & \ldots & \bm{A}\bm{v}_{n}
      \end{bmatrix}
      =
      \begin{bmatrix}
        \bm{v}_{1} & \bm{v}_{2} & \ldots & \bm{v}_{n}
      \end{bmatrix}
      \begin{bmatrix}
        \lambda_{1} & 0 & \dots & 0 \\
        0 & \lambda_{2} & \dots & 0 \\
        \vdots & \vdots & \ddots & \vdots \\
        0 & 0 & \dots & \lambda_{n} \\
      \end{bmatrix}
    \end{equation}
    or $\bm{A}\bm{V} = \bm{V}\bm{\Lambda}$
    
  \item Thus, $\bm{A}$ is diagonalizable whenever $\bm{V}$ is non-singular
  \item $\bm{V}$ is non-singular if and only if its column vectors are linearly independent
  \item $\bm{A}$ is diagonalizable whenever \textbf{one can find $n$ linearly independent eigenvectors of $\bm{A}$}
  \end{itemize}

\item \textbf{A sufficient condition for diagonalization}
  \begin{itemize}
  \item A square matrix $\bm{A}\in \R^{n\times n}$ is diagonalizable
    if it has $n$ distinct eigenvalues\footnote{%
      This is a sufficient, but not necessary, condition for a matrix to be diagonalizable.
    }
  \item Let $\lambda_{1},\lambda_{2},\ldots, \lambda_{n}$ be the $n$ distinct eigenvalues of $\bm{A}$
    and let $\bm{v}_{1}, \bm{v}_{2}, \ldots, \bm{v}_{n}$ be the associated eigenvectors
    so that
    \begin{equation}\nonumber%\label{eq:}%
      \bm{A}\bm{V} = \bm{V}\bm{\Lambda}
    \end{equation}
  \item Since $\lambda_{1},\ldots, \lambda_{n}$ are distinct from each other,
    $\bm{v}_{1}, \ldots, \bm{v}_{n}$ are linearly independent:
    \begin{itemize}
    \item If $\bm{v}_{i}$ and $\bm{v}_{j}$ are linearly dependent,
      there exists $c\neq 0$ such that $c\bm{v}_{i}=\bm{v}_{j}$ and thus
      \begin{equation}\nonumber%\label{eq:}%
        c(\lambda_{i}\bm{v}_{i}) =
        c(\bm{A}\bm{v}_{i}) = 
        \bm{A}(c\bm{v}_{i}) = 
        \bm{A}\bm{v}_{j} = 
        \lambda_{j}\bm{v}_{j} = 
        \lambda_{j}(c\bm{v}_{i}) =
        c(\lambda_{j}\bm{v}_{i}),
      \end{equation}
      which, because $c\neq 0$ and $\bm{v}_{i}\neq \bm{0}$, implies $\lambda_{i}=\lambda_{j}$, a contradiction
    \item By induction, one can conclude $\bm{v}_{1}, \ldots, \bm{v}_{n}$ are linearly independent
    \end{itemize}
  \item Since $\bm{v}_{1}, \ldots, \bm{v}_{n}$ are linearly independent,
    the matrix $\bm{V}=[\bm{v}_{1}\ldots \bm{v}_{n}]$ is nonsingular
  \item Then there exists the inverse $\bm{V}^{-1}$ and therefore
    \begin{equation}\nonumber%\label{eq:}%
      \bm{A} = \bm{A}\bm{V}\bm{V}^{-1} = \bm{V}\bm{\Lambda}\bm{V}^{-1},
    \end{equation}
    meaning that $\bm{A}$ is diagonalizable
  \end{itemize}

\item \textbf{Similar matrices}
  \begin{itemize}
  \item We say that two matrices, $\bm{A}$ and $\bm{B}$, are \emph{similar}
    if there exits a non-singular $\bm{C}$ such that
    \begin{equation}\nonumber%\label{eq:}%
      \bm{A} = \bm{C}\bm{B}\bm{C}^{-1}
    \end{equation}
  \item If $\bm{A}$ and $\bm{B}$ are similar, then
    \begin{itemize}
    \item $|\bm{A}| = |\bm{B}|$ because
      \begin{equation}\nonumber%\label{eq:}%
        |\bm{A}|
        =
        |\bm{C}\bm{B}\bm{C}^{-1}|
        =
        |\bm{C}||\bm{B}||\bm{C}^{-1}|
        =
        |\bm{C}||\bm{B}||\bm{C}|^{-1}
        = |\bm{B}|
      \end{equation}
    \item $\bm{A}$ and $\bm{B}$ have the same characteristic polynomial because
      \begin{equation}\nonumber%\label{eq:}%
        \phi_{\bm{A}}(t)
        = |\bm{A}-t\bm{I}|
        = |\bm{C}\bm{B}\bm{C}^{-1}-t\bm{I}|
        = |\bm{C}||\bm{B}-t\bm{I}||\bm{C}^{-1}|
        = |\bm{B}-t\bm{I}|
        =
        \phi_{\bm{B}}(t)
        \quad \forall t \in \R
      \end{equation}
    \item $\bm{A}$ and $\bm{B}$ have the same set of eigenvalues
    \item $\bm{A}$ is diagonalizable if and only if $\bm{B}$ is diagonalizable
    \end{itemize}

  \end{itemize}

\end{itemize}

\subsection{Examples}

\begin{itemize}

\item \textbf{Example~1}
  \begin{itemize}
  \item Consider a square matrix
    \begin{equation}\nonumber%\label{eq:}%
      \bm{A} :=
      \begin{bmatrix}
        5/2 & -1 \\
        1 & 0
      \end{bmatrix},
    \end{equation}
    which we know has $\lambda_{1}=2, \lambda_{2}=1/2$ as eigenvalues
    and $\bm{v}_{1}=(2, 1)^{\top}, \bm{v}_{2}=(1, 2)^{\top}$ as associated eigenvectors
  \item Define
    \begin{equation}\nonumber%\label{eq:}%
      \bm{V}:=
      \begin{bmatrix}
        \bm{v}_{1} & \bm{v}_{2}
      \end{bmatrix}
      =
      \begin{bmatrix}
        2 & 1 \\
        1 & 2 \\
      \end{bmatrix},
      \quad
      \bm{\Lambda}:=
      \begin{bmatrix}
        \lambda_{1} & 0 \\
        0 & \lambda_{2} \\
      \end{bmatrix}
      =
      \begin{bmatrix}
        2 & 0 \\
        0 & 1/2 \\
      \end{bmatrix}
    \end{equation}
  \item Then
    \begin{equation}\nonumber%\label{eq:}%
      |\bm{V}| = 3,
      \quad
      \bm{V}^{-1}
      = \frac{1}{|\bm{V}|}
      \begin{bmatrix}
        2 & -1 \\
        -1 & 2 \\
      \end{bmatrix}
      =
      \begin{bmatrix}
        2/3 & -1/3 \\
        -1/3 & 2/3 \\
      \end{bmatrix}
    \end{equation}
    and therefore
    \begin{equation}\nonumber%\label{eq:}%
      \bm{V}\bm{\Lambda}\bm{V}^{-1}
      = 
      \begin{bmatrix}
        2 & 1 \\
        1 & 2 \\
      \end{bmatrix}
      \begin{bmatrix}
        2 & 0 \\
        0 & 1/2 \\
      \end{bmatrix}
      \begin{bmatrix}
        2/3 & -1/3 \\
        -1/3 & 2/3 \\
      \end{bmatrix}
      = 
      \begin{bmatrix}
        5/2 & -1 \\
        1 & 0 \\
      \end{bmatrix}
      = \bm{A}
    \end{equation}
  \end{itemize}
  
\item \textbf{Example~2}
  \begin{itemize}
  \item Consider a square matrix
    \begin{equation}\nonumber%\label{eq:}%
      \bm{A} :=
      \begin{bmatrix}
        1 & 0 & -1 \\
        1 & 2 & 1 \\
        2 & 2 & 3 \\
      \end{bmatrix}
    \end{equation}
  \item The characteristic polynomial is
    \begin{equation}\nonumber%\label{eq:}%
      \phi_{\bm{A}}(t) =
      \begin{vmatrix}
        1-t & 0 & -1 \\
        1 & 2-t & 1 \\
        2 & 2 & 3-t \\
      \end{vmatrix}
      = (1-t)(2-t)(3-t),
    \end{equation}
    which means that
    $\lambda_{1}:=1$,
    $\lambda_{2}:=2$,
    $\lambda_{3}:=3$
    are the eigenvalues of $\bm{A}$
  \item Solving $\bm{A}\bm{v}=\lambda_{1}\bm{v}$ for $\bm{v}$ yields
    \begin{equation}\nonumber%\label{eq:}%
      \begin{bmatrix}
        1-1 & 0 & -1 \\
        1 & 2-1 & 1 \\
        2 & 2 & 3-1 \\
      \end{bmatrix}
      \begin{bmatrix}
        v_{1} \\
        v_{2} \\
        v_{3} \\
      \end{bmatrix}
      = \bm{0}
      \iff
      \begin{matrix*}[l]
        v_{2} = -v_{1} \\
        v_{3} = 0 \\
      \end{matrix*}
      \iff
      \begin{bmatrix}
        v_{1} \\
        v_{2} \\
        v_{3} \\
      \end{bmatrix}
      =
      \alpha
      \begin{bmatrix}
        1 \\
        -1 \\
        0 \\
      \end{bmatrix},
      \quad \forall \alpha \in \R
    \end{equation}
    so we choose the following $\bm{v}_{1}$ as an eigenvector associated with $\lambda_{1}$:
    \begin{equation}\nonumber%\label{eq:}%
      \bm{v}_{1} := 
      \begin{bmatrix}
        1 \\
        -1 \\
        0 \\
      \end{bmatrix}
    \end{equation}
  \item Solving $\bm{A}\bm{v}=\lambda_{2}\bm{v}$ for $\bm{v}$ yields
    \begin{equation}\nonumber%\label{eq:}%
      \begin{bmatrix}
        1-2 & 0 & -1 \\
        1 & 2-2 & 1 \\
        2 & 2 & 3-2 \\
      \end{bmatrix}
      \begin{bmatrix}
        v_{1} \\
        v_{2} \\
        v_{3} \\
      \end{bmatrix}
      = \bm{0}
      \iff
      \begin{matrix*}[l]
        v_{1} = -2v_{2} \\
        v_{3} = -v_{1} \\
      \end{matrix*}
      \iff
      \begin{bmatrix}
        v_{1} \\
        v_{2} \\
        v_{3} \\
      \end{bmatrix}
      =
      \alpha
      \begin{bmatrix}
        -2 \\
        1 \\
        2 \\
      \end{bmatrix},
      \quad \forall \alpha \in \R
    \end{equation}
    so we choose the following $\bm{v}_{2}$ as an eigenvector associated with $\lambda_{2}$:
    \begin{equation}\nonumber%\label{eq:}%
      \bm{v}_{2} := 
      \begin{bmatrix}
        -2 \\
        1 \\
        2 \\
      \end{bmatrix}
    \end{equation}

  \item Solving $\bm{A}\bm{v}=\lambda_{3}\bm{v}$ for $\bm{v}$ yields
    \begin{equation}\nonumber%\label{eq:}%
      \begin{bmatrix}
        1-3 & 0 & -1 \\
        1 & 2-3 & 1 \\
        2 & 2 & 3-3 \\
      \end{bmatrix}
      \begin{bmatrix}
        v_{1} \\
        v_{2} \\
        v_{3} \\
      \end{bmatrix}
      = \bm{0}
      \iff
      \begin{matrix*}[l]
        v_{2} = -v_{1} \\
        v_{3} = -2v_{1} \\
      \end{matrix*}
      \iff
      \begin{bmatrix}
        v_{1} \\
        v_{2} \\
        v_{3} \\
      \end{bmatrix}
      =
      \alpha
      \begin{bmatrix}
        1 \\
        -1 \\
        -2 \\
      \end{bmatrix},
      \quad \forall \alpha \in \R
    \end{equation}
    so we choose the following $\bm{v}_{3}$ as an eigenvector associated with $\lambda_{3}$:
    \begin{equation}\nonumber%\label{eq:}%
      \bm{v}_{3} := 
      \begin{bmatrix}
        1 \\
        -1 \\
        -2 \\
      \end{bmatrix}
    \end{equation}

  \item Define
    \begin{equation}\nonumber%\label{eq:}%
      \bm{V}:=
      \begin{bmatrix}
        \bm{v}_{1} & \bm{v}_{2} & \bm{v}_{3}
      \end{bmatrix}
      =
      \begin{bmatrix}
        1 & -2 & 1 \\
        -1 & 1 & -1\\
        0 & 2 & -2 \\
      \end{bmatrix},
      \quad
      \bm{\Lambda}:=
      \begin{bmatrix}
        \lambda_{1} & 0 & 0 \\
        0 & \lambda_{2} & 0 \\
        0 & 0 & \lambda_{3} \\
      \end{bmatrix}
      =
      \begin{bmatrix}
        1 & 0 & 0 \\
        0 & 2 & 0 \\
        0 & 0 & 3 \\
      \end{bmatrix}
    \end{equation}

  \item Then
    \begin{equation}\nonumber%\label{eq:}%
      \bm{V}^{-1}
      =
      \begin{bmatrix}
        0 & -1 & 1/2 \\
        -1 & -1 & 0 \\
        -1 & -1 & -1/2 \\
      \end{bmatrix}
    \end{equation}
    and therefore
    \begin{equation}\nonumber%\label{eq:}%
      \bm{V}\bm{\Lambda}\bm{V}^{-1}
      = 
      \begin{bmatrix}
        1 & -2 & 1 \\
        -1 & 1 & -1\\
        0 & 2 & -2 \\
      \end{bmatrix}
      \begin{bmatrix}
        1 & 0 & 0 \\
        0 & 2 & 0 \\
        0 & 0 & 3 \\
      \end{bmatrix}
      \begin{bmatrix}
        0 & -1 & 1/2 \\
        -1 & -1 & 0 \\
        -1 & -1 & -1/2 \\
      \end{bmatrix}
      =
      \begin{bmatrix}
        1 & 0 & -1 \\
        1 & 2 & 1 \\
        2 & 2 & 3 \\
      \end{bmatrix}
      = \bm{A}
    \end{equation}

  \end{itemize}

\clearpage
\item \textbf{Example~3}
  \begin{itemize}
  \item Consider a square matrix
    \begin{equation}\nonumber%\label{eq:}%
      \bm{A} :=
      \begin{bmatrix}
        5 & 4 \\
        -1 & 1 \\
      \end{bmatrix}
    \end{equation}
  \item The characteristic polynomial is
    \begin{equation}\nonumber%\label{eq:}%
      \phi_{\bm{A}}(t) =
      \begin{vmatrix}
        5-t & 4 \\
        -1 & 1-t \\
      \end{vmatrix}
      =
      (3-t)^{2}
    \end{equation}
    which means that
    $\lambda_{1}:=3$
    is the unique eigenvalue of $\bm{A}$
  \item Solving $\bm{A}\bm{v}=\lambda_{1}\bm{v}$ for $\bm{v}$ yields
    \begin{equation}\nonumber%\label{eq:}%
      \begin{bmatrix}
        5-3 & 4 \\
        -1 & 1-3 \\
      \end{bmatrix}
      \begin{bmatrix}
        v_{1} \\
        v_{2} \\
      \end{bmatrix}
      = \bm{0}
      \iff
      v_{1} = -2v_{2}
      \iff
      \begin{bmatrix}
        v_{1} \\
        v_{2} \\
      \end{bmatrix}
      =
      \alpha
      \begin{bmatrix}
        2 \\
        -1 \\
      \end{bmatrix},
      \quad \forall \alpha \in \R,
    \end{equation}
    meaning that 
    we can only choose one linearly independent eigenvector

  \item Matrix $\bm{A}$ is not diagonalizable

  \end{itemize}

\item \textbf{Example~4}
  \begin{itemize}
  \item Consider a square matrix
    \begin{equation}\nonumber%\label{eq:}%
      \bm{A} :=
      \begin{bmatrix}
        1 & 3 & 0 \\
        0 & 1 & 0 \\
        2 & 1 & 5 \\
      \end{bmatrix}
    \end{equation}
  \item The characteristic polynomial is
    \begin{equation}\nonumber%\label{eq:}%
      \phi_{\bm{A}}(t) =
      \begin{vmatrix}
        1-t & 3 & 0 \\
        0 & 1-t & 0 \\
        2 & 1 & 5-t \\
      \end{vmatrix}
      =
      (1-t)^{2}(5-t)
    \end{equation}
    which means that
    $\lambda_{1}:=1$,
    $\lambda_{2}:=5$
    are the eigenvalues of $\bm{A}$
  \item Solving $\bm{A}\bm{v}=\lambda_{1}\bm{v}$ for $\bm{v}$ yields
    \begin{equation}\nonumber%\label{eq:}%
      \begin{bmatrix}
        1-1 & 3 & 0 \\
        0 & 1-1 & 0 \\
        2 & 1 & 5-1 \\
      \end{bmatrix}
      \begin{bmatrix}
        v_{1} \\
        v_{2} \\
        v_{3} \\
      \end{bmatrix}
      = \bm{0}
      \iff
      \begin{matrix*}[l]
        v_{2} = 0 \\
        v_{1} = -2v_{3} \\
      \end{matrix*}
      \iff
      \begin{bmatrix}
        v_{1} \\
        v_{2} \\
        v_{3} \\
      \end{bmatrix}
      =
      \alpha
      \begin{bmatrix}
        2 \\
        0 \\
        -1 \\
      \end{bmatrix},
      \quad \forall \alpha \in \R
    \end{equation}
    so we choose the following $\bm{v}_{1}$ as an eigenvector associated with $\lambda_{1}$:
    \begin{equation}\nonumber%\label{eq:}%
      \bm{v}_{1} := 
      \begin{bmatrix}
        2 \\
        0 \\
        -1 \\
      \end{bmatrix}
    \end{equation}

  \item Solving $\bm{A}\bm{v}=\lambda_{2}\bm{v}$ for $\bm{v}$ yields
    \begin{equation}\nonumber%\label{eq:}%
      \begin{bmatrix}
        1-5 & 3 & 0 \\
        0 & 1-5 & 0 \\
        2 & 1 & 5-5 \\
      \end{bmatrix}
      \begin{bmatrix}
        v_{1} \\
        v_{2} \\
        v_{3} \\
      \end{bmatrix}
      = \bm{0}
      \iff
      \begin{matrix*}[l]
        4v_{1} = 3v_{2} \\
        v_{2} = 0 \\
      \end{matrix*}
      \iff
      \begin{bmatrix}
        v_{1} \\
        v_{2} \\
        v_{3} \\
      \end{bmatrix}
      =
      \alpha
      \begin{bmatrix}
        0 \\
        0 \\
        1 \\
      \end{bmatrix},
      \quad \forall \alpha \in \R
    \end{equation}
    so we choose the following $\bm{v}_{2}$ as an eigenvector associated with $\lambda_{1}$:
    \begin{equation}\nonumber%\label{eq:}%
      \bm{v}_{2} := 
      \begin{bmatrix}
        0 \\
        0 \\
        1 \\
      \end{bmatrix}
    \end{equation}
  \item Matrix $\bm{A}$ is not diagonalizable

  \end{itemize}

\clearpage
\item \textbf{Example~5}
  \begin{itemize}
  \item Consider a square matrix
    \begin{equation}\nonumber%\label{eq:}%
      \bm{A} :=
      \begin{bmatrix}
        1 & 0 & 0 \\
        6 & -2 & -6 \\
        -2 & 1 & 3 \\
      \end{bmatrix}
    \end{equation}
  \item The characteristic polynomial is
    \begin{equation}\nonumber%\label{eq:}%
      \phi_{\bm{A}}(t) =
      \begin{vmatrix}
        1-t & 0 & 0 \\
        6 & -2-t & -6 \\
        -2 & 1 & 3-t \\
      \end{vmatrix}
      =
      -t(1-t)^{2}
    \end{equation}
    which means that
    $\lambda_{1}:=0$,
    $\lambda_{2}:=1$
    are the eigenvalues of $\bm{A}$
  \item Solving $\bm{A}\bm{v}=\lambda_{1}\bm{v}$ for $\bm{v}$ yields
    \begin{equation}\nonumber%\label{eq:}%
      \begin{bmatrix}
        1-0 & 0 & 0 \\
        6 & -2-0 & -6 \\
        -2 & 1 & 3-0 \\
      \end{bmatrix}
      \begin{bmatrix}
        v_{1} \\
        v_{2} \\
        v_{3} \\
      \end{bmatrix}
      = \bm{0}
      \iff
      \begin{matrix*}[l]
        v_{1} = 0 \\
        v_{2} = -3v_{3} \\
      \end{matrix*}
      \iff
      \begin{bmatrix}
        v_{1} \\
        v_{2} \\
        v_{3} \\
      \end{bmatrix}
      =
      \alpha
      \begin{bmatrix}
        0 \\
        -3 \\
        1 \\
      \end{bmatrix},
      \quad \forall \alpha \in \R
    \end{equation}
    so we choose the following $\bm{v}_{1}$ as an eigenvector associated with $\lambda_{1}$:
    \begin{equation}\nonumber%\label{eq:}%
      \bm{v}_{1} := 
      \begin{bmatrix}
        0 \\
        -3 \\
        1 \\
      \end{bmatrix}
    \end{equation}

  \item Solving $\bm{A}\bm{v}=\lambda_{2}\bm{v}$ for $\bm{v}$ yields
    \begin{equation}\nonumber%\label{eq:}%
      \begin{bmatrix}
        1-1 & 0 & 0 \\
        6 & -2-1 & -6 \\
        -2 & 1 & 3-1 \\
      \end{bmatrix}
      \begin{bmatrix}
        v_{1} \\
        v_{2} \\
        v_{3} \\
      \end{bmatrix}
      = \bm{0}
      \iff
      \begin{matrix*}[l]
        v_{1} =  \frac{1}{2}v_{2} + v_{3} \\
      \end{matrix*}
      \iff
      \begin{bmatrix}
        v_{1} \\
        v_{2} \\
        v_{3} \\
      \end{bmatrix}
      =
      \alpha
      \begin{bmatrix}
        1 \\
        2 \\
        0 \\
      \end{bmatrix}
      +
      \beta
      \begin{bmatrix}
        1 \\
        0 \\
        1 \\
      \end{bmatrix},
      \quad \forall \alpha, \beta
    \end{equation}
    so we choose the following as
    two linearly independent eigenvectors associated with $\lambda_{2}$:
    \begin{equation}\nonumber%\label{eq:}%
      \bm{v}_{2} := 
      \begin{bmatrix}
        1 \\
        2 \\
        0 \\
      \end{bmatrix},
      \quad
      \bm{v}_{3} := 
      \begin{bmatrix}
        1 \\
        0 \\
        1 \\
      \end{bmatrix}
    \end{equation}

  \item Note that
    \begin{itemize}
    \item $\bm{A}$ does not have $3$ distinct eigenvalues
    \item yet $\bm{A}$ has $3$ linearly independent eigenvectors (one for $\lambda_{1}$, two for $\lambda_{2}$)
    \item we say that $\lambda_{1}$ and $\lambda_{2}$ has \emph{geometric multiplicity} of $1$ and $2$, respectively
    \end{itemize}

  \item In fact, $\bm{A}$ is diagonalizable by defining
    \begin{equation}\nonumber%\label{eq:}%
      \bm{V}:=
      \begin{bmatrix}
        \bm{v}_{1} & \bm{v}_{2} & \bm{v}_{3}
      \end{bmatrix}
      =
      \begin{bmatrix}
        0 & 1 & 1 \\
        -3 & 2 & 0\\
        1 & 0 & 1 \\
      \end{bmatrix},
      \quad
      \bm{\Lambda}:=
      \begin{bmatrix}
        \lambda_{1} & 0 & 0 \\
        0 & \lambda_{2} & 0 \\
        0 & 0 & \lambda_{2} \\
      \end{bmatrix}
      =
      \begin{bmatrix}
        0 & 0 & 0 \\
        0 & 1 & 0 \\
        0 & 0 & 1 \\
      \end{bmatrix}
    \end{equation}

  \item Then
    \begin{equation}\nonumber%\label{eq:}%
      \bm{V}^{-1}
      =
      \begin{bmatrix}
        2 & -1 & -2 \\
        3 & -1 & -3 \\
        -2 & 1 & 3 \\
      \end{bmatrix}
    \end{equation}
    and therefore
    \begin{equation}\nonumber%\label{eq:}%
      \bm{V}\bm{\Lambda}\bm{V}^{-1}
      = 
      \begin{bmatrix}
        0 & 1 & 1 \\
        -3 & 2 & 0\\
        1 & 0 & 1 \\
      \end{bmatrix}
      \begin{bmatrix}
        0 & 0 & 0 \\
        0 & 1 & 0 \\
        0 & 0 & 1 \\
      \end{bmatrix}
      \begin{bmatrix}
        2 & -1 & -2 \\
        3 & -1 & -3 \\
        -2 & 1 & 3 \\
      \end{bmatrix}
      =
      \begin{bmatrix}
        1 & 0 & 0 \\
        6 & -2 & -6 \\
        -2 & 1 & 3 \\
      \end{bmatrix}
      = \bm{A}
    \end{equation}
  \end{itemize}

\end{itemize}

\end{document}
